%%%%%%%%%%%%%%%%%%%%%%%%%%%%%%%%%%%%%%%%%%%%%%%%%%%%%%%%
%%
\clearpage
\newpage
\section{The validation recipe}
\label{ch:the_recipes:cal_validate_spirou}
%%
%%%%%%%%%%%%%%%%%%%%%%%%%%%%%%%%%%%%%%%%%%%%%%%%%%%%%%%%

This recipe validates that the DRS has be installed correctly and insures that all recipes should run. This is mentioned in Section \ref{ch:install:validating_installunix}.
\begin{note}
The validation recipe does not protect against incorrect or missing constants and keywords. If all constants are defined as they were when installed and if all paths were set up correctly, then running the validation recipe should be enough to confirm that the DRS installed correctly. Again this does not protect against invalid files and other user inputs.
\end{note}

% -------------------------------------------------------
\subsection{The inputs}
% -------------------------------------------------------
The \calvalidate recipe requires no input. Thus run the validation as follows:
\begin{cmdbox}
cal_validate_spirou.py
\end{cmdbox}
\noindent or
\begin{pythonbox}
import cal_validate_spirou
cal_validate_spirou.main()
\end{pythonbox}
\noindent One can also define an optional argument to put the validation recipe into debug mode with:
\begin{pythonbox}
import cal_validate_spirou
cal_validate_spirou.main(DEBUG=1)
\end{pythonbox}
\noindent where a value of True or 1 runs the validation script in debugging mode. This lists all sub-module tests that are performed during the validation and thus allows problems to be identified more easily.

% -------------------------------------------------------
\subsection{The outputs}
% -------------------------------------------------------

There are no outputs to \calvalidate other than printing to screen and the log file (see Section \ref{ch:the_recipes:cal_validate_spirou:example_run}).

% -------------------------------------------------------
\subsection{Summary of procedure}
% -------------------------------------------------------
\begin{enumerate}
\item checks core module imports:
	\begin{itemize}
	\item SpirouDRS
	\item \spirouConfig
	\item \spirouCore
	\end{itemize}
\item checks that configuration files can be read
\item checks recipe modules
	\begin{itemize}
	\item \spirouBACK
	\item \spirouCDB
	\item \spirouEXTOR
	\item \spirouFLAT
	\item \spirouImage
	\item \spirouLOCOR
	\item \spirouRV
	\item \spirouStartup
	\item \spirouTHORCA
	\end{itemize}
\item displays and prints the configuration file paths
\item confirms validation of the DRS installation
\end{enumerate}

% -------------------------------------------------------
\newpage
\subsection{Example working run}
\label{ch:the_recipes:cal_validate_spirou:example_run}
% -------------------------------------------------------

An example run where everything worked is below:

\begin{cmdbox}
cal_validate_spirou.py
\end{cmdbox}
\begin{cmdboxprintspecial}[fontupper=\tiny, fontlower=\tiny]
 *****************************************
 *        VALIDATING DRS 
 *****************************************

1) Running core module tests

2) Running config test

         Testing /drs/INTROOT/config/config.py

                Congratulations all paths in /drs/INTROOT/config/config.py set up correctly.

2) Running sub-module tests

4) Running recipe test

@gHH:MM:SS.S -   || *****************************************@g
@gHH:MM:SS.S -   || * SPIROU \@(#) Geneva Observatory (VERSION)@g
@gHH:MM:SS.S -   || *****************************************@g
@gHH:MM:SS.S -   ||(dir_data_raw)      DRS_DATA_RAW=/drs/data/raw@g
@gHH:MM:SS.S -   ||(dir_data_reduc)    DRS_DATA_REDUC=/drs/data/reduced@g
@gHH:MM:SS.S -   ||(dir_calib_db)      DRS_CALIB_DB=/drs/data/calibDB@g
@gHH:MM:SS.S -   ||(dir_data_msg)      DRS_DATA_MSG=/drs/data/msg@g
@gHH:MM:SS.S -   ||(print_level)       PRINT_LEVEL=all         %(error/warning/info/all)@g
@gHH:MM:SS.S -   ||(log_level)         LOG_LEVEL=all         %(error/warning/info/all)@g
@gHH:MM:SS.S -   ||(plot_graph)        DRS_PLOT=1            %(def/undef/trigger)@g
@gHH:MM:SS.S -   ||(used_date)         DRS_USED_DATE=undefined@g
@gHH:MM:SS.S -   ||(working_dir)       DRS_DATA_WORKING=/drs/data/tmp@g
@gHH:MM:SS.S -   ||                    DRS_INTERACTIVE is not set, running on-line mode@g
@gHH:MM:SS.S -   ||                    DRS_DEBUG is set, debug mode level:1@g
@gHH:MM:SS.S -   ||@g
@gHH:MM:SS.S -   ||Validation successful. DRS installed corrected.@g

\end{cmdboxprintspecial}
