%%%%%%%%%%%%%%%%%%%%%%%%%%%%%%%%%%%%%%%%%%%%%%%%%%%%%%%%
%%
\chapter{Writing the documentation}
\label{ch:documentation}
%%
%%%%%%%%%%%%%%%%%%%%%%%%%%%%%%%%%%%%%%%%%%%%%%%%%%%%%%%%

%%%%%%%%%%%%%%%%%%%%%%%%%%%%%%%%%%%%%%%%%%%%%%%%%%%%%%%%
%%
\section{Documentation Architecture}
\label{ch:documentation:architecture}
%%
%%%%%%%%%%%%%%%%%%%%%%%%%%%%%%%%%%%%%%%%%%%%%%%%%%%%%%%%

The documentation is written in \LaTeX and to ease writing many packages and customizations are used. The documentation is located in the \textcolor{blue}{./documentation} folder. Both the user documentation and the developer documentation (this document) are written together. 

The main \textcolor{blue}{.tex} files are \textcolor{blue}{User\_guide\_spirou\_drs.tex} and \textcolor{blue}{Dev\_guide\_spirou\_drs.tex} these are the files that should be compiled by \LaTeX. As well as this file there are four directories, the `Chapters' directory (containing the content of each chapter), the `Config' directory (containing custom commands, formatting and constants - see Section \ref{ch:documentation:commands}, Section \ref{ch:documentation:constants}, and Section \ref{ch:documentation:code_format}), the `Figures' directory (containing all figures and graphics) and the `Tables' directory containing table \textcolor{blue}{.tex} files. \\

\noindent The documentation is currently written using the `memoir' class file (\url{texdoc.net/texmf-dist/doc/latex/memoir/memman.pdf}) and uses custom chapter styles from \url{ctan.org/pkg/memoirchapterstyles} (Contained within the \textcolor{blue}{.documentation/Config/preamble.tex} file). 

%%%%%%%%%%%%%%%%%%%%%%%%%%%%%%%%%%%%%%%%%%%%%%%%%%%%%%%%
%%
\section{Required \LaTeX packages}
\label{ch:documentation:packages}
%%
%%%%%%%%%%%%%%%%%%%%%%%%%%%%%%%%%%%%%%%%%%%%%%%%%%%%%%%%

To compile in \LaTeX one needs the following document class:
\begin{itemize}
	\item memoir \dotfill Typeset fiction, non-fiction and mathematical books
\end{itemize}

To compile in \LaTeX\, one needs the following packages:
\begin{itemize}
	\item inputenc \dotfill utf8 encoding
	\item fontenc \dotfill Standard package for selecting font encodings
	\item babel \dotfill Multilingual support for Plain \TeX or \LaTeX
	\item microtype \dotfill Sublim­i­nal re­fine­ments to­wards ty­po­graph­i­cal per­fec­tion
	\item amsmath \dotfill AMS mathematical facilities for \LaTeX
	\item amssymb \dotfill AMS symbols for \LaTeX
	\item mathtools \dotfill Mathematical tools to use with amsmath
	\item memhfixc \dotfill Adjustment for using hyperref in memoir documents
	\item graphicx \dotfill Enhanced support for graphics
	\item listings \dotfill Typeset source code listings using \LaTeX
	\item xcolor \dotfill Driver-independent colour extensions for \LaTeX\, and pdf\LaTeX
	\item hyperref \dotfill Extensive support for hypertext in \LaTeX
	\item dirtree \dotfill Display trees in the style of windows explorer
	\item framed \dotfill Framed or shaded regions that can break across pages
	\item multirow \dotfill Create tabular cells spanning multiple rows
	\item float \dotfill Improved interface for floating objects
	\item background \dotfill Placement of background material on pages of a document
	\item tcolorbox \dotfill Coloured boxes, for \LaTeX\, examples and theorems
	\item eso-pic \dotfill Add picture commands (or backgrounds) to every page
	\item ulem \dotfill Package for underlining
	\item tocloft \dotfill Con­trol ta­ble of con­tents, fig­ures, etc
	\item caption \dotfill Cus­tomis­ing cap­tions in float­ing en­vi­ron­ments
	\item pdflscape \dotfill Make land­scape pages dis­play as land­scape
	\item xifthen \dotfill Ex­tended con­di­tional com­mands
\end{itemize}


%%%%%%%%%%%%%%%%%%%%%%%%%%%%%%%%%%%%%%%%%%%%%%%%%%%%%%%%
%%
\section{Developer documentation content}
\label{ch:documentation:devguide}
%%
%%%%%%%%%%%%%%%%%%%%%%%%%%%%%%%%%%%%%%%%%%%%%%%%%%%%%%%%

As mentioned above we write the developer documentation and user guide using the same files, for this reason one needs a way to distinguish content that is unique to the user documentation or the developer documentation. This is done using the boolean statement `\textcolor{red}{\textbackslash{ifdevguide}}' (defined in the main \textcolor{blue}{.tex} files for the user documentation - \textcolor{red}{\textbackslash{devguidefalse}} - and developer documentation \textcolor{red}{\textbackslash{devguidetrue}}). 

\noindent An example of a different content for each type of documentation is below:
\begin{latexbox}
\ifdevguide
This is the developer guide.
\else
This is the user guide.
\fi
\end{latexbox}

\noindent an example of content only for the developer guide is below:
\begin{latexbox}
\ifdevguide
This section is only for developers
\fi
\end{latexbox}

\noindent an example of content only for the user guide is below:
\begin{latexbox}
\ifdevguide
\else
This section is only for developers
\fi
\end{latexbox}
\begin{note}
As this is the developer guide the content for the user guide only will not be present.
\end{note}
\begin{note}
It is probably never the case where the user documentation will have content that the developer documentation does not need.
\end{note}


%%%%%%%%%%%%%%%%%%%%%%%%%%%%%%%%%%%%%%%%%%%%%%%%%%%%%%%%
%%
\clearpage
\newpage
\section{Custom Commands}
\label{ch:documentation:commands}
%%
%%%%%%%%%%%%%%%%%%%%%%%%%%%%%%%%%%%%%%%%%%%%%%%%%%%%%%%%

To ease writing the documentation some custom commands are defined in \textcolor{blue}{./documentation/Config/commands.tex}. These include the following:
\begin{itemize}
	
	\item \namedlabel{text:definevariable} \textcolor{red}{\textbackslash{definevariable}\{label reference\}\{text\}} - used to create in-text variables (that link to the variables chapter) i.e.:
	\begin{latexbox}
	The variable \definevariable{ch:variables}{VARIABLE} can be used.
	\end{latexbox}

	
	\item \namedlabel{text:defineinkeyword} \textcolor{red}{\textbackslash{defineinkeyword}\{label reference\}\{text\}}, \textcolor{red}{\textbackslash{defineoutkeyword}\{label reference\}\{text\}} - used to create in-text keywords (that link to the keywords chapter) i.e.:
	\begin{latexbox}
	The keyword \defineinkeyword{ch:input_keywords}{IN\_KEYWORD} can be used.
	The keyword \defineoutkeyword{ch:output_keywords}{OUT\_KEYWORD} can be used.
	\end{latexbox}

	\item \textcolor{red}{\textbackslash{Program}} - used to highligh a program (writen in small caps). i.e.:
	\begin{latexbox}
	The program \Program{AstroPy} can be used.
	\end{latexbox}

	\newpage

	\item \begin{minipage}[t]{\textwidth}
	\namedlabel{text:variable_name1}
	\textcolor{red}{\textbackslash{ParameterEntry}} - used to define a parameter entry. It requires 8 arguments (Variable title, Description, variable name, default value, which recipe it is used in, the place the variable is defined, the code/module/function it is used in -- dev only, and the visibility level -- dev only). i.e.:
	\begin{latexbox}
	\namedlabel{text:variablename1}
	\ParameterEntry{Variable title}
	{Description of the variable}
	{VARIABLE\_NAME}
	{Default Value}{The recipe used the variable is used in.}
	{The place where the variable is defined.}
	{The code (module + function) where variable is used.}
	{
	Who should be able to change this variable, levels are as follows:
	\begin{itemize}
		\item Public: Everyone (including the user)
		\item Private: Only the developer
	\end{itemize}
	}
	\end{latexbox}
	\begin{note}
	the \textcolor{red}{\textbackslash{label}} here is used to link variables with this name (i.e. via \definevariable{text:definevariable}{definevariable})
	\end{note}
	\end{minipage}

	\item \begin{minipage}[t]{\textwidth}
	\namedlabel{text:variable_name2}
	\textcolor{red}{\textbackslash{PseudoParamEntry}} - used to define a pseudo parameter entry. It requires 6 arguments (Variable title, Description, variable name, which recipe it is used in, the place the variable is defined, the code/module/function it is used in). It is only available for the developer guide. i.e.:
	\begin{latexbox}
	\namedlabel{text:variablename2}
	\PseudoParamEntry{Variable title}
	{Description of the variable}
	{VARIABLE\_NAME}
	{The recipe used the variable is used in.}
	{The place where the variable is defined.}
	{The code (module + function) where variable is used.}
	\end{latexbox}
	\begin{note}
	\textcolor{red}{\textbackslash{PseudoParamEntry}} is identical to \textcolor{red}{\textbackslash{ParameterEntry}} other than not requiring a default value and not requiring a visibility level (as it is generally used for code thus a simple value can not be given cleanly and will always be a private variable).
	\end{note}
	\begin{note}
	the \textcolor{red}{\textbackslash{label}} here is used to link variables with this name (i.e. via \definevariable{text:definevariable}{definevariable})
	\end{note}
	\end{minipage}

	\item \begin{minipage}[t]{\textwidth}
	\namedlabel{text:definekeyword}
	\textcolor{red}{\textbackslash{KeywordEntry}} - used to define a keyword entry. It requires 9 arguments (Keyword title, Description, keyword name, HEADER key name, default HEADER value, HEADER comment, which recipe it is used in, the place the variable is defined, the code/module/function it is used in). It is only available for the developer guide.
	\begin{latexbox}
	\namedlabel{text:keywordname}
	\KeywordEntry{Keyword title}
	{Description of the keyword}
	{kw\_variable}{HEADER key}
	{Default HEADER value}{HEADER comment}
	{The recipe the keyword is used in}
	{The place where the keyword is defined}
	{The code where the keyword is used.}
	\end{latexbox}
	\begin{note}
	the \textcolor{red}{\textbackslash{label}} here is used to link variables with this name (i.e. via \definevariable{text:definekeyword}{definekeyword})
	\end{note}
	\end{minipage}

	\item \begin{minipage}[t]{\textwidth} 
	\textcolor{red}{\textbackslash{customdirtree}} - used to create a directory tree, so add background use with the \textcolor{red}{tcustomdir} environment (see \ref{ch:documentation:code_format}). The format of each line is 
	\begin{textbox}
	.{level} {directory}.
	\end{textbox}
	\noindent each line must start with a period and end with a period, comments can be added using the \textcolor{red}{\textbackslash{DTcomment}} command.

	\noindent an example is shown below:
	\begin{latexbox}
	The file structure should look as follows:
\begin{tcustomdir}
\customdirtree{%
.1 home.
.2 user1.
.3 Downloads\DTcomment{User 1 downloads}.
.3 Documents.
.4 \DTcomment{Many documents in here}.
.2 user2.
.3 Downloads.
.3 Documents\DTcomment{User 2 documents}.
}
\end{tcustomdir}
	\end{latexbox}
	\end{minipage}

\end{itemize}


%%%%%%%%%%%%%%%%%%%%%%%%%%%%%%%%%%%%%%%%%%%%%%%%%%%%%%%%
%%
\clearpage
\newpage
\section{Constants}
\label{ch:documentation:constants}
%%
%%%%%%%%%%%%%%%%%%%%%%%%%%%%%%%%%%%%%%%%%%%%%%%%%%%%%%%%

Many constants are setup to ease the writing of this documentation. These can be found in the \textcolor{blue}{./documentation/Config/constants.tex} file. These are defined and use in the form:
\begin{latexbox}
% define constant
\newcommand{\ConstantName}{ConstantName}
% user constant
The constant is called \ConstantName
\end{latexbox}

\noindent Please check out the \textcolor{blue}{constants.tex} file for the list of which constants are currently defined.

%%%%%%%%%%%%%%%%%%%%%%%%%%%%%%%%%%%%%%%%%%%%%%%%%%%%%%%%
%%
\section{Code formatting}
\label{ch:documentation:code_format}
%%
%%%%%%%%%%%%%%%%%%%%%%%%%%%%%%%%%%%%%%%%%%%%%%%%%%%%%%%%

This section deals with the textbox, cmdbox, pythonbox and \LaTeX boxes seen throughout the documentation. These are defined in \textcolor{blue}{./documentation/Config/code\_format.tex} along with many style definitions (that only need to be changed to change colours/box styles) - this is left out of this guide for berivity.

\begin{itemize}


\item \begin{minipage}[t]{\textwidth} 
A line/lines of text (that are to be edited in a text editor):
\begin{latexbox}
\begin{textbox}
<# A variable name that can be changes to a specific value>
@VARIABLE_NAME@ = "Variable Value"
\end{textbox}
\end{latexbox}
\end{minipage}

\begin{note}
A custom title can be added to any code box and any other tcolorbox parameters can be overwritten as follows:

\begin{latexbox}
\begin{textbox}[title={Custom title}, colback=blue!30!white]
<# A variable name that can be changes to a specific value>
@VARIABLE_NAME@ = "Variable Value"
\end{textbox}
\end{latexbox}

For keywords and options see the \Program{tcolorbox} documentation here: \url{http://texdoc.net/texmf-dist/doc/latex/tcolorbox/tcolorbox.pdf}. 
\end{note}


\item \begin{minipage}[t]{\textwidth} 
A line/lines of text (that are to be edited in bash):
\begin{latexbox}
\begin{bashbox}
#!/usr/bin/bash
# Find out which console you are using
echo $0
# Set environment Hello
export Hello="Hello"
\end{bashbox}
\end{latexbox}
\end{minipage}


\item \begin{minipage}[t]{\textwidth}
A line/lines of text (that are to be edited in bash):
\begin{latexbox}
\begin{cshbox}
#!/usr/bin/tcsh
# Find out which console you are using
echo $0
# Set environment Hello
setenv Hello "Hello"
\end{cshbox}
\end{latexbox}
\end{minipage}


\item \begin{minipage}[t]{\textwidth} 
A line/lines of text to be run in the command shell:
\begin{latexbox}
\begin{cmdbox}
cd (*$\sim$*)/Downloads
\end{cmdbox}
\end{latexbox}
\end{minipage}


\item \begin{minipage}[t]{\textwidth} 
A line/lines of text that is print out to the screen (standard output):
\begin{latexbox}
\begin{cmdboxprint}
 This is a print out in the command line
 produced by using the echo command
\end{cmdboxprint}
\end{latexbox}
\end{minipage}


\item \begin{minipage}[t]{\textwidth} 
A line/lines of text in the python terminal or python script
\begin{latexbox}
\begin{pythonbox}
import numpy as np
print("Hello world")
print("{0} seconds".format(np.sqrt(25)))
\end{pythonbox}
\end{latexbox}
\end{minipage}


\item \begin{minipage}[t]{\textwidth} 
A line/lines of text in \LaTeX code (in raw form and then compiled form).
\begin{latexbox1}
\begin{latexbox}[colframe=blue!75!black,]
This is my \LaTeX code.
\end{latexbox}
\end{latexbox1}
\end{minipage}

\item \begin{minipage}[t]{\textwidth} 
highlighted textbox:
\begin{latexbox}
\begin{thighlight}
Highlighted section
\end{thighlight}
\end{latexbox}
\end{minipage}


\item \begin{minipage}[t]{\textwidth} 
Custom directory tree (highlighted section):
\begin{latexbox}
\begin{tcustomdir}
\customdirtree{%
.1 home.
.2 user1\DTcomment{User directories here}.
.3 Downloads\DTcomment{Sub-directories have increasing numbers}.
.2 user2 \DTcomment{Each line needs terminating period (`.')}.
}
\end{tcustomdir}
\end{latexbox}
\end{minipage}


\item \begin{minipage}[t]{\textwidth} 
note box:
\begin{latexbox}
\begin{note}
This is a Note
\end{note}
\end{latexbox}
\end{minipage}


\item \begin{minipage}[t]{\textwidth} 
todo box:
\begin{latexbox}
\begin{todo}
This is a to do statement (temporary)
\end{todo}
\end{latexbox}
\end{minipage}

\end{itemize}