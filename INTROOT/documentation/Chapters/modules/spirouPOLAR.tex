%%%%%%%%%%%%%%%%%%%%%%%%%%%%%%%%%%%%%%%%%%%%%%%%%%%%%%%%
%%
\clearpage
\newpage
\noindent\begin{minipage}{\textwidth}
\section{The spirouPOLAR module}
\label{ch:the_module:spirouPOLAR}
%%
%%%%%%%%%%%%%%%%%%%%%%%%%%%%%%%%%%%%%%%%%%%%%%%%%%%%%%%%

%----------------------------------------------------------------------------------------
\subsection{SortPolarFiles}

Defined in \spirouPOLAR\path{.spirouPOLAR.sort_polar_files}

\begin{pythonbox}
from SpirouDRS import spirou
spirouPOLAR.SortPolarFiles(p, polardict)
spirouPOLAR.spirouPOLAR.sort_polar_files(p, polardict)
\end{pythonbox}

\begin{pythondocstring}
Function to sort input data for polarimetry.
    
:param p: parameter dictionary, ParamDict containing constants
    Must contain at least:
        LOG_OPT: string, option for logging
        REDUCED_DIR: string, directory path where reduced data are stored
        ARG_FILE_NAMES: list, list of input filenames
        KW_CMMTSEQ: string, FITS keyword where to find polarimetry 
                    information

:return polardict: dictionary, ParamDict containing information on the
                   input data
\end{pythondocstring}
\end{minipage}


%----------------------------------------------------------------------------------------
\noindent\begin{minipage}{\textwidth}
\subsection{LoadPolarData}

Defined in \spirouPOLAR\path{.spirouPOLAR.load_data}

\begin{pythonbox}
from SpirouDRS import spirou
spirouPOLAR.LoadPolarData(p, polardict, loc)
spirouPOLAR.spirouPOLAR.load_data(p, polardict, loc)
\end{pythonbox}

\begin{pythondocstring}
Function to load input E2DS data for polarimetry.
    
:param p: parameter dictionary, ParamDict containing constants
    Must contain at least:
        LOG_OPT: string, option for logging
        IC_POLAR_STOKES_PARAMS: list, list of stokes parameters
        IC_POLAR_FIBERS: list, list of fiber types used in polarimetry

:param polardict: dictionary, ParamDict containing information on the
                  input data
    
:param loc: parameter dictionary, ParamDict to store data
    
:return p, loc: parameter dictionaries,
    The updated parameter dictionary adds/updates the following:
        FIBER: saves reference fiber used for base file in polar sequence
               The updated data dictionary adds/updates the following:
        DATA: array of numpy arrays (2D), E2DS data from all fibers in
              all input exposures.
        BASENAME, string, basename for base FITS file
        HDR: dictionary, header from base FITS file
        CDR: dictionary, header comments from base FITS file
        STOKES: string, stokes parameter detected in sequence
        NEXPOSURES: int, number of exposures in polar sequence
\end{pythondocstring}
\end{minipage}


%----------------------------------------------------------------------------------------

\noindent\begin{minipage}{\textwidth}
\subsection{CalculatePolarimetry}

Defined in \spirouPOLAR\path{.spirouPOLAR.calculate_polarimetry_wrapper}

\begin{pythonbox}
from SpirouDRS import spirou
spirouPOLAR.CalculatePolarimetry(p, loc)
spirouPOLAR.spirouPOLAR.calculate_polarimetry_wrapper(p, loc)
\end{pythonbox}

\begin{pythondocstring}
Function to call functions to calculate polarimetry either using
the Ratio or Difference methods.
    
:param p: parameter dictionary, ParamDict containing constants
    Must contain at least:
        LOG_OPT: string, option for logging
        IC_POLAR_METHOD: string, to define polar method "Ratio" or 
                         "Difference"

:param loc: parameter dictionary, ParamDict containing data
    
:return polarfunc: function, either polarimetry_diff_method(p, loc) 
                   or polarimetry_ratio_method(p, loc)
\end{pythondocstring}
\end{minipage}


%----------------------------------------------------------------------------------------

\noindent\begin{minipage}{\textwidth}
\subsection{CalculateContinuum}

Defined in \spirouPOLAR\path{.spirouPOLAR.calculate_continuum}

\begin{pythonbox}
from SpirouDRS import spirou
spirouPOLAR.CalculateContinuum(p, loc, in_wavelength=True)
spirouPOLAR.spirouPOLAR.calculate_continuum(p, loc, in_wavelength=True)
\end{pythonbox}

\begin{pythondocstring}
Function to calculate the continuum polarization
    
:param p: parameter dictionary, ParamDict containing constants
    Must contain at least:
        LOG_OPT: string, option for logging
        IC_POLAR_CONT_BINSIZE: int, number of points in each sample bin
        IC_POLAR_CONT_OVERLAP: int, number of points to overlap before and
                               after each sample bin
        IC_POLAR_CONT_TELLMASK: list of float pairs, list of telluric bands,
                                i.e, a list of wavelength ranges ([wl0,wlf]) 
                                for telluric absorption
    
:param loc: parameter dictionary, ParamDict containing data
    Must contain at least:
        POL: numpy array (2D), e2ds degree of polarization data
        POLERR: numpy array (2D), e2ds errors of degree of polarization
        NULL1: numpy array (2D), e2ds 1st null polarization
        NULL2: numpy array (2D), e2ds 2nd null polarization
        STOKESI: numpy array (2D), e2ds Stokes I data
        STOKESIERR: numpy array (2D), e2ds errors of Stokes I
    
:param in_wavelength: bool, to indicate whether or not there is wave cal

:return loc: parameter dictionary, the updated parameter dictionary
    Adds/updates the following:
        FLAT_X: numpy array (1D), flatten polarimetric x data
        FLAT_POL: numpy array (1D), flatten polarimetric pol data
        FLAT_POLERR: numpy array (1D), flatten polarimetric pol error data
        FLAT_STOKESI: numpy array (1D), flatten polarimetric stokes I data
        FLAT_STOKESIERR: numpy array (1D), flatten polarimetric stokes I 
                         error data
        FLAT_NULL1: numpy array (1D), flatten polarimetric null1 data
        FLAT_NULL2: numpy array (1D), flatten polarimetric null2 data
        CONT_POL: numpy array (1D), e2ds continuum polarization data
                  interpolated from xbin, ybin points, same shape as 
                  FLAT_POL
        CONT_XBIN: numpy array (1D), continuum in x polarization samples 
        CONT_YBIN: numpy array (1D), continuum in y polarization samples
\end{pythondocstring}
\end{minipage}


%----------------------------------------------------------------------------------------

\noindent\begin{minipage}{\textwidth}
\subsection{CalculateStokesI}

Defined in \spirouPOLAR\path{.spirouPOLAR.calculate_stokes_I}

\begin{pythonbox}
from SpirouDRS import spirou
spirouPOLAR.CalculateStokesI(p, loc)
spirouPOLAR.spirouPOLAR.calculate_stokes_I(p, loc)
\end{pythonbox}

\begin{pythondocstring}
Function to calculate the Stokes I polarization

:param p: parameter dictionary, ParamDict containing constants
    Must contain at least:
        LOG_OPT: string, option for logging
    
:param loc: parameter dictionary, ParamDict containing data
    Must contain at least:
        DATA: array of numpy arrays (2D), E2DS data from all fibers in
              all input exposures.
        NEXPOSURES: int, number of exposures in polar sequence
    
:return loc: parameter dictionary, the updated parameter dictionary
    Adds/updates the following:
        STOKESI: numpy array (2D), the Stokes I parameters, same shape as
                 DATA
        STOKESIERR: numpy array (2D), the Stokes I error parameters, same 
                    shape as DATA
\end{pythondocstring}
\end{minipage}


%----------------------------------------------------------------------------------------
