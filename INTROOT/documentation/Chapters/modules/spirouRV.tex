
%%%%%%%%%%%%%%%%%%%%%%%%%%%%%%%%%%%%%%%%%%%%%%%%%%%%%%%%
%%
\clearpage
\newpage
\noindent\begin{minipage}{\textwidth}
\section{The spirouRV module}
\label{ch:the_module:spirouRV}
%%
%%%%%%%%%%%%%%%%%%%%%%%%%%%%%%%%%%%%%%%%%%%%%%%%%%%%%%%%

%----------------------------------------------------------------------------------------
\subsection{CalcRVdrift2D}

Defined in \spirouRV\path{.spirouRV.calculate_rv_drifts_2d}

\begin{pythonbox}
from SpirouDRS import spirouRV
spirouRV.CalcRVdrift2D(speref, spe, wave, sigdet, threshold, size)
spirouRV.spirouRV.calculate_rv_drifts_2d(speref, spe, wave, sigdet, threshold, size)
\end{pythonbox}

\begin{pythondocstring}
Calculate the RV drift between the REFERENCE (speref) and COMPARISON (spe)
extracted spectra.

:param speref: numpy array (2D), the REFERENCE extracted spectrum
               size = (number of orders by number of columns (x-axis))
:param spe:  numpy array (2D), the COMPARISON extracted spectrum
             size = (number of orders by number of columns (x-axis))
:param wave: numpy array (2D), the wave solution for each pixel
:param sigdet: float, the read noise (sigdet) for calculating the
               noise array
:param threshold: float, upper limit for pixel values, above this limit
                  pixels are regarded as saturated
:param size: int, size (in pixels) around saturated pixels to also regard
             as bad pixels

:return rvdrift: numpy array (1D), the RV drift between REFERENCE and
                 COMPARISON spectrum for each order
\end{pythondocstring}
\end{minipage}

%----------------------------------------------------------------------------------------
\noindent\begin{minipage}{\textwidth}
\subsection{Coravelation}

Defined in \spirouRV\path{.spirouRV.coravelation}

\begin{pythonbox}
from SpirouDRS import spirouRV
spirouRV.Coravelation(p, loc)
spirouRV.spirouRV.coravelation(p, loc)
\end{pythonbox}

\begin{pythondocstring}
Calculate the CCF and fit it with a Gaussian profile

:param p: parameter dictionary, ParamDict containing constants
    Must contain at least:
            ccf_berv: float, the barycentric Earth RV (berv)
            ccf_berv_max: float, the maximum barycentric Earth RV
            target_rv: float, the target RV
            ccf_width: float, the CCF width
            ccf_step: float, the CCF step
            ccf_det_noise: float, the detector noise to use in the ccf
            ccf_fit_type: int, the type of fit for the CCF fit
            log_opt: string, log option, normally the program name
            DRS_DEBUG: int, Whether to run in debug mode
                            0: no debug
                            1: basic debugging on errors
                            2: recipes specific (plots and some code runs)
            DRS_PLOT: bool, Whether to plot (True to plot)

:param loc: parameter dictionary, ParamDict containing data
        Must contain at least:
            wave_ll: numpy array (1D), the line list values
            param_ll: numpy array (1d), the line list fit coefficients
                      (used to generate line list - read from file defined)
            ll_mask_d: numpy array (1D), the size of each line
                       (in wavelengths)
            ll_mask_ctr: numpy array (1D), the central point of each line
                         (in wavelengths)
            w_mask: numpy array (1D), the weight mask
            e2dsff: numpy array (2D), the flat fielded E2DS spectrum
                    shape = (number of orders x number of columns in image
                                                  (x-axis dimension) )
            blaze: numpy array (2D), the blaze function
                    shape = (number of orders x number of columns in image
                                                  (x-axis dimension) )

:return loc: parameter dictionary, the updated parameter dictionary
        Adds/updates the following:
            rv_ccf: numpy array (1D), the radial velocities for the CCF
            ccf: numpy array (2D), the CCF for each order and each RV
                 shape = (number of orders x number of RV points)
            ccf_max: float, numpy array (1D), the max value of the CCF for
                     each order
            pix_passed_all: numpy array (1D), the weighted line list
                            position for each order?
            tot_line: numpy array (1D), the total number of lines for each
                      order
            ll_range_all: numpy array (1D), the weighted line list width for
                      each order
            ccf_noise: numpy array (2D), the CCF noise for each order and
                       each RV
                       shape = (number of orders x number of RV points)
\end{pythondocstring}
\end{minipage}

%----------------------------------------------------------------------------------------
\noindent\begin{minipage}{\textwidth}
\subsection{CreateDriftFile}

Defined in \spirouRV\path{.spirouRV.create_drift_file}

\begin{pythonbox}
from SpirouDRS import spirouRV
spirouRV.CreateDriftFile(p, loc)
spirouRV.spirouRV.create_drift_file(p, loc)
\end{pythonbox}

\begin{pythondocstring}
Creates a reference ascii file that contains the positions of the FP peaks
Returns the pixels positions and Nth order of each FP peak

:param p: parameter dictionary, ParamDict containing constants
    Must contain at least:
            drift_peak_border_size: int, the border size (edges in
                                    x-direction) for the FP fitting
                                    algorithm
            drift_peak_fpbox_size: int, the box half-size (in pixels) to
                                   fit an individual FP peak to - a
                                   Gaussian will be fit to +/- this size
                                   from the centre of the FP peak
            drift_peak_peak_sig_lim: dictionary, the sigma above the median
                                     that a peak must have to be recognised
                                     as a valid peak (before fitting a
                                     Gaussian) dictionary must have keys
                                     equal to the lamp types (hc, fp)
            drift_peak_inter_peak_spacing: int, the minimum spacing between
                                           peaks in order to be recognised
                                           as a valid peak (before fitting
                                           a Gaussian)
            log_opt: string, log option, normally the program name

:param loc: parameter dictionary, ParamDict containing data
        Must contain at least:
            speref: numpy array (2D), the reference spectrum
            wave: numpy array (2D), the wave solution image
            lamp: string, the lamp type (either 'hc' or 'fp')

:return loc: parameter dictionary, the updated parameter dictionary
        Adds/updates the following:
            ordpeak: numpy array (1D), the order number for each valid FP
                     peak
            xpeak: numpy array (1D), the central position each Gaussian fit
                   to valid FP peak
            ewpeak: numpy array (1D), the FWHM of each Gaussian fit
                    to valid FP peak
            vrpeak: numpy array (1D), the radial velocity drift for each
                    valid FP peak
            llpeak: numpy array (1D), the delta wavelength for each valid
                    FP peak
            amppeak: numpy array (1D), the amplitude for each valid FP peak
\end{pythondocstring}
\end{minipage}

%----------------------------------------------------------------------------------------
\noindent\begin{minipage}{\textwidth}
\subsection{DeltaVrms2D}

Defined in \spirouRV\path{.spirouRV.delta_v_rms_2d}

\begin{pythonbox}
from SpirouDRS import spirouRV
spirouRV.DeltaVrms2D(spe, wave, sigdet, threshold, size)
spirouRV.spirouRV.delta_v_rms_2d(spe, wave, sigdet, threshold, size)
\end{pythonbox}

\begin{pythondocstring}
Compute the photon noise uncertainty for all orders (for the 2D image)

:param spe: numpy array (2D), the extracted spectrum
            size = (number of orders by number of columns (x-axis))
:param wave: numpy array (2D), the wave solution for each pixel
:param sigdet: float, the read noise (sigdet) for calculating the
               noise array
:param threshold: float, upper limit for pixel values, above this limit
                  pixels are regarded as saturated
:param size: int, size (in pixels) around saturated pixels to also regard
             as bad pixels

:return dvrms2: numpy array (1D), the photon noise for each pixel (squared)
:return weightedmean: float, weighted mean photon noise across all orders
\end{pythondocstring}
\end{minipage}

%----------------------------------------------------------------------------------------
\noindent\begin{minipage}{\textwidth}
\subsection{DriftPerOrder}

Defined in \spirouRV\path{.spirouRV.drift_per_order}

\begin{pythonbox}
from SpirouDRS import spirouRV
spirouRV.DriftPerOrder(loc, fileno)
spirouRV.spirouRV.drift_per_order(loc, fileno)
\end{pythonbox}

\begin{pythondocstring}

\end{pythondocstring}
\end{minipage}

%----------------------------------------------------------------------------------------
\noindent\begin{minipage}{\textwidth}
\subsection{DriftAllOrders}

Defined in \spirouRV\path{.spirouRV.drift_all_orders}

\begin{pythonbox}
from SpirouDRS import spirouRV
spirouRV.DriftAllOrders(loc, fileno, nomin, nomax)
spirouRV.spirouRV.drift_all_orders(loc, fileno, nomin, nomax)
\end{pythonbox}

\begin{pythondocstring}
Work out the weighted mean drift across all orders

:param loc: parameter dictionary, ParamDict containing data
        Must contain at least:
            drift: numpy array (2D), the median drift values for each
                   file and each order
                   shape = (number of files x number of orders)
            drift_left: numpy array (2D), the median drift values for the
                        left half of each order (for each file and each
                        order)
                        shape = (number of files x number of orders)
            drift_right: numpy array (2D), the median drift values for the
                         right half of each order (for each file and each
                         order)
                         shape = (number of files x number of orders)
            errdrift: numpy array (2D), the error in the drift for each
                      file and each order
                      shape = (number of files x number of orders)

:param fileno: int, the file number (iterator number)
:param nomin: int, the first order to use (i.e. from nomin to nomax)
:param nomax: int, the last order to use (i.e. from nomin to nomax)

:return loc: parameter dictionary, the updated parameter dictionary
        Adds/updates the following:
            meanrv: numpy array (1D), the weighted mean drift, for each file
                    shape = (number of files)
            meanrv_left: numpy array (1D), the weighted mean drift for the
                         left half of each order, for each file
                         shape = (number of files)
            meanrv_right: numpy array (1D), the weighted mean drift for the
                          right half of each order, for each file
                          shape = (number of files)
            merrdrift: numpy array (1D), the error in weighted mean for
                       each file
                       shape = (number of files)
\end{pythondocstring}
\end{minipage}

%----------------------------------------------------------------------------------------
\noindent\begin{minipage}{\textwidth}
\subsection{FitCCF}

Defined in \spirouRV\path{.spirouRV.fit_ccf}

\begin{pythonbox}
from SpirouDRS import spirouRV
spirouRV.FitCCF(rv, ccf, fit_type)
spirouRV.spirouRV.fit_ccf(rv, ccf, fit_type)
\end{pythonbox}

\begin{pythondocstring}
Fit the CCF to a guassian function

:param rv: numpy array (1D), the radial velocities for the line
:param ccf: numpy array (1D), the CCF values for the line
:param fit_type: int, if "0" then we have an absorption line
                      if "1" then we have an emission line

:return result: numpy array (1D), the fit parameters in the
                following order:

            [amplitude, center, fwhm, offset from 0 (in y-direction)]

:return ccf_fit: numpy array (1D), the fit values, i.e. the Gaussian values
                 for the fit parameters in "result"
\end{pythondocstring}
\end{minipage}

%----------------------------------------------------------------------------------------
\noindent\begin{minipage}{\textwidth}
\subsection{GetDrift}

Defined in \spirouRV\path{.spirouRV.get_drift}

\begin{pythonbox}
from SpirouDRS import spirouRV
spirouRV.GetDrift(p, sp, ordpeak, xpeak0, gaussfit=False)
spirouRV.spirouRV.get_drift(p, sp, ordpeak, xpeak0, gaussfit=False)
\end{pythonbox}

\begin{pythondocstring}
Get the centroid of all peaks provided an input peak position

:param p: parameter dictionary, ParamDict containing constants
    Must contain at least:
            drift_peak_fpbox_size: int, the box half-size (in pixels) to
                                   fit an individual FP peak to - a
                                   Gaussian will be fit to +/- this size
                                   from the centre of the FP peak
            drift_peak_exp_width: float, the expected width of FP peaks -
                                  used to "normalise" peaks (which are then
                                  subsequently removed if >
                                  drift_peak_norm_width_cut
            log_opt: string, log option, normally the program name

:param sp: numpy array (2D), E2DS fits file with FP peaks
           size = (number of orders x number of pixels in x-dim of image)
:param ordpeak: numpy array (1D), order of each peak
:param xpeak0: numpy array (1D), position in the x dimension of all peaks
:param gaussfit: bool, if True uses a Gaussian fit to get each centroid
                 (slow) or adjusts a barycentre (gaussfit=False)

:return xpeak: numpy array (1D), the central positions of the peaks
\end{pythondocstring}
\end{minipage}

%----------------------------------------------------------------------------------------
\noindent\begin{minipage}{\textwidth}
\subsection{GetCCFMask}

Defined in \spirouRV\path{.spirouRV.get_ccf_mask}

\begin{pythonbox}
from SpirouDRS import spirouRV
spirouRV.GetCCFMask(p, loc, filename=None)
spirouRV.spirouRV.get_ccf_mask(p, loc, filename=None)
\end{pythonbox}

\begin{pythondocstring}
Get the CCF mask

:param p: parameter dictionary, ParamDict containing constants
    Must contain at least:
            ccf_mask: string, the name (and or location) of the CCF
                      mask file
            ic_w_mask_min: float, the weight of the CCF mask (if 1 force
                           all weights equal)
            ic_mask_width: float, the width of the template line
                           (if 0 use natural
            log_opt: string, log option, normally the program name

:param loc: parameter dictionary, ParamDict containing data

:param filename: string or None, the filename and location of the ccf mask
                 file, if None then file names is gotten from p["ccf_mask"]

:return loc: parameter dictionary, the updated parameter dictionary
        Adds/updates the following:
            ll_mask_d: numpy array (1D), the size of each pixel
                       (in wavelengths)
            ll_mask_ctr: numpy array (1D), the central point of each pixel
                         (in wavelengths)
            w_mask: numpy array (1D), the weight mask
\end{pythondocstring}
\end{minipage}

%----------------------------------------------------------------------------------------
\noindent\begin{minipage}{\textwidth}
\subsection{PearsonRtest}

Defined in \spirouRV\path{.spirouRV.pearson_rtest}

\begin{pythonbox}
from SpirouDRS import spirouRV
spirouRV.PearsonRtest(nbo, spe, speref)
spirouRV.spirouRV.spirouRV.pearson_rtest(nbo, spe, speref)
\end{pythonbox}

\begin{pythondocstring}
Perform a Pearson R test on each order in spe against speref

:param nbo: int, the number of orders
:param spe: numpy array (2D), the extracted array for this iteration
            size = (number of orders x number of pixels in x-dim)
:param speref: numpy array (2D), the extracted array for the reference
               image, size = (number of orders x number of pixels in x-dim)

:return cc_orders: numpy array (1D), the Pearson correlation coefficients
                   for each order, size = (number of orders)
\end{pythondocstring}
\end{minipage}

%----------------------------------------------------------------------------------------
\noindent\begin{minipage}{\textwidth}
\subsection{RemoveWidePeaks}

Defined in \spirouRV\path{.spirouRV.remove_wide_peaks}

\begin{pythonbox}
from SpirouDRS import spirouRV
spirouRV.RemoveWidePeaks(p, loc, expwidth=None, cutwidth=None)
spirouRV.spirouRV.remove_wide_peaks(p, loc, expwidth=None, cutwidth=None)
\end{pythonbox}

\begin{pythondocstring}
Remove peaks that are too wide

:param p: parameter dictionary, ParamDict containing constants
    Must contain at least:
            drift_peak_exp_width: float, the expected width of FP peaks -
                                  used to "normalise" peaks (which are then
                                  subsequently removed if >
                                  drift_peak_norm_width_cut
            drift_peak_norm_width_cut: float, the "normalised" width of
                                       FP peaks that is too large
                                       normalised width = FP FWHM -
                                       drift_peak_exp_width cut is
                                       essentially:=
                                       FP FWHM < (drift_peak_exp_width +
                                       drift_peak_norm_width_cut)
            log_opt: string, log option, normally the program name

:param loc: parameter dictionary, ParamDict containing data
        Must contain at least:
            ordpeak: numpy array (1D), the order number for each valid FP
                     peak
            xpeak: numpy array (1D), the central position each Gaussian fit
                   to valid FP peak
            ewpeak: numpy array (1D), the FWHM of each Gaussian fit
                    to valid FP peak
            vrpeak: numpy array (1D), the radial velocity drift for each
                    valid FP peak
            llpeak: numpy array (1D), the delta wavelength for each valid
                    FP peak
            amppeak: numpy array (1D), the amplitude for each valid FP peak

:param expwidth: float or None, the expected width of FP peaks - used to
                 "normalise" peaks (which are then subsequently removed
                 if > "cutwidth") if expwidth is None taken from
                 p["drift_peak_exp_width"]
:param cutwidth: float or None, the normalised width of FP peaks that is too
                 large normalised width FP FWHM - expwidth
                 cut is essentially: FP FWHM < (expwidth + cutwidth), if
                 cutwidth is None taken from p["drift_peak_norm_width_cut"]

:return loc: parameter dictionary, the updated parameter dictionary
        Adds/updates the following:
            ordpeak: numpy array (1D), the order number for each valid FP
                     peak (masked to remove wide peaks)
            xpeak: numpy array (1D), the central position each Gaussian fit
                   to valid FP peak (masked to remove wide peaks)
            ewpeak: numpy array (1D), the FWHM of each Gaussian fit
                    to valid FP peak (masked to remove wide peaks)
            vrpeak: numpy array (1D), the radial velocity drift for each
                    valid FP peak (masked to remove wide peaks)
            llpeak: numpy array (1D), the delta wavelength for each valid
                    FP peak (masked to remove wide peaks)
            amppeak: numpy array (1D), the amplitude for each valid FP peak
                     (masked to remove wide peaks)
\end{pythondocstring}
\end{minipage}

%----------------------------------------------------------------------------------------
\noindent\begin{minipage}{\textwidth}
\subsection{RemoveZeroPeaks}

Defined in \spirouRV\path{.remove_zero_peaks}

\begin{pythonbox}
from SpirouDRS import spirouRV
spirouRV.RemoveZeroPeaks(p, loc)
spirouRV.spirouRV.remove_zero_peaks(p, loc)
\end{pythonbox}

\begin{pythondocstring}
Remove peaks that have a value of zero

:param p: parameter dictionary, ParamDict containing constants
    Must contain at least:
            log_opt: string, log option, normally the program name

:param loc: parameter dictionary, ParamDict containing data
        Must contain at least:
            xref: numpy array (1D), the central positions of the peaks
            ordpeak: numpy array (1D), the order number for each valid FP
                     peak
            xpeak: numpy array (1D), the central position each Gaussian fit
                   to valid FP peak
            ewpeak: numpy array (1D), the FWHM of each Gaussian fit
                    to valid FP peak
            vrpeak: numpy array (1D), the radial velocity drift for each
                    valid FP peak
            llpeak: numpy array (1D), the delta wavelength for each valid
                    FP peak
            amppeak: numpy array (1D), the amplitude for each valid FP peak

:return loc: parameter dictionary, the updated parameter dictionary
        Adds/updates the following:
            xref: numpy array (1D), the central positions of the peaks
                  (masked with zero peaks removed)
            ordpeak: numpy array (1D), the order number for each valid FP
                     peak (masked with zero peaks removed)
            xpeak: numpy array (1D), the central position each Gaussian fit
                   to valid FP peak (masked with zero peaks removed)
            ewpeak: numpy array (1D), the FWHM of each Gaussian fit
                    to valid FP peak (masked with zero peaks removed)
            vrpeak: numpy array (1D), the radial velocity drift for each
                    valid FP peak (masked with zero peaks removed)
            llpeak: numpy array (1D), the delta wavelength for each valid
                    FP peak (masked with zero peaks removed)
            amppeak: numpy array (1D), the amplitude for each valid FP peak
                     (masked with zero peaks removed)
\end{pythondocstring}
\end{minipage}

%----------------------------------------------------------------------------------------
\noindent\begin{minipage}{\textwidth}
\subsection{ReNormCosmic2D}

Defined in \spirouRV\path{.spirouRV.remove_zero_peaks}

\begin{pythonbox}
from SpirouDRS import spirouRV
spirouRV.ReNormCosmic2D(p, loc)
spirouRV.spirouRV.remove_zero_peaks(p, loc)
\end{pythonbox}

\begin{pythondocstring}
Remove peaks that have a value of zero

:param p: parameter dictionary, ParamDict containing constants
    Must contain at least:
            log_opt: string, log option, normally the program name

:param loc: parameter dictionary, ParamDict containing data
        Must contain at least:
            xref: numpy array (1D), the central positions of the peaks
            ordpeak: numpy array (1D), the order number for each valid FP
                     peak
            xpeak: numpy array (1D), the central position each Gaussian fit
                   to valid FP peak
            ewpeak: numpy array (1D), the FWHM of each Gaussian fit
                    to valid FP peak
            vrpeak: numpy array (1D), the radial velocity drift for each
                    valid FP peak
            llpeak: numpy array (1D), the delta wavelength for each valid
                    FP peak
            amppeak: numpy array (1D), the amplitude for each valid FP peak

:return loc: parameter dictionary, the updated parameter dictionary
        Adds/updates the following:
            xref: numpy array (1D), the central positions of the peaks
                  (masked with zero peaks removed)
            ordpeak: numpy array (1D), the order number for each valid FP
                     peak (masked with zero peaks removed)
            xpeak: numpy array (1D), the central position each Gaussian fit
                   to valid FP peak (masked with zero peaks removed)
            ewpeak: numpy array (1D), the FWHM of each Gaussian fit
                    to valid FP peak (masked with zero peaks removed)
            vrpeak: numpy array (1D), the radial velocity drift for each
                    valid FP peak (masked with zero peaks removed)
            llpeak: numpy array (1D), the delta wavelength for each valid
                    FP peak (masked with zero peaks removed)
            amppeak: numpy array (1D), the amplitude for each valid FP peak
                     (masked with zero peaks removed)
\end{pythondocstring}
\end{minipage}

%----------------------------------------------------------------------------------------
\noindent\begin{minipage}{\textwidth}
\subsection{ReNormCosmic2D}

Defined in \spirouRV\path{.spirouRV.renormalise_cosmic2d}

\begin{pythonbox}
from SpirouDRS import spirouRV
spirouRV.ReNormCosmic2D(speref, spe, threshold, size, cut)
spirouRV.spirouRV.renormalise_cosmic2d(speref, spe, threshold, size, cut)
\end{pythonbox}

\begin{pythondocstring}
Correction of the cosmics and re-normalisation by comparison with
reference spectrum (for the 2D image)

:param speref: numpy array (2D), the REFERENCE extracted spectrum
               size = (number of orders by number of columns (x-axis))
:param spe:  numpy array (2D), the COMPARISON extracted spectrum
             size = (number of orders by number of columns (x-axis))
:param threshold: float, upper limit for pixel values, above this limit
                  pixels are regarded as saturated
:param size: int, size (in pixels) around saturated pixels to also regard
             as bad pixels
:param cut: float, define the number of standard deviations cut at in             -                  cosmic renormalisation

:return spen: numpy array (2D), the corrected normalised COMPARISON
              extracted spectrum
:return cnormspe: numpy array (1D), the flux ratio for each order between
                  corrected normalised COMPARISON extracted spectrum and
                  REFERENCE extracted spectrum
:return cpt: float, the total flux above the "cut" parameter
             (cut * standard deviations above median)
\end{pythondocstring}
\end{minipage}


%----------------------------------------------------------------------------------------
\noindent\begin{minipage}{\textwidth}
\subsection{SigmaClip}

Defined in \spirouRV\path{.spirouRV.sigma_clip}

\begin{pythonbox}
from SpirouDRS import spirouRV
spirouRV.SigmaClip(loc, sigma=1.0)
spirouRV.spirouRV.sigma_clip(loc, sigma=1.0)
\end{pythonbox}

\begin{pythondocstring}
Perform a sigma clip on dv

:param loc: parameter dictionary, ParamDict containing data
        Must contain at least:
            dv: numpy array (1D), the drift values
            ordpeak: numpy array (1D), the order number for each drift
                     value

:param sigma: float, the sigma of the clip (away from the median)

:return loc: parameter dictionary, the updated parameter dictionary
        Adds/updates the following:
            dvc: numpy array (1D), the sigma clipped drift values
            orderpeakc: numpy array (1D), the order numbers for the sigma
                        clipped drift values
\end{pythondocstring}
\end{minipage}

%----------------------------------------------------------------------------------------
