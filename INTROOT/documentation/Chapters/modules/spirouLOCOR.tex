
%%%%%%%%%%%%%%%%%%%%%%%%%%%%%%%%%%%%%%%%%%%%%%%%%%%%%%%%
%%
\clearpage
\newpage
\noindent\begin{minipage}{\textwidth}
\section{The spirouLOCOR module}
\label{ch:the_module:spirouLOCOR}
%%
%%%%%%%%%%%%%%%%%%%%%%%%%%%%%%%%%%%%%%%%%%%%%%%%%%%%%%%%

%----------------------------------------------------------------------------------------
\subsection{BoxSmoothedImage}

Defined in \spirouLOCOR\path{.spirouLOCOR.smoothed_boxmean_image}

\begin{pythonbox}
from SpirouDRS import spirouLOCOR
spirouLOCOR.BoxSmoothedImage(image, size, weighted=True, mode='convolve')
spirouLOCOR.spirouLOCOR.smoothed_boxmean_image(image, size, weighted=True, mode='convolve')
\end{pythonbox}

\begin{pythondocstring}
Produce a (box) smoothed image, smoothed by the mean of a box of
    size=2*"size" pixels.

    if mode='convolve' (default) then this is done
    by convolving a top-hat function with the image (FAST)
    - note produces small inconsistencies due to FT of top-hat function

    if mode='manual' then this is done by working out the mean in each
    box manually (SLOW)

:param image: numpy array (2D), the image
:param size: int, the number of pixels to mask before and after pixel
             (for every row)
             i.e. box runs from  "pixel-size" to "pixel+size" unless
             near an edge
:param weighted: bool, if True pixel values less than zero are weighted to
                 a value of 1e-6 and values above 0 are weighted to a value
                 of 1
:param mode: string, if 'convolve' convoles with a top-hat function of the
                     size "box" for each column (FAST) - note produces small
                     inconsistencies due to FT of top-hat function

                     if 'manual' calculates every box individually (SLOW)

:return newimage: numpy array (2D), the smoothed image
\end{pythondocstring}
\end{minipage}

%----------------------------------------------------------------------------------------
\noindent\begin{minipage}{\textwidth}
\subsection{CalcLocoFits}

Defined in \spirouLOCOR\path{.spirouLOCOR.calculate_location_fits}

\begin{pythonbox}
from SpirouDRS import spirouLOCOR
spirouLOCOR.CalcLocoFits(coeffs, dim)
spirouLOCOR.spirouLOCOR.calculate_location_fits(coeffs, dim)
\end{pythonbox}

\begin{pythondocstring}
Calculates all fits in coeffs array across pixels of size=dim

:param coeffs: coefficient array,
               size = (number of orders x number of coefficients in fit)
               output array will be size = (number of orders x dim)
:param dim: int, number of pixels to calculate fit for
            fit will be done over x = 0 to dim in steps of 1
:return yfits: array,
               size = (number of orders x dim)
               the fit for each order at each pixel values from 0 to dim
\end{pythondocstring}
\end{minipage}

%----------------------------------------------------------------------------------------
\noindent\begin{minipage}{\textwidth}
\subsection{FiberParams}

Defined in \spirouLOCOR\path{.spirouLOCOR.fiber_params}

\begin{pythonbox}
from SpirouDRS import spirouLOCOR
spirouLOCOR.FiberParams(pp, fiber, merge=False)
spirouLOCOR.spirouLOCOR.fiber_params(pp, fiber, merge=False)
\end{pythonbox}

\begin{pythondocstring}
Takes the parameters defined in FIBER_PARAMS from parameter dictionary
(i.e. from config files) and adds the correct parameter to a fiber
parameter dictionary

:param p: parameter dictionary, ParamDict containing constants
    Must contain at least:
            log_opt: string, log option, normally the program name

:param fiber: string, the fiber type (and suffix used in confiruation file)
              i.e. for fiber AB fiber="AB" and nbfib_AB should be present
              in config if "nbfib" is in FIBER_PARAMS
:param merge: bool, if True merges with pp and returns

:return fparam: dictionary, the fiber parameter dictionary (if merge False)
:treun pp: dictionary, paramter dictionary (if merge True)
\end{pythondocstring}
\end{minipage}

%----------------------------------------------------------------------------------------
\noindent\begin{minipage}{\textwidth}
\subsection{FindPosCentCol}

Defined in \spirouLOCOR\path{.spirouLOCOR.find_position_of_cent_col}

\begin{pythonbox}
from SpirouDRS import spirouLOCOR
spirouLOCOR.FindPosCentCol(values, threshold)
spirouLOCOR.spirouLOCOR.find_position_of_cent_col(values, threshold)
\end{pythonbox}

\begin{pythondocstring}
Finds the central positions based on the central column values

:param values: numpy array (1D) size = number of rows,
                the central column values
:param threshold: float, the threshold above which to find pixels as being
                  part of an order

:return position: numpy array (1D), size= number of rows,
                  the pixel positions in cvalues where the centers of each
                  order should be
\end{pythondocstring}
\end{minipage}

%----------------------------------------------------------------------------------------
\noindent\begin{minipage}{\textwidth}
\subsection{FindOrderCtrs}

Defined in \spirouLOCOR\path{.spirouLOCOR.find_order_centers}

\begin{pythonbox}
from SpirouDRS import spirouLOCOR
spirouLOCOR.FindOrderCtrs(pp, image, loc, order_num)
spirouLOCOR.spirouLOCOR.find_order_centers(pp, image, loc, order_num)
\end{pythonbox}

\begin{pythondocstring}
Find the center pixels and widths of this order at specific points
along this order="order_num"

specific points are defined by steps (ic_locstepc) away from the
central pixel (ic_cent_col)

:param p: parameter dictionary, ParamDict containing constants
    Must contain at least:
            IC_LOCSTEPC: int, the column separation for fitting orders
            IC_CENT_COL: int, the column number (x-axis) of the central
                         column
            IC_EXT_WINDOW: int, extraction window size (half size)
            IC_IMAGE_GAP: int, the gap index in the selected area
            sigdet: float, the read noise of the image
            IC_LOCSEUIL: float, Normalised amplitude threshold to accept
                         pixels for background calculation
            IC_WIDTHMIN: int, minimum width of order to be accepted
            DRS_DEBUG: int, Whether to run in debug mode
                            0: no debug
                            1: basic debugging on errors
                            2: recipes specific (plots and some code runs)
            DRS_PLOT: bool, Whether to plot (True to plot)

:param image: numpy array (2D), the image

:param loc: parameter dictionary, ParamDict containing data
        Must contain at least:
            ctro: numpy array (2D), storage for the center positions
                  shape = (number of orders x number of columns (x-axis)

:param order_num: int, the current order to process

:return loc: parameter dictionary, the updated parameter dictionary
        Adds/updates the following:
            ctro: numpy array (2D), storage for the center positions
                  shape = (number of orders x number of columns (x-axis)
                  updated the values for "order_num"
            sigo: numpy array (2D), storage for the width positions
                  shape = (number of orders x number of columns (x-axis)
                  updated the values for "order_num"
\end{pythondocstring}
\end{minipage}

%----------------------------------------------------------------------------------------
\noindent\begin{minipage}{\textwidth}
\subsection{GetCoeffs}

Defined in \spirouLOCOR\path{.spirouLOCOR.get_loc_coefficients}

\begin{pythonbox}
from SpirouDRS import spirouLOCOR
spirouLOCOR.GetCoeffs(p, hdr=None, loc=None)
spirouLOCOR.spirouLOCOR.get_loc_coefficients(p, hdr=None, loc=None)
\end{pythonbox}

\begin{pythondocstring}
Extracts loco coefficients from parameters keys (uses header="hdr" provided
to get acquisition time or uses p['fitsfilename'] to get acquisition time if
"hdr" is None

:param p: parameter dictionary, ParamDict containing constants
    Must contain at least:
            fitsfilename: string, the full path of for the main raw fits
                          file for a recipe
                          i.e. /data/raw/20170710/filename.fits
            kw_LOCO_NBO: list, keyword store for the number of orders
                         located
            kw_LOCO_DEG_C: list, keyword store for the fit degree for
                           order centers
            kw_LOCO_DEG_W: list, keyword store for the fit degree for
                           order widths
            kw_LOCO_CTR_COEFF: list, keyword store for the Coeff center
                               order
            kw_LOCO_FWHM_COEFF: list, keyword store for the Coeff width
                                order
            LOC_FILE: string, the suffix for the location calibration
                      database key (usually the fiber type)
                         - read from "loc_file_fpall", if not defined
                           uses p["fiber"]
            fiber: string, the fiber used for this recipe (eg. AB or A or C)
            calibDB: dictionary, the calibration database dictionary
            reduced_dir: string, the reduced data directory
                         (i.e. p['DRS_DATA_REDUC']/p['arg_night_name'])
            log_opt: string, log option, normally the program name

:param hdr: dictionary, header file from FITS rec (opened by spirouFITS)
:param loc: parameter dictionary, ParamDict containing data

:return loc: parameter dictionary, the updated parameter dictionary
        Adds/updates the following:
            number_orders: int, the number of orders in reference spectrum
            nbcoeff_ctr: int, number of coefficients for the center fit
            nbcoeff_wid: int, number of coefficients for the width fit
            acc: numpy array (2D), the fit coefficients array for
                  the centers fit
                  shape = (number of orders x number of fit coefficients)
            ass: numpy array (2D), the fit coefficients array for
                  the widths fit
\end{pythondocstring}
\end{minipage}

%----------------------------------------------------------------------------------------
\noindent\begin{minipage}{\textwidth}
\subsection{ImageLocSuperimp}

Defined in \spirouLOCOR\path{.spirouLOCOR.image_localization_superposition}

\begin{pythonbox}
from SpirouDRS import spirouLOCOR
spirouLOCOR.ImageLocSuperimp(image, coeffs)
spirouLOCOR.spirouLOCOR.image_localization_superposition(image, coeffs)
\end{pythonbox}

\begin{pythondocstring}
Take an image and superimpose zeros over the positions in the image where
the central fits where found to be

:param image: numpy array (2D), the image
:param coeffs: coefficient array,
               size = (number of orders x number of coefficients in fit)
               output array will be size = (number of orders x dim)
:return newimage: numpy array (2D), the image with super-imposed zero filled
                  fits
\end{pythondocstring}
\end{minipage}

%----------------------------------------------------------------------------------------
\noindent\begin{minipage}{\textwidth}
\subsection{InitialOrderFit}

Defined in \spirouLOCOR\path{.spirouLOCOR.initial_order_fit}

\begin{pythonbox}
from SpirouDRS import spirouLOCOR
spirouLOCOR.InitialOrderFit(pp, loc, mask, onum, rnum, kind, fig=None, frame=None)
spirouLOCOR.spirouLOCOR.initial_order_fit(pp, loc, mask, onum, rnum, kind, fig=None, frame=None)
\end{pythonbox}

\begin{pythondocstring}
Performs a crude initial fit for this order, uses the ctro positions or sigo 
width values found in "FindOrderCtrs" or "find_order_centers" to do the fit

:param p: parameter dictionary, ParamDict containing constants
    Must contain at least:
            log_opt: string, log option, normally the program name
            IC_LOCDFITC: int, order of polynomial to fit for positions
            IC_LOCDFITW: int, order of polynomial to fit for widths
            DRS_PLOT: bool, Whether to plot (True to plot)
            IC_CENT_COL: int, Definition of the central column

:param loc: parameter dictionary, ParamDict containing data
        Must contain at least:
            x: numpy array (1D), the order numbers
            ctro: numpy array (2D), storage for the center positions
                  shape = (number of orders x number of columns (x-axis)
            sigo: numpy array (2D), storage for the width positions
                  shape = (number of orders x number of columns (x-axis)

:param mask: numpy array (1D) of booleans, True where we have non-zero
             widths
:param onum: int, order iteration number (running number over all
             iterations)
:param rnum: int, order number (running number of successful order
             iterations only)
:param kind: string, 'center' or 'fwhm', if 'center' then this fit is for
             the central positions, if 'fwhm' this fit is for the width of
             the orders
:param fig: plt.figure, the figure to plot initial fit on
:param frame: matplotlib axis i.e. plt.subplot(), the axis on which to plot
              the initial fit on (carries the plt.imshow(image))
:return fitdata: dictionary, contains the fit data key value pairs for this
                 initial fit. keys are as follows:

        a = coefficients of the fit from key
        size = 'ic_locdfitc' [for kind='center'] or
             = 'ic_locdftiw' [for kind='fwhm']
        fit = the fity values for the fit (for x = loc['x'])
            where fity = Sum(a[i] * x^i)
        res = the residuals from y - fity
             where y = ctro [kind='center'] or 
                     = sigo [kind='fwhm'])
        abs_res = abs(res)
        rms = the standard deviation of the residuals
        max_ptp = maximum residual value max(res)
        max_ptp_frac = max_ptp / rms  [kind='center']
                     = max(abs_res/y) * 100   [kind='fwhm']
\end{pythondocstring}
\end{minipage}

%----------------------------------------------------------------------------------------
\noindent\begin{minipage}{\textwidth}
\subsection{LocCentralOrderPos}

Defined in \spirouLOCOR\path{.spirouLOCOR.locate_order_center}

\begin{pythonbox}
from SpirouDRS import spirouLOCOR
spirouLOCOR.LocCentralOrderPos(values, threshold, min_width=None)
spirouLOCOR.spirouLOCOR.locate_order_center(values, threshold, min_width=None)
\end{pythonbox}

\begin{pythondocstring}
Takes the values across the oder and finds the order center by looking for
the start and end of the order (and thus the center) above threshold

:param values: numpy array (1D) size = number of rows, the pixels in an
                order

:param threshold: float, the threshold above which to find pixels as being
                  part of an order

:param min_width: float, the minimum width for an order to be accepted

:return positions: numpy array (1D), size= number of rows,
                   the pixel positions in cvalues where the centers of each
                   order should be

:return widths:    numpy array (1D), size= number of rows,
                   the pixel positions in cvalues where the centers of each
                   order should be
\end{pythondocstring}
\end{minipage}

%----------------------------------------------------------------------------------------
\noindent\begin{minipage}{\textwidth}
\subsection{MergeCoefficients}

Defined in \spirouLOCOR\path{.spirouLOCOR.merge_coefficients}

\begin{pythonbox}
from SpirouDRS import spirouLOCOR
spirouLOCOR.MergeCoefficients(loc, coeffs, step)
spirouLOCOR.spirouLOCOR.merge_coefficients(loc, coeffs, step)
\end{pythonbox}

\begin{pythondocstring}
Takes a list of coefficients "coeffs" and merges them based on "step"
using the mean of "step" blocks

i.e. shrinks a list of N coefficients to N/2 (if step = 2) where
     indices 0 and 1 are averaged, indices 2 and 3 are averaged etc

:param loc: parameter dictionary, ParamDict containing data
        Must contain at least:
            number_orders: int, the number of orders in reference spectrum

:param coeffs: numpy array (2D), the list of coefficients
               shape = (number of orders x number of fit parameters)

:param step: int, the step between merges
             i.e. total size before = "number_orders"
                  total size after = "number_orders"/step

:return newcoeffs: numpy array (2D), the new list of coefficients
            shape = (number of orders/step x number of fit parmaeters)
\end{pythondocstring}
\end{minipage}

%----------------------------------------------------------------------------------------
\noindent\begin{minipage}{\textwidth}
\subsection{SigClipOrderFit}

Defined in \spirouLOCOR\path{.spirouLOCOR.sigmaclip_order_fit}

\begin{pythonbox}
from SpirouDRS import spirouLOCOR
spirouLOCOR.SigClipOrderFit(pp, loc, fitdata, mask, onum, rnum, kind)
spirouLOCOR.spirouLOCOR.sigmaclip_order_fit(pp, loc, fitdata, mask, onum, rnum, kind)
\end{pythonbox}

\begin{pythondocstring}
Performs a sigma clip fit for this order, uses the ctro positions or
sigo width values found in "FindOrderCtrs" or "find_order_centers" to do
the fit. Removes the largest residual from the initial fit (or subsequent
sigmaclips) value in x and y and recalculates the fit.

Does this until all the following conditions are NOT met:
       rms > 'ic_max_rms'   [kind='center' or kind='fwhm']
    or max_ptp > 'ic_max_ptp [kind='center']
    or max_ptp_frac > 'ic_ptporms_center'   [kind='center']
    or max_ptp_frac > 'ic_max_ptp_frac'     [kind='fwhm'

:param p: parameter dictionary, ParamDict containing constants
    Must contain at least:
            log_opt: string, log option, normally the program name
            IC_MAX_RMS_CENTER: required when kind="center", float, Maximum
                               rms for sigma-clip order fit (center
                               positions)
            IC_MAX_RMS_FWHM: required when kind="fwhm", float, Maximum
                             rms for sigma-clip order fit (width)
            IC_LOCDFITC: int, order of polynomial to fit for positions
            IC_MAX_PTP_CENTER: required when kind="center", float, Maximum
                               peak-to-peak for sigma-clip order fit
                               (center positions)
            IC_PTPORMS_CENTER: required when kind="center", float, Maximum
                               frac ptp/rms for sigma-clip order fit
                               (center positions)
            IC_LOCDFITW: int, order of polynomial to fit for widths
            IC_MAX_PTP_FRAC_FWHM: required when kind="fwhm", float, Maximum
                                  fractional peak-to-peak for sigma-clip
                                  order fit (width)
            DRS_DEBUG: int, Whether to run in debug mode
                            0: no debug
                            1: basic debugging on errors
                            2: recipes specific (plots and some code runs)
            DRS_PLOT: bool, Whether to plot (True to plot)

:param loc: parameter dictionary, ParamDict containing data
        Must contain at least:
            ctro: numpy array (2D), storage for the center positions
                  shape = (number of orders x number of columns (x-axis)
            sigo: numpy array (2D), storage for the width positions
                  shape = (number of orders x number of columns (x-axis)
            max_rmpts_pos: int, maximum number of removed points in sigma
                           clipping process, for center fits
            max_rmpts_wid: int, maximum number of removed poitns in sigma
                           clipping process, for width fits

\end{pythondocstring}
\end{minipage}

\noindent\begin{minipage}{\textwidth}                        
\begin{pythondocstring}
sigmaclip_order_fit (continued)

:param fitdata: dictionary, contains the fit data key value pairs for this
                 initial fit. keys are as follows:

        a = coefficients of the fit from key
        size = 'ic_locdfitc' [for kind='center'] or
             = 'ic_locdftiw' [for kind='fwhm']
        fit = the fity values for the fit (for x = loc['x'])
            where fity = Sum(a[i] * x^i)
        res = the residuals from y - fity
             where y = ctro [kind='center'] or
                     = sigo [kind='fwhm'])
        abs_res = abs(res)
        rms = the standard deviation of the residuals
        max_ptp = maximum residual value max(res)
        max_ptp_frac = max_ptp / rms  [kind='center']
                     = max(abs_res/y) * 100   [kind='fwhm']

:param mask: numpy array (1D) of booleans, True where we have non-zero
             widths
:param onum: int, order iteration number (running number over all
             iterations)
:param rnum: int, order number (running number of successful order
             iterations only)
:param kind: string, 'center' or 'fwhm', if 'center' then this fit is for
             the central p

:return fitdata: dictionary, contains the fit data key value pairs for this
                 initial fit. keys are as follows:

        a = coefficients of the fit from key
        size = 'ic_locdfitc' [for kind='center'] or
             = 'ic_locdftiw' [for kind='fwhm']
        fit = the fity values for the fit (for x = loc['x'])
            where fity = Sum(a[i] * x^i)
        res = the residuals from y - fity
             where y = ctro [kind='center'] or
                     = sigo [kind='fwhm'])
        abs_res = abs(res)
        rms = the standard deviation of the residuals
        max_ptp = maximum residual value max(res)
        max_ptp_frac = max_ptp / rms  [kind='center']
                     = max(abs_res/y) * 100   [kind='fwhm']
\end{pythondocstring}
\end{minipage}

%----------------------------------------------------------------------------------------