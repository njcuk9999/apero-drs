
%%%%%%%%%%%%%%%%%%%%%%%%%%%%%%%%%%%%%%%%%%%%%%%%%%%%%%%%
%%
\clearpage
\newpage
\noindent\begin{minipage}{\textwidth}
\section{The spirouTHORCA module}
\label{ch:the_module:spirouTHORCA}
%%
%%%%%%%%%%%%%%%%%%%%%%%%%%%%%%%%%%%%%%%%%%%%%%%%%%%%%%%%

%----------------------------------------------------------------------------------------
\subsection{CalcInstrumentDrift}

Defined in \spirouTHORCA\path{.spirouWAVE.calculate_instrument_drift}

\begin{pythonbox}
from SpirouDRS import spirouTHORCA
spirouTHORCA.CalcInstrumentDrift
spirouTHORCA.spirouWAVE.calculate_instrument_drift

\end{pythonbox}

\begin{pythondocstring}
Calculate the instrumental drift between the reference file and the current
file

:param p: parameter dictionary, ParamDict containing constants
    Must contain at least:
            log_opt: string, log option, normally the program name
            DRS_PLOT: bool, if True do plots else do not
            FIBER: string, the fiber type (i.e. AB or A or B or C)
            IC_HC_N_ORD_FINAL: int, the order to which the solution is
                               calculated
            IC_WAVE_IDRIFT_NOISE: float, the noise used in drift calculation
            IC_WAVE_IDRIFT_BOXSIZE: int, the size around a saturated pixel
                                    to count as bad
            IC_WAVE_IDRIFT_MAXFLUX: int, the maximum flux for a good
                                    (unsaturated) pixel
            IC_WAVE_IDRIFT_RV_CUT: float, the rv cut above which rv from
                                   orders are not used
            QC_WAVE_IDRIFT_NBORDEROUT: int, the maximum number of orders to
                                       remove from RV calculation
            QC_WAVE_IDRIFT_RV_MAX: float, the maximum allowed drift (in m/s)

:param loc: parameter dictionary, ParamDict containing data
    Must contain at least:
        HCDATA: numpy array (2D), the image data
        HCHDR: dictionary, the header dictionary for HCDATA

:return loc: parameter dictionary, the updated parameter dictionary
        Adds/updates the following:
            DRIFT_REF: string, the filename of the reference data
            DRIFT_RV: float, the mean instrumental RV
            DRIFT_NBCOS: int, the number of cosmic pixels found
            DRIFT_RFLUX: float, the mean flux ratio
            DRIFT_NBORDKILL: int, the number of orders removed
            DRIFT_NOISE: flat, the weighted mean uncertainty on the RV
\end{pythondocstring}
\end{minipage}


%----------------------------------------------------------------------------------------
\noindent\begin{minipage}{\textwidth}
\subsection{CalcLittrowSolution}

Defined in \spirouTHORCA\path{.spirouTHORCA.calculate_littrow_sol}

\begin{pythonbox}
from SpirouDRS import spirouTHORCA
spirouTHORCA.CalcLittrowSolution
spirouTHORCA.spirouTHORCA.calculate_littrow_sol

\end{pythonbox}

\begin{pythondocstring}
Calculate the Littrow solution for this iteration for a set of cut points

Uses ALL_LINES_i  where i = iteration to calculate the littrow solutions
for defiend cut points (given a cut_step and fit_deg of
IC_LITTROW_CUT_STEP_i and IC_LITTROW_FIT_DEG_i where i = iteration)

:param p: parameter dictionary, ParamDict containing constants
    Must contain at least:
        LOG_OPT: string, log option, normally the program name
        FIBER: string, the fiber type (i.e. AB or A or B or C)
        IC_LITTROW_REMOVE_ORDERS: list of ints, if not empty removes these
                                  orders from influencing the fit 
        IC_LITTROW_ORDER_INIT: int, defines the first order to for the fit
                               solution
        IC_HC_N_ORD_START: int, defines first order HC solution was
                           calculated from
        IC_HC_N_ORD_FINAL: int, defines last order HC solution was
                           calculated to
        IC_LITTROW_CUT_STEP_i: int, defines the step to use between 
                               cut points 
        IC_LITTROW_FIT_DEG_i: int, defines the polynomial fit degree

        where i = iteration

:param loc: parameter dictionary, ParamDict containing data
    Must contain at least:
        ECHELLE_ORDERS: numpy array (1D), the echelle order numbers
        HCDATA: numpy array (2D), the image data (used for shape)
        ALL_LINES_i: list of numpy arrays, length = number of orders
                     each numpy array contains gaussian parameters
                     for each found line in that order

        where i = iteration

:param ll: numpy array (1D), the initial guess wavelengths for each line
:param iteration: int, the iteration number (used so we can store multiple
                  calculations in loc, defines "i" in input and outputs
                  from p and loc
:param log: bool, if True will print a final log message on completion with
            some stats

:return loc: parameter dictionary, the updated parameter dictionary
        Adds/updates the following:
            X_CUT_POINTS_i: numpy array (1D), the x pixel cut points
            LITTROW_MEAN_i: list, the mean position of each cut point
            LITTROW_SIG_i: list, the mean FWHM of each cut point
            LITTROW_MINDEV_i: list, the minimum deviation of each cut point
            LITTROW_MAXDEV_i: list, the maximum deviation of each cut point
            LITTROW_PARAM_i: list of numpy arrays, the gaussian fit
                             coefficients of each cut point
            LITTROW_XX_i: list, the order positions of each cut point
            LITTROW_YY_i: list, the residual fit of each cut point

            where i = iteration

ALL_LINES_i definition:
    ALL_LINES_i[row] = [gparams1, gparams2, ..., gparamsN]

                where:
                    gparams[0] = output wavelengths
                    gparams[1] = output sigma (gauss fit width)
                    gparams[2] = output amplitude (gauss fit)
                    gparams[3] = difference in input/output wavelength
                    gparams[4] = input amplitudes
                    gparams[5] = output pixel positions
                    gparams[6] = output pixel sigma width
                                      (gauss fit width in pixels)
                    gparams[7] = output weights for the pixel position
\end{pythondocstring}
\end{minipage}


%----------------------------------------------------------------------------------------
\noindent\begin{minipage}{\textwidth}
\subsection{DetectBadLines}

Defined in \spirouTHORCA\path{.spirouTHORCA.detect_bad_lines}

\begin{pythonbox}
from SpirouDRS import spirouTHORCA
spirouTHORCA.DetectBadLines(p, loc, key=None, iteration=0)
spirouTHORCA.spirouTHORCA.detect_bad_lines(p, loc, key=None, iteration=0)
\end{pythonbox}

\begin{pythondocstring}
Detect and filter out the bad lines for this iteration

:param p: parameter dictionary, ParamDict containing constants
    Must contain at least:
        IC_MAX_ERRW_ONFIT: float, the maximum error on the first guess
                           lines
        IC_MAX_SIGLL_CAL_LINES: float, the maximum sigma fit of the guessed
                                lines
        IC_MAX_AMPL_LINE: float, the maximum amplitude the guessed lines
                          can have
        IC_HC_T_ORDER_START: int, the echelle number of the first
                             extracted order

:param loc: parameter dictionary, ParamDict containing data
    Must contain at least:
        ECHELLE_ORDERS: numpy array (1D), the echelle order numbers
        ALL_LINES_i or "key": if key is None use ALL_LINES_i
                              where i = iteration
                     list of numpy arrays, length = number of orders
                     each numpy array contains gaussian parameters
                     for each found line in that order

:param key: string or None, if string should exist in "loc" and replaces
            "ALL_LINES_i" where i = iteration, and should point to a list
            of numpy arrays, length = number of orders, each numpy array
            contains gaussian parameters for each found line in that order
:param iteration: int, the iteration number (used so we can store multiple
                  calculations in loc, defines "i" in input and outputs
                  from p and loc

:return loc: parameter dictionary, the updated parameter dictionary
    Adds/updates the following:
        ALL_LINES_i or "key": list of numpy arrays, ALL_LINES_i or key input
                              with the bad lines filtered out

ALL_LINES_i definition:
    ALL_LINES_i[row] = [gparams1, gparams2, ..., gparamsN]

                where:
                    gparams[0] = output wavelengths
                    gparams[1] = output sigma (gauss fit width)
                    gparams[2] = output amplitude (gauss fit)
                    gparams[3] = difference in input/output wavelength
                    gparams[4] = input amplitudes
                    gparams[5] = output pixel positions
                    gparams[6] = output pixel sigma width
                                      (gauss fit width in pixels)
                    gparams[7] = output weights for the pixel position
\end{pythondocstring}
\end{minipage}

%----------------------------------------------------------------------------------------
\noindent\begin{minipage}{\textwidth}
\subsection{ExtrapolateLittrowSolution}

Defined in \spirouTHORCA\path{.spirouTHORCA.extrapolate_littrow_sol}

\begin{pythonbox}
from SpirouDRS import spirouTHORCA
spirouTHORCA.ExtrapolateLittrowSolution(p, loc, ll, iteration=0)
spirouTHORCA.spirouTHORCA.extrapolate_littrow_sol(p, loc, ll, iteration=0)
\end{pythonbox}

\begin{pythondocstring}
Extrapolate and fit the Littrow solution at defined points and return
the wavelengths, solutions, and cofficients of the littorw fits

:param p: parameter dictionary, ParamDict containing constants
    Must contain at least:
        IC_LITTROW_ORDER_FIT_DEG: int, defines the polynomial fit degree
        IC_HC_T_ORDER_START
        IC_LITTROW_ORDER_INIT int, defines the first order to for the fit
                               solution
:param loc: parameter dictionary, ParamDict containing data
    Must contain at least:
        HCDATA: numpy array (2D), the image data (used for shape)
        LITTROW_PARAM_i: list of numpy arrays, the gaussian fit
                         coefficients of each cut point
        X_CUT_POINTS_i: numpy array (1D), the x pixel cut points
    
        where i = iteration
        
:param ll: numpy array (1D), the initial guess wavelengths for each line
:param iteration: int, the iteration number (used so we can store multiple
                  calculations in loc, defines "i" in input and outputs
                  from p and loc
                  
:return loc: parameter dictionary, the updated parameter dictionary
        Adds/updates the following:
            LITTROW_EXTRAP_i: numpy array (2D), 
                              size=([no. orders] by [no. cut points])
                              the wavelength values at each cut point for
                              each order
            LITTROW_EXTRAP_SOL_i: numpy array (2D), 
                              size=([no. orders] by [no. cut points])
                              the wavelength solution at each cut point for
                              each order
            LITTROW_EXTRAP_PARAM_i: numy array (2D),
                              size=([no. orders] by [the fit degree +1])
                              the coefficients of the fits for each cut 
                              point for each order
            
            where i = iteration
\end{pythondocstring}
\end{minipage}


%----------------------------------------------------------------------------------------


\noindent\begin{minipage}{\textwidth}
\subsection{FirstGuessSolution}

Defined in \spirouTHORCA\path{.spirouTHORCA.first_guess_at_wave_solution}

\begin{pythonbox}
from SpirouDRS import spirouTHORCA
spirouTHORCA.FirstGuessSolution(p, loc, mode=0)
spirouTHORCA.spirouTHORCA.first_guess_at_wave_solution(p, loc, mode=0)
\end{pythonbox}

\begin{pythondocstring}
First guess at wave solution, consistency check, using the wavelength
solutions line list

:param p: parameter dictionary, ParamDict containing constants
    Must contain at least:
        IC_HC_N_ORD_START: int, defines first order solution is calculated
        IC_HC_N_ORD_FINAL: int, defines last order solution is calculated
                            from
        IC_HC_T_ORDER_START: int, defines the echelle order of
                            the first e2ds order
        log_opt: string, log option, normally the program name
        fiber: string, the fiber number

:param loc: parameter dictionary, ParamDict containing data
    Must contain at least:
        ECHELLE_ORDERS: numpy array (1D), the echelle order numbers

:param mode: string, if mode="new" uses python to work out gaussian fit
                     if mode="old" uses FORTRAN (testing only) requires
                     compiling of FORTRAN fitgaus into spirouTHORCA dir

:return loc: parameter dictionary, the updated parameter dictionary
        Adds/updates the following:
            ECHELLE_ORDERS: numpy array, the echelle orders to fit
            LL_INIT: numpy array, the initial guess at the line list
            LL_LINE: numpy array, the line list wavelengths from file
            AMPL_LINE: numpy array, the line list amplitudes from file
            ALL_LINES: list of numpy arrays, length = number of orders
                        each numpy array contains gaussian parameters
                        for each found line in that order

ALL_LINES_i definition:
    ALL_LINES_i[row] = [gparams1, gparams2, ..., gparamsN]

                where:
                    gparams[0] = output wavelengths
                    gparams[1] = output sigma (gauss fit width)
                    gparams[2] = output amplitude (gauss fit)
                    gparams[3] = difference in input/output wavelength
                    gparams[4] = input amplitudes
                    gparams[5] = output pixel positions
                    gparams[6] = output pixel sigma width
                                      (gauss fit width in pixels)
                    gparams[7] = output weights for the pixel position
\end{pythondocstring}
\end{minipage}

%----------------------------------------------------------------------------------------
\noindent\begin{minipage}{\textwidth}
\subsection{Fit1DSolution}

Defined in \spirouTHORCA\path{.spirouTHORCA.fit_1d_solution}

\begin{pythonbox}
from SpirouDRS import spirouTHORCA
spirouTHORCA.Fit1DSolution(p, loc, ll, iteration=0)
spirouTHORCA.spirouTHORCA.fit_1d_solution(p, loc, ll, iteration=0)
\end{pythonbox}

\begin{pythondocstring}
Fits the 1D solution between wavelength and pixel postion and inverts it
into a fit between pixel position and wavelength

:param p: parameter dictionary, ParamDict containing constants
    Must contain at least:
        LOG_OPT: string, log option, normally the program name
        FIBER: string, the fiber type (i.e. AB or A or B or C)

:param loc: parameter dictionary, ParamDict containing data
    Must contain at least:
        ECHELLE_ORDERS: numpy array (1D), the echelle order numbers
        HCDATA: numpy array (2D), the image data (used for shape)
        ALL_LINES_i: list of numpy arrays, length = number of orders
                     each numpy array contains gaussian parameters
                     for each found line in that order
        IC_ERRX_MIN: float, the minimum instrumental error
        IC_LL_DEGR_FIT: int, the wavelength fit polynomial order
        IC_MAX_LLFIT_RMS: float, the max rms for the wavelength
                          sigma-clip fit
        IC_HC_T_ORDER_START: int, defines the echelle order of
                            the first e2ds order

:param ll: numpy array (1D), the initial guess wavelengths for each line
:param iteration: int, the iteration number (used so we can store multiple
                  calculations in loc, defines "i" in input and outputs
                  from p and loc

:return loc: parameter dictionary, the updated parameter dictionary
        Adds/updates the following:
            X_MEAN_i: float, the mean residual from the fit [km/s]
            X_VAR_i: float, the rms residual from the fit [km/s]
            X_ITER_i: numpy array(1D), the last line central position, FWHM
            and the number of lines in each order
            X_PARAM_i: numpy array (1D), the coefficients of the fit to x
                       pixel position as a function of wavelength
            X_DETAILS_i: list, [lines, xfit, cfit, weight] where
                         lines= original wavelength-centers used for the fit
                         xfit= original pixel-centers used for the fit
                         cfit= fitted pixel-centers using fit coefficients
                         weight=the line weights used
            SCALE_i: list, the convertion for each line into wavelength
            LL_MEAN_i: float, the mean residual after inversion [km/s]
            LL_VAR_i: float, the rms residual after inversion [km/s]
            LL_PARAM_i: numpy array (1D), the cofficients of the inverted
                        fit (i.e. wavelength as a function of x pixel
                        position)
            LL_DETAILS_i: numpy array (1D), the [nres, wei] where
                          nres = normalised residuals in km/s
                          wei = the line weights
            LL_OUT_i: numpy array (2D), the output wavelengths for each
                      pixel and each order (in the shape of original image)
            DLL_OUT_i: numpy array (2D), the output delta wavelengths for
                       each pixel and each order (in the shape of original
                       image)

            where i = iteration

ALL_LINES_i definition:
    ALL_LINES_i[row] = [gparams1, gparams2, ..., gparamsN]

                where:
                    gparams[0] = output wavelengths
                    gparams[1] = output sigma (gauss fit width)
                    gparams[2] = output amplitude (gauss fit)
                    gparams[3] = difference in input/output wavelength
                    gparams[4] = input amplitudes
                    gparams[5] = output pixel positions
                    gparams[6] = output pixel sigma width
                                      (gauss fit width in pixels)
                    gparams[7] = output weights for the pixel position
\end{pythondocstring}
\end{minipage}

%----------------------------------------------------------------------------------------
\noindent\begin{minipage}{\textwidth}
\subsection{FPWaveSolution}

Defined in \spirouTHORCA\path{.spirouWAVE.fp_wavelength_sol}

\begin{pythonbox}
from SpirouDRS import spirouTHORCA
spirouTHORCA.FPWaveSolution(p, loc, mode='new')
spirouTHORCA.spirouWAVE.fp_wavelength_sol(p, loc, mode='new')
\end{pythonbox}

\begin{pythondocstring}
Derives the FP line wavelengths from the first solution
Follows the Bauer et al 2015 procedure

:param p: parameter dictionary, ParamDict containing constants
    Must contain at least:
        IC_FP_N_ORD_START: int, defines first order FP solution is
                           calculated from

        IC_FP_N_ORD_FINAL: int, defines last order FP solution is
                           calculated to

        IC_HC_N_ORD_START_2: int, defines first order HC solution was
                             calculated from

        IC_HC_N_ORD_FINAL_2: int, defines last order HC solution was
                             calculated to

        IC_FP_THRESHOLD: float, defines threshold for detecting FP lines

        IC_FP_SIZE: int, defines size of region where each line is fitted

        IC_FP_DOPD0: float, initial value of FP effective cavity width

        IC_FP_FIT_DEGREE: int, degree of polynomial fit

        log_opt: string, log option, normally the program name

:param loc: parameter dictionary, ParamDict containing data
    Must contain at least:
        FP_DATA: the FP e2ds data
        LITTROW_EXTRAP_SOL_1: the wavelength solution derived from the HC
                              and Littrow-constrained
        ALL_LINES: list of numpy arrays, length = number of orders
                   each numpy array contains gaussian parameters
                   for each found line in that order
        BLAZE: numpy array (2D), the blaze data

:return loc: parameter dictionary, the updated parameter dictionary
        Adds/updates the following:
            FP_LL_POS: numpy array, the initial wavelengths of the FP lines
            FP_XX_POS: numpy array, the pixel positions of the FP lines
            FP_M: numpy array, the FP line numbers
            FP_DOPD_T: numpy array, the measured cavity width for each line
            FP_AMPL: numpy array, the FP line amplitudes
            FP_LL_POS_NEW: numpy array, the corrected wavelengths of the
                           FP lines
            ALL_LINES_2: list of numpy arrays, length = number of orders
                         each numpy array contains gaussian parameters
                         for each found line in that order
\end{pythondocstring}
\end{minipage}

%----------------------------------------------------------------------------------------
\noindent\begin{minipage}{\textwidth}
\subsection{GetE2DSll}

Defined in \spirouTHORCA\path{.spirouTHORCA.get_e2ds_ll}

\begin{pythonbox}
from SpirouDRS import spirouTHORCA
spirouTHORCA.GetE2DSll(p, hdr=None, filename=None, key=None)
spirouTHORCA.spirouTHORCA.get_e2ds_ll(p, hdr=None, filename=None, key=None)
\end{pythonbox}

\begin{pythondocstring}
Get the line list for the e2ds file from "filename" or from calibration
database using hdr (aqctime) and key. Line list is constructed from
fit coefficients stored in keywords:
    'kw_TH_ORD_N', 'kw_TH_LL_D', 'kw_TH_NAXIS1'

:param pp: parameter dictionary, ParamDict containing constants
    Must contain at least:
            log_opt: string, log option, normally the program name
            kw_TH_COEFF_PREFIX: list, the keyword store for the prefix to
                                use to get the TH line list fit coefficients

:param hdr: dictionary or None, the HEADER dictionary with the acquisition
            time in to use in the calibration database to get the filename
            with key=key (or if None key='WAVE_AB')
:param filename: string or None, the file to get the line list from
                 (overrides getting the filename from calibration database)
:param key: string or None, if defined the key in the calibration database
            to get the file from (using the HEADER dictionary to deal with
            calibration database time constraints for duplicated keys.

:return ll: numpy array (1D), the line list values
:return param_ll: numpy array (1d), the line list fit coefficients (used to
                  generate line list - read from file defined)
\end{pythondocstring}
\end{minipage}

%----------------------------------------------------------------------------------------
\noindent\begin{minipage}{\textwidth}
\subsection{GetLampParams}

Defined in \spirouTHORCA\path{.spirouTHORCA.get_lamp_parameters}

\begin{pythonbox}
from SpirouDRS import spirouTHORCA
spirouTHORCA.GetLampParams(p, filename=None, kind=None)
spirouTHORCA.spirouTHORCA.get_lamp_parameters(p, filename=None, kind=None)
\end{pythonbox}

\begin{pythondocstring}
Get lamp parameters from either a specified lamp type="kind" or a filename
or from p['ARG_FILE_NAMES'][0] (if no filename or kind defined)

:param p: parameter dictionary, ParamDict containing constants
    Must contain at least:
        IC_LAMPS: list of strings, the different allowed lamp types
        log_opt: string, log option, normally the program name
:param filename: string or None, the filename to check for the lamp
                 substring in
:param kind: string or None, the lamp type

:return p: parameter dictionary, the updated parameter dictionary
        Adds the following:
            LAMP_TYPE: string, the type of lamp (e.g. UNe or TH)
            IC_LL_LINE_FILE: string, the file name of the line list to use
            IC_CAT_TYPE: string, the line list catalogue type
\end{pythondocstring}
\end{minipage}

%----------------------------------------------------------------------------------------
\noindent\begin{minipage}{\textwidth}
\subsection{JoinOrders}

Defined in \spirouTHORCA\path{.spirouTHORCA.join_orders}

\begin{pythonbox}
from SpirouDRS import spirouTHORCA
spirouTHORCA.JoinOrders
spirouTHORCA.spirouTHORCA.join_orders
\end{pythonbox}

\begin{pythondocstring}
Merge the littrow extrapolated solutions with the fitted line solutions

:param p: parameter dictionary, ParamDict containing constants
    Must contain at least:
        IC_HC_N_ORD_START: int, defines first order solution is calculated
        IC_HC_N_ORD_FINAL: int, defines last order solution is calculated
                            from

:param loc: parameter dictionary, ParamDict containing data
    Must contain at least:
        LL_OUT_2: numpy array (2D), the output wavelengths for each
                  pixel and each order (in the shape of original image)
        DLL_OUT_2: numpy array (2D), the output delta wavelengths for
                   each pixel and each order (in the shape of original
                   image)
        LITTROW_EXTRAP_SOL_2: numpy array (2D),
                          size=([no. orders] by [no. cut points])
                          the wavelength solution at each cut point for
                          each order
        LITTROW_EXTRAP_PARAM_2: numy array (2D),
                          size=([no. orders] by [the fit degree +1])
                          the coefficients of the fits for each cut
                          point for each order
    
:return loc: parameter dictionary, the updated parameter dictionary
    Adds/updates the following:
        LL_FINAL: numpy array, the joined littrow extrapolated and fitted
                  solution wavelengths
        LL_PARAM_FINAL: numpy array, the joined littrow extrapolated and
                        fitted fit coefficients
\end{pythondocstring}
\end{minipage}

%----------------------------------------------------------------------------------------
\noindent\begin{minipage}{\textwidth}
\subsection{SecondGuessSolution}

Defined in \spirouTHORCA\path{.spirouTHORCA.second_guess_at_wave_solution}

\begin{pythonbox}
from SpirouDRS import spirouTHORCA
spirouTHORCA.SecondGuessSolution(p, loc, mode=0)
spirouTHORCA.spirouTHORCA.second_guess_at_wave_solution(p, loc, mode=0)
\end{pythonbox}

\begin{pythondocstring}
Second guess at wave solution, consistency check, using the wavelength
solutions line list

:param p: parameter dictionary, ParamDict containing constants
    Must contain at least:
        IC_LL_SP_MIN: int, minimum wavelength of the catalog
        IC_LL_SP_MAX: int, maximum wavelength of the catalog
        IC_RESOL: int, Resolution of spectrograph
        IC_LL_FREE_SPAN_2: int, window size in sigma unit
        IC_HC_N_ORD_START_2: int, defines first order solution is
                             calculated from
        IC_HC_N_ORD_FINAL_2: int, defines last order solution is
                             calculated from
        IC_HC_T_ORDER_START: int, defines the echelle order of
                            the first e2ds order

:param loc: parameter dictionary, ParamDict containing data
    Must contain at least:
        ECHELLE_ORDERS: numpy array (1D), the echelle order numbers
        HCDATA: numpy array (2D), the image data
        LL_LINE: numpy array, the line list wavelengths from file
        AMPL_LINE: numpy array, the line list amplitudes from file
        LITTROW_EXTRAP_SOL_1: numpy array (2D),
                          size=([no. orders] by [no. cut points])
                          the wavelength solution at each cut point for
                          each order

:param mode: string, if mode="new" uses python to work out gaussian fit
                     if mode="old" uses FORTRAN (testing only) requires
                     compiling of FORTRAN fitgaus into spirouTHORCA dir

:return loc: parameter dictionary, the updated parameter dictionary
        Adds/updates the following:
            ALL_LINES_2: list of numpy arrays, length = number of orders
                         each numpy array contains gaussian parameters
                         for each found line in that order

ALL_LINES_2 definition:
    ALL_LINES_2[row] = [gparams1, gparams2, ..., gparamsN]
                where:
                    gparams[0] = output wavelengths
                    gparams[1] = output sigma (gauss fit width)
                    gparams[2] = output amplitude (gauss fit)
                    gparams[3] = difference in input/output wavelength
                    gparams[4] = input amplitudes
                    gparams[5] = output pixel positions
                    gparams[6] = output pixel sigma width
                                      (gauss fit width in pixels)
                    gparams[7] = output weights for the pixel position
\end{pythondocstring}
\end{minipage}

%----------------------------------------------------------------------------------------