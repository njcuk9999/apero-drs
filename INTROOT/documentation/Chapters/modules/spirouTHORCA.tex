
%%%%%%%%%%%%%%%%%%%%%%%%%%%%%%%%%%%%%%%%%%%%%%%%%%%%%%%%
%%
\clearpage
\newpage
\noindent\begin{minipage}{\textwidth}
\section{The spirouTHORCA module}
\label{ch:the_module:spirouTHORCA}
%%
%%%%%%%%%%%%%%%%%%%%%%%%%%%%%%%%%%%%%%%%%%%%%%%%%%%%%%%%

%----------------------------------------------------------------------------------------
\subsection{GetE2DSll}

Defined in \spirouTHORCA\path{.spirouTHORCA.get_e2ds_ll}

\begin{pythonbox}
from SpirouDRS import spirouTHORCA
spirouTHORCA.GetE2DSll(p, hdr=None, filename=None, key=None)
spirouTHORCA.spirouTHORCA.get_e2ds_ll(p, hdr=None, filename=None, key=None)
\end{pythonbox}

\begin{pythondocstring}
Get the line list for the e2ds file from "filename" or from calibration
database using hdr (aqctime) and key. Line list is constructed from
fit coefficients stored in keywords:
    'kw_TH_ORD_N', 'kw_TH_LL_D', 'kw_TH_NAXIS1'

:param pp: parameter dictionary, ParamDict containing constants
    Must contain at least:
            log_opt: string, log option, normally the program name
            kw_TH_COEFF_PREFIX: list, the keyword store for the prefix to
                                use to get the TH line list fit coefficients

:param hdr: dictionary or None, the HEADER dictionary with the acquisition
            time in to use in the calibration database to get the filename
            with key=key (or if None key='WAVE_AB')
:param filename: string or None, the file to get the line list from
                 (overrides getting the filename from calibration database)
:param key: string or None, if defined the key in the calibration database
            to get the file from (using the HEADER dictionary to deal with
            calibration database time constraints for duplicated keys.

:return ll: numpy array (1D), the line list values
:return param_ll: numpy array (1d), the line list fit coefficients (used to
                  generate line list - read from file defined)
\end{pythondocstring}
\end{minipage}

%----------------------------------------------------------------------------------------
\noindent\begin{minipage}{\textwidth}
\subsection{Getll}

Defined in \spirouTHORCA\path{.spirouTHORCA.get_ll_from_coefficients}

\begin{pythonbox}
from SpirouDRS import spirouTHORCA
spirouTHORCA.Getll(params, nx, nbo)
spirouTHORCA.spirouTHORCA.get_ll_from_coefficients(params, nx, nbo)
\end{pythonbox}

\begin{pythondocstring}
Use the coefficient matrix "params" to construct fit values for each order
(dimension 0 of coefficient matrix) for values of x from 0 to nx
(interger steps)

:param params: numpy array (2D), the coefficient matrix
               size = (number of orders x number of fit coefficients)

:param nx: int, the number of values and the maximum value of x to use
           the coefficients for, where x is such that

            yfit = p[0]*x**(N-1) + p[1]*x**(N-2) + ... + p[N-2]*x + p[N-1]

            N = number of fit coefficients
            and p is the coefficients for one order
            (i.e. params = [ p_1, p_2, p_3, p_4, p_5, ... p_nbo]

:param nbo: int, the number of orders to use

:return ll: numpy array (2D): the yfit values for each order
            (i.e. ll = [yfit_1, yfit_2, yfit_3, ..., yfit_nbo] )
\end{pythondocstring}
\end{minipage}

%----------------------------------------------------------------------------------------
\noindent\begin{minipage}{\textwidth}
\subsection{Getdll}

Defined in \spirouTHORCA\path{.spirouTHORCA.get_dll_from_coefficients}

\begin{pythonbox}
from SpirouDRS import spirouTHORCA
spirouTHORCA.Getdll(params, nx, nbo)
spirouTHORCA.spirouTHORCA.get_dll_from_coefficients(params, nx, nbo)
\end{pythonbox}

\begin{pythondocstring}
Derivative of the coefficients, using the coefficient matrix "params"
to construct the derivative of the fit values for each order
(dimension 0 of coefficient matrix) for values of x from 0 to nx
(interger steps)

:param params: numpy array (2D), the coefficient matrix
               size = (number of orders x number of fit coefficients)

:param nx: int, the number of values and the maximum value of x to use
           the coefficients for, where x is such that

            yfit = p[0]*x**(N-1) + p[1]*x**(N-2) + ... + p[N-2]*x + p[N-1]

            dyfit = p[0]*(N-1)*x**(N-2) + p[1]*(N-2)*x**(N-3) + ... +
                    p[N-3]*x + p[N-2]

            N = number of fit coefficients
            and p is the coefficients for one order
            (i.e. params = [ p_1, p_2, p_3, p_4, p_5, ... p_nbo]

:param nbo: int, the number of orders to use

:return ll: numpy array (2D): the yfit values for each order
            (i.e. ll = [dyfit_1, dyfit_2, dyfit_3, ..., dyfit_nbo] )
\end{pythondocstring}
\end{minipage}

%----------------------------------------------------------------------------------------


\noindent\begin{minipage}{\textwidth}
\subsection{FirstGuessSolution}

Defined in \spirouTHORCA\path{.spirouTHORCA.first_guess_at_wave_solution}

\begin{pythonbox}
from SpirouDRS import spirouTHORCA
spirouTHORCA.FirstGuessSolution(p, loc)
spirouTHORCA.spirouTHORCA.first_guess_at_wave_solution(p, loc)
\end{pythonbox}

\begin{pythondocstring}
First guess at wave solution, consistency check, using the wavelength
solutions line list

:param p: parameter dictionary, ParamDict containing constants
    Must contain at least:
        CAL_HC_N_ORD_FINAL: int, defines first order solution is calculated
                            from
        CAL_HC_T_ORDER_START: int, defines the first order solution is 
                              calculated from
        log_opt: string, log option, normally the program name
        fiber: string, the fiber number

:param loc: parameter dictionary, ParamDict containing data
    Must contain at least:

:return loc: parameter dictionary, the updated parameter dictionary
        Adds/updates the following:
            FIT_ORDERS: numpy array, the orders to fit
            LL_INIT: numpy array, the initial guess at the line list
            LL_LINE: numpy array, the line list wavelengths from file
            AMPL_LINE: numpy array, the line list amplitudes from file
\end{pythondocstring}
\end{minipage}

%----------------------------------------------------------------------------------------

\noindent\begin{minipage}{\textwidth}
\subsection{GetLampParams}

Defined in \spirouTHORCA\path{.spirouTHORCA.get_lamp_parameters}

\begin{pythonbox}
from SpirouDRS import spirouTHORCA
spirouTHORCA.GetLampParams(p, filename=None, kind=None)
spirouTHORCA.spirouTHORCA.get_lamp_parameters(p, filename=None, kind=None)
\end{pythonbox}

\begin{pythondocstring}
Get lamp parameters from either a specified lamp type="kind" or a filename
or from p['ARG_FILE_NAMES'][0] (if no filename or kind defined)

:param p: parameter dictionary, ParamDict containing constants
    Must contain at least:
        IC_LAMPS: list of strings, the different allowed lamp types
        log_opt: string, log option, normally the program name
:param filename: string or None, the filename to check for the lamp
                 substring in
:param kind: string or None, the lamp type

:return p: parameter dictionary, the updated parameter dictionary
        Adds the following:
            LAMP_TYPE: string, the type of lamp (e.g. UNe or TH)
            IC_LL_LINE_FILE: string, the file name of the line list to use
            IC_CAT_TYPE: string, the line list catalogue type
\end{pythondocstring}
\end{minipage}

%----------------------------------------------------------------------------------------