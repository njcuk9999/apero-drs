%%%%%%%%%%%%%%%%%%%%%%%%%%%%%%%%%%%%%%%%%%%%%%%%%%%%%%%%
%%
\chapter{Using the DRS}
\label{chapter:using_the_drs}
%%
%%%%%%%%%%%%%%%%%%%%%%%%%%%%%%%%%%%%%%%%%%%%%%%%%%%%%%%%

There are two ways to run the DRS recipes. The first (described in Section \ref{chapter:using_the_drs:direct}) directly calls the code and inputs arguments (either from the command line or from python), the second way is to import the recipes in a python script and define arguments in a call to a function (see Section \ref{chapter:using_the_drs:script}).

%%%%%%%%%%%%%%%%%%%%%%%%%%%%%%%%%%%%%%%%%%%%%%%%%%%%%%%%
%%
\section{Running the DRS recipes directly}
\label{chapter:using_the_drs:direct}
%%
%%%%%%%%%%%%%%%%%%%%%%%%%%%%%%%%%%%%%%%%%%%%%%%%%%%%%%%%

As in Chapter \ref{chapter:installation}, using Linux or \mac one can run DRS recipes from the command line or from python, in windows one is required to be in python before running the scripts. Below we use \calDARK as an example:
\begin{itemize}
\item To run from command line type:
\begin{cmdbox}
(*\calDARK*) YYMMDD Filenames
\end{cmdbox}

\item To run from python/ipython from the command line type:
\begin{cmdbox}
python (*\calDARK*) YYMMDD Filenames
ipython (*\calDARK*) YYMMDD Filenames
\end{cmdbox}

\item To run from ipython, open ipython and type:
\begin{pythonbox}
@run@ (*\calDARK*) YYMMDD Filenames
\end{pythonbox}
\end{itemize}

%%%%%%%%%%%%%%%%%%%%%%%%%%%%%%%%%%%%%%%%%%%%%%%%%%%%%%%%
%%
\section{Running the DRS recipes from a python script}
\label{chapter:using_the_drs:script}
%%
%%%%%%%%%%%%%%%%%%%%%%%%%%%%%%%%%%%%%%%%%%%%%%%%%%%%%%%%

In any operating system one can also import a recipe and call a function to run the code. This is useful in batch operations, timing tests and unit tests for example. Below we use \calDARK as an example:

\begin{pythonbox}
# import the recipe
import cal_DARK_spirou
# define the night folder name
night_name = "20170710"
# define the file(s) to run through the code
files = ['dark_dark02d406.fits']
# run code
cal_validate_spirou.main(night_name=night_name, files=files)
\end{pythonbox}

%%%%%%%%%%%%%%%%%%%%%%%%%%%%%%%%%%%%%%%%%%%%%%%%%%%%%%%%
%%
\clearpage
\newpage
\section{Working example of the code for SPIRou}
\label{chapter:using_the_drs:working_example}
%%
%%%%%%%%%%%%%%%%%%%%%%%%%%%%%%%%%%%%%%%%%%%%%%%%%%%%%%%%

% ----------------------------------------------
\subsection{Overview}
\label{chapter:using_the_drs:working_example:overview}
% ----------------------------------------------

For this example all files are from 2018-04-09.

\noindent following our example data architecture (from Section \ref{ch:install:setup} and shown explicitly in Section \ref{ch:data_architecture:folder_layout}) all files should be places in the \definevariable{text:drs_data_raw}{DRS\_DATA\_RAW} (\textcolor{blue}{/drs/data/raw} in our case) and placed into a night directory: `20180409'.

\noindent Starting with RAMP files and ending with extracted orders and calculated drifts we need to run six codes:
\begin{enumerate}
\item \calDARK \hfill (See Section \ref{ch:the_recipes:cal_DARK_spirou})
\item \calbadpix \hfill (See Section \ref{ch:the_recipes:cal_BADPIX_spirou})
\item \callocRAW ($\times$2) \hfill (See Section \ref{ch:the_recipes:cal_loc_RAW_spirou})
\item \calSLIT \hfill (See Section \ref{ch:the_recipes:cal_SLIT_spirou})
\item \calFFraw ($\times$2) \hfill (See Section \ref{ch:the_recipes:cal_FF_RAW_spirou})
\item \calextractRAW \hfill (See Section \ref{ch:the_recipes:cal_extract_RAW_spirou})
\item \calDRIFTE \hfill (See Section \ref{ch:the_recipes:cal_DRIFT_RAW_spirou})
\item \calDRIFTPEAK \hfill (See Section \ref{ch:the_recipes:cal_DRIFT_RAW_spirou})
\item \calCCF \hfill (See Section \ref{ch:the_recipes:cal_CCF_E2DS_spirou})
\end{enumerate}

% ----------------------------------------------
\subsection{Run through from command line/python shell (Linux and macOS)}
\label{chapter:using_the_drs:working_example:run_cmd}
% ----------------------------------------------

As long as all codes are executable (see Section \ref{ch:install:installunix:executable}) one can run all codes from the command line or if not executable or one has a preference for python one can run the following with `python \{command\}', `ipython \{command\}' or indeed through an interactive ipython session using `run \{command\}'.

\begin{enumerate}

\item run the pre-processing on all files:
\begin{cmdbox}
cal_preprocess_spirou 20180409 "*.fits"
\end{cmdbox}
\begin{note}
This adds the `\_pp' suffix on the end of all pre-processed files.
\end{note}

\item run the dark extraction on the `dark\_dark' file:
\begin{cmdbox}
cal_DARK_spirou.py 20180409 dark_dark_001_pp.fits
\end{cmdbox}

\item run the bad pixel identification on the `flat\_flat' and `dark\_dark' files:
\begin{cmdbox}
cal_BADPIX_spirou.py 20180409 flat_flat_001_pp.fits dark_dark_001_pp.fits
\end{cmdbox}

\item run the order localisation on the `dark\_flat' files:
\begin{cmdbox}
cal_loc_RAW_spirou.py 20180409 dark_flat_001_pp.fits
\end{cmdbox}

\item run the order localisation on the `flat\_dark' files:
\begin{cmdbox}
cal_loc_RAW_spirou.py 20180409 flat_dark_001_pp.fits
\end{cmdbox}

\item run the slit calibration on the `fp\_fp' files.
\begin{cmdbox}
cal_SLIT_spirou.py 20180409 fp_fp_001_pp.fits
\end{cmdbox}

\item run the flat field creation on the `flat\_flat' files:
\begin{cmdbox}
cal_FF_RAW_spirou.py 20180409 flat_flat_001_pp.fits
\end{cmdbox}

\item run the extraction on the `fp\_fp' and `hcone' files:
\begin{cmdbox}
cal_extract_RAW_spirou.py 20180409 fp_fp_001_pp.fits
cal_extract_RAW_spirou.py 20180409 fp_fp_002_pp.fits
cal_extract_RAW_spirou.py 20180409 fp_fp_003_pp.fits
cal_extract_RAW_spirou.py 20180409 hcone_hcone_001_pp.fits
\end{cmdbox}

\item run the drift calculation on the `fp\_fp' files:
\begin{cmdbox}
cal_DRIFT_E2DS_spirou.py 20180409 fp_fp_001_pp_e2ds_AB.fits
cal_DRIFTPEAK_E2DS_spirou.py 20180409 fp_fp_001_pp_e2ds_AB.fits
\end{cmdbox}

\item run the CCF calculation on an extracted file:
\begin{cmdbox}
cal_CCF_E2DS_spirou 20180409 hcone_hcone_001_pp_e2ds_C.fits UrNe.mas 0 10 0.1
\end{cmdbox}

\end{enumerate}

% ----------------------------------------------
\clearpage
\newpage
\subsection{Run through python script}
\label{chapter:using_the_drs:working_example:run_python}
% ----------------------------------------------

The process is in the same order as Section \ref{chapter:using_the_drs:working_example:run_cmd}.

\begin{pythonbox}
import cal_DARK_spirou, cal_loc_RAW_spirou
import cal_SLIT_spirou, cal_FF_RAW_spirou
import cal_extract_RAW_spirou, cal_DRIFT_RAW_spirou
import matplotlib.pyplot as plt
# define constants
NIGHT_NAME = '20180409'
# cal dark
files = ['dark_dark_001_pp.fits']
cal_DARK_spirou.main(NIGHT_NAME, files)
# cal badpix
flatfile = 'flat_flat_001_pp.fits'
darkfile = 'dark_dark_001_pp.fits'
cal_BADPIX_spirou.main(NIGHT_NAME, flatfile, darkfile)
# cal loc
files = ['dark_flat_001_pp.fits']
cal_loc_RAW_spirou.main(NIGHT_NAME, files)
# cal loc
files = ['flat_dark_001_pp.fits']
cal_loc_RAW_spirou.main(NIGHT_NAME, files)
# cal slit
files = ['fp_fp_001_pp.fits']
cal_SLIT_spirou.main(NIGHT_NAME, files)
# cal ff
files = ['flat_flat_001_pp.fits']
cal_FF_RAW_spirou.main(NIGHT_NAME, files)
# cal extract
files = ['fp_fp_001_pp.fits']
cal_extract_RAW_spirou.main(NIGHT_NAME, files)
files = ['fp_fp_002_pp.fits']
cal_extract_RAW_spirou.main(NIGHT_NAME, files)
files = ['fp_fp_003_pp.fits']
cal_extract_RAW_spirou.main(NIGHT_NAME, files)
files = ['hcone_hcone_001_pp.fits']
cal_extract_RAW_spirou.main(NIGHT_NAME, files)
# cal drift
reffile = ['fp_fp_001_pp_e2ds_AB.fits']
cal_DRIFT_E2DS_spirou.main(NIGHT_NAME, reffile)
cal_DRIFTPEAK_E2DS_spirou(NIGHT_NAME, reffile)
# cal ccf
reffile = 'hcone_hcone_001_pp_e2ds_C.fits'
mask = 'UrNe.mas'
cal_CCF_E2DS_spirou.main(NIGHT_NAME, reffile, mask, 0, 10, 0.1)
# close all the plots
plt.close('all')

\end{pythonbox}
