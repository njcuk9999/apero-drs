%%%%%%%%%%%%%%%%%%%%%%%%%%%%%%%%%%%%%%%%%%%%%%%%%%%%%%%%
%%
\chapter{Coding style and standardization}
\label{ch:rules}
%%
%%%%%%%%%%%%%%%%%%%%%%%%%%%%%%%%%%%%%%%%%%%%%%%%%%%%%%%%

To keep the code neat, tidy, consistent and professional the following sections suggest guideline by which the DRS should conform to.

%%%%%%%%%%%%%%%%%%%%%%%%%%%%%%%%%%%%%%%%%%%%%%%%%%%%%%%%
%%
\section{PEP 8 - A style guide for python code}
\label{ch:rules:pep8}
%%
%%%%%%%%%%%%%%%%%%%%%%%%%%%%%%%%%%%%%%%%%%%%%%%%%%%%%%%%

PEP 8 is a style guide for python it lays out a specific way to format python code, a full guide can be found here: \url{https://www.python.org/dev/peps/pep-0008/} but the following summarizes the main points used in the DRS.

\begin{itemize}
	\item Code lay-out
	\begin{itemize}
		\item 4 spaces per indentation level (spaces not tabs)
		\item Continuation lines should align wrapped elements
		\item Maximum line length of 79 characters
		\item Surround top-level functions and class definitions with two blank lines (methods with one blank line and all other code with one blank line maximum)
		\item imports should usually be on separate lines
	\end{itemize}

	\item White space in expressions and statements
	\begin{itemize}
		\item No white spaces immediately inside parentheses, brackets or braces
		\item No white spaces immediately before a comma, semicolon, or colon (exception for slicing)
		\item No white spaces immediately before the open parenthesis that starts the argument list of a function call
		\item No white spaces immediately before the open parenthesis that starts an indexing or slicing
		\item Exactly one white space around an assignment (or other) operator
		\item No space around the = sign when used to indicate a keyword argument or a default parameter value
		\item Avoid compound statements (multiple statements on the same line)
	\end{itemize}

	\item Comments
	\begin{itemize}
		\item Comments should start with a \# and be followed by a single white space
		\item In-line comments should be used sparingly
		\item All functions, classes and methods should have a valid document string (see here: \url{https://www.python.org/dev/peps/pep-0257})
	\end{itemize}

	\item Naming conventions
	\begin{itemize}
		\item Never use lower-case letter el `l', upper-case letter `oh' `O', or upper-case letter `eye' `I' as single character variables names
		\item Class Names should normally use CamelCase (words should be Capitalized)
		\item Functions names should be lower-case with words separated by underscores as necessary (same is true for global variable names)
		\item Constants defined on a module level should be written in capital letters with underscores separating words
	\end{itemize}

\end{itemize}


%%%%%%%%%%%%%%%%%%%%%%%%%%%%%%%%%%%%%%%%%%%%%%%%%%%%%%%%
%%
\section{DRS specific style and standardization}
\label{ch:rules:drs_specific}
%%
%%%%%%%%%%%%%%%%%%%%%%%%%%%%%%%%%%%%%%%%%%%%%%%%%%%%%%%%

In addition to PEP-8 we stick to some extra style and standardization points, these include some custom objects to help the ease of development and user experience.

% -------------------------------------------------------
\subsection{Functions from sub-modules}
\label{ch:rules:drs_specific:sub-module_functions}
% -------------------------------------------------------

Unlike `normal' functions these are written in CamelCase without underscores between words. This is done to distinguish them from standard functions. They are always defined in a module (or sub-modules) \INIT\, code and are essentially public aliases to module level code. An example is presented below.

\begin{pythonbox}
# --------------------------
# in the module file spirouMath.py
# --------------------------
def add_x_to_y(x, y):
	"""
	Returns the summation of x and y
	:param x: float, the first term to add
	:param y: float, the second term to add
	:return z: float, the summation of x and y
	"""
	# add x to y
	z = x + y
	# return z
	return z

# --------------------------
# in the __init__ file for spirouCore
# --------------------------
# import from local code
from . import spirouMath
# publicly defined alias to local code function
AddXtoY = spirouMath.add_x_to_y

# --------------------------
# in the recipe
# --------------------------
# import sub-module
from SpirouDRS import spirouCore
# set up constants
x = 4.123
y = 5.234
# add via function
z = spirouCore.AddXtoY(x, y)
\end{pythonbox}


% -------------------------------------------------------
\clearpage
\newpage
\subsection{The logger (WLOG)}
\label{ch:rules:drs_specific:logger}
% -------------------------------------------------------

As in previous version of the DRS the printing and logging is controlled by a function. In this version of the DRS this is in \spirouLog.logger but in most recipes/modules this is aliased to \WLOG. The \WLOG function controls both the printing to the screen (standard output) and to a log file. Where and how this is done is controlled by several variables.

The format of the log entry (whether it is printed to the standard output or to the logging file) is as follows:
\begin{pythonbox}
WLOG(level, program, message)
\end{pythonbox}
\noindent and produces the following entry (in log or standard output)
\begin{cmdboxprint}
HH:MM:SS.s - char | program | message
\end{cmdboxprint}
\noindent where the `char' is dependent on the input level. 

\noindent The `char' and level are a dictionary pair in the form `level = char' and is controlled by \definevariable{text:trig_keys}{trig\_key} (see section \ref{ch:variables:log_print}) i.e. by default the level char pairs are:
\begin{pythonbox}
dict(all=' ', error='!', warning='@', info='*', graph='~')
\end{pythonbox}

\noindent The level also determines whether or not a message is shown in the screen (standard output) or in the log. A log message will be shown if it has a numeric value (defined in \definevariable{text:write_level}{write\_level}) higher than that set in \definevariable{text:print_level}{PRINT\_LEVEL} for printing to the screen (standard output) or set in \definevariable{text:log_level}{LOG\_LEVEL} for printing to the log.

i.e.: 
\begin{pythonbox}
write_level = dict(error=3, warning=2, info=1)
trig_key = dict(all=' ', error='!', warning='@', info='*', graph='~')
PRINT_LEVEL = 'warning'

WLOG('info', 'program', 'Info message')
WLOG('warning', 'program', 'Warning message')
WLOG('error', 'program', 'Error message')
\end{pythonbox}
returns
\begin{cmdboxprint}
HH:MM:SS.s - @ |program|Warning message
HH:MM:SS.s - ! |program|Error message
\end{cmdboxprint}
\begin{note}
Note the info message was not shown as info=1 and \definevariable{text:print_level}{PRINT\_LEVEL} is set to warning=2.
\end{note}

\newpage 

\noindent In addition to logging the certain levels can be set to exit the DRS recipe when they are used. They are defined in \definevariable{text:exit_levels}{exit\_levels} and exiting python is controlled via \definevariable{text:exit_controller}{exit} and \definevariable{text:log_exit_type}{log\_exit\_type}.

i.e. 
\begin{pythonbox}
write_level = dict(error=3, warning=2, info=1)
trig_key = dict(all=' ', error='!', warning='@', info='*', graph='~')
exit_levels = ['error']
PRINT_LEVEL = 'warning'

WLOG('error', 'program', 'Error message')
WLOG('info', 'program', 'Info message')
WLOG('warning', 'program', 'Warning message')

\end{pythonbox}
returns
\begin{cmdboxprint}
HH:MM:SS.s - ! |program|Error message
\end{cmdboxprint}
\begin{note}
Note that `WLOG('error')' triggered the recipe/module to exit python, thus no other logs were printed.
\end{note}

% -------------------------------------------------------
\subsection{The coloured log}
\label{ch:rules:drs_specific:coloured_log}
% -------------------------------------------------------

In addition to the features above the log can be coloured to aid usability.
Currently errors are coloured red, warnings are coloured yellow and all other text is coloured green. These colours can be changed using \definevariable{text:colouredlevels}{clevels} or turned on/off using \definevariable{text:coloured_log}{COLOURED\_LOG}.

\vspace{0.5cm}

\noindent An example of each is shown below:
\begin{pythonbox}
WLOG('all', 'program', 'All message')
WLOG('info', 'program', 'Info message')
WLOG('warning', 'program', 'Warning message')
WLOG('error', 'program', 'Error message')
WLOG('all', 'program', 'All message')
\end{pythonbox}
\begin{cmdboxprintspecial}
@gHH:MM:SS.s -   |program|All message@g
@gHH:MM:SS.s - * |program|Info message@g
@yHH:MM:SS.s - \@ |program|Warning message@y
@rHH:MM:SS.s - ! |program|Error message@r
@gHH:MM:SS.s -   |program|All message@g
\end{cmdboxprintspecial}

\newpage

% -------------------------------------------------------
\subsection{The Parameter Dictionary Object}
\label{ch:rules:drs_specific:param_dict}
% -------------------------------------------------------

While running the DRS there are many variables defined in many places that are used throughout the recipes, DRS module and sub-modules, defined in configuration files and from certain sub-modules and recipes. It is important as a developer (and for proper error handling) to keep track of where this variables are being defined and changed in the DRS. \\

\noindent For this reason, and for convenience for passing between functions and recipes, a new object, based on a dictionary has been defined to handle all variables defined throughout the DRS. This is the parameter dictionary (\ParamDict) class (defined in \spirouConfig). \\

\noindent The \ParamDict is a custom dictionary class (that inherits all attributes and methods from the standard python dictionary object), with the ability to get and set a source for each key value pair. In addition to this all variables stored are \textbf{insensitive to case} (i.e. upper-case variables, lower-case variables and mixed case variables are stored as the \textbf{same} variable). \\

\noindent Construct/initiate the \ParamDict in the same way one would a python dictionary:
\begin{pythonbox}
# as an empty dictionary
p1 = ParamDict()
# from a list of keys and values (using zip)
p2 = ParamDict(zip(keys, values))
\end{pythonbox}

\noindent Once created key, value pairs are created the same way one would with a python dictionary.
\begin{pythonbox}
# set a key, value pair
p1['test'] = 1
# ParamDict are case insensitive 'Test' overwrites 'test' and 'teST' 
p1['Test'] = 99
\end{pythonbox}

\vspace{0.5cm}
\noindent After creating a key the source should be set. This can be done as follows:
\begin{pythonbox}
# -----------------------
# Set a single source
# -----------------------
# set the key value pair
p1['test'] = 1
# set the source
p1.set_source('test', 'test.py/__main__()')
\end{pythonbox}

\newpage

\noindent One can also add a set of sources (after creating multiple key value pairs)
\begin{pythonbox}
# -----------------------
# Set a list of sources
# -----------------------
# set the key value pairs
p1['a'] = 1
p1['b'] = 2
p1['c'] = 3
# set the sources
p1.set_sources(['a', 'b', 'c'], 'SpirouConfig.DefineConstants()')
\end{pythonbox}

\vspace{0.5cm}
\noindent or one can set all sources in the \ParamDict to a specific source
\begin{note}
Note set all sources will change every source in the \ParamDict so should only be used after \ParamDict created from a set of key value pairs
\end{note}

\begin{pythonbox}
# -----------------------
# Set all sources
# -----------------------
# create ParamDict
keys = ['a', 'b', 'c', 'd', 'e']
values = [1, 2, 3, 4, 5]
p3 = ParamDict(zip(keys, values))
# set all sources
p3.set_all_sources('SpirouMath.LetterNumbers()')
\end{pythonbox}

The parameter dictionary also contains some helper functions (as the parameter dictionary can get quite large).

\begin{itemize}
	\item startswith - This allows one to search for all keys that start with a defined string

	\begin{pythonbox}
	p.startswith('DRS')
	\end{pythonbox}
	\begin{cmdboxprint}
	['DRS_PLOT',
	 'DRS_DEBUG',
	 'DRS_ROOT',
	 'DRS_DATA_RAW',
	 'DRS_DATA_REDUC',
	 'DRS_CALIB_DB',
	 'DRS_DATA_MSG',
	 'DRS_DATA_WORKING',
	 'DRS_NAME',
	 'DRS_VERSION',
	 'DRS_CONFIG',
	 'DRS_MAN',
	 'DRS_USED_DATE',
	 'DRS_INTERACTIVE']
	 \end{cmdboxprint}


	\item contains - This allows one to search for all keys that contain a defined string
	\begin{pythonbox}
	p.contains('calib')
	\end{pythonbox}
	\begin{cmdboxprint}
	['DRS_CALIB_DB', 'IC_CALIBDB_FILENAME', 'CALIB_MAX_WAIT', 'CALIB_DB_MATCH']
	 \end{cmdboxprint}

	\item endswith - This allows one to search for all keys that end with a defined string
	\begin{pythonbox}
	p.endswith('ALL')
	\end{pythonbox}
	\begin{cmdboxprint}
	['NBFIB_FPALL',
	 'IC_FIRST_ORDER_JUMP_FPALL',
	 'IC_LOCNBMAXO_FPALL',
	 'QC_LOC_NBO_FPALL',
	 'FIB_TYPE_FPALL',
	 'IC_EXT_RANGE1_FPALL',
	 'IC_EXT_RANGE2_FPALL',
	 'IC_EXT_RANGE_FPALL',
	 'LOC_FILE_FPALL',
	 'ORDERP_FILE_FPALL',
	 'IC_EXT_D_RANGE_FPALL']
	 \end{cmdboxprint}

\end{itemize}


% -------------------------------------------------------
\clearpage
\newpage
\subsection{Configuration Error and Exception}
\label{ch:rules:drs_specific:config_error}
% -------------------------------------------------------

As mentioned above in section \ref{ch:rules:drs_specific:param_dict} it is important to handle errors caused by variable definition. Included in the parameter dictionary definitions are a new set of exception handlers to be used with \ParamDict and the \spirouLog.logger (aliased to \WLOG in most modules/recipes). It is very similar to standard python Exceptions but adds some new methods that can be accessed to be used with \WLOG. \\

An example is below of the \ConfigError exception (without using \ParamDict)

\begin{pythonbox}
def a_function():
    try:
        # some_code that causes an exception
        x = dict()
        y = x['a']
        return y
    except KeyError:
        # define a log message
        message = 'a was not found in dictionary x'
        raise ConfigError(message, level='error')

# Main code:
try:
    a_function()
except ConfigError as e:
    WLOG(e.level, 'program', e.message)
\end{pythonbox}
\vspace{0.5cm}
\noindent This functionality is coded into \ParamDict (with a \WLOG level set to `error') thus one only needs the following code:
\begin{pythonbox}
# set up the ParamDict
x = ParamDict()
# Main code:
try:
    y = x['add']
except ConfigError as e:
    WLOG(e.level, 'program', e.message)
\end{pythonbox}
\noindent and the result will be as follows:
\begin{cmdboxprint}
HH:MM:SS.s - ! |program|Parameter "add" not found in parameter dictionary
\end{cmdboxprint}
\begin{note}
Due to WLOG `error' currently meaning the code is exited a missing parameter will print the above message and then exit using the \definevariable{text:log_exit_type}{log\_exit\_type} exit strategy (see section \ref{ch:rules:drs_specific:logger}).
\end{note}