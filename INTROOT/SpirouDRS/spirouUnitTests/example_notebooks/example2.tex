
% Default to the notebook output style

    


% Inherit from the specified cell style.




    
\documentclass[11pt]{article}

    
    
    \usepackage[T1]{fontenc}
    % Nicer default font (+ math font) than Computer Modern for most use cases
    \usepackage{mathpazo}

    % Basic figure setup, for now with no caption control since it's done
    % automatically by Pandoc (which extracts ![](path) syntax from Markdown).
    \usepackage{graphicx}
    % We will generate all images so they have a width \maxwidth. This means
    % that they will get their normal width if they fit onto the page, but
    % are scaled down if they would overflow the margins.
    \makeatletter
    \def\maxwidth{\ifdim\Gin@nat@width>\linewidth\linewidth
    \else\Gin@nat@width\fi}
    \makeatother
    \let\Oldincludegraphics\includegraphics
    % Set max figure width to be 80% of text width, for now hardcoded.
    \renewcommand{\includegraphics}[1]{\Oldincludegraphics[width=.8\maxwidth]{#1}}
    % Ensure that by default, figures have no caption (until we provide a
    % proper Figure object with a Caption API and a way to capture that
    % in the conversion process - todo).
    \usepackage{caption}
    \DeclareCaptionLabelFormat{nolabel}{}
    \captionsetup{labelformat=nolabel}

    \usepackage{adjustbox} % Used to constrain images to a maximum size 
    \usepackage{xcolor} % Allow colors to be defined
    \usepackage{enumerate} % Needed for markdown enumerations to work
    \usepackage{geometry} % Used to adjust the document margins
    \usepackage{amsmath} % Equations
    \usepackage{amssymb} % Equations
    \usepackage{textcomp} % defines textquotesingle
    % Hack from http://tex.stackexchange.com/a/47451/13684:
    \AtBeginDocument{%
        \def\PYZsq{\textquotesingle}% Upright quotes in Pygmentized code
    }
    \usepackage{upquote} % Upright quotes for verbatim code
    \usepackage{eurosym} % defines \euro
    \usepackage[mathletters]{ucs} % Extended unicode (utf-8) support
    \usepackage[utf8x]{inputenc} % Allow utf-8 characters in the tex document
    \usepackage{fancyvrb} % verbatim replacement that allows latex
    \usepackage{grffile} % extends the file name processing of package graphics 
                         % to support a larger range 
    % The hyperref package gives us a pdf with properly built
    % internal navigation ('pdf bookmarks' for the table of contents,
    % internal cross-reference links, web links for URLs, etc.)
    \usepackage{hyperref}
    \usepackage{longtable} % longtable support required by pandoc >1.10
    \usepackage{booktabs}  % table support for pandoc > 1.12.2
    \usepackage[inline]{enumitem} % IRkernel/repr support (it uses the enumerate* environment)
    \usepackage[normalem]{ulem} % ulem is needed to support strikethroughs (\sout)
                                % normalem makes italics be italics, not underlines
    

    
    
    % Colors for the hyperref package
    \definecolor{urlcolor}{rgb}{0,.145,.698}
    \definecolor{linkcolor}{rgb}{.71,0.21,0.01}
    \definecolor{citecolor}{rgb}{.12,.54,.11}

    % ANSI colors
    \definecolor{ansi-black}{HTML}{3E424D}
    \definecolor{ansi-black-intense}{HTML}{282C36}
    \definecolor{ansi-red}{HTML}{E75C58}
    \definecolor{ansi-red-intense}{HTML}{B22B31}
    \definecolor{ansi-green}{HTML}{00A250}
    \definecolor{ansi-green-intense}{HTML}{007427}
    \definecolor{ansi-yellow}{HTML}{DDB62B}
    \definecolor{ansi-yellow-intense}{HTML}{B27D12}
    \definecolor{ansi-blue}{HTML}{208FFB}
    \definecolor{ansi-blue-intense}{HTML}{0065CA}
    \definecolor{ansi-magenta}{HTML}{D160C4}
    \definecolor{ansi-magenta-intense}{HTML}{A03196}
    \definecolor{ansi-cyan}{HTML}{60C6C8}
    \definecolor{ansi-cyan-intense}{HTML}{258F8F}
    \definecolor{ansi-white}{HTML}{C5C1B4}
    \definecolor{ansi-white-intense}{HTML}{A1A6B2}

    % commands and environments needed by pandoc snippets
    % extracted from the output of `pandoc -s`
    \providecommand{\tightlist}{%
      \setlength{\itemsep}{0pt}\setlength{\parskip}{0pt}}
    \DefineVerbatimEnvironment{Highlighting}{Verbatim}{commandchars=\\\{\}}
    % Add ',fontsize=\small' for more characters per line
    \newenvironment{Shaded}{}{}
    \newcommand{\KeywordTok}[1]{\textcolor[rgb]{0.00,0.44,0.13}{\textbf{{#1}}}}
    \newcommand{\DataTypeTok}[1]{\textcolor[rgb]{0.56,0.13,0.00}{{#1}}}
    \newcommand{\DecValTok}[1]{\textcolor[rgb]{0.25,0.63,0.44}{{#1}}}
    \newcommand{\BaseNTok}[1]{\textcolor[rgb]{0.25,0.63,0.44}{{#1}}}
    \newcommand{\FloatTok}[1]{\textcolor[rgb]{0.25,0.63,0.44}{{#1}}}
    \newcommand{\CharTok}[1]{\textcolor[rgb]{0.25,0.44,0.63}{{#1}}}
    \newcommand{\StringTok}[1]{\textcolor[rgb]{0.25,0.44,0.63}{{#1}}}
    \newcommand{\CommentTok}[1]{\textcolor[rgb]{0.38,0.63,0.69}{\textit{{#1}}}}
    \newcommand{\OtherTok}[1]{\textcolor[rgb]{0.00,0.44,0.13}{{#1}}}
    \newcommand{\AlertTok}[1]{\textcolor[rgb]{1.00,0.00,0.00}{\textbf{{#1}}}}
    \newcommand{\FunctionTok}[1]{\textcolor[rgb]{0.02,0.16,0.49}{{#1}}}
    \newcommand{\RegionMarkerTok}[1]{{#1}}
    \newcommand{\ErrorTok}[1]{\textcolor[rgb]{1.00,0.00,0.00}{\textbf{{#1}}}}
    \newcommand{\NormalTok}[1]{{#1}}
    
    % Additional commands for more recent versions of Pandoc
    \newcommand{\ConstantTok}[1]{\textcolor[rgb]{0.53,0.00,0.00}{{#1}}}
    \newcommand{\SpecialCharTok}[1]{\textcolor[rgb]{0.25,0.44,0.63}{{#1}}}
    \newcommand{\VerbatimStringTok}[1]{\textcolor[rgb]{0.25,0.44,0.63}{{#1}}}
    \newcommand{\SpecialStringTok}[1]{\textcolor[rgb]{0.73,0.40,0.53}{{#1}}}
    \newcommand{\ImportTok}[1]{{#1}}
    \newcommand{\DocumentationTok}[1]{\textcolor[rgb]{0.73,0.13,0.13}{\textit{{#1}}}}
    \newcommand{\AnnotationTok}[1]{\textcolor[rgb]{0.38,0.63,0.69}{\textbf{\textit{{#1}}}}}
    \newcommand{\CommentVarTok}[1]{\textcolor[rgb]{0.38,0.63,0.69}{\textbf{\textit{{#1}}}}}
    \newcommand{\VariableTok}[1]{\textcolor[rgb]{0.10,0.09,0.49}{{#1}}}
    \newcommand{\ControlFlowTok}[1]{\textcolor[rgb]{0.00,0.44,0.13}{\textbf{{#1}}}}
    \newcommand{\OperatorTok}[1]{\textcolor[rgb]{0.40,0.40,0.40}{{#1}}}
    \newcommand{\BuiltInTok}[1]{{#1}}
    \newcommand{\ExtensionTok}[1]{{#1}}
    \newcommand{\PreprocessorTok}[1]{\textcolor[rgb]{0.74,0.48,0.00}{{#1}}}
    \newcommand{\AttributeTok}[1]{\textcolor[rgb]{0.49,0.56,0.16}{{#1}}}
    \newcommand{\InformationTok}[1]{\textcolor[rgb]{0.38,0.63,0.69}{\textbf{\textit{{#1}}}}}
    \newcommand{\WarningTok}[1]{\textcolor[rgb]{0.38,0.63,0.69}{\textbf{\textit{{#1}}}}}
    
    
    % Define a nice break command that doesn't care if a line doesn't already
    % exist.
    \def\br{\hspace*{\fill} \\* }
    % Math Jax compatability definitions
    \def\gt{>}
    \def\lt{<}
    % Document parameters
    \title{example2}
    
    
    

    % Pygments definitions
    
\makeatletter
\def\PY@reset{\let\PY@it=\relax \let\PY@bf=\relax%
    \let\PY@ul=\relax \let\PY@tc=\relax%
    \let\PY@bc=\relax \let\PY@ff=\relax}
\def\PY@tok#1{\csname PY@tok@#1\endcsname}
\def\PY@toks#1+{\ifx\relax#1\empty\else%
    \PY@tok{#1}\expandafter\PY@toks\fi}
\def\PY@do#1{\PY@bc{\PY@tc{\PY@ul{%
    \PY@it{\PY@bf{\PY@ff{#1}}}}}}}
\def\PY#1#2{\PY@reset\PY@toks#1+\relax+\PY@do{#2}}

\expandafter\def\csname PY@tok@w\endcsname{\def\PY@tc##1{\textcolor[rgb]{0.73,0.73,0.73}{##1}}}
\expandafter\def\csname PY@tok@c\endcsname{\let\PY@it=\textit\def\PY@tc##1{\textcolor[rgb]{0.25,0.50,0.50}{##1}}}
\expandafter\def\csname PY@tok@cp\endcsname{\def\PY@tc##1{\textcolor[rgb]{0.74,0.48,0.00}{##1}}}
\expandafter\def\csname PY@tok@k\endcsname{\let\PY@bf=\textbf\def\PY@tc##1{\textcolor[rgb]{0.00,0.50,0.00}{##1}}}
\expandafter\def\csname PY@tok@kp\endcsname{\def\PY@tc##1{\textcolor[rgb]{0.00,0.50,0.00}{##1}}}
\expandafter\def\csname PY@tok@kt\endcsname{\def\PY@tc##1{\textcolor[rgb]{0.69,0.00,0.25}{##1}}}
\expandafter\def\csname PY@tok@o\endcsname{\def\PY@tc##1{\textcolor[rgb]{0.40,0.40,0.40}{##1}}}
\expandafter\def\csname PY@tok@ow\endcsname{\let\PY@bf=\textbf\def\PY@tc##1{\textcolor[rgb]{0.67,0.13,1.00}{##1}}}
\expandafter\def\csname PY@tok@nb\endcsname{\def\PY@tc##1{\textcolor[rgb]{0.00,0.50,0.00}{##1}}}
\expandafter\def\csname PY@tok@nf\endcsname{\def\PY@tc##1{\textcolor[rgb]{0.00,0.00,1.00}{##1}}}
\expandafter\def\csname PY@tok@nc\endcsname{\let\PY@bf=\textbf\def\PY@tc##1{\textcolor[rgb]{0.00,0.00,1.00}{##1}}}
\expandafter\def\csname PY@tok@nn\endcsname{\let\PY@bf=\textbf\def\PY@tc##1{\textcolor[rgb]{0.00,0.00,1.00}{##1}}}
\expandafter\def\csname PY@tok@ne\endcsname{\let\PY@bf=\textbf\def\PY@tc##1{\textcolor[rgb]{0.82,0.25,0.23}{##1}}}
\expandafter\def\csname PY@tok@nv\endcsname{\def\PY@tc##1{\textcolor[rgb]{0.10,0.09,0.49}{##1}}}
\expandafter\def\csname PY@tok@no\endcsname{\def\PY@tc##1{\textcolor[rgb]{0.53,0.00,0.00}{##1}}}
\expandafter\def\csname PY@tok@nl\endcsname{\def\PY@tc##1{\textcolor[rgb]{0.63,0.63,0.00}{##1}}}
\expandafter\def\csname PY@tok@ni\endcsname{\let\PY@bf=\textbf\def\PY@tc##1{\textcolor[rgb]{0.60,0.60,0.60}{##1}}}
\expandafter\def\csname PY@tok@na\endcsname{\def\PY@tc##1{\textcolor[rgb]{0.49,0.56,0.16}{##1}}}
\expandafter\def\csname PY@tok@nt\endcsname{\let\PY@bf=\textbf\def\PY@tc##1{\textcolor[rgb]{0.00,0.50,0.00}{##1}}}
\expandafter\def\csname PY@tok@nd\endcsname{\def\PY@tc##1{\textcolor[rgb]{0.67,0.13,1.00}{##1}}}
\expandafter\def\csname PY@tok@s\endcsname{\def\PY@tc##1{\textcolor[rgb]{0.73,0.13,0.13}{##1}}}
\expandafter\def\csname PY@tok@sd\endcsname{\let\PY@it=\textit\def\PY@tc##1{\textcolor[rgb]{0.73,0.13,0.13}{##1}}}
\expandafter\def\csname PY@tok@si\endcsname{\let\PY@bf=\textbf\def\PY@tc##1{\textcolor[rgb]{0.73,0.40,0.53}{##1}}}
\expandafter\def\csname PY@tok@se\endcsname{\let\PY@bf=\textbf\def\PY@tc##1{\textcolor[rgb]{0.73,0.40,0.13}{##1}}}
\expandafter\def\csname PY@tok@sr\endcsname{\def\PY@tc##1{\textcolor[rgb]{0.73,0.40,0.53}{##1}}}
\expandafter\def\csname PY@tok@ss\endcsname{\def\PY@tc##1{\textcolor[rgb]{0.10,0.09,0.49}{##1}}}
\expandafter\def\csname PY@tok@sx\endcsname{\def\PY@tc##1{\textcolor[rgb]{0.00,0.50,0.00}{##1}}}
\expandafter\def\csname PY@tok@m\endcsname{\def\PY@tc##1{\textcolor[rgb]{0.40,0.40,0.40}{##1}}}
\expandafter\def\csname PY@tok@gh\endcsname{\let\PY@bf=\textbf\def\PY@tc##1{\textcolor[rgb]{0.00,0.00,0.50}{##1}}}
\expandafter\def\csname PY@tok@gu\endcsname{\let\PY@bf=\textbf\def\PY@tc##1{\textcolor[rgb]{0.50,0.00,0.50}{##1}}}
\expandafter\def\csname PY@tok@gd\endcsname{\def\PY@tc##1{\textcolor[rgb]{0.63,0.00,0.00}{##1}}}
\expandafter\def\csname PY@tok@gi\endcsname{\def\PY@tc##1{\textcolor[rgb]{0.00,0.63,0.00}{##1}}}
\expandafter\def\csname PY@tok@gr\endcsname{\def\PY@tc##1{\textcolor[rgb]{1.00,0.00,0.00}{##1}}}
\expandafter\def\csname PY@tok@ge\endcsname{\let\PY@it=\textit}
\expandafter\def\csname PY@tok@gs\endcsname{\let\PY@bf=\textbf}
\expandafter\def\csname PY@tok@gp\endcsname{\let\PY@bf=\textbf\def\PY@tc##1{\textcolor[rgb]{0.00,0.00,0.50}{##1}}}
\expandafter\def\csname PY@tok@go\endcsname{\def\PY@tc##1{\textcolor[rgb]{0.53,0.53,0.53}{##1}}}
\expandafter\def\csname PY@tok@gt\endcsname{\def\PY@tc##1{\textcolor[rgb]{0.00,0.27,0.87}{##1}}}
\expandafter\def\csname PY@tok@err\endcsname{\def\PY@bc##1{\setlength{\fboxsep}{0pt}\fcolorbox[rgb]{1.00,0.00,0.00}{1,1,1}{\strut ##1}}}
\expandafter\def\csname PY@tok@kc\endcsname{\let\PY@bf=\textbf\def\PY@tc##1{\textcolor[rgb]{0.00,0.50,0.00}{##1}}}
\expandafter\def\csname PY@tok@kd\endcsname{\let\PY@bf=\textbf\def\PY@tc##1{\textcolor[rgb]{0.00,0.50,0.00}{##1}}}
\expandafter\def\csname PY@tok@kn\endcsname{\let\PY@bf=\textbf\def\PY@tc##1{\textcolor[rgb]{0.00,0.50,0.00}{##1}}}
\expandafter\def\csname PY@tok@kr\endcsname{\let\PY@bf=\textbf\def\PY@tc##1{\textcolor[rgb]{0.00,0.50,0.00}{##1}}}
\expandafter\def\csname PY@tok@bp\endcsname{\def\PY@tc##1{\textcolor[rgb]{0.00,0.50,0.00}{##1}}}
\expandafter\def\csname PY@tok@fm\endcsname{\def\PY@tc##1{\textcolor[rgb]{0.00,0.00,1.00}{##1}}}
\expandafter\def\csname PY@tok@vc\endcsname{\def\PY@tc##1{\textcolor[rgb]{0.10,0.09,0.49}{##1}}}
\expandafter\def\csname PY@tok@vg\endcsname{\def\PY@tc##1{\textcolor[rgb]{0.10,0.09,0.49}{##1}}}
\expandafter\def\csname PY@tok@vi\endcsname{\def\PY@tc##1{\textcolor[rgb]{0.10,0.09,0.49}{##1}}}
\expandafter\def\csname PY@tok@vm\endcsname{\def\PY@tc##1{\textcolor[rgb]{0.10,0.09,0.49}{##1}}}
\expandafter\def\csname PY@tok@sa\endcsname{\def\PY@tc##1{\textcolor[rgb]{0.73,0.13,0.13}{##1}}}
\expandafter\def\csname PY@tok@sb\endcsname{\def\PY@tc##1{\textcolor[rgb]{0.73,0.13,0.13}{##1}}}
\expandafter\def\csname PY@tok@sc\endcsname{\def\PY@tc##1{\textcolor[rgb]{0.73,0.13,0.13}{##1}}}
\expandafter\def\csname PY@tok@dl\endcsname{\def\PY@tc##1{\textcolor[rgb]{0.73,0.13,0.13}{##1}}}
\expandafter\def\csname PY@tok@s2\endcsname{\def\PY@tc##1{\textcolor[rgb]{0.73,0.13,0.13}{##1}}}
\expandafter\def\csname PY@tok@sh\endcsname{\def\PY@tc##1{\textcolor[rgb]{0.73,0.13,0.13}{##1}}}
\expandafter\def\csname PY@tok@s1\endcsname{\def\PY@tc##1{\textcolor[rgb]{0.73,0.13,0.13}{##1}}}
\expandafter\def\csname PY@tok@mb\endcsname{\def\PY@tc##1{\textcolor[rgb]{0.40,0.40,0.40}{##1}}}
\expandafter\def\csname PY@tok@mf\endcsname{\def\PY@tc##1{\textcolor[rgb]{0.40,0.40,0.40}{##1}}}
\expandafter\def\csname PY@tok@mh\endcsname{\def\PY@tc##1{\textcolor[rgb]{0.40,0.40,0.40}{##1}}}
\expandafter\def\csname PY@tok@mi\endcsname{\def\PY@tc##1{\textcolor[rgb]{0.40,0.40,0.40}{##1}}}
\expandafter\def\csname PY@tok@il\endcsname{\def\PY@tc##1{\textcolor[rgb]{0.40,0.40,0.40}{##1}}}
\expandafter\def\csname PY@tok@mo\endcsname{\def\PY@tc##1{\textcolor[rgb]{0.40,0.40,0.40}{##1}}}
\expandafter\def\csname PY@tok@ch\endcsname{\let\PY@it=\textit\def\PY@tc##1{\textcolor[rgb]{0.25,0.50,0.50}{##1}}}
\expandafter\def\csname PY@tok@cm\endcsname{\let\PY@it=\textit\def\PY@tc##1{\textcolor[rgb]{0.25,0.50,0.50}{##1}}}
\expandafter\def\csname PY@tok@cpf\endcsname{\let\PY@it=\textit\def\PY@tc##1{\textcolor[rgb]{0.25,0.50,0.50}{##1}}}
\expandafter\def\csname PY@tok@c1\endcsname{\let\PY@it=\textit\def\PY@tc##1{\textcolor[rgb]{0.25,0.50,0.50}{##1}}}
\expandafter\def\csname PY@tok@cs\endcsname{\let\PY@it=\textit\def\PY@tc##1{\textcolor[rgb]{0.25,0.50,0.50}{##1}}}

\def\PYZbs{\char`\\}
\def\PYZus{\char`\_}
\def\PYZob{\char`\{}
\def\PYZcb{\char`\}}
\def\PYZca{\char`\^}
\def\PYZam{\char`\&}
\def\PYZlt{\char`\<}
\def\PYZgt{\char`\>}
\def\PYZsh{\char`\#}
\def\PYZpc{\char`\%}
\def\PYZdl{\char`\$}
\def\PYZhy{\char`\-}
\def\PYZsq{\char`\'}
\def\PYZdq{\char`\"}
\def\PYZti{\char`\~}
% for compatibility with earlier versions
\def\PYZat{@}
\def\PYZlb{[}
\def\PYZrb{]}
\makeatother


    % Exact colors from NB
    \definecolor{incolor}{rgb}{0.0, 0.0, 0.5}
    \definecolor{outcolor}{rgb}{0.545, 0.0, 0.0}



    
    % Prevent overflowing lines due to hard-to-break entities
    \sloppy 
    % Setup hyperref package
    \hypersetup{
      breaklinks=true,  % so long urls are correctly broken across lines
      colorlinks=true,
      urlcolor=urlcolor,
      linkcolor=linkcolor,
      citecolor=citecolor,
      }
    % Slightly bigger margins than the latex defaults
    
    \geometry{verbose,tmargin=1in,bmargin=1in,lmargin=1in,rmargin=1in}
    
    

    \begin{document}
    
    
    \maketitle
    
    

    
    \section{What is a parameter
dictionary?}\label{what-is-a-parameter-dictionary}

    In the main recipes a variable "p" is used often. This is a instance of
a custom dictionary called a parameter dictionary or "ParamDict" for
short.

We use this to store all constants and keep track (as a developer) of
where constants were defined or modified.

Below we explore "p" in more detail.

    First we begin as if we were in cal\_DARK\_spirou, importing all modules
required to run the "set up" procedure.

    \begin{Verbatim}[commandchars=\\\{\}]
{\color{incolor}In [{\color{incolor}21}]:} \PY{c+c1}{\PYZsh{} imports as in cal\PYZus{}DARK\PYZus{}spirou.py}
         \PY{k+kn}{from} \PY{n+nn}{\PYZus{}\PYZus{}future\PYZus{}\PYZus{}} \PY{k+kn}{import} \PY{n}{division}
         
         \PY{k+kn}{from} \PY{n+nn}{SpirouDRS} \PY{k+kn}{import} \PY{n}{spirouConfig}
         \PY{k+kn}{from} \PY{n+nn}{SpirouDRS} \PY{k+kn}{import} \PY{n}{spirouCore}
         \PY{k+kn}{from} \PY{n+nn}{SpirouDRS} \PY{k+kn}{import} \PY{n}{spirouStartup}
         
         \PY{c+c1}{\PYZsh{} ========================================}
         \PY{c+c1}{\PYZsh{} Define variables}
         \PY{c+c1}{\PYZsh{} ========================================}
         \PY{c+c1}{\PYZsh{} Name of program}
         \PY{n}{\PYZus{}\PYZus{}NAME\PYZus{}\PYZus{}} \PY{o}{=} \PY{l+s+s1}{\PYZsq{}}\PY{l+s+s1}{cal\PYZus{}DARK\PYZus{}spirou.py}\PY{l+s+s1}{\PYZsq{}}
         \PY{c+c1}{\PYZsh{} Get version and author}
         \PY{n}{\PYZus{}\PYZus{}version\PYZus{}\PYZus{}} \PY{o}{=} \PY{n}{spirouConfig}\PY{o}{.}\PY{n}{Constants}\PY{o}{.}\PY{n}{VERSION}\PY{p}{(}\PY{p}{)}
         \PY{n}{\PYZus{}\PYZus{}author\PYZus{}\PYZus{}} \PY{o}{=} \PY{n}{spirouConfig}\PY{o}{.}\PY{n}{Constants}\PY{o}{.}\PY{n}{AUTHORS}\PY{p}{(}\PY{p}{)}
         \PY{n}{\PYZus{}\PYZus{}date\PYZus{}\PYZus{}} \PY{o}{=} \PY{n}{spirouConfig}\PY{o}{.}\PY{n}{Constants}\PY{o}{.}\PY{n}{LATEST\PYZus{}EDIT}\PY{p}{(}\PY{p}{)}
         \PY{n}{\PYZus{}\PYZus{}release\PYZus{}\PYZus{}} \PY{o}{=} \PY{n}{spirouConfig}\PY{o}{.}\PY{n}{Constants}\PY{o}{.}\PY{n}{RELEASE}\PY{p}{(}\PY{p}{)}
         \PY{c+c1}{\PYZsh{} Get Logging function}
         \PY{n}{WLOG} \PY{o}{=} \PY{n}{spirouCore}\PY{o}{.}\PY{n}{wlog}
         \PY{c+c1}{\PYZsh{} Get plotting functions}
         \PY{n}{sPlt} \PY{o}{=} \PY{n}{spirouCore}\PY{o}{.}\PY{n}{sPlt}
\end{Verbatim}


    The next section is generally inside the "main" function of a recipe,
here we keep it outside so we can show the features easily. The setup of
normal routines uses three functions "Begin", "LoadArguments" and
"InitialFileSetup". We will go through each using cal\_DARK\_spirou.py
as an example.

    \subsection{The begin function}\label{the-begin-function}

    This is the first function that will be run in any recipe "main"
function. It loads all the primary constants (from the config.py file)
as well as displaying the startup title and paths defined in the
config.py file.

    \begin{Verbatim}[commandchars=\\\{\}]
{\color{incolor}In [{\color{incolor}2}]:} \PY{c+c1}{\PYZsh{} Begin function}
        \PY{n}{p} \PY{o}{=} \PY{n}{spirouStartup}\PY{o}{.}\PY{n}{Begin}\PY{p}{(}\PY{p}{)}
\end{Verbatim}


    \begin{Verbatim}[commandchars=\\\{\}]
\textcolor{ansi-green-intense}{\textbf{12:05:18.0 -   || *****************************************}}
\textcolor{ansi-green-intense}{\textbf{12:05:18.0 -   || * SPIROU @(\#) Geneva Observatory (0.1.029)}}
\textcolor{ansi-green-intense}{\textbf{12:05:18.0 -   || *****************************************}}
\textcolor{ansi-green-intense}{\textbf{12:05:18.0 -   ||(dir\_data\_raw)      DRS\_DATA\_RAW=/scratch/Projects/spirou\_py3/data/raw}}
\textcolor{ansi-green-intense}{\textbf{12:05:18.0 -   ||(dir\_data\_reduc)    DRS\_DATA\_REDUC=/scratch/Projects/spirou\_py3/data/reduced}}
\textcolor{ansi-green-intense}{\textbf{12:05:18.0 -   ||(dir\_calib\_db)      DRS\_CALIB\_DB=/scratch/Projects/spirou\_py3/data/calibDB}}
\textcolor{ansi-green-intense}{\textbf{12:05:18.0 -   ||(dir\_data\_msg)      DRS\_DATA\_MSG=/scratch/Projects/spirou\_py3/data/msg}}
\textcolor{ansi-green-intense}{\textbf{12:05:18.0 -   ||(print\_level)       PRINT\_LEVEL=all         \%(error/warning/info/all)}}
\textcolor{ansi-green-intense}{\textbf{12:05:18.0 -   ||(log\_level)         LOG\_LEVEL=all         \%(error/warning/info/all)}}
\textcolor{ansi-green-intense}{\textbf{12:05:18.0 -   ||(plot\_graph)        DRS\_PLOT=1            \%(def/undef/trigger)}}
\textcolor{ansi-green-intense}{\textbf{12:05:18.0 -   ||(used\_date)         DRS\_USED\_DATE=undefined}}
\textcolor{ansi-green-intense}{\textbf{12:05:18.0 -   ||(working\_dir)       DRS\_DATA\_WORKING=/scratch/Projects/spirou\_py3/data/tmp}}
\textcolor{ansi-green-intense}{\textbf{12:05:18.0 -   ||                    DRS\_INTERACTIVE is not set, running on-line mode}}
\textcolor{ansi-green-intense}{\textbf{12:05:18.0 -   ||                    DRS\_DEBUG is set, debug mode level:1}}

    \end{Verbatim}

    Taking a look at what is contained in "p" at this point is shown below.
It should be representative of the constants defined in constants.py
where type casting to floats/ints and booleans has been done where
possible.

    \begin{Verbatim}[commandchars=\\\{\}]
{\color{incolor}In [{\color{incolor}3}]:} \PY{n}{p}
\end{Verbatim}


\begin{Verbatim}[commandchars=\\\{\}]
{\color{outcolor}Out[{\color{outcolor}3}]:} \{'COLOURED\_LOG': True,
         'DRS\_CALIB\_DB': '/scratch/Projects/spirou\_py3/data/calibDB',
         'DRS\_CONFIG': '/scratch/Projects/spirou\_py3/spirou\_py3/INTROOT/config/',
         'DRS\_DATA\_MSG': '/scratch/Projects/spirou\_py3/data/msg',
         'DRS\_DATA\_RAW': '/scratch/Projects/spirou\_py3/data/raw',
         'DRS\_DATA\_REDUC': '/scratch/Projects/spirou\_py3/data/reduced',
         'DRS\_DATA\_WORKING': '/scratch/Projects/spirou\_py3/data/tmp',
         'DRS\_DEBUG': 1,
         'DRS\_INTERACTIVE': 0,
         'DRS\_MAN': '/scratch/Projects/spirou\_py3/spirou\_py3/INTROOT/man/',
         'DRS\_NAME': 'SPIROU',
         'DRS\_PLOT': 1,
         'DRS\_ROOT': '/scratch/Projects/spirou\_py3/spirou\_py3/INTROOT/',
         'DRS\_USED\_DATE': 'undefined',
         'DRS\_VERSION': '0.1.029',
         'ICDP\_NAME': 'constants\_SPIROU.py',
         'LOG\_LEVEL': 'all',
         'PRINT\_LEVEL': 'all',
         'SPECIAL\_NAME': 'special\_config\_SPIROU.py',
         'TDATA': '/scratch/Projects/spirou\_py3/data/'\}
\end{Verbatim}
            
    Note we can also look at where each key was defined:

    \begin{Verbatim}[commandchars=\\\{\}]
{\color{incolor}In [{\color{incolor}4}]:} \PY{n}{p}\PY{o}{.}\PY{n}{sources}
\end{Verbatim}


\begin{Verbatim}[commandchars=\\\{\}]
{\color{outcolor}Out[{\color{outcolor}4}]:} \{'COLOURED\_LOG': '/scratch/Projects/spirou\_py3/spirou\_py3/INTROOT/config/config.py',
         'DRS\_CALIB\_DB': '/scratch/Projects/spirou\_py3/spirou\_py3/INTROOT/config/config.py',
         'DRS\_CONFIG': 'spirouConfig.py.check\_params()',
         'DRS\_DATA\_MSG': '/scratch/Projects/spirou\_py3/spirou\_py3/INTROOT/config/config.py',
         'DRS\_DATA\_RAW': '/scratch/Projects/spirou\_py3/spirou\_py3/INTROOT/config/config.py',
         'DRS\_DATA\_REDUC': '/scratch/Projects/spirou\_py3/spirou\_py3/INTROOT/config/config.py',
         'DRS\_DATA\_WORKING': '/scratch/Projects/spirou\_py3/spirou\_py3/INTROOT/config/config.py',
         'DRS\_DEBUG': '/scratch/Projects/spirou\_py3/spirou\_py3/INTROOT/config/config.py',
         'DRS\_INTERACTIVE': 'spirouConfig.py.check\_params()',
         'DRS\_MAN': 'spirouConfig.py.check\_params()',
         'DRS\_NAME': 'spirouConfig.Constants',
         'DRS\_PLOT': '/scratch/Projects/spirou\_py3/spirou\_py3/INTROOT/config/config.py',
         'DRS\_ROOT': '/scratch/Projects/spirou\_py3/spirou\_py3/INTROOT/config/config.py',
         'DRS\_USED\_DATE': 'spirouConfig.py.check\_params()',
         'DRS\_VERSION': 'spirouConfig.Constants',
         'ICDP\_NAME': '/scratch/Projects/spirou\_py3/spirou\_py3/INTROOT/config/config.py',
         'LOG\_LEVEL': '/scratch/Projects/spirou\_py3/spirou\_py3/INTROOT/config/config.py',
         'PRINT\_LEVEL': '/scratch/Projects/spirou\_py3/spirou\_py3/INTROOT/config/config.py',
         'SPECIAL\_NAME': '/scratch/Projects/spirou\_py3/spirou\_py3/INTROOT/config/config.py',
         'TDATA': '/scratch/Projects/spirou\_py3/spirou\_py3/INTROOT/config/config.py'\}
\end{Verbatim}
            
    \subsection{LoadArguments}\label{loadarguments}

    This function is the second argument to run. It deals with loading of
the arguments both from the recipes main() arguments and from run time
(i.e. after the file name when called from python/ipython or a console).

By default the user will either have to define arguements from the
command line or in the function call to the recipes "main" function.

i.e. for cal\_DARK\_spirou.py one could call the following from the
command line: \textgreater{}\textgreater{} cal\_DARK\_Spirou.py 20170710
dark\_dark

or in python:

cal\_DARK\_spirou.main(night\_name=nightname, files=filenames)

    \begin{Verbatim}[commandchars=\\\{\}]
{\color{incolor}In [{\color{incolor}5}]:} \PY{c+c1}{\PYZsh{} we will assume we are inside the cal\PYZus{}DARK\PYZus{}Spirou.main function and have called }
        \PY{c+c1}{\PYZsh{}   the arguments correctly:}
        \PY{n}{night\PYZus{}name} \PY{o}{=} \PY{l+s+s1}{\PYZsq{}}\PY{l+s+s1}{20170710}\PY{l+s+s1}{\PYZsq{}}
        \PY{n}{files} \PY{o}{=} \PY{p}{[}\PY{l+s+s1}{\PYZsq{}}\PY{l+s+s1}{dark\PYZus{}dark02d406.fits}\PY{l+s+s1}{\PYZsq{}}\PY{p}{]}
        
        \PY{c+c1}{\PYZsh{} Notebook only code}
        \PY{k+kn}{import} \PY{n+nn}{sys}
        \PY{n}{sys}\PY{o}{.}\PY{n}{argv} \PY{o}{=} \PY{p}{[}\PY{l+s+s1}{\PYZsq{}}\PY{l+s+s1}{cal\PYZus{}DARK\PYZus{}spirou.py}\PY{l+s+s1}{\PYZsq{}}\PY{p}{]}
\end{Verbatim}


    we can then call the LoadArguments function

    \begin{Verbatim}[commandchars=\\\{\}]
{\color{incolor}In [{\color{incolor}6}]:} \PY{n}{p} \PY{o}{=} \PY{n}{spirouStartup}\PY{o}{.}\PY{n}{LoadArguments}\PY{p}{(}\PY{n}{p}\PY{p}{,} \PY{n}{night\PYZus{}name}\PY{p}{,} \PY{n}{files}\PY{p}{)}
\end{Verbatim}


    \begin{Verbatim}[commandchars=\\\{\}]
\textcolor{ansi-green-intense}{\textbf{12:05:27.0 -   |cal\_DARK\_spirou|Now running : cal\_DARK\_spirou on file(s): }}
\textcolor{ansi-green-intense}{\textbf{12:05:27.0 -   |cal\_DARK\_spirou|On directory /scratch/Projects/spirou\_py3/data/raw/20170710}}
\textcolor{ansi-green-intense}{\textbf{12:05:27.0 -   |cal\_DARK\_spirou|ICDP\_NAME loaded from: /scratch/Projects/spirou\_py3/spirou\_py3/INTROOT/config/constants\_SPIROU.py}}

    \end{Verbatim}

    Taking a look a "p" now we can see it has added the "run time"
arguements and additional constants from the constants\_SPIROU.py file.
Note below we only print the key names.

    \begin{Verbatim}[commandchars=\\\{\}]
{\color{incolor}In [{\color{incolor}7}]:} \PY{k}{print}\PY{p}{(}\PY{n+nb}{list}\PY{p}{(}\PY{n}{p}\PY{o}{.}\PY{n}{keys}\PY{p}{(}\PY{p}{)}\PY{p}{)}\PY{p}{)}
\end{Verbatim}


    \begin{Verbatim}[commandchars=\\\{\}]
['DRS\_PLOT', 'DRS\_DEBUG', 'TDATA', 'DRS\_ROOT', 'DRS\_DATA\_RAW', 'DRS\_DATA\_REDUC', 'DRS\_CALIB\_DB', 'DRS\_DATA\_MSG', 'DRS\_DATA\_WORKING', 'PRINT\_LEVEL', 'LOG\_LEVEL', 'COLOURED\_LOG', 'SPECIAL\_NAME', 'ICDP\_NAME', 'DRS\_NAME', 'DRS\_VERSION', 'DRS\_CONFIG', 'DRS\_MAN', 'DRS\_USED\_DATE', 'DRS\_INTERACTIVE', 'ARG\_NIGHT\_NAME', 'PROGRAM', 'ARG\_FILE\_NAMES', 'STR\_FILE\_NAMES', 'FITSFILENAME', 'LOG\_OPT', 'NBFRAMES', 'REDUCED\_DIR', 'RAW\_DIR', 'IC\_DISPLAY\_TIMEOUT', 'IC\_CCDX\_BLUE\_LOW', 'IC\_CCDX\_BLUE\_HIGH', 'IC\_CCDY\_BLUE\_LOW', 'IC\_CCDY\_BLUE\_HIGH', 'IC\_CCDX\_RED\_LOW', 'IC\_CCDX\_RED\_HIGH', 'IC\_CCDY\_RED\_LOW', 'IC\_CCDY\_RED\_HIGH', 'IC\_CCDX\_LOW', 'IC\_CCDX\_HIGH', 'IC\_CCDY\_LOW', 'IC\_CCDY\_HIGH', 'FIBER\_TYPES', 'NBFIB\_FPALL', 'IC\_FIRST\_ORDER\_JUMP\_FPALL', 'IC\_LOCNBMAXO\_FPALL', 'QC\_LOC\_NBO\_FPALL', 'FIB\_TYPE\_FPALL', 'IC\_EXT\_RANGE1\_FPALL', 'IC\_EXT\_RANGE2\_FPALL', 'IC\_EXT\_RANGE\_FPALL', 'LOC\_FILE\_FPALL', 'ORDERP\_FILE\_FPALL', 'IC\_EXT\_D\_RANGE\_FPALL', 'DARK\_QMIN', 'DARK\_QMAX', 'HISTO\_BINS', 'HISTO\_RANGE\_LOW', 'HISTO\_RANGE\_HIGH', 'DARK\_CUTLIMIT', 'LOC\_BOX\_SIZE', 'IC\_OFFSET', 'IC\_CENT\_COL', 'IC\_EXT\_WINDOW', 'IC\_IMAGE\_GAP', 'IC\_LOCSTEPC', 'IC\_WIDTHMIN', 'IC\_LOCNBPIX', 'IC\_MIN\_AMPLITUDE', 'IC\_LOCSEUIL', 'IC\_SATSEUIL', 'IC\_LOCDFITC', 'IC\_LOCDFITW', 'IC\_LOCDFITP', 'IC\_MAX\_RMS\_CENTER', 'IC\_MAX\_PTP\_CENTER', 'IC\_PTPORMS\_CENTER', 'IC\_MAX\_RMS\_FWHM', 'IC\_MAX\_PTP\_FRACFWHM', 'IC\_LOC\_DELTA\_WIDTH', 'IC\_LOCOPT1', 'IC\_TILT\_COI', 'IC\_FACDEC', 'IC\_TILT\_FIT', 'IC\_SLIT\_ORDER\_PLOT', 'IC\_DO\_BKGR\_SUBTRACTION', 'IC\_BKGR\_WINDOW', 'IC\_TILT\_NBO', 'IC\_FF\_SIGDET', 'IC\_EXTFBLAZ', 'IC\_BLAZE\_FITN', 'IC\_FF\_ORDER\_PLOT', 'IC\_FF\_PLOT\_ALL\_ORDERS', 'IC\_EXTOPT', 'IC\_EXTNBSIG', 'IC\_EXTRACT\_TYPE', 'IC\_EXT\_SIGDET', 'IC\_EXT\_ORDER\_PLOT', 'IC\_DRIFT\_NOISE', 'IC\_DRIFT\_MAXFLUX', 'IC\_DRIFT\_BOXSIZE', 'DRIFT\_NLARGE', 'DRIFT\_FILE\_SKIP', 'DRIFT\_E2DS\_FILE\_SKIP', 'DRIFT\_PEAK\_FILE\_SKIP', 'IC\_DRIFT\_CUT\_RAW', 'IC\_DRIFT\_CUT\_E2DS', 'IC\_DRIFT\_N\_ORDER\_MAX', 'IC\_DRIFT\_PEAK\_N\_ORDER\_MIN', 'IC\_DRIFT\_PEAK\_N\_ORDER\_MAX', 'DRIFT\_TYPE\_RAW', 'DRIFT\_TYPE\_E2DS', 'IC\_DRIFT\_ORDER\_PLOT', 'DRIFT\_PEAK\_MINMAX\_BOXSIZE', 'DRIFT\_PEAK\_BORDER\_SIZE', 'DRIFT\_PEAK\_FPBOX\_SIZE', 'DRIFT\_PEAK\_PEAK\_SIG\_LIM', 'DRIFT\_PEAK\_INTER\_PEAK\_SPACING', 'DRIFT\_PEAK\_EXP\_WIDTH', 'DRIFT\_PEAK\_NORM\_WIDTH\_CUT', 'DRIFT\_PEAK\_GETDRIFT\_GAUSSFIT', 'DRIFT\_PEAK\_PEARSONR\_CUT', 'DRIFT\_PEAK\_SIGMACLIP', 'DRIFT\_PEAK\_PLOT\_LINE\_LOG\_AMP', 'DRIFT\_PEAK\_SELECTED\_ORDER', 'BADPIX\_FLAT\_MED\_WID', 'BADPIX\_ILLUM\_CUT', 'BADPIX\_FLAT\_CUT\_RATIO', 'BADPIX\_MAX\_HOTPIX', 'BADPIX\_NORM\_PERCENTILE', 'IC\_CCF\_WIDTH', 'IC\_CCF\_STEP', 'IC\_W\_MASK\_MIN', 'IC\_MASK\_WIDTH', 'CCF\_BERV', 'CCF\_BERV\_MAX', 'CCF\_DET\_NOISE', 'CCF\_FIT\_TYPE', 'CCF\_NUM\_ORDERS\_MAX', 'QC\_MAX\_DARKLEVEL', 'QC\_MAX\_DEAD', 'QC\_MAX\_DARK', 'QC\_DARK\_TIME', 'QC\_LOC\_MAXLOCFIT\_REMOVED\_CTR', 'QC\_LOC\_MAXLOCFIT\_REMOVED\_WID', 'QC\_LOC\_RMSMAX\_CENTER', 'QC\_LOC\_RMSMAX\_FWHM', 'QC\_FF\_RMS', 'QC\_LOC\_FLUMAX', 'QC\_SLIT\_RMS', 'QC\_SLIT\_MIN', 'QC\_SLIT\_MAX', 'QC\_MAX\_SIGNAL', 'IC\_CALIBDB\_FILENAME', 'CALIB\_MAX\_WAIT', 'CALIB\_DB\_MATCH', 'KW\_ACQTIME\_KEY', 'KW\_ACQTIME\_KEY\_UNIX', 'KW\_BBAD', 'KW\_BBFLAT', 'KW\_BHOT', 'KW\_BNDARK', 'KW\_BNFLAT', 'KW\_CCD\_CONAD', 'KW\_CCD\_SIGDET', 'KW\_CCF\_CDELT', 'KW\_CCF\_CONTRAST', 'KW\_CCF\_CRVAL', 'KW\_CCF\_CTYPE', 'KW\_CCF\_FWHM', 'KW\_CCF\_LINES', 'KW\_CCF\_MASK', 'KW\_CCF\_MAXCPP', 'KW\_CCF\_RV', 'KW\_CCF\_RVC', 'KW\_DARK\_B\_DEAD', 'KW\_DARK\_B\_MED', 'KW\_DARK\_CUT', 'KW\_DARK\_DEAD', 'KW\_DARK\_MED', 'KW\_DARK\_R\_DEAD', 'KW\_DARK\_R\_MED', 'KW\_DPRTYPE', 'KW\_EXPTIME', 'KW\_EXTRA\_SN', 'KW\_FLAT\_RMS', 'KW\_GAIN', 'KW\_LOCO\_BCKGRD', 'KW\_LOCO\_CTR\_COEFF', 'KW\_LOCO\_DEG\_C', 'KW\_LOCO\_DEG\_E', 'KW\_LOCO\_DEG\_W', 'KW\_LOCO\_DELTA', 'KW\_LOCO\_FILE', 'KW\_LOCO\_FWHM\_COEFF', 'KW\_LOCO\_NBO', 'KW\_LOC\_MAXFLX', 'KW\_LOC\_RMS\_CTR', 'KW\_LOC\_RMS\_WID', 'KW\_LOC\_SMAXPTS\_CTR', 'KW\_LOC\_SMAXPTS\_WID', 'KW\_RDNOISE', 'KW\_TH\_COEFF\_PREFIX', 'KW\_TH\_LL\_D', 'KW\_TH\_NAXIS1', 'KW\_TH\_NAXIS2', 'KW\_TH\_ORD\_N', 'KW\_TILT', 'KW\_DRS\_QC', 'KW\_ROOT\_DRS\_FLAT', 'KW\_ROOT\_DRS\_HC', 'KW\_ROOT\_DRS\_LOC', 'KW\_VERSION']

    \end{Verbatim}

    Sources are set for all variables, some examples are shown below. Note
that strings are evaluated (i.e. "IC\_CCDX\_RED\_LOW = 2028" value is a
int not a string.

    \begin{Verbatim}[commandchars=\\\{\}]
{\color{incolor}In [{\color{incolor}8}]:} \PY{n}{keys} \PY{o}{=} \PY{p}{[}\PY{l+s+s1}{\PYZsq{}}\PY{l+s+s1}{DRS\PYZus{}NAME}\PY{l+s+s1}{\PYZsq{}}\PY{p}{,} \PY{l+s+s1}{\PYZsq{}}\PY{l+s+s1}{IC\PYZus{}CCDX\PYZus{}RED\PYZus{}LOW}\PY{l+s+s1}{\PYZsq{}}\PY{p}{,} \PY{l+s+s1}{\PYZsq{}}\PY{l+s+s1}{IC\PYZus{}CCDX\PYZus{}BLUE\PYZus{}LOW}\PY{l+s+s1}{\PYZsq{}}\PY{p}{,} \PY{l+s+s1}{\PYZsq{}}\PY{l+s+s1}{KW\PYZus{}VERSION}\PY{l+s+s1}{\PYZsq{}}\PY{p}{]}
        
        \PY{k}{for} \PY{n}{key} \PY{o+ow}{in} \PY{n}{keys}\PY{p}{:}
            \PY{n}{args} \PY{o}{=} \PY{p}{[}\PY{n}{key}\PY{p}{,} \PY{n}{p}\PY{p}{[}\PY{n}{key}\PY{p}{]}\PY{p}{,} \PY{n+nb}{type}\PY{p}{(}\PY{n}{p}\PY{p}{[}\PY{n}{key}\PY{p}{]}\PY{p}{)}\PY{p}{,} \PY{n}{p}\PY{o}{.}\PY{n}{sources}\PY{p}{[}\PY{n}{key}\PY{p}{]}\PY{p}{]}
            \PY{k}{print}\PY{p}{(}\PY{l+s+s1}{\PYZsq{}}\PY{l+s+se}{\PYZbs{}n}\PY{l+s+s1}{ \PYZhy{} key = }\PY{l+s+s1}{\PYZdq{}}\PY{l+s+s1}{\PYZob{}0\PYZcb{}}\PY{l+s+s1}{\PYZdq{}}\PY{l+s+se}{\PYZbs{}n}\PY{l+s+se}{\PYZbs{}t}\PY{l+s+s1}{value=\PYZob{}1\PYZcb{}}\PY{l+s+se}{\PYZbs{}n}\PY{l+s+se}{\PYZbs{}t}\PY{l+s+s1}{type=\PYZob{}2\PYZcb{}}\PY{l+s+se}{\PYZbs{}n}\PY{l+s+se}{\PYZbs{}t}\PY{l+s+s1}{source=\PYZob{}3\PYZcb{}}\PY{l+s+s1}{\PYZsq{}}\PY{o}{.}\PY{n}{format}\PY{p}{(}\PY{o}{*}\PY{n}{args}\PY{p}{)}\PY{p}{)}
\end{Verbatim}


    \begin{Verbatim}[commandchars=\\\{\}]

 - key = "DRS\_NAME"
	value=SPIROU
	type=<class 'str'>
	source=spirouConfig.Constants

 - key = "IC\_CCDX\_RED\_LOW"
	value=2028
	type=<class 'int'>
	source=/scratch/Projects/spirou\_py3/spirou\_py3/INTROOT/config/constants\_SPIROU.py

 - key = "IC\_CCDX\_BLUE\_LOW"
	value=1848
	type=<class 'int'>
	source=/scratch/Projects/spirou\_py3/spirou\_py3/INTROOT/config/constants\_SPIROU.py

 - key = "KW\_VERSION"
	value=['VERSION', 'SPIROU\_0.1.029', 'DRS version']
	type=<class 'list'>
	source=spirouKeywords.py

    \end{Verbatim}

    \section{InitialFileSetup}\label{initialfilesetup}

    This is used to set up files and check them for required prefixes, and
loads the calibration database if it is reuiqred. Below we use the
example of cal\_loc\_RAW\_spirou.py where either "dark\_flat" or
"flat\_dark" is required (and a set of different fiber parameter are
required for each - i.e. "dark\_flat" means we are using fiber "C" and
"flat\_dark" means we are using fibers "AB")

    \begin{Verbatim}[commandchars=\\\{\}]
{\color{incolor}In [{\color{incolor}10}]:} \PY{c+c1}{\PYZsh{} cal\PYZus{}loc\PYZus{}RAW requires some additional imports}
         \PY{k+kn}{from} \PY{n+nn}{\PYZus{}\PYZus{}future\PYZus{}\PYZus{}} \PY{k+kn}{import} \PY{n}{division}
         
         \PY{k+kn}{from} \PY{n+nn}{SpirouDRS} \PY{k+kn}{import} \PY{n}{spirouImage}
         \PY{k+kn}{from} \PY{n+nn}{SpirouDRS} \PY{k+kn}{import} \PY{n}{spirouLOCOR}
         \PY{k+kn}{from} \PY{n+nn}{SpirouDRS} \PY{k+kn}{import} \PY{n}{spirouStartup}
         
         \PY{c+c1}{\PYZsh{} redefine arguments}
         \PY{n}{night\PYZus{}name} \PY{o}{=} \PY{l+s+s1}{\PYZsq{}}\PY{l+s+s1}{20170710}\PY{l+s+s1}{\PYZsq{}}
         \PY{n}{files} \PY{o}{=} \PY{p}{[}\PY{l+s+s1}{\PYZsq{}}\PY{l+s+s1}{flat\PYZus{}dark02f10.fits}\PY{l+s+s1}{\PYZsq{}}\PY{p}{,} \PY{l+s+s1}{\PYZsq{}}\PY{l+s+s1}{flat\PYZus{}dark03f10.fits}\PY{l+s+s1}{\PYZsq{}}\PY{p}{,}
                  \PY{l+s+s1}{\PYZsq{}}\PY{l+s+s1}{flat\PYZus{}dark04f10.fits}\PY{l+s+s1}{\PYZsq{}}\PY{p}{,} \PY{l+s+s1}{\PYZsq{}}\PY{l+s+s1}{flat\PYZus{}dark05f10.fits}\PY{l+s+s1}{\PYZsq{}}\PY{p}{,}
                  \PY{l+s+s1}{\PYZsq{}}\PY{l+s+s1}{flat\PYZus{}dark06f10.fits}\PY{l+s+s1}{\PYZsq{}}\PY{p}{]}
         
         \PY{c+c1}{\PYZsh{} \PYZhy{}\PYZhy{}\PYZhy{}\PYZhy{}\PYZhy{}\PYZhy{}\PYZhy{}\PYZhy{}\PYZhy{}\PYZhy{}\PYZhy{}\PYZhy{}\PYZhy{}\PYZhy{}\PYZhy{}\PYZhy{}\PYZhy{}\PYZhy{}\PYZhy{}\PYZhy{}\PYZhy{}\PYZhy{}\PYZhy{}\PYZhy{}\PYZhy{}\PYZhy{}\PYZhy{}\PYZhy{}\PYZhy{}\PYZhy{}\PYZhy{}\PYZhy{}\PYZhy{}\PYZhy{}\PYZhy{}\PYZhy{}\PYZhy{}\PYZhy{}\PYZhy{}\PYZhy{}\PYZhy{}\PYZhy{}\PYZhy{}\PYZhy{}\PYZhy{}\PYZhy{}\PYZhy{}\PYZhy{}\PYZhy{}\PYZhy{}\PYZhy{}\PYZhy{}\PYZhy{}\PYZhy{}\PYZhy{}\PYZhy{}\PYZhy{}\PYZhy{}\PYZhy{}\PYZhy{}\PYZhy{}\PYZhy{}\PYZhy{}\PYZhy{}\PYZhy{}\PYZhy{}\PYZhy{}\PYZhy{}\PYZhy{}\PYZhy{}}
         \PY{c+c1}{\PYZsh{} Set up}
         \PY{c+c1}{\PYZsh{} \PYZhy{}\PYZhy{}\PYZhy{}\PYZhy{}\PYZhy{}\PYZhy{}\PYZhy{}\PYZhy{}\PYZhy{}\PYZhy{}\PYZhy{}\PYZhy{}\PYZhy{}\PYZhy{}\PYZhy{}\PYZhy{}\PYZhy{}\PYZhy{}\PYZhy{}\PYZhy{}\PYZhy{}\PYZhy{}\PYZhy{}\PYZhy{}\PYZhy{}\PYZhy{}\PYZhy{}\PYZhy{}\PYZhy{}\PYZhy{}\PYZhy{}\PYZhy{}\PYZhy{}\PYZhy{}\PYZhy{}\PYZhy{}\PYZhy{}\PYZhy{}\PYZhy{}\PYZhy{}\PYZhy{}\PYZhy{}\PYZhy{}\PYZhy{}\PYZhy{}\PYZhy{}\PYZhy{}\PYZhy{}\PYZhy{}\PYZhy{}\PYZhy{}\PYZhy{}\PYZhy{}\PYZhy{}\PYZhy{}\PYZhy{}\PYZhy{}\PYZhy{}\PYZhy{}\PYZhy{}\PYZhy{}\PYZhy{}\PYZhy{}\PYZhy{}\PYZhy{}\PYZhy{}\PYZhy{}\PYZhy{}\PYZhy{}\PYZhy{}}
         \PY{c+c1}{\PYZsh{} get parameters from config files/run time args/load paths + calibdb}
         \PY{n}{p} \PY{o}{=} \PY{n}{spirouStartup}\PY{o}{.}\PY{n}{Begin}\PY{p}{(}\PY{p}{)}
         \PY{n}{p} \PY{o}{=} \PY{n}{spirouStartup}\PY{o}{.}\PY{n}{LoadArguments}\PY{p}{(}\PY{n}{p}\PY{p}{,} \PY{n}{night\PYZus{}name}\PY{p}{,} \PY{n}{files}\PY{p}{)}
         \PY{c+c1}{\PYZsh{} run specific start up}
         \PY{n}{params2add} \PY{o}{=} \PY{n+nb}{dict}\PY{p}{(}\PY{p}{)}
         \PY{n}{params2add}\PY{p}{[}\PY{l+s+s1}{\PYZsq{}}\PY{l+s+s1}{dark\PYZus{}flat}\PY{l+s+s1}{\PYZsq{}}\PY{p}{]} \PY{o}{=} \PY{n}{spirouLOCOR}\PY{o}{.}\PY{n}{FiberParams}\PY{p}{(}\PY{n}{p}\PY{p}{,} \PY{l+s+s1}{\PYZsq{}}\PY{l+s+s1}{C}\PY{l+s+s1}{\PYZsq{}}\PY{p}{)}
         \PY{n}{params2add}\PY{p}{[}\PY{l+s+s1}{\PYZsq{}}\PY{l+s+s1}{flat\PYZus{}dark}\PY{l+s+s1}{\PYZsq{}}\PY{p}{]} \PY{o}{=} \PY{n}{spirouLOCOR}\PY{o}{.}\PY{n}{FiberParams}\PY{p}{(}\PY{n}{p}\PY{p}{,} \PY{l+s+s1}{\PYZsq{}}\PY{l+s+s1}{AB}\PY{l+s+s1}{\PYZsq{}}\PY{p}{)}
         \PY{n}{p} \PY{o}{=} \PY{n}{spirouStartup}\PY{o}{.}\PY{n}{InitialFileSetup}\PY{p}{(}\PY{n}{p}\PY{p}{,} \PY{n}{kind}\PY{o}{=}\PY{l+s+s1}{\PYZsq{}}\PY{l+s+s1}{localisation}\PY{l+s+s1}{\PYZsq{}}\PY{p}{,}
                                            \PY{n}{prefixes}\PY{o}{=}\PY{p}{[}\PY{l+s+s1}{\PYZsq{}}\PY{l+s+s1}{dark\PYZus{}flat}\PY{l+s+s1}{\PYZsq{}}\PY{p}{,} \PY{l+s+s1}{\PYZsq{}}\PY{l+s+s1}{flat\PYZus{}dark}\PY{l+s+s1}{\PYZsq{}}\PY{p}{]}\PY{p}{,}
                                            \PY{n}{add\PYZus{}to\PYZus{}p}\PY{o}{=}\PY{n}{params2add}\PY{p}{,} \PY{n}{calibdb}\PY{o}{=}\PY{n+nb+bp}{True}\PY{p}{)}
\end{Verbatim}


    \begin{Verbatim}[commandchars=\\\{\}]
\textcolor{ansi-green-intense}{\textbf{12:05:37.0 -   || *****************************************}}
\textcolor{ansi-green-intense}{\textbf{12:05:37.0 -   || * SPIROU @(\#) Geneva Observatory (0.1.029)}}
\textcolor{ansi-green-intense}{\textbf{12:05:37.0 -   || *****************************************}}
\textcolor{ansi-green-intense}{\textbf{12:05:37.0 -   ||(dir\_data\_raw)      DRS\_DATA\_RAW=/scratch/Projects/spirou\_py3/data/raw}}
\textcolor{ansi-green-intense}{\textbf{12:05:37.0 -   ||(dir\_data\_reduc)    DRS\_DATA\_REDUC=/scratch/Projects/spirou\_py3/data/reduced}}
\textcolor{ansi-green-intense}{\textbf{12:05:37.0 -   ||(dir\_calib\_db)      DRS\_CALIB\_DB=/scratch/Projects/spirou\_py3/data/calibDB}}
\textcolor{ansi-green-intense}{\textbf{12:05:37.0 -   ||(dir\_data\_msg)      DRS\_DATA\_MSG=/scratch/Projects/spirou\_py3/data/msg}}
\textcolor{ansi-green-intense}{\textbf{12:05:37.0 -   ||(print\_level)       PRINT\_LEVEL=all         \%(error/warning/info/all)}}
\textcolor{ansi-green-intense}{\textbf{12:05:37.0 -   ||(log\_level)         LOG\_LEVEL=all         \%(error/warning/info/all)}}
\textcolor{ansi-green-intense}{\textbf{12:05:37.0 -   ||(plot\_graph)        DRS\_PLOT=1            \%(def/undef/trigger)}}
\textcolor{ansi-green-intense}{\textbf{12:05:37.0 -   ||(used\_date)         DRS\_USED\_DATE=undefined}}
\textcolor{ansi-green-intense}{\textbf{12:05:37.0 -   ||(working\_dir)       DRS\_DATA\_WORKING=/scratch/Projects/spirou\_py3/data/tmp}}
\textcolor{ansi-green-intense}{\textbf{12:05:37.0 -   ||                    DRS\_INTERACTIVE is not set, running on-line mode}}
\textcolor{ansi-green-intense}{\textbf{12:05:37.0 -   ||                    DRS\_DEBUG is set, debug mode level:1}}
\textcolor{ansi-green-intense}{\textbf{12:05:37.0 -   |cal\_DARK\_spirou|Now running : cal\_DARK\_spirou on file(s): }}
\textcolor{ansi-green-intense}{\textbf{12:05:37.0 -   |cal\_DARK\_spirou|On directory /scratch/Projects/spirou\_py3/data/raw/20170710}}
\textcolor{ansi-green-intense}{\textbf{12:05:37.0 -   |cal\_DARK\_spirou|ICDP\_NAME loaded from: /scratch/Projects/spirou\_py3/spirou\_py3/INTROOT/config/constants\_SPIROU.py}}
\textcolor{ansi-green-intense}{\textbf{12:05:37.0 - * |cal\_DARK\_spirou|Correct type of image for localisation (dark\_flat or flat\_dark)}}
\textcolor{ansi-green-intense}{\textbf{12:05:38.0 -   |cal\_DARK\_spirou|Calibration file: 20170710\_flat\_flat02f10\_badpixel.fits already exists - not copied}}
\textcolor{ansi-green-intense}{\textbf{12:05:38.0 -   |cal\_DARK\_spirou|Calibration file: 20170710\_flat\_dark02f10\_blaze\_AB.fits already exists - not copied}}
\textcolor{ansi-green-intense}{\textbf{12:05:38.0 -   |cal\_DARK\_spirou|Calibration file: 20170710\_dark\_flat02f10\_blaze\_C.fits already exists - not copied}}
\textcolor{ansi-green-intense}{\textbf{12:05:39.0 -   |cal\_DARK\_spirou|Calibration file: 20170710\_dark\_dark02d406.fits already exists - not copied}}
\textcolor{ansi-green-intense}{\textbf{12:05:39.0 -   |cal\_DARK\_spirou|Calibration file: 20170710\_flat\_dark02f10\_flat\_AB.fits already exists - not copied}}
\textcolor{ansi-green-intense}{\textbf{12:05:39.0 -   |cal\_DARK\_spirou|Calibration file: 20170710\_dark\_flat02f10\_flat\_C.fits already exists - not copied}}
\textcolor{ansi-green-intense}{\textbf{12:05:39.0 -   |cal\_DARK\_spirou|Calibration file: 20170710\_flat\_dark02f10\_loco\_AB.fits already exists - not copied}}
\textcolor{ansi-green-intense}{\textbf{12:05:39.0 -   |cal\_DARK\_spirou|Calibration file: 20170710\_dark\_flat02f10\_loco\_C.fits already exists - not copied}}
\textcolor{ansi-green-intense}{\textbf{12:05:39.0 -   |cal\_DARK\_spirou|Calibration file: 20170710\_flat\_dark02f10\_order\_profile\_AB.fits already exists - not copied}}
\textcolor{ansi-green-intense}{\textbf{12:05:40.0 -   |cal\_DARK\_spirou|Calibration file: 20170710\_dark\_flat02f10\_order\_profile\_C.fits already exists - not copied}}
\textcolor{ansi-green-intense}{\textbf{12:05:40.0 -   |cal\_DARK\_spirou|Calibration file: 20170710\_fp\_fp02a203\_tilt.fits already exists - not copied}}
\textcolor{ansi-green-intense}{\textbf{12:05:40.0 -   |cal\_DARK\_spirou|Calibration file: spirou\_wave\_ini3.fits already exists - not copied}}
\textcolor{ansi-green-intense}{\textbf{12:05:40.0 -   |cal\_DARK\_spirou|Calibration file: 2017-10-11\_21-32-17\_hcone\_hcone02c406\_wave\_AB.fits already exists - not copied}}
\textcolor{ansi-green-intense}{\textbf{12:05:40.0 -   |cal\_DARK\_spirou|Calibration file: spirou\_wave\_ini3.fits already exists - not copied}}
\textcolor{ansi-green-intense}{\textbf{12:05:40.0 -   |cal\_DARK\_spirou|Calibration file: 2017-10-11\_21-32-17\_hcone\_hcone02c406\_wave\_C.fits already exists - not copied}}

    \end{Verbatim}

    The DRS is now set up and if "calibdb" was True then we have access to
the full calibration database (using "fitsfilename" as the reference to
sort between files)

    \begin{Verbatim}[commandchars=\\\{\}]
{\color{incolor}In [{\color{incolor}12}]:} \PY{k}{print}\PY{p}{(}\PY{l+s+s1}{\PYZsq{}}\PY{l+s+se}{\PYZbs{}n}\PY{l+s+s1}{fitsfilename: \PYZob{}0\PYZcb{}}\PY{l+s+s1}{\PYZsq{}}\PY{o}{.}\PY{n}{format}\PY{p}{(}\PY{n}{p}\PY{p}{[}\PY{l+s+s1}{\PYZsq{}}\PY{l+s+s1}{fitsfilename}\PY{l+s+s1}{\PYZsq{}}\PY{p}{]}\PY{p}{)}\PY{p}{)}
         \PY{k}{print}\PY{p}{(}\PY{l+s+s1}{\PYZsq{}}\PY{l+s+se}{\PYZbs{}n}\PY{l+s+s1}{arg\PYZus{}file\PYZus{}names: \PYZob{}0\PYZcb{}}\PY{l+s+s1}{\PYZsq{}}\PY{o}{.}\PY{n}{format}\PY{p}{(}\PY{n}{p}\PY{p}{[}\PY{l+s+s1}{\PYZsq{}}\PY{l+s+s1}{arg\PYZus{}file\PYZus{}names}\PY{l+s+s1}{\PYZsq{}}\PY{p}{]}\PY{p}{)}\PY{p}{)}
         \PY{k}{print}\PY{p}{(}\PY{l+s+s1}{\PYZsq{}}\PY{l+s+se}{\PYZbs{}n}\PY{l+s+s1}{calib db: \PYZob{}0\PYZcb{}}\PY{l+s+s1}{\PYZsq{}}\PY{o}{.}\PY{n}{format}\PY{p}{(}\PY{n}{p}\PY{p}{[}\PY{l+s+s1}{\PYZsq{}}\PY{l+s+s1}{calibdb}\PY{l+s+s1}{\PYZsq{}}\PY{p}{]}\PY{p}{)}\PY{p}{)}
\end{Verbatim}


    \begin{Verbatim}[commandchars=\\\{\}]

fitsfilename: /scratch/Projects/spirou\_py3/data/raw/20170710/flat\_dark02f10.fits

arg\_file\_names: ['flat\_dark02f10.fits', 'flat\_dark03f10.fits', 'flat\_dark04f10.fits', 'flat\_dark05f10.fits', 'flat\_dark06f10.fits']

calib db: \{'BADPIX': ['20170710', '20170710\_flat\_flat02f10\_badpixel.fits'], 'BLAZE\_AB': ['20170710', '20170710\_flat\_dark02f10\_blaze\_AB.fits'], 'BLAZE\_C': ['20170710', '20170710\_dark\_flat02f10\_blaze\_C.fits'], 'DARK': ['20170710', '20170710\_dark\_dark02d406.fits'], 'FLAT\_AB': ['20170710', '20170710\_flat\_dark02f10\_flat\_AB.fits'], 'FLAT\_C': ['20170710', '20170710\_dark\_flat02f10\_flat\_C.fits'], 'LOC\_AB': ['20170710', '20170710\_flat\_dark02f10\_loco\_AB.fits'], 'LOC\_C': ['20170710', '20170710\_dark\_flat02f10\_loco\_C.fits'], 'ORDER\_PROFILE\_AB': ['20170710', '20170710\_flat\_dark02f10\_order\_profile\_AB.fits'], 'ORDER\_PROFILE\_C': ['20170710', '20170710\_dark\_flat02f10\_order\_profile\_C.fits'], 'TILT': ['20170710', '20170710\_fp\_fp02a203\_tilt.fits'], 'WAVE\_A': ['20170710', 'spirou\_wave\_ini3.fits'], 'WAVE\_AB': ['AT4-04/2017-10-11\_21-32-17', '2017-10-11\_21-32-17\_hcone\_hcone02c406\_wave\_AB.fits'], 'WAVE\_B': ['20170710', 'spirou\_wave\_ini3.fits'], 'WAVE\_C': ['AT4-04/2017-10-11\_21-32-17', '2017-10-11\_21-32-17\_hcone\_hcone02c406\_wave\_C.fits']\}

    \end{Verbatim}

    \subsection{Defining and updating variables using
ParamDict}\label{defining-and-updating-variables-using-paramdict}

One will need to update and set variables in the parameter dictionary at
various points in a recipe. When one does this one should update the
source(s). This is shown below. It is also possible to update/add
multiple variables with one source.

    \begin{Verbatim}[commandchars=\\\{\}]
{\color{incolor}In [{\color{incolor}13}]:} \PY{c+c1}{\PYZsh{} set new variable}
         \PY{n}{p}\PY{p}{[}\PY{l+s+s1}{\PYZsq{}}\PY{l+s+s1}{MY\PYZus{}NEW\PYZus{}VARIABLE}\PY{l+s+s1}{\PYZsq{}}\PY{p}{]} \PY{o}{=} \PY{l+m+mf}{3.141592}
         \PY{n}{p}\PY{o}{.}\PY{n}{set\PYZus{}source}\PY{p}{(}\PY{l+s+s1}{\PYZsq{}}\PY{l+s+s1}{MY\PYZus{}NEW\PYZus{}VARIABLE}\PY{l+s+s1}{\PYZsq{}}\PY{p}{,} \PY{l+s+s1}{\PYZsq{}}\PY{l+s+s1}{myprogram.myfunction1()}\PY{l+s+s1}{\PYZsq{}}\PY{p}{)}
         
         \PY{c+c1}{\PYZsh{} update current variable and append source}
         \PY{n}{p}\PY{p}{[}\PY{l+s+s1}{\PYZsq{}}\PY{l+s+s1}{FITSFILENAME}\PY{l+s+s1}{\PYZsq{}}\PY{p}{]} \PY{o}{=} \PY{l+s+s1}{\PYZsq{}}\PY{l+s+s1}{newfile2}\PY{l+s+s1}{\PYZsq{}}
         \PY{n}{p}\PY{o}{.}\PY{n}{append\PYZus{}source}\PY{p}{(}\PY{l+s+s1}{\PYZsq{}}\PY{l+s+s1}{FITSFILENAME}\PY{l+s+s1}{\PYZsq{}}\PY{p}{,} \PY{l+s+s1}{\PYZsq{}}\PY{l+s+s1}{myprogram.myfunction2()}\PY{l+s+s1}{\PYZsq{}}\PY{p}{)}
         
         \PY{c+c1}{\PYZsh{} add multiple variables with same source name}
         \PY{n}{p}\PY{p}{[}\PY{l+s+s1}{\PYZsq{}}\PY{l+s+s1}{VAR1}\PY{l+s+s1}{\PYZsq{}}\PY{p}{]} \PY{o}{=} \PY{l+m+mi}{1}
         \PY{n}{p}\PY{p}{[}\PY{l+s+s1}{\PYZsq{}}\PY{l+s+s1}{VAR2}\PY{l+s+s1}{\PYZsq{}}\PY{p}{]} \PY{o}{=} \PY{l+m+mi}{2}
         \PY{n}{p}\PY{p}{[}\PY{l+s+s1}{\PYZsq{}}\PY{l+s+s1}{VAR3}\PY{l+s+s1}{\PYZsq{}}\PY{p}{]} \PY{o}{=} \PY{l+m+mi}{3}
         \PY{n}{p}\PY{o}{.}\PY{n}{set\PYZus{}sources}\PY{p}{(}\PY{p}{[}\PY{l+s+s1}{\PYZsq{}}\PY{l+s+s1}{VAR1}\PY{l+s+s1}{\PYZsq{}}\PY{p}{,} \PY{l+s+s1}{\PYZsq{}}\PY{l+s+s1}{VAR2}\PY{l+s+s1}{\PYZsq{}}\PY{p}{,} \PY{l+s+s1}{\PYZsq{}}\PY{l+s+s1}{VAR3}\PY{l+s+s1}{\PYZsq{}}\PY{p}{]}\PY{p}{,} \PY{l+s+s1}{\PYZsq{}}\PY{l+s+s1}{myprogram.myfunction3()}\PY{l+s+s1}{\PYZsq{}}\PY{p}{)}
         
         \PY{n}{p}\PY{o}{.}\PY{n}{append\PYZus{}sources}\PY{p}{(}\PY{p}{[}\PY{l+s+s1}{\PYZsq{}}\PY{l+s+s1}{VAR1}\PY{l+s+s1}{\PYZsq{}}\PY{p}{,} \PY{l+s+s1}{\PYZsq{}}\PY{l+s+s1}{VAR3}\PY{l+s+s1}{\PYZsq{}}\PY{p}{]}\PY{p}{,} \PY{l+s+s1}{\PYZsq{}}\PY{l+s+s1}{myprogram.main()}\PY{l+s+s1}{\PYZsq{}}\PY{p}{)}
\end{Verbatim}


    Below we show the output of each new key in "p"

    \begin{Verbatim}[commandchars=\\\{\}]
{\color{incolor}In [{\color{incolor}14}]:} \PY{n}{keys} \PY{o}{=} \PY{p}{[}\PY{l+s+s1}{\PYZsq{}}\PY{l+s+s1}{MY\PYZus{}NEW\PYZus{}VARIABLE}\PY{l+s+s1}{\PYZsq{}}\PY{p}{,} \PY{l+s+s1}{\PYZsq{}}\PY{l+s+s1}{FITSFILENAME}\PY{l+s+s1}{\PYZsq{}}\PY{p}{,} \PY{l+s+s1}{\PYZsq{}}\PY{l+s+s1}{VAR1}\PY{l+s+s1}{\PYZsq{}}\PY{p}{,} \PY{l+s+s1}{\PYZsq{}}\PY{l+s+s1}{VAR2}\PY{l+s+s1}{\PYZsq{}}\PY{p}{,} \PY{l+s+s1}{\PYZsq{}}\PY{l+s+s1}{VAR3}\PY{l+s+s1}{\PYZsq{}}\PY{p}{]}
         
         \PY{k}{for} \PY{n}{key} \PY{o+ow}{in} \PY{n}{keys}\PY{p}{:}
             \PY{n}{args} \PY{o}{=} \PY{p}{[}\PY{n}{key}\PY{p}{,} \PY{n}{p}\PY{p}{[}\PY{n}{key}\PY{p}{]}\PY{p}{,} \PY{n+nb}{type}\PY{p}{(}\PY{n}{p}\PY{p}{[}\PY{n}{key}\PY{p}{]}\PY{p}{)}\PY{p}{,} \PY{n}{p}\PY{o}{.}\PY{n}{sources}\PY{p}{[}\PY{n}{key}\PY{p}{]}\PY{p}{]}
             \PY{k}{print}\PY{p}{(}\PY{l+s+s1}{\PYZsq{}}\PY{l+s+se}{\PYZbs{}n}\PY{l+s+s1}{ \PYZhy{} key = }\PY{l+s+s1}{\PYZdq{}}\PY{l+s+s1}{\PYZob{}0\PYZcb{}}\PY{l+s+s1}{\PYZdq{}}\PY{l+s+se}{\PYZbs{}n}\PY{l+s+se}{\PYZbs{}t}\PY{l+s+s1}{value=\PYZob{}1\PYZcb{}}\PY{l+s+se}{\PYZbs{}n}\PY{l+s+se}{\PYZbs{}t}\PY{l+s+s1}{type=\PYZob{}2\PYZcb{}}\PY{l+s+se}{\PYZbs{}n}\PY{l+s+se}{\PYZbs{}t}\PY{l+s+s1}{source=\PYZob{}3\PYZcb{}}\PY{l+s+s1}{\PYZsq{}}\PY{o}{.}\PY{n}{format}\PY{p}{(}\PY{o}{*}\PY{n}{args}\PY{p}{)}\PY{p}{)}
\end{Verbatim}


    \begin{Verbatim}[commandchars=\\\{\}]

 - key = "MY\_NEW\_VARIABLE"
	value=3.141592
	type=<class 'float'>
	source=myprogram.myfunction1()

 - key = "FITSFILENAME"
	value=newfile2
	type=<class 'str'>
	source=spirouStarup.py.get\_call\_arg\_files\_fitsfilename() myprogram.myfunction2() myprogram.myfunction2()

 - key = "VAR1"
	value=1
	type=<class 'int'>
	source=myprogram.myfunction3() myprogram.main()

 - key = "VAR2"
	value=2
	type=<class 'int'>
	source=myprogram.myfunction3()

 - key = "VAR3"
	value=3
	type=<class 'int'>
	source=myprogram.myfunction3() myprogram.main()

    \end{Verbatim}

    \subsection{Error and exception
handling}\label{error-and-exception-handling}

Constants and variables is one large area where errors and exceptions
can be caused, be it from incorrectly defined constants entered by the
users in the confirugations files, configuration files missing, illegal
characters or a developer trying to access a key that does not exist.
The parameter dictionary was designed with this in mind. It will use a
special custom exception class (ConfigError) to guide the user/developer
to a solution when an error/exception is generated, it is also designed
to work with the WLOG logging function to provide easy feedback to the
user.

    \begin{Verbatim}[commandchars=\\\{\}]
{\color{incolor}In [{\color{incolor}15}]:} \PY{c+c1}{\PYZsh{} try to access a undefined variable}
         \PY{n}{my\PYZus{}var} \PY{o}{=} \PY{n}{p}\PY{p}{[}\PY{l+s+s1}{\PYZsq{}}\PY{l+s+s1}{random}\PY{l+s+s1}{\PYZsq{}}\PY{p}{]}
\end{Verbatim}


    \begin{Verbatim}[commandchars=\\\{\}]

        ---------------------------------------------------------------------------

        KeyError                                  Traceback (most recent call last)

        /scratch/Projects/spirou\_py3/spirou\_py3/INTROOT/SpirouDRS/spirouConfig/spirouConfig.py in \_\_getitem\_\_(self, key)
        145         try:
    --> 146             return super(ParamDict, self).\_\_getitem\_\_(key)
        147         except KeyError:


        KeyError: 'RANDOM'

        
    During handling of the above exception, another exception occurred:


        ConfigError                               Traceback (most recent call last)

        <ipython-input-15-9d5ebfbc5eff> in <module>()
          1 \# try to access a undefined variable
    ----> 2 my\_var = p['random']
    

        /scratch/Projects/spirou\_py3/spirou\_py3/INTROOT/SpirouDRS/spirouConfig/spirouConfig.py in \_\_getitem\_\_(self, key)
        148             emsg = ('Config Error: Parameter "\{0\}" not found in parameter '
        149                     'dictionary')
    --> 150             raise ConfigError(emsg.format(oldkey), level='error')
        151 
        152     def \_\_setitem\_\_(self, key, value, source=None):


        ConfigError: level=error: Config Error: Parameter "random" not found in parameter dictionary

    \end{Verbatim}

    This can be caught and piped cleanly to the logging system as below.

When a ConfigError is raise it has two attributes: "level" and "message"

(i.e. ConfigError as e -\/-\textgreater{} e.level and e.message)

Note that the "level" of an exception is set to "error" and thus WLOG
will exit.

    \begin{Verbatim}[commandchars=\\\{\}]
{\color{incolor}In [{\color{incolor}22}]:} \PY{k+kn}{from} \PY{n+nn}{SpirouDRS.spirouConfig} \PY{k+kn}{import} \PY{n}{ConfigError}
         
         \PY{c+c1}{\PYZsh{} try to get a constant}
         \PY{k}{try}\PY{p}{:}
             \PY{n}{my\PYZus{}var} \PY{o}{=} \PY{n}{p}\PY{p}{[}\PY{l+s+s1}{\PYZsq{}}\PY{l+s+s1}{random}\PY{l+s+s1}{\PYZsq{}}\PY{p}{]}
         \PY{c+c1}{\PYZsh{} catch the ConfigError}
         \PY{k}{except} \PY{n}{ConfigError} \PY{k}{as} \PY{n}{e}\PY{p}{:}
             \PY{k}{print}\PY{p}{(}\PY{l+s+s2}{\PYZdq{}}\PY{l+s+se}{\PYZbs{}n}\PY{l+s+s2}{level = \PYZob{}0\PYZcb{}}\PY{l+s+s2}{\PYZdq{}}\PY{o}{.}\PY{n}{format}\PY{p}{(}\PY{n}{e}\PY{o}{.}\PY{n}{level}\PY{p}{)}\PY{p}{)}
             \PY{k}{print}\PY{p}{(}\PY{l+s+s2}{\PYZdq{}}\PY{l+s+se}{\PYZbs{}n}\PY{l+s+s2}{message = \PYZob{}0\PYZcb{}}\PY{l+s+s2}{\PYZdq{}}\PY{o}{.}\PY{n}{format}\PY{p}{(}\PY{n}{e}\PY{o}{.}\PY{n}{message}\PY{p}{)}\PY{p}{)}
             \PY{c+c1}{\PYZsh{} using with log}
             \PY{k}{print}\PY{p}{(}\PY{l+s+s2}{\PYZdq{}}\PY{l+s+se}{\PYZbs{}n}\PY{l+s+s2}{ Log message:}\PY{l+s+se}{\PYZbs{}n}\PY{l+s+s2}{\PYZdq{}}\PY{p}{)}
             \PY{n}{WLOG}\PY{p}{(}\PY{n}{e}\PY{o}{.}\PY{n}{level}\PY{p}{,} \PY{l+s+s1}{\PYZsq{}}\PY{l+s+s1}{program}\PY{l+s+s1}{\PYZsq{}}\PY{p}{,} \PY{n}{e}\PY{o}{.}\PY{n}{message}\PY{p}{)}
\end{Verbatim}


    \begin{Verbatim}[commandchars=\\\{\}]

level = error

message = Config Error: Parameter "random" not found in parameter dictionary

 Log message:

\textcolor{ansi-red-intense}{\textbf{12:06:54.0 - ! |program|Config Error: Parameter "random" not found in parameter dictionary}}
\textcolor{ansi-red-intense}{\textbf{

	Error found and running in DEBUG mode
}}

    \end{Verbatim}

    \begin{Verbatim}[commandchars=\\\{\}]

        An exception has occurred, use \%tb to see the full traceback.


        SystemExit: 1


    \end{Verbatim}

    \begin{Verbatim}[commandchars=\\\{\}]
/scratch/bin/anaconda3/lib/python3.6/site-packages/IPython/core/interactiveshell.py:2918: UserWarning: To exit: use 'exit', 'quit', or Ctrl-D.
  warn("To exit: use 'exit', 'quit', or Ctrl-D.", stacklevel=1)

    \end{Verbatim}


    % Add a bibliography block to the postdoc
    
    
    
    \end{document}
