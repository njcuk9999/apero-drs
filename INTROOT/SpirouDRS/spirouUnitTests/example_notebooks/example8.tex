
% Default to the notebook output style

    


% Inherit from the specified cell style.




    
\documentclass[11pt]{article}

    
    
    \usepackage[T1]{fontenc}
    % Nicer default font (+ math font) than Computer Modern for most use cases
    \usepackage{mathpazo}

    % Basic figure setup, for now with no caption control since it's done
    % automatically by Pandoc (which extracts ![](path) syntax from Markdown).
    \usepackage{graphicx}
    % We will generate all images so they have a width \maxwidth. This means
    % that they will get their normal width if they fit onto the page, but
    % are scaled down if they would overflow the margins.
    \makeatletter
    \def\maxwidth{\ifdim\Gin@nat@width>\linewidth\linewidth
    \else\Gin@nat@width\fi}
    \makeatother
    \let\Oldincludegraphics\includegraphics
    % Set max figure width to be 80% of text width, for now hardcoded.
    \renewcommand{\includegraphics}[1]{\Oldincludegraphics[width=.8\maxwidth]{#1}}
    % Ensure that by default, figures have no caption (until we provide a
    % proper Figure object with a Caption API and a way to capture that
    % in the conversion process - todo).
    \usepackage{caption}
    \DeclareCaptionLabelFormat{nolabel}{}
    \captionsetup{labelformat=nolabel}

    \usepackage{adjustbox} % Used to constrain images to a maximum size 
    \usepackage{xcolor} % Allow colors to be defined
    \usepackage{enumerate} % Needed for markdown enumerations to work
    \usepackage{geometry} % Used to adjust the document margins
    \usepackage{amsmath} % Equations
    \usepackage{amssymb} % Equations
    \usepackage{textcomp} % defines textquotesingle
    % Hack from http://tex.stackexchange.com/a/47451/13684:
    \AtBeginDocument{%
        \def\PYZsq{\textquotesingle}% Upright quotes in Pygmentized code
    }
    \usepackage{upquote} % Upright quotes for verbatim code
    \usepackage{eurosym} % defines \euro
    \usepackage[mathletters]{ucs} % Extended unicode (utf-8) support
    \usepackage[utf8x]{inputenc} % Allow utf-8 characters in the tex document
    \usepackage{fancyvrb} % verbatim replacement that allows latex
    \usepackage{grffile} % extends the file name processing of package graphics 
                         % to support a larger range 
    % The hyperref package gives us a pdf with properly built
    % internal navigation ('pdf bookmarks' for the table of contents,
    % internal cross-reference links, web links for URLs, etc.)
    \usepackage{hyperref}
    \usepackage{longtable} % longtable support required by pandoc >1.10
    \usepackage{booktabs}  % table support for pandoc > 1.12.2
    \usepackage[inline]{enumitem} % IRkernel/repr support (it uses the enumerate* environment)
    \usepackage[normalem]{ulem} % ulem is needed to support strikethroughs (\sout)
                                % normalem makes italics be italics, not underlines
    

    
    
    % Colors for the hyperref package
    \definecolor{urlcolor}{rgb}{0,.145,.698}
    \definecolor{linkcolor}{rgb}{.71,0.21,0.01}
    \definecolor{citecolor}{rgb}{.12,.54,.11}

    % ANSI colors
    \definecolor{ansi-black}{HTML}{3E424D}
    \definecolor{ansi-black-intense}{HTML}{282C36}
    \definecolor{ansi-red}{HTML}{E75C58}
    \definecolor{ansi-red-intense}{HTML}{B22B31}
    \definecolor{ansi-green}{HTML}{00A250}
    \definecolor{ansi-green-intense}{HTML}{007427}
    \definecolor{ansi-yellow}{HTML}{DDB62B}
    \definecolor{ansi-yellow-intense}{HTML}{B27D12}
    \definecolor{ansi-blue}{HTML}{208FFB}
    \definecolor{ansi-blue-intense}{HTML}{0065CA}
    \definecolor{ansi-magenta}{HTML}{D160C4}
    \definecolor{ansi-magenta-intense}{HTML}{A03196}
    \definecolor{ansi-cyan}{HTML}{60C6C8}
    \definecolor{ansi-cyan-intense}{HTML}{258F8F}
    \definecolor{ansi-white}{HTML}{C5C1B4}
    \definecolor{ansi-white-intense}{HTML}{A1A6B2}

    % commands and environments needed by pandoc snippets
    % extracted from the output of `pandoc -s`
    \providecommand{\tightlist}{%
      \setlength{\itemsep}{0pt}\setlength{\parskip}{0pt}}
    \DefineVerbatimEnvironment{Highlighting}{Verbatim}{commandchars=\\\{\}}
    % Add ',fontsize=\small' for more characters per line
    \newenvironment{Shaded}{}{}
    \newcommand{\KeywordTok}[1]{\textcolor[rgb]{0.00,0.44,0.13}{\textbf{{#1}}}}
    \newcommand{\DataTypeTok}[1]{\textcolor[rgb]{0.56,0.13,0.00}{{#1}}}
    \newcommand{\DecValTok}[1]{\textcolor[rgb]{0.25,0.63,0.44}{{#1}}}
    \newcommand{\BaseNTok}[1]{\textcolor[rgb]{0.25,0.63,0.44}{{#1}}}
    \newcommand{\FloatTok}[1]{\textcolor[rgb]{0.25,0.63,0.44}{{#1}}}
    \newcommand{\CharTok}[1]{\textcolor[rgb]{0.25,0.44,0.63}{{#1}}}
    \newcommand{\StringTok}[1]{\textcolor[rgb]{0.25,0.44,0.63}{{#1}}}
    \newcommand{\CommentTok}[1]{\textcolor[rgb]{0.38,0.63,0.69}{\textit{{#1}}}}
    \newcommand{\OtherTok}[1]{\textcolor[rgb]{0.00,0.44,0.13}{{#1}}}
    \newcommand{\AlertTok}[1]{\textcolor[rgb]{1.00,0.00,0.00}{\textbf{{#1}}}}
    \newcommand{\FunctionTok}[1]{\textcolor[rgb]{0.02,0.16,0.49}{{#1}}}
    \newcommand{\RegionMarkerTok}[1]{{#1}}
    \newcommand{\ErrorTok}[1]{\textcolor[rgb]{1.00,0.00,0.00}{\textbf{{#1}}}}
    \newcommand{\NormalTok}[1]{{#1}}
    
    % Additional commands for more recent versions of Pandoc
    \newcommand{\ConstantTok}[1]{\textcolor[rgb]{0.53,0.00,0.00}{{#1}}}
    \newcommand{\SpecialCharTok}[1]{\textcolor[rgb]{0.25,0.44,0.63}{{#1}}}
    \newcommand{\VerbatimStringTok}[1]{\textcolor[rgb]{0.25,0.44,0.63}{{#1}}}
    \newcommand{\SpecialStringTok}[1]{\textcolor[rgb]{0.73,0.40,0.53}{{#1}}}
    \newcommand{\ImportTok}[1]{{#1}}
    \newcommand{\DocumentationTok}[1]{\textcolor[rgb]{0.73,0.13,0.13}{\textit{{#1}}}}
    \newcommand{\AnnotationTok}[1]{\textcolor[rgb]{0.38,0.63,0.69}{\textbf{\textit{{#1}}}}}
    \newcommand{\CommentVarTok}[1]{\textcolor[rgb]{0.38,0.63,0.69}{\textbf{\textit{{#1}}}}}
    \newcommand{\VariableTok}[1]{\textcolor[rgb]{0.10,0.09,0.49}{{#1}}}
    \newcommand{\ControlFlowTok}[1]{\textcolor[rgb]{0.00,0.44,0.13}{\textbf{{#1}}}}
    \newcommand{\OperatorTok}[1]{\textcolor[rgb]{0.40,0.40,0.40}{{#1}}}
    \newcommand{\BuiltInTok}[1]{{#1}}
    \newcommand{\ExtensionTok}[1]{{#1}}
    \newcommand{\PreprocessorTok}[1]{\textcolor[rgb]{0.74,0.48,0.00}{{#1}}}
    \newcommand{\AttributeTok}[1]{\textcolor[rgb]{0.49,0.56,0.16}{{#1}}}
    \newcommand{\InformationTok}[1]{\textcolor[rgb]{0.38,0.63,0.69}{\textbf{\textit{{#1}}}}}
    \newcommand{\WarningTok}[1]{\textcolor[rgb]{0.38,0.63,0.69}{\textbf{\textit{{#1}}}}}
    
    
    % Define a nice break command that doesn't care if a line doesn't already
    % exist.
    \def\br{\hspace*{\fill} \\* }
    % Math Jax compatability definitions
    \def\gt{>}
    \def\lt{<}
    % Document parameters
    \title{example8}
    
    
    

    % Pygments definitions
    
\makeatletter
\def\PY@reset{\let\PY@it=\relax \let\PY@bf=\relax%
    \let\PY@ul=\relax \let\PY@tc=\relax%
    \let\PY@bc=\relax \let\PY@ff=\relax}
\def\PY@tok#1{\csname PY@tok@#1\endcsname}
\def\PY@toks#1+{\ifx\relax#1\empty\else%
    \PY@tok{#1}\expandafter\PY@toks\fi}
\def\PY@do#1{\PY@bc{\PY@tc{\PY@ul{%
    \PY@it{\PY@bf{\PY@ff{#1}}}}}}}
\def\PY#1#2{\PY@reset\PY@toks#1+\relax+\PY@do{#2}}

\expandafter\def\csname PY@tok@w\endcsname{\def\PY@tc##1{\textcolor[rgb]{0.73,0.73,0.73}{##1}}}
\expandafter\def\csname PY@tok@c\endcsname{\let\PY@it=\textit\def\PY@tc##1{\textcolor[rgb]{0.25,0.50,0.50}{##1}}}
\expandafter\def\csname PY@tok@cp\endcsname{\def\PY@tc##1{\textcolor[rgb]{0.74,0.48,0.00}{##1}}}
\expandafter\def\csname PY@tok@k\endcsname{\let\PY@bf=\textbf\def\PY@tc##1{\textcolor[rgb]{0.00,0.50,0.00}{##1}}}
\expandafter\def\csname PY@tok@kp\endcsname{\def\PY@tc##1{\textcolor[rgb]{0.00,0.50,0.00}{##1}}}
\expandafter\def\csname PY@tok@kt\endcsname{\def\PY@tc##1{\textcolor[rgb]{0.69,0.00,0.25}{##1}}}
\expandafter\def\csname PY@tok@o\endcsname{\def\PY@tc##1{\textcolor[rgb]{0.40,0.40,0.40}{##1}}}
\expandafter\def\csname PY@tok@ow\endcsname{\let\PY@bf=\textbf\def\PY@tc##1{\textcolor[rgb]{0.67,0.13,1.00}{##1}}}
\expandafter\def\csname PY@tok@nb\endcsname{\def\PY@tc##1{\textcolor[rgb]{0.00,0.50,0.00}{##1}}}
\expandafter\def\csname PY@tok@nf\endcsname{\def\PY@tc##1{\textcolor[rgb]{0.00,0.00,1.00}{##1}}}
\expandafter\def\csname PY@tok@nc\endcsname{\let\PY@bf=\textbf\def\PY@tc##1{\textcolor[rgb]{0.00,0.00,1.00}{##1}}}
\expandafter\def\csname PY@tok@nn\endcsname{\let\PY@bf=\textbf\def\PY@tc##1{\textcolor[rgb]{0.00,0.00,1.00}{##1}}}
\expandafter\def\csname PY@tok@ne\endcsname{\let\PY@bf=\textbf\def\PY@tc##1{\textcolor[rgb]{0.82,0.25,0.23}{##1}}}
\expandafter\def\csname PY@tok@nv\endcsname{\def\PY@tc##1{\textcolor[rgb]{0.10,0.09,0.49}{##1}}}
\expandafter\def\csname PY@tok@no\endcsname{\def\PY@tc##1{\textcolor[rgb]{0.53,0.00,0.00}{##1}}}
\expandafter\def\csname PY@tok@nl\endcsname{\def\PY@tc##1{\textcolor[rgb]{0.63,0.63,0.00}{##1}}}
\expandafter\def\csname PY@tok@ni\endcsname{\let\PY@bf=\textbf\def\PY@tc##1{\textcolor[rgb]{0.60,0.60,0.60}{##1}}}
\expandafter\def\csname PY@tok@na\endcsname{\def\PY@tc##1{\textcolor[rgb]{0.49,0.56,0.16}{##1}}}
\expandafter\def\csname PY@tok@nt\endcsname{\let\PY@bf=\textbf\def\PY@tc##1{\textcolor[rgb]{0.00,0.50,0.00}{##1}}}
\expandafter\def\csname PY@tok@nd\endcsname{\def\PY@tc##1{\textcolor[rgb]{0.67,0.13,1.00}{##1}}}
\expandafter\def\csname PY@tok@s\endcsname{\def\PY@tc##1{\textcolor[rgb]{0.73,0.13,0.13}{##1}}}
\expandafter\def\csname PY@tok@sd\endcsname{\let\PY@it=\textit\def\PY@tc##1{\textcolor[rgb]{0.73,0.13,0.13}{##1}}}
\expandafter\def\csname PY@tok@si\endcsname{\let\PY@bf=\textbf\def\PY@tc##1{\textcolor[rgb]{0.73,0.40,0.53}{##1}}}
\expandafter\def\csname PY@tok@se\endcsname{\let\PY@bf=\textbf\def\PY@tc##1{\textcolor[rgb]{0.73,0.40,0.13}{##1}}}
\expandafter\def\csname PY@tok@sr\endcsname{\def\PY@tc##1{\textcolor[rgb]{0.73,0.40,0.53}{##1}}}
\expandafter\def\csname PY@tok@ss\endcsname{\def\PY@tc##1{\textcolor[rgb]{0.10,0.09,0.49}{##1}}}
\expandafter\def\csname PY@tok@sx\endcsname{\def\PY@tc##1{\textcolor[rgb]{0.00,0.50,0.00}{##1}}}
\expandafter\def\csname PY@tok@m\endcsname{\def\PY@tc##1{\textcolor[rgb]{0.40,0.40,0.40}{##1}}}
\expandafter\def\csname PY@tok@gh\endcsname{\let\PY@bf=\textbf\def\PY@tc##1{\textcolor[rgb]{0.00,0.00,0.50}{##1}}}
\expandafter\def\csname PY@tok@gu\endcsname{\let\PY@bf=\textbf\def\PY@tc##1{\textcolor[rgb]{0.50,0.00,0.50}{##1}}}
\expandafter\def\csname PY@tok@gd\endcsname{\def\PY@tc##1{\textcolor[rgb]{0.63,0.00,0.00}{##1}}}
\expandafter\def\csname PY@tok@gi\endcsname{\def\PY@tc##1{\textcolor[rgb]{0.00,0.63,0.00}{##1}}}
\expandafter\def\csname PY@tok@gr\endcsname{\def\PY@tc##1{\textcolor[rgb]{1.00,0.00,0.00}{##1}}}
\expandafter\def\csname PY@tok@ge\endcsname{\let\PY@it=\textit}
\expandafter\def\csname PY@tok@gs\endcsname{\let\PY@bf=\textbf}
\expandafter\def\csname PY@tok@gp\endcsname{\let\PY@bf=\textbf\def\PY@tc##1{\textcolor[rgb]{0.00,0.00,0.50}{##1}}}
\expandafter\def\csname PY@tok@go\endcsname{\def\PY@tc##1{\textcolor[rgb]{0.53,0.53,0.53}{##1}}}
\expandafter\def\csname PY@tok@gt\endcsname{\def\PY@tc##1{\textcolor[rgb]{0.00,0.27,0.87}{##1}}}
\expandafter\def\csname PY@tok@err\endcsname{\def\PY@bc##1{\setlength{\fboxsep}{0pt}\fcolorbox[rgb]{1.00,0.00,0.00}{1,1,1}{\strut ##1}}}
\expandafter\def\csname PY@tok@kc\endcsname{\let\PY@bf=\textbf\def\PY@tc##1{\textcolor[rgb]{0.00,0.50,0.00}{##1}}}
\expandafter\def\csname PY@tok@kd\endcsname{\let\PY@bf=\textbf\def\PY@tc##1{\textcolor[rgb]{0.00,0.50,0.00}{##1}}}
\expandafter\def\csname PY@tok@kn\endcsname{\let\PY@bf=\textbf\def\PY@tc##1{\textcolor[rgb]{0.00,0.50,0.00}{##1}}}
\expandafter\def\csname PY@tok@kr\endcsname{\let\PY@bf=\textbf\def\PY@tc##1{\textcolor[rgb]{0.00,0.50,0.00}{##1}}}
\expandafter\def\csname PY@tok@bp\endcsname{\def\PY@tc##1{\textcolor[rgb]{0.00,0.50,0.00}{##1}}}
\expandafter\def\csname PY@tok@fm\endcsname{\def\PY@tc##1{\textcolor[rgb]{0.00,0.00,1.00}{##1}}}
\expandafter\def\csname PY@tok@vc\endcsname{\def\PY@tc##1{\textcolor[rgb]{0.10,0.09,0.49}{##1}}}
\expandafter\def\csname PY@tok@vg\endcsname{\def\PY@tc##1{\textcolor[rgb]{0.10,0.09,0.49}{##1}}}
\expandafter\def\csname PY@tok@vi\endcsname{\def\PY@tc##1{\textcolor[rgb]{0.10,0.09,0.49}{##1}}}
\expandafter\def\csname PY@tok@vm\endcsname{\def\PY@tc##1{\textcolor[rgb]{0.10,0.09,0.49}{##1}}}
\expandafter\def\csname PY@tok@sa\endcsname{\def\PY@tc##1{\textcolor[rgb]{0.73,0.13,0.13}{##1}}}
\expandafter\def\csname PY@tok@sb\endcsname{\def\PY@tc##1{\textcolor[rgb]{0.73,0.13,0.13}{##1}}}
\expandafter\def\csname PY@tok@sc\endcsname{\def\PY@tc##1{\textcolor[rgb]{0.73,0.13,0.13}{##1}}}
\expandafter\def\csname PY@tok@dl\endcsname{\def\PY@tc##1{\textcolor[rgb]{0.73,0.13,0.13}{##1}}}
\expandafter\def\csname PY@tok@s2\endcsname{\def\PY@tc##1{\textcolor[rgb]{0.73,0.13,0.13}{##1}}}
\expandafter\def\csname PY@tok@sh\endcsname{\def\PY@tc##1{\textcolor[rgb]{0.73,0.13,0.13}{##1}}}
\expandafter\def\csname PY@tok@s1\endcsname{\def\PY@tc##1{\textcolor[rgb]{0.73,0.13,0.13}{##1}}}
\expandafter\def\csname PY@tok@mb\endcsname{\def\PY@tc##1{\textcolor[rgb]{0.40,0.40,0.40}{##1}}}
\expandafter\def\csname PY@tok@mf\endcsname{\def\PY@tc##1{\textcolor[rgb]{0.40,0.40,0.40}{##1}}}
\expandafter\def\csname PY@tok@mh\endcsname{\def\PY@tc##1{\textcolor[rgb]{0.40,0.40,0.40}{##1}}}
\expandafter\def\csname PY@tok@mi\endcsname{\def\PY@tc##1{\textcolor[rgb]{0.40,0.40,0.40}{##1}}}
\expandafter\def\csname PY@tok@il\endcsname{\def\PY@tc##1{\textcolor[rgb]{0.40,0.40,0.40}{##1}}}
\expandafter\def\csname PY@tok@mo\endcsname{\def\PY@tc##1{\textcolor[rgb]{0.40,0.40,0.40}{##1}}}
\expandafter\def\csname PY@tok@ch\endcsname{\let\PY@it=\textit\def\PY@tc##1{\textcolor[rgb]{0.25,0.50,0.50}{##1}}}
\expandafter\def\csname PY@tok@cm\endcsname{\let\PY@it=\textit\def\PY@tc##1{\textcolor[rgb]{0.25,0.50,0.50}{##1}}}
\expandafter\def\csname PY@tok@cpf\endcsname{\let\PY@it=\textit\def\PY@tc##1{\textcolor[rgb]{0.25,0.50,0.50}{##1}}}
\expandafter\def\csname PY@tok@c1\endcsname{\let\PY@it=\textit\def\PY@tc##1{\textcolor[rgb]{0.25,0.50,0.50}{##1}}}
\expandafter\def\csname PY@tok@cs\endcsname{\let\PY@it=\textit\def\PY@tc##1{\textcolor[rgb]{0.25,0.50,0.50}{##1}}}

\def\PYZbs{\char`\\}
\def\PYZus{\char`\_}
\def\PYZob{\char`\{}
\def\PYZcb{\char`\}}
\def\PYZca{\char`\^}
\def\PYZam{\char`\&}
\def\PYZlt{\char`\<}
\def\PYZgt{\char`\>}
\def\PYZsh{\char`\#}
\def\PYZpc{\char`\%}
\def\PYZdl{\char`\$}
\def\PYZhy{\char`\-}
\def\PYZsq{\char`\'}
\def\PYZdq{\char`\"}
\def\PYZti{\char`\~}
% for compatibility with earlier versions
\def\PYZat{@}
\def\PYZlb{[}
\def\PYZrb{]}
\makeatother


    % Exact colors from NB
    \definecolor{incolor}{rgb}{0.0, 0.0, 0.5}
    \definecolor{outcolor}{rgb}{0.545, 0.0, 0.0}



    
    % Prevent overflowing lines due to hard-to-break entities
    \sloppy 
    % Setup hyperref package
    \hypersetup{
      breaklinks=true,  % so long urls are correctly broken across lines
      colorlinks=true,
      urlcolor=urlcolor,
      linkcolor=linkcolor,
      citecolor=citecolor,
      }
    % Slightly bigger margins than the latex defaults
    
    \geometry{verbose,tmargin=1in,bmargin=1in,lmargin=1in,rmargin=1in}
    
    

    \begin{document}
    
    
    \maketitle
    
    

    
    \section{Using the logging (WLOG)}\label{using-the-logging-wlog}

Printing and logging is controlled via the \texttt{WLOG} function
(\texttt{SpirouDRS.spirouCore.spirouLog.logger} aliased to
\texttt{WLOG})

The first thing that is needed is the import and alias to the logger
function (\texttt{WLOG}):

    \begin{Verbatim}[commandchars=\\\{\}]
{\color{incolor}In [{\color{incolor}3}]:} \PY{k+kn}{from} \PY{n+nn}{SpirouDRS} \PY{k+kn}{import} \PY{n}{spirouCore}
        
        \PY{c+c1}{\PYZsh{} Get Logging function}
        \PY{n}{WLOG} \PY{o}{=} \PY{n}{spirouCore}\PY{o}{.}\PY{n}{wlog}
\end{Verbatim}


    Then we can use the logger as follows:

\texttt{WLOG(level,\ program,\ message)}

and produces the following entry:

\begin{verbatim}
`HH:MM:SS.s - char | program | message`
\end{verbatim}

where:

\begin{itemize}
\item
  level: (string) is the type of log message we are producing by default
  it can be one of the following "all", "info", 'warning", "error" or ""
  (a blank string). It also sets the "char" in the printed entry. This
  chooses the colour of the text
\item
  program: (string) is the "program" message that is printed/logged
  (i.e. the recipe that is being run or any other custom string that is
  required)
\item
  message: (string) this is the message part of the print statement/log
  statement.
\end{itemize}

Below we show each of these in use:

    \begin{Verbatim}[commandchars=\\\{\}]
{\color{incolor}In [{\color{incolor}5}]:} \PY{c+c1}{\PYZsh{} set up a program name}
        \PY{n}{program} \PY{o}{=} \PY{l+s+s2}{\PYZdq{}}\PY{l+s+s2}{test program}\PY{l+s+s2}{\PYZdq{}}
\end{Verbatim}


    \subsubsection{General/info message:}\label{generalinfo-message}

    \begin{Verbatim}[commandchars=\\\{\}]
{\color{incolor}In [{\color{incolor}7}]:} \PY{n}{WLOG}\PY{p}{(}\PY{l+s+s1}{\PYZsq{}}\PY{l+s+s1}{\PYZsq{}}\PY{p}{,} \PY{n}{program}\PY{p}{,} \PY{l+s+s1}{\PYZsq{}}\PY{l+s+s1}{This is a general message}\PY{l+s+s1}{\PYZsq{}}\PY{p}{)}
        \PY{n}{WLOG}\PY{p}{(}\PY{l+s+s1}{\PYZsq{}}\PY{l+s+s1}{all}\PY{l+s+s1}{\PYZsq{}}\PY{p}{,} \PY{n}{program}\PY{p}{,} \PY{l+s+s1}{\PYZsq{}}\PY{l+s+s1}{This is a general message}\PY{l+s+s1}{\PYZsq{}}\PY{p}{)}
        \PY{n}{WLOG}\PY{p}{(}\PY{l+s+s1}{\PYZsq{}}\PY{l+s+s1}{info}\PY{l+s+s1}{\PYZsq{}}\PY{p}{,} \PY{n}{program}\PY{p}{,} \PY{l+s+s1}{\PYZsq{}}\PY{l+s+s1}{This is an info message}\PY{l+s+s1}{\PYZsq{}}\PY{p}{)}
\end{Verbatim}


    \begin{Verbatim}[commandchars=\\\{\}]
\textcolor{ansi-green-intense}{\textbf{11:22:16.0 -   |test program|This is a general message}}
\textcolor{ansi-green-intense}{\textbf{11:22:16.0 -   |test program|This is a general message}}
\textcolor{ansi-green-intense}{\textbf{11:22:16.0 - * |test program|This is an info message}}

    \end{Verbatim}

    \subsubsection{Warning message:}\label{warning-message}

    \begin{Verbatim}[commandchars=\\\{\}]
{\color{incolor}In [{\color{incolor}9}]:} \PY{n}{WLOG}\PY{p}{(}\PY{l+s+s1}{\PYZsq{}}\PY{l+s+s1}{warning}\PY{l+s+s1}{\PYZsq{}}\PY{p}{,} \PY{n}{program}\PY{p}{,} \PY{l+s+s1}{\PYZsq{}}\PY{l+s+s1}{This is a warning messagge}\PY{l+s+s1}{\PYZsq{}}\PY{p}{)}
\end{Verbatim}


    \begin{Verbatim}[commandchars=\\\{\}]
\textcolor{ansi-yellow-intense}{\textbf{11:23:07.0 - @ |test program|This is a warning messagge}}

    \end{Verbatim}

    \subsubsection{Error message:}\label{error-message}

An error message (by default) has an added feature. An error will
automatically stop the recipe. Therefore using a error logging should
only be used when the recipes cannot continue afterwards.

In the DRS (in none interactive mode) this exit will be silent. If
\texttt{DEBUG=1} the error will be handled and the user will be able to
debug from the point of exit (unlike uncatch python errors).

    \begin{Verbatim}[commandchars=\\\{\}]
{\color{incolor}In [{\color{incolor}12}]:} \PY{n}{WLOG}\PY{p}{(}\PY{l+s+s1}{\PYZsq{}}\PY{l+s+s1}{error}\PY{l+s+s1}{\PYZsq{}}\PY{p}{,} \PY{n}{program}\PY{p}{,} \PY{l+s+s1}{\PYZsq{}}\PY{l+s+s1}{This is an error message}\PY{l+s+s1}{\PYZsq{}}\PY{p}{)}
\end{Verbatim}


    \begin{Verbatim}[commandchars=\\\{\}]
\textcolor{ansi-red-intense}{\textbf{11:28:45.0 - ! |test program|This is an error message}}

    \end{Verbatim}

    \begin{Verbatim}[commandchars=\\\{\}]

        An exception has occurred, use \%tb to see the full traceback.


        SystemExit: 1


    \end{Verbatim}

    \subsection{Customising the logger}\label{customising-the-logger}

There are several ways these default settings can be changed/modified.
They are: - changing the levels - changing the levels that exit python -
changing which levels write to the standard output (the screen) -
changing which levels write to the log file - changing the colours of
the levels (when coloured text is enabled)

These are shown below and are changable in
\texttt{SpirouDRS.spirouConfig.spirouConst.py}

    \begin{Verbatim}[commandchars=\\\{\}]
{\color{incolor}In [{\color{incolor}14}]:} \PY{c+c1}{\PYZsh{} noinspection PyPep8Naming}
         \PY{k}{def} \PY{n+nf}{LOG\PYZus{}TRIG\PYZus{}KEYS}\PY{p}{(}\PY{p}{)}\PY{p}{:}
             \PY{l+s+sd}{\PYZdq{}\PYZdq{}\PYZdq{}}
         \PY{l+s+sd}{    The log trigger key characters to use in log. Keys must be the same as}
         \PY{l+s+sd}{    spirouConst.WRITE\PYZus{}LEVELS()}
         
         \PY{l+s+sd}{    i.e.}
         
         \PY{l+s+sd}{    if the following is defined:}
         \PY{l+s+sd}{    \PYZgt{}\PYZgt{} trig\PYZus{}key[error] = \PYZsq{}!\PYZsq{}}
         \PY{l+s+sd}{    and the following log is used:}
         \PY{l+s+sd}{    \PYZgt{}\PYZgt{} WLOG(\PYZsq{}error\PYZsq{}, \PYZsq{}program\PYZsq{}, \PYZsq{}message\PYZsq{})}
         \PY{l+s+sd}{    the output is:}
         \PY{l+s+sd}{    \PYZgt{}\PYZgt{} print(\PYZdq{}TIMESTAMP \PYZhy{} ! |program|message\PYZdq{})}
         
         \PY{l+s+sd}{    :return trig\PYZus{}key: dictionary, contains all the trigger keys and the}
         \PY{l+s+sd}{                      characters/strings to use in logging. Keys must be the}
         \PY{l+s+sd}{                      same as spirouConst.WRITE\PYZus{}LEVELS()}
         \PY{l+s+sd}{    \PYZdq{}\PYZdq{}\PYZdq{}}
             \PY{c+c1}{\PYZsh{} The trigger character to display for each}
             \PY{n}{trig\PYZus{}key} \PY{o}{=} \PY{n+nb}{dict}\PY{p}{(}\PY{n+nb}{all}\PY{o}{=}\PY{l+s+s1}{\PYZsq{}}\PY{l+s+s1}{ }\PY{l+s+s1}{\PYZsq{}}\PY{p}{,} \PY{n}{error}\PY{o}{=}\PY{l+s+s1}{\PYZsq{}}\PY{l+s+s1}{!}\PY{l+s+s1}{\PYZsq{}}\PY{p}{,} \PY{n}{warning}\PY{o}{=}\PY{l+s+s1}{\PYZsq{}}\PY{l+s+s1}{@}\PY{l+s+s1}{\PYZsq{}}\PY{p}{,} \PY{n}{info}\PY{o}{=}\PY{l+s+s1}{\PYZsq{}}\PY{l+s+s1}{*}\PY{l+s+s1}{\PYZsq{}}\PY{p}{,} \PY{n}{graph}\PY{o}{=}\PY{l+s+s1}{\PYZsq{}}\PY{l+s+s1}{\PYZti{}}\PY{l+s+s1}{\PYZsq{}}\PY{p}{)}
             \PY{k}{return} \PY{n}{trig\PYZus{}key}
         
         
         \PY{c+c1}{\PYZsh{} noinspection PyPep8Naming}
         \PY{k}{def} \PY{n+nf}{WRITE\PYZus{}LEVEL}\PY{p}{(}\PY{p}{)}\PY{p}{:}
             \PY{l+s+sd}{\PYZdq{}\PYZdq{}\PYZdq{}}
         \PY{l+s+sd}{    The write levels. Keys must be the same as spirouConst.LOG\PYZus{}TRIG\PYZus{}KEYS()}
         
         \PY{l+s+sd}{    The write levels define which levels are logged and printed (based on}
         \PY{l+s+sd}{    constants \PYZdq{}PRINT\PYZus{}LEVEL\PYZdq{} and \PYZdq{}LOG\PYZus{}LEVEL\PYZdq{} in the primary config file}
         
         \PY{l+s+sd}{    i.e. if}
         \PY{l+s+sd}{    \PYZgt{}\PYZgt{} PRINT\PYZus{}LEVEL = \PYZsq{}warning\PYZsq{}}
         \PY{l+s+sd}{    then no level with a numerical value less than}
         \PY{l+s+sd}{    \PYZgt{}\PYZgt{} write\PYZus{}level[\PYZsq{}warning\PYZsq{}]}
         \PY{l+s+sd}{    will be printed to the screen}
         
         \PY{l+s+sd}{    similarly if}
         \PY{l+s+sd}{    \PYZgt{}\PYZgt{} LOG\PYZus{}LEVEL = \PYZsq{}error\PYZsq{}}
         \PY{l+s+sd}{    then no level with a numerical value less than}
         \PY{l+s+sd}{    \PYZgt{}\PYZgt{} write\PYZus{}level[\PYZsq{}error\PYZsq{}]}
         \PY{l+s+sd}{    will be printed to the log file}
         
         \PY{l+s+sd}{    :return write\PYZus{}level: dictionary, contains the keys and numerical levels}
         \PY{l+s+sd}{                         of each trigger level. Keys must be the same as}
         \PY{l+s+sd}{                         spirouConst.LOG\PYZus{}TRIG\PYZus{}KEYS()}
         \PY{l+s+sd}{    \PYZdq{}\PYZdq{}\PYZdq{}}
             \PY{n}{write\PYZus{}level} \PY{o}{=} \PY{n+nb}{dict}\PY{p}{(}\PY{n}{error}\PY{o}{=}\PY{l+m+mi}{3}\PY{p}{,} \PY{n}{warning}\PY{o}{=}\PY{l+m+mi}{2}\PY{p}{,} \PY{n}{info}\PY{o}{=}\PY{l+m+mi}{1}\PY{p}{,} \PY{n}{graph}\PY{o}{=}\PY{l+m+mi}{0}\PY{p}{,} \PY{n+nb}{all}\PY{o}{=}\PY{l+m+mi}{0}\PY{p}{)}
             \PY{k}{return} \PY{n}{write\PYZus{}level}
         
         
         \PY{c+c1}{\PYZsh{} noinspection PyPep8Naming}
         \PY{k}{def} \PY{n+nf}{LOG\PYZus{}CAUGHT\PYZus{}WARNINGS}\PY{p}{(}\PY{p}{)}\PY{p}{:}
             \PY{l+s+sd}{\PYZdq{}\PYZdq{}\PYZdq{}}
         \PY{l+s+sd}{    Defines a master switch, whether to report warnings that are caught in}
         
         \PY{l+s+sd}{    \PYZgt{}\PYZgt{} with warnings.catch\PYZus{}warnings(record=True) as w:}
         \PY{l+s+sd}{    \PYZgt{}\PYZgt{}     code\PYZus{}that\PYZus{}may\PYZus{}gen\PYZus{}warnings}
         
         \PY{l+s+sd}{    :return warn: bool, if True reports warnings, if False does not}
         \PY{l+s+sd}{    \PYZdq{}\PYZdq{}\PYZdq{}}
             \PY{c+c1}{\PYZsh{} Define whether we warn}
             \PY{n}{warn} \PY{o}{=} \PY{n+nb+bp}{True}
             \PY{k}{return} \PY{n}{warn}
         
         
         \PY{c+c1}{\PYZsh{} noinspection PyPep8Naming}
         \PY{k}{def} \PY{n+nf}{COLOUREDLEVELS}\PY{p}{(}\PY{p}{)}\PY{p}{:}
             \PY{l+s+sd}{\PYZdq{}\PYZdq{}\PYZdq{}}
         \PY{l+s+sd}{    Defines the colours if using coloured log.}
         \PY{l+s+sd}{    Allowed colour strings are found here:}
         \PY{l+s+sd}{            see here:}
         \PY{l+s+sd}{            http://ozzmaker.com/add\PYZhy{}colour\PYZhy{}to\PYZhy{}text\PYZhy{}in\PYZhy{}python/}
         \PY{l+s+sd}{            or in spirouConst.bcolors (colour class):}
         \PY{l+s+sd}{                HEADER, OKBLUE, OKGREEN, WARNING, FAIL,}
         \PY{l+s+sd}{                BOLD, UNDERLINE}
         
         \PY{l+s+sd}{    :return clevels: dictionary, containing all the keys identical to}
         \PY{l+s+sd}{                     LOG\PYZus{}TRIG\PYZus{}KEYS or WRITE\PYZus{}LEVEL, values must be strings}
         \PY{l+s+sd}{                     that prodive colour information to python print statement}
         \PY{l+s+sd}{                     see here:}
         \PY{l+s+sd}{                         http://ozzmaker.com/add\PYZhy{}colour\PYZhy{}to\PYZhy{}text\PYZhy{}in\PYZhy{}python/}
         \PY{l+s+sd}{    \PYZdq{}\PYZdq{}\PYZdq{}}
             \PY{c+c1}{\PYZsh{} reference:}
             \PY{c+c1}{\PYZsh{} http://ozzmaker.com/add\PYZhy{}colour\PYZhy{}to\PYZhy{}text\PYZhy{}in\PYZhy{}python/}
             \PY{n}{clevels} \PY{o}{=} \PY{n+nb}{dict}\PY{p}{(}\PY{n}{error}\PY{o}{=}\PY{n}{BColors}\PY{o}{.}\PY{n}{FAIL}\PY{p}{,}  \PY{c+c1}{\PYZsh{} red}
                            \PY{n}{warning}\PY{o}{=}\PY{n}{BColors}\PY{o}{.}\PY{n}{WARNING}\PY{p}{,}  \PY{c+c1}{\PYZsh{} yellow}
                            \PY{n}{info}\PY{o}{=}\PY{n}{BColors}\PY{o}{.}\PY{n}{OKGREEN}\PY{p}{,}  \PY{c+c1}{\PYZsh{} green}
                            \PY{n}{graph}\PY{o}{=}\PY{n}{BColors}\PY{o}{.}\PY{n}{OKBLUE}\PY{p}{,}  \PY{c+c1}{\PYZsh{} green}
                            \PY{n+nb}{all}\PY{o}{=}\PY{n}{BColors}\PY{o}{.}\PY{n}{OKGREEN}\PY{p}{)}  \PY{c+c1}{\PYZsh{} green}
             \PY{k}{return} \PY{n}{clevels}
         
         
         \PY{c+c1}{\PYZsh{} defines the colours}
         \PY{k}{class} \PY{n+nc}{BColors}\PY{p}{:}
             \PY{n}{HEADER} \PY{o}{=} \PY{l+s+s1}{\PYZsq{}}\PY{l+s+se}{\PYZbs{}033}\PY{l+s+s1}{[95;1m}\PY{l+s+s1}{\PYZsq{}}
             \PY{n}{OKBLUE} \PY{o}{=} \PY{l+s+s1}{\PYZsq{}}\PY{l+s+se}{\PYZbs{}033}\PY{l+s+s1}{[94;1m}\PY{l+s+s1}{\PYZsq{}}
             \PY{n}{OKGREEN} \PY{o}{=} \PY{l+s+s1}{\PYZsq{}}\PY{l+s+se}{\PYZbs{}033}\PY{l+s+s1}{[92;1m}\PY{l+s+s1}{\PYZsq{}}
             \PY{n}{WARNING} \PY{o}{=} \PY{l+s+s1}{\PYZsq{}}\PY{l+s+se}{\PYZbs{}033}\PY{l+s+s1}{[93;1m}\PY{l+s+s1}{\PYZsq{}}
             \PY{n}{FAIL} \PY{o}{=} \PY{l+s+s1}{\PYZsq{}}\PY{l+s+se}{\PYZbs{}033}\PY{l+s+s1}{[91;1m}\PY{l+s+s1}{\PYZsq{}}
             \PY{n}{ENDC} \PY{o}{=} \PY{l+s+s1}{\PYZsq{}}\PY{l+s+se}{\PYZbs{}033}\PY{l+s+s1}{[0;0m}\PY{l+s+s1}{\PYZsq{}}
             \PY{n}{BOLD} \PY{o}{=} \PY{l+s+s1}{\PYZsq{}}\PY{l+s+se}{\PYZbs{}033}\PY{l+s+s1}{[1m}\PY{l+s+s1}{\PYZsq{}}
             \PY{n}{UNDERLINE} \PY{o}{=} \PY{l+s+s1}{\PYZsq{}}\PY{l+s+se}{\PYZbs{}033}\PY{l+s+s1}{[4m}\PY{l+s+s1}{\PYZsq{}}
\end{Verbatim}



    % Add a bibliography block to the postdoc
    
    
    
    \end{document}
