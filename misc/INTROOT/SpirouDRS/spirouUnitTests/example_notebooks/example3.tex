
% Default to the notebook output style

    


% Inherit from the specified cell style.




    
\documentclass[11pt]{article}

    
    
    \usepackage[T1]{fontenc}
    % Nicer default font (+ math font) than Computer Modern for most use cases
    \usepackage{mathpazo}

    % Basic figure setup, for now with no caption control since it's done
    % automatically by Pandoc (which extracts ![](path) syntax from Markdown).
    \usepackage{graphicx}
    % We will generate all images so they have a width \maxwidth. This means
    % that they will get their normal width if they fit onto the page, but
    % are scaled down if they would overflow the margins.
    \makeatletter
    \def\maxwidth{\ifdim\Gin@nat@width>\linewidth\linewidth
    \else\Gin@nat@width\fi}
    \makeatother
    \let\Oldincludegraphics\includegraphics
    % Set max figure width to be 80% of text width, for now hardcoded.
    \renewcommand{\includegraphics}[1]{\Oldincludegraphics[width=.8\maxwidth]{#1}}
    % Ensure that by default, figures have no caption (until we provide a
    % proper Figure object with a Caption API and a way to capture that
    % in the conversion process - todo).
    \usepackage{caption}
    \DeclareCaptionLabelFormat{nolabel}{}
    \captionsetup{labelformat=nolabel}

    \usepackage{adjustbox} % Used to constrain images to a maximum size 
    \usepackage{xcolor} % Allow colors to be defined
    \usepackage{enumerate} % Needed for markdown enumerations to work
    \usepackage{geometry} % Used to adjust the document margins
    \usepackage{amsmath} % Equations
    \usepackage{amssymb} % Equations
    \usepackage{textcomp} % defines textquotesingle
    % Hack from http://tex.stackexchange.com/a/47451/13684:
    \AtBeginDocument{%
        \def\PYZsq{\textquotesingle}% Upright quotes in Pygmentized code
    }
    \usepackage{upquote} % Upright quotes for verbatim code
    \usepackage{eurosym} % defines \euro
    \usepackage[mathletters]{ucs} % Extended unicode (utf-8) support
    \usepackage[utf8x]{inputenc} % Allow utf-8 characters in the tex document
    \usepackage{fancyvrb} % verbatim replacement that allows latex
    \usepackage{grffile} % extends the file name processing of package graphics 
                         % to support a larger range 
    % The hyperref package gives us a pdf with properly built
    % internal navigation ('pdf bookmarks' for the table of contents,
    % internal cross-reference links, web links for URLs, etc.)
    \usepackage{hyperref}
    \usepackage{longtable} % longtable support required by pandoc >1.10
    \usepackage{booktabs}  % table support for pandoc > 1.12.2
    \usepackage[inline]{enumitem} % IRkernel/repr support (it uses the enumerate* environment)
    \usepackage[normalem]{ulem} % ulem is needed to support strikethroughs (\sout)
                                % normalem makes italics be italics, not underlines
    

    
    
    % Colors for the hyperref package
    \definecolor{urlcolor}{rgb}{0,.145,.698}
    \definecolor{linkcolor}{rgb}{.71,0.21,0.01}
    \definecolor{citecolor}{rgb}{.12,.54,.11}

    % ANSI colors
    \definecolor{ansi-black}{HTML}{3E424D}
    \definecolor{ansi-black-intense}{HTML}{282C36}
    \definecolor{ansi-red}{HTML}{E75C58}
    \definecolor{ansi-red-intense}{HTML}{B22B31}
    \definecolor{ansi-green}{HTML}{00A250}
    \definecolor{ansi-green-intense}{HTML}{007427}
    \definecolor{ansi-yellow}{HTML}{DDB62B}
    \definecolor{ansi-yellow-intense}{HTML}{B27D12}
    \definecolor{ansi-blue}{HTML}{208FFB}
    \definecolor{ansi-blue-intense}{HTML}{0065CA}
    \definecolor{ansi-magenta}{HTML}{D160C4}
    \definecolor{ansi-magenta-intense}{HTML}{A03196}
    \definecolor{ansi-cyan}{HTML}{60C6C8}
    \definecolor{ansi-cyan-intense}{HTML}{258F8F}
    \definecolor{ansi-white}{HTML}{C5C1B4}
    \definecolor{ansi-white-intense}{HTML}{A1A6B2}

    % commands and environments needed by pandoc snippets
    % extracted from the output of `pandoc -s`
    \providecommand{\tightlist}{%
      \setlength{\itemsep}{0pt}\setlength{\parskip}{0pt}}
    \DefineVerbatimEnvironment{Highlighting}{Verbatim}{commandchars=\\\{\}}
    % Add ',fontsize=\small' for more characters per line
    \newenvironment{Shaded}{}{}
    \newcommand{\KeywordTok}[1]{\textcolor[rgb]{0.00,0.44,0.13}{\textbf{{#1}}}}
    \newcommand{\DataTypeTok}[1]{\textcolor[rgb]{0.56,0.13,0.00}{{#1}}}
    \newcommand{\DecValTok}[1]{\textcolor[rgb]{0.25,0.63,0.44}{{#1}}}
    \newcommand{\BaseNTok}[1]{\textcolor[rgb]{0.25,0.63,0.44}{{#1}}}
    \newcommand{\FloatTok}[1]{\textcolor[rgb]{0.25,0.63,0.44}{{#1}}}
    \newcommand{\CharTok}[1]{\textcolor[rgb]{0.25,0.44,0.63}{{#1}}}
    \newcommand{\StringTok}[1]{\textcolor[rgb]{0.25,0.44,0.63}{{#1}}}
    \newcommand{\CommentTok}[1]{\textcolor[rgb]{0.38,0.63,0.69}{\textit{{#1}}}}
    \newcommand{\OtherTok}[1]{\textcolor[rgb]{0.00,0.44,0.13}{{#1}}}
    \newcommand{\AlertTok}[1]{\textcolor[rgb]{1.00,0.00,0.00}{\textbf{{#1}}}}
    \newcommand{\FunctionTok}[1]{\textcolor[rgb]{0.02,0.16,0.49}{{#1}}}
    \newcommand{\RegionMarkerTok}[1]{{#1}}
    \newcommand{\ErrorTok}[1]{\textcolor[rgb]{1.00,0.00,0.00}{\textbf{{#1}}}}
    \newcommand{\NormalTok}[1]{{#1}}
    
    % Additional commands for more recent versions of Pandoc
    \newcommand{\ConstantTok}[1]{\textcolor[rgb]{0.53,0.00,0.00}{{#1}}}
    \newcommand{\SpecialCharTok}[1]{\textcolor[rgb]{0.25,0.44,0.63}{{#1}}}
    \newcommand{\VerbatimStringTok}[1]{\textcolor[rgb]{0.25,0.44,0.63}{{#1}}}
    \newcommand{\SpecialStringTok}[1]{\textcolor[rgb]{0.73,0.40,0.53}{{#1}}}
    \newcommand{\ImportTok}[1]{{#1}}
    \newcommand{\DocumentationTok}[1]{\textcolor[rgb]{0.73,0.13,0.13}{\textit{{#1}}}}
    \newcommand{\AnnotationTok}[1]{\textcolor[rgb]{0.38,0.63,0.69}{\textbf{\textit{{#1}}}}}
    \newcommand{\CommentVarTok}[1]{\textcolor[rgb]{0.38,0.63,0.69}{\textbf{\textit{{#1}}}}}
    \newcommand{\VariableTok}[1]{\textcolor[rgb]{0.10,0.09,0.49}{{#1}}}
    \newcommand{\ControlFlowTok}[1]{\textcolor[rgb]{0.00,0.44,0.13}{\textbf{{#1}}}}
    \newcommand{\OperatorTok}[1]{\textcolor[rgb]{0.40,0.40,0.40}{{#1}}}
    \newcommand{\BuiltInTok}[1]{{#1}}
    \newcommand{\ExtensionTok}[1]{{#1}}
    \newcommand{\PreprocessorTok}[1]{\textcolor[rgb]{0.74,0.48,0.00}{{#1}}}
    \newcommand{\AttributeTok}[1]{\textcolor[rgb]{0.49,0.56,0.16}{{#1}}}
    \newcommand{\InformationTok}[1]{\textcolor[rgb]{0.38,0.63,0.69}{\textbf{\textit{{#1}}}}}
    \newcommand{\WarningTok}[1]{\textcolor[rgb]{0.38,0.63,0.69}{\textbf{\textit{{#1}}}}}
    
    
    % Define a nice break command that doesn't care if a line doesn't already
    % exist.
    \def\br{\hspace*{\fill} \\* }
    % Math Jax compatability definitions
    \def\gt{>}
    \def\lt{<}
    % Document parameters
    \title{example3}
    
    
    

    % Pygments definitions
    
\makeatletter
\def\PY@reset{\let\PY@it=\relax \let\PY@bf=\relax%
    \let\PY@ul=\relax \let\PY@tc=\relax%
    \let\PY@bc=\relax \let\PY@ff=\relax}
\def\PY@tok#1{\csname PY@tok@#1\endcsname}
\def\PY@toks#1+{\ifx\relax#1\empty\else%
    \PY@tok{#1}\expandafter\PY@toks\fi}
\def\PY@do#1{\PY@bc{\PY@tc{\PY@ul{%
    \PY@it{\PY@bf{\PY@ff{#1}}}}}}}
\def\PY#1#2{\PY@reset\PY@toks#1+\relax+\PY@do{#2}}

\expandafter\def\csname PY@tok@w\endcsname{\def\PY@tc##1{\textcolor[rgb]{0.73,0.73,0.73}{##1}}}
\expandafter\def\csname PY@tok@c\endcsname{\let\PY@it=\textit\def\PY@tc##1{\textcolor[rgb]{0.25,0.50,0.50}{##1}}}
\expandafter\def\csname PY@tok@cp\endcsname{\def\PY@tc##1{\textcolor[rgb]{0.74,0.48,0.00}{##1}}}
\expandafter\def\csname PY@tok@k\endcsname{\let\PY@bf=\textbf\def\PY@tc##1{\textcolor[rgb]{0.00,0.50,0.00}{##1}}}
\expandafter\def\csname PY@tok@kp\endcsname{\def\PY@tc##1{\textcolor[rgb]{0.00,0.50,0.00}{##1}}}
\expandafter\def\csname PY@tok@kt\endcsname{\def\PY@tc##1{\textcolor[rgb]{0.69,0.00,0.25}{##1}}}
\expandafter\def\csname PY@tok@o\endcsname{\def\PY@tc##1{\textcolor[rgb]{0.40,0.40,0.40}{##1}}}
\expandafter\def\csname PY@tok@ow\endcsname{\let\PY@bf=\textbf\def\PY@tc##1{\textcolor[rgb]{0.67,0.13,1.00}{##1}}}
\expandafter\def\csname PY@tok@nb\endcsname{\def\PY@tc##1{\textcolor[rgb]{0.00,0.50,0.00}{##1}}}
\expandafter\def\csname PY@tok@nf\endcsname{\def\PY@tc##1{\textcolor[rgb]{0.00,0.00,1.00}{##1}}}
\expandafter\def\csname PY@tok@nc\endcsname{\let\PY@bf=\textbf\def\PY@tc##1{\textcolor[rgb]{0.00,0.00,1.00}{##1}}}
\expandafter\def\csname PY@tok@nn\endcsname{\let\PY@bf=\textbf\def\PY@tc##1{\textcolor[rgb]{0.00,0.00,1.00}{##1}}}
\expandafter\def\csname PY@tok@ne\endcsname{\let\PY@bf=\textbf\def\PY@tc##1{\textcolor[rgb]{0.82,0.25,0.23}{##1}}}
\expandafter\def\csname PY@tok@nv\endcsname{\def\PY@tc##1{\textcolor[rgb]{0.10,0.09,0.49}{##1}}}
\expandafter\def\csname PY@tok@no\endcsname{\def\PY@tc##1{\textcolor[rgb]{0.53,0.00,0.00}{##1}}}
\expandafter\def\csname PY@tok@nl\endcsname{\def\PY@tc##1{\textcolor[rgb]{0.63,0.63,0.00}{##1}}}
\expandafter\def\csname PY@tok@ni\endcsname{\let\PY@bf=\textbf\def\PY@tc##1{\textcolor[rgb]{0.60,0.60,0.60}{##1}}}
\expandafter\def\csname PY@tok@na\endcsname{\def\PY@tc##1{\textcolor[rgb]{0.49,0.56,0.16}{##1}}}
\expandafter\def\csname PY@tok@nt\endcsname{\let\PY@bf=\textbf\def\PY@tc##1{\textcolor[rgb]{0.00,0.50,0.00}{##1}}}
\expandafter\def\csname PY@tok@nd\endcsname{\def\PY@tc##1{\textcolor[rgb]{0.67,0.13,1.00}{##1}}}
\expandafter\def\csname PY@tok@s\endcsname{\def\PY@tc##1{\textcolor[rgb]{0.73,0.13,0.13}{##1}}}
\expandafter\def\csname PY@tok@sd\endcsname{\let\PY@it=\textit\def\PY@tc##1{\textcolor[rgb]{0.73,0.13,0.13}{##1}}}
\expandafter\def\csname PY@tok@si\endcsname{\let\PY@bf=\textbf\def\PY@tc##1{\textcolor[rgb]{0.73,0.40,0.53}{##1}}}
\expandafter\def\csname PY@tok@se\endcsname{\let\PY@bf=\textbf\def\PY@tc##1{\textcolor[rgb]{0.73,0.40,0.13}{##1}}}
\expandafter\def\csname PY@tok@sr\endcsname{\def\PY@tc##1{\textcolor[rgb]{0.73,0.40,0.53}{##1}}}
\expandafter\def\csname PY@tok@ss\endcsname{\def\PY@tc##1{\textcolor[rgb]{0.10,0.09,0.49}{##1}}}
\expandafter\def\csname PY@tok@sx\endcsname{\def\PY@tc##1{\textcolor[rgb]{0.00,0.50,0.00}{##1}}}
\expandafter\def\csname PY@tok@m\endcsname{\def\PY@tc##1{\textcolor[rgb]{0.40,0.40,0.40}{##1}}}
\expandafter\def\csname PY@tok@gh\endcsname{\let\PY@bf=\textbf\def\PY@tc##1{\textcolor[rgb]{0.00,0.00,0.50}{##1}}}
\expandafter\def\csname PY@tok@gu\endcsname{\let\PY@bf=\textbf\def\PY@tc##1{\textcolor[rgb]{0.50,0.00,0.50}{##1}}}
\expandafter\def\csname PY@tok@gd\endcsname{\def\PY@tc##1{\textcolor[rgb]{0.63,0.00,0.00}{##1}}}
\expandafter\def\csname PY@tok@gi\endcsname{\def\PY@tc##1{\textcolor[rgb]{0.00,0.63,0.00}{##1}}}
\expandafter\def\csname PY@tok@gr\endcsname{\def\PY@tc##1{\textcolor[rgb]{1.00,0.00,0.00}{##1}}}
\expandafter\def\csname PY@tok@ge\endcsname{\let\PY@it=\textit}
\expandafter\def\csname PY@tok@gs\endcsname{\let\PY@bf=\textbf}
\expandafter\def\csname PY@tok@gp\endcsname{\let\PY@bf=\textbf\def\PY@tc##1{\textcolor[rgb]{0.00,0.00,0.50}{##1}}}
\expandafter\def\csname PY@tok@go\endcsname{\def\PY@tc##1{\textcolor[rgb]{0.53,0.53,0.53}{##1}}}
\expandafter\def\csname PY@tok@gt\endcsname{\def\PY@tc##1{\textcolor[rgb]{0.00,0.27,0.87}{##1}}}
\expandafter\def\csname PY@tok@err\endcsname{\def\PY@bc##1{\setlength{\fboxsep}{0pt}\fcolorbox[rgb]{1.00,0.00,0.00}{1,1,1}{\strut ##1}}}
\expandafter\def\csname PY@tok@kc\endcsname{\let\PY@bf=\textbf\def\PY@tc##1{\textcolor[rgb]{0.00,0.50,0.00}{##1}}}
\expandafter\def\csname PY@tok@kd\endcsname{\let\PY@bf=\textbf\def\PY@tc##1{\textcolor[rgb]{0.00,0.50,0.00}{##1}}}
\expandafter\def\csname PY@tok@kn\endcsname{\let\PY@bf=\textbf\def\PY@tc##1{\textcolor[rgb]{0.00,0.50,0.00}{##1}}}
\expandafter\def\csname PY@tok@kr\endcsname{\let\PY@bf=\textbf\def\PY@tc##1{\textcolor[rgb]{0.00,0.50,0.00}{##1}}}
\expandafter\def\csname PY@tok@bp\endcsname{\def\PY@tc##1{\textcolor[rgb]{0.00,0.50,0.00}{##1}}}
\expandafter\def\csname PY@tok@fm\endcsname{\def\PY@tc##1{\textcolor[rgb]{0.00,0.00,1.00}{##1}}}
\expandafter\def\csname PY@tok@vc\endcsname{\def\PY@tc##1{\textcolor[rgb]{0.10,0.09,0.49}{##1}}}
\expandafter\def\csname PY@tok@vg\endcsname{\def\PY@tc##1{\textcolor[rgb]{0.10,0.09,0.49}{##1}}}
\expandafter\def\csname PY@tok@vi\endcsname{\def\PY@tc##1{\textcolor[rgb]{0.10,0.09,0.49}{##1}}}
\expandafter\def\csname PY@tok@vm\endcsname{\def\PY@tc##1{\textcolor[rgb]{0.10,0.09,0.49}{##1}}}
\expandafter\def\csname PY@tok@sa\endcsname{\def\PY@tc##1{\textcolor[rgb]{0.73,0.13,0.13}{##1}}}
\expandafter\def\csname PY@tok@sb\endcsname{\def\PY@tc##1{\textcolor[rgb]{0.73,0.13,0.13}{##1}}}
\expandafter\def\csname PY@tok@sc\endcsname{\def\PY@tc##1{\textcolor[rgb]{0.73,0.13,0.13}{##1}}}
\expandafter\def\csname PY@tok@dl\endcsname{\def\PY@tc##1{\textcolor[rgb]{0.73,0.13,0.13}{##1}}}
\expandafter\def\csname PY@tok@s2\endcsname{\def\PY@tc##1{\textcolor[rgb]{0.73,0.13,0.13}{##1}}}
\expandafter\def\csname PY@tok@sh\endcsname{\def\PY@tc##1{\textcolor[rgb]{0.73,0.13,0.13}{##1}}}
\expandafter\def\csname PY@tok@s1\endcsname{\def\PY@tc##1{\textcolor[rgb]{0.73,0.13,0.13}{##1}}}
\expandafter\def\csname PY@tok@mb\endcsname{\def\PY@tc##1{\textcolor[rgb]{0.40,0.40,0.40}{##1}}}
\expandafter\def\csname PY@tok@mf\endcsname{\def\PY@tc##1{\textcolor[rgb]{0.40,0.40,0.40}{##1}}}
\expandafter\def\csname PY@tok@mh\endcsname{\def\PY@tc##1{\textcolor[rgb]{0.40,0.40,0.40}{##1}}}
\expandafter\def\csname PY@tok@mi\endcsname{\def\PY@tc##1{\textcolor[rgb]{0.40,0.40,0.40}{##1}}}
\expandafter\def\csname PY@tok@il\endcsname{\def\PY@tc##1{\textcolor[rgb]{0.40,0.40,0.40}{##1}}}
\expandafter\def\csname PY@tok@mo\endcsname{\def\PY@tc##1{\textcolor[rgb]{0.40,0.40,0.40}{##1}}}
\expandafter\def\csname PY@tok@ch\endcsname{\let\PY@it=\textit\def\PY@tc##1{\textcolor[rgb]{0.25,0.50,0.50}{##1}}}
\expandafter\def\csname PY@tok@cm\endcsname{\let\PY@it=\textit\def\PY@tc##1{\textcolor[rgb]{0.25,0.50,0.50}{##1}}}
\expandafter\def\csname PY@tok@cpf\endcsname{\let\PY@it=\textit\def\PY@tc##1{\textcolor[rgb]{0.25,0.50,0.50}{##1}}}
\expandafter\def\csname PY@tok@c1\endcsname{\let\PY@it=\textit\def\PY@tc##1{\textcolor[rgb]{0.25,0.50,0.50}{##1}}}
\expandafter\def\csname PY@tok@cs\endcsname{\let\PY@it=\textit\def\PY@tc##1{\textcolor[rgb]{0.25,0.50,0.50}{##1}}}

\def\PYZbs{\char`\\}
\def\PYZus{\char`\_}
\def\PYZob{\char`\{}
\def\PYZcb{\char`\}}
\def\PYZca{\char`\^}
\def\PYZam{\char`\&}
\def\PYZlt{\char`\<}
\def\PYZgt{\char`\>}
\def\PYZsh{\char`\#}
\def\PYZpc{\char`\%}
\def\PYZdl{\char`\$}
\def\PYZhy{\char`\-}
\def\PYZsq{\char`\'}
\def\PYZdq{\char`\"}
\def\PYZti{\char`\~}
% for compatibility with earlier versions
\def\PYZat{@}
\def\PYZlb{[}
\def\PYZrb{]}
\makeatother


    % Exact colors from NB
    \definecolor{incolor}{rgb}{0.0, 0.0, 0.5}
    \definecolor{outcolor}{rgb}{0.545, 0.0, 0.0}



    
    % Prevent overflowing lines due to hard-to-break entities
    \sloppy 
    % Setup hyperref package
    \hypersetup{
      breaklinks=true,  % so long urls are correctly broken across lines
      colorlinks=true,
      urlcolor=urlcolor,
      linkcolor=linkcolor,
      citecolor=citecolor,
      }
    % Slightly bigger margins than the latex defaults
    
    \geometry{verbose,tmargin=1in,bmargin=1in,lmargin=1in,rmargin=1in}
    
    

    \begin{document}
    
    
    \maketitle
    
    

    
    \section{Debug mode log error exception
handling}\label{debug-mode-log-error-exception-handling}

    Below we show how the logger (WLOG) is handled in debugging mode, using
cal\_DARK\_spirou as an example:

    \begin{Verbatim}[commandchars=\\\{\}]
{\color{incolor}In [{\color{incolor}14}]:} \PY{c+c1}{\PYZsh{} simulate cal\PYZus{}DARK\PYZus{}spirou}
         \PY{k+kn}{from} \PY{n+nn}{\PYZus{}\PYZus{}future\PYZus{}\PYZus{}} \PY{k+kn}{import} \PY{n}{division}
         
         \PY{k+kn}{from} \PY{n+nn}{SpirouDRS} \PY{k+kn}{import} \PY{n}{spirouConfig}
         \PY{k+kn}{from} \PY{n+nn}{SpirouDRS} \PY{k+kn}{import} \PY{n}{spirouCore}
         \PY{k+kn}{from} \PY{n+nn}{SpirouDRS} \PY{k+kn}{import} \PY{n}{spirouStartup}
         
         \PY{c+c1}{\PYZsh{} =============================================================================}
         \PY{c+c1}{\PYZsh{} Define variables}
         \PY{c+c1}{\PYZsh{} =============================================================================}
         \PY{c+c1}{\PYZsh{} Name of program}
         \PY{n}{\PYZus{}\PYZus{}NAME\PYZus{}\PYZus{}} \PY{o}{=} \PY{l+s+s1}{\PYZsq{}}\PY{l+s+s1}{cal\PYZus{}DARK\PYZus{}spirou.py}\PY{l+s+s1}{\PYZsq{}}
         \PY{c+c1}{\PYZsh{} Get version and author}
         \PY{n}{\PYZus{}\PYZus{}version\PYZus{}\PYZus{}} \PY{o}{=} \PY{n}{spirouConfig}\PY{o}{.}\PY{n}{Constants}\PY{o}{.}\PY{n}{VERSION}\PY{p}{(}\PY{p}{)}
         \PY{n}{\PYZus{}\PYZus{}author\PYZus{}\PYZus{}} \PY{o}{=} \PY{n}{spirouConfig}\PY{o}{.}\PY{n}{Constants}\PY{o}{.}\PY{n}{AUTHORS}\PY{p}{(}\PY{p}{)}
         \PY{n}{\PYZus{}\PYZus{}date\PYZus{}\PYZus{}} \PY{o}{=} \PY{n}{spirouConfig}\PY{o}{.}\PY{n}{Constants}\PY{o}{.}\PY{n}{LATEST\PYZus{}EDIT}\PY{p}{(}\PY{p}{)}
         \PY{n}{\PYZus{}\PYZus{}release\PYZus{}\PYZus{}} \PY{o}{=} \PY{n}{spirouConfig}\PY{o}{.}\PY{n}{Constants}\PY{o}{.}\PY{n}{RELEASE}\PY{p}{(}\PY{p}{)}
         \PY{c+c1}{\PYZsh{} Get Logging function}
         \PY{n}{WLOG} \PY{o}{=} \PY{n}{spirouCore}\PY{o}{.}\PY{n}{wlog}
         \PY{c+c1}{\PYZsh{} Get plotting functions}
         \PY{n}{sPlt} \PY{o}{=} \PY{n}{spirouCore}\PY{o}{.}\PY{n}{sPlt}
         
         \PY{c+c1}{\PYZsh{} Notebook only code}
         \PY{k+kn}{import} \PY{n+nn}{sys}
         \PY{n}{sys}\PY{o}{.}\PY{n}{argv} \PY{o}{=} \PY{p}{[}\PY{l+s+s1}{\PYZsq{}}\PY{l+s+s1}{cal\PYZus{}DARK\PYZus{}spirou.py}\PY{l+s+s1}{\PYZsq{}}\PY{p}{]}
         \PY{n}{night\PYZus{}name}\PY{p}{,} \PY{n}{files} \PY{o}{=} \PY{n+nb+bp}{None}\PY{p}{,} \PY{n+nb+bp}{None}
         
         \PY{c+c1}{\PYZsh{} \PYZhy{}\PYZhy{}\PYZhy{}\PYZhy{}\PYZhy{}\PYZhy{}\PYZhy{}\PYZhy{}\PYZhy{}\PYZhy{}\PYZhy{}\PYZhy{}\PYZhy{}\PYZhy{}\PYZhy{}\PYZhy{}\PYZhy{}\PYZhy{}\PYZhy{}\PYZhy{}\PYZhy{}\PYZhy{}\PYZhy{}\PYZhy{}\PYZhy{}\PYZhy{}\PYZhy{}\PYZhy{}\PYZhy{}\PYZhy{}\PYZhy{}\PYZhy{}\PYZhy{}\PYZhy{}\PYZhy{}\PYZhy{}\PYZhy{}\PYZhy{}\PYZhy{}\PYZhy{}\PYZhy{}\PYZhy{}\PYZhy{}\PYZhy{}\PYZhy{}\PYZhy{}\PYZhy{}\PYZhy{}\PYZhy{}\PYZhy{}\PYZhy{}\PYZhy{}\PYZhy{}\PYZhy{}\PYZhy{}\PYZhy{}\PYZhy{}\PYZhy{}\PYZhy{}\PYZhy{}\PYZhy{}\PYZhy{}\PYZhy{}\PYZhy{}\PYZhy{}\PYZhy{}\PYZhy{}\PYZhy{}\PYZhy{}\PYZhy{}}
         \PY{c+c1}{\PYZsh{} Set up}
         \PY{c+c1}{\PYZsh{} \PYZhy{}\PYZhy{}\PYZhy{}\PYZhy{}\PYZhy{}\PYZhy{}\PYZhy{}\PYZhy{}\PYZhy{}\PYZhy{}\PYZhy{}\PYZhy{}\PYZhy{}\PYZhy{}\PYZhy{}\PYZhy{}\PYZhy{}\PYZhy{}\PYZhy{}\PYZhy{}\PYZhy{}\PYZhy{}\PYZhy{}\PYZhy{}\PYZhy{}\PYZhy{}\PYZhy{}\PYZhy{}\PYZhy{}\PYZhy{}\PYZhy{}\PYZhy{}\PYZhy{}\PYZhy{}\PYZhy{}\PYZhy{}\PYZhy{}\PYZhy{}\PYZhy{}\PYZhy{}\PYZhy{}\PYZhy{}\PYZhy{}\PYZhy{}\PYZhy{}\PYZhy{}\PYZhy{}\PYZhy{}\PYZhy{}\PYZhy{}\PYZhy{}\PYZhy{}\PYZhy{}\PYZhy{}\PYZhy{}\PYZhy{}\PYZhy{}\PYZhy{}\PYZhy{}\PYZhy{}\PYZhy{}\PYZhy{}\PYZhy{}\PYZhy{}\PYZhy{}\PYZhy{}\PYZhy{}\PYZhy{}\PYZhy{}\PYZhy{}}
         \PY{c+c1}{\PYZsh{} get parameters from config files/run time args/load paths + calibdb}
         \PY{n}{p} \PY{o}{=} \PY{n}{spirouStartup}\PY{o}{.}\PY{n}{Begin}\PY{p}{(}\PY{p}{)}
         \PY{n}{p} \PY{o}{=} \PY{n}{spirouStartup}\PY{o}{.}\PY{n}{LoadArguments}\PY{p}{(}\PY{n}{p}\PY{p}{,} \PY{n}{night\PYZus{}name}\PY{p}{,} \PY{n}{files}\PY{p}{)}
         \PY{n}{p} \PY{o}{=} \PY{n}{spirouStartup}\PY{o}{.}\PY{n}{InitialFileSetup}\PY{p}{(}\PY{n}{p}\PY{p}{,} \PY{n}{kind}\PY{o}{=}\PY{l+s+s1}{\PYZsq{}}\PY{l+s+s1}{dark}\PY{l+s+s1}{\PYZsq{}}\PY{p}{,} \PY{n}{prefixes}\PY{o}{=}\PY{p}{[}\PY{l+s+s1}{\PYZsq{}}\PY{l+s+s1}{dark\PYZus{}dark}\PY{l+s+s1}{\PYZsq{}}\PY{p}{]}\PY{p}{)}
\end{Verbatim}


    \begin{Verbatim}[commandchars=\\\{\}]
\textcolor{ansi-green-intense}{\textbf{13:44:04.0 -   || *****************************************}}
\textcolor{ansi-green-intense}{\textbf{13:44:04.0 -   || * SPIROU @(\#) Geneva Observatory (0.1.029)}}
\textcolor{ansi-green-intense}{\textbf{13:44:04.0 -   || *****************************************}}
\textcolor{ansi-green-intense}{\textbf{13:44:04.0 -   ||(dir\_data\_raw)      DRS\_DATA\_RAW=/scratch/Projects/spirou\_py3/data/raw}}
\textcolor{ansi-green-intense}{\textbf{13:44:04.0 -   ||(dir\_data\_reduc)    DRS\_DATA\_REDUC=/scratch/Projects/spirou\_py3/data/reduced}}
\textcolor{ansi-green-intense}{\textbf{13:44:04.0 -   ||(dir\_calib\_db)      DRS\_CALIB\_DB=/scratch/Projects/spirou\_py3/data/calibDB}}
\textcolor{ansi-green-intense}{\textbf{13:44:04.0 -   ||(dir\_data\_msg)      DRS\_DATA\_MSG=/scratch/Projects/spirou\_py3/data/msg}}
\textcolor{ansi-green-intense}{\textbf{13:44:04.0 -   ||(print\_level)       PRINT\_LEVEL=all         \%(error/warning/info/all)}}
\textcolor{ansi-green-intense}{\textbf{13:44:04.0 -   ||(log\_level)         LOG\_LEVEL=all         \%(error/warning/info/all)}}
\textcolor{ansi-green-intense}{\textbf{13:44:04.0 -   ||(plot\_graph)        DRS\_PLOT=1            \%(def/undef/trigger)}}
\textcolor{ansi-green-intense}{\textbf{13:44:04.0 -   ||(used\_date)         DRS\_USED\_DATE=undefined}}
\textcolor{ansi-green-intense}{\textbf{13:44:04.0 -   ||(working\_dir)       DRS\_DATA\_WORKING=/scratch/Projects/spirou\_py3/data/tmp}}
\textcolor{ansi-green-intense}{\textbf{13:44:04.0 -   ||                    DRS\_INTERACTIVE is not set, running on-line mode}}
\textcolor{ansi-green-intense}{\textbf{13:44:04.0 -   ||                    DRS\_DEBUG is set, debug mode level:1}}
\textcolor{ansi-green-intense}{\textbf{13:44:04.0 -   |cal\_DARK\_spirou|Now running : cal\_DARK\_spirou on file(s): }}
\textcolor{ansi-green-intense}{\textbf{13:44:04.0 -   |cal\_DARK\_spirou|On directory /scratch/Projects/spirou\_py3/data/raw/}}
\textcolor{ansi-green-intense}{\textbf{13:44:04.0 -   |cal\_DARK\_spirou|ICDP\_NAME loaded from: /scratch/Projects/spirou\_py3/spirou\_py3/INTROOT/config/constants\_SPIROU.py}}
\textcolor{ansi-red-intense}{\textbf{13:44:04.0 - ! |cal\_DARK\_spirou|Argument Error: No fits file defined at run time argument}}
\textcolor{ansi-red-intense}{\textbf{13:44:04.0 - ! |cal\_DARK\_spirou|    format must be:}}
\textcolor{ansi-red-intense}{\textbf{13:44:04.0 - ! |cal\_DARK\_spirou|    >>> cal\_DARK\_spirou.py [FOLDER] [FILES]}}
\textcolor{ansi-red-intense}{\textbf{

	Error found and running in DEBUG mode
}}

    \end{Verbatim}

    \begin{Verbatim}[commandchars=\\\{\}]

        An exception has occurred, use \%tb to see the full traceback.


        SystemExit: 1


    \end{Verbatim}

    \begin{Verbatim}[commandchars=\\\{\}]
/scratch/bin/anaconda3/lib/python3.6/site-packages/IPython/core/interactiveshell.py:2918: UserWarning: To exit: use 'exit', 'quit', or Ctrl-D.
  warn("To exit: use 'exit', 'quit', or Ctrl-D.", stacklevel=1)

    \end{Verbatim}

    In the console the following message will be displayed:

\begin{verbatim}
"Enter python debugger? [Y]es or [N]o?"

If "N" is selected the recipe will exit. If "Y" is selected the recipe will go into a python debugger:



        Error found and running in DEBUG mode

        Enter python debugger? [Y]es or [N]o?   Y

         ==== DEBUGGER ====

         - type "list" to list code
         - type "up" to go up a level
         - type "interact" to go to an interactive shell
         - type "print(variable)" to print variable
         - type "print(dir())" to list available variables
         - type "continue" to exit
         - type "help" to see all commands

         ==================


    > /scratch/Projects/spirou_py3/spirou_py3/INTROOT/SpirouDRS/spirouCore/spirouLog.py(217)debug_start()
    -> print(cc + '\n\nCode Exited' + nocol)
    (Pdb) 
\end{verbatim}

This starts a basic python debugger with command such as: - list: which
lists the last 10 lines of code before this error was generated (will
normally just show the "error" catching in the logger (WLOG) function)

\begin{itemize}
\item
  up: this command goes "up" a level in the code, going up a couple of
  levels should get to the WLOG message which caused the crash.
\item
  interact: this starts a python interactive session which can be used
  to access all variables currently stored in the memory and to see the
  currently assigned values of each variable
\end{itemize}

Shown below is a typical use for the debug based on the code above
(unfortunately notebooks cannot run interactive code so this is copied
as text)

    \begin{itemize}
\tightlist
\item
  Here the code is "listed" which caused the recipe to exit (this level
  is where the debugger was entered so isn't the level we want to
  debug).
\end{itemize}

As one can see we are in the \textbf{spirouLog.py debug\_start()}
function

\begin{verbatim}
        > /scratch/Projects/spirou_py3/spirou_py3/INTROOT/SpirouDRS/spirouCore/spirouLog.py(217)debug_start()
        (Pdb) list
        212                       '\n\n\t ==================\n\n' + nocol)
        213
        214                 import pdb
        215                 pdb.set_trace()
        216
        217  ->             print(cc + '\n\nCode Exited' + nocol)
        218                 EXIT_TYPE(1)
        219             else:
        220                 EXIT_TYPE(1)
        221         except:
        222             EXIT_TYPE(1)
\end{verbatim}

\begin{itemize}
\item
  Here the "up" command was used to go to up a level (the function that
  called the \textbf{debug\_start()} function (\textbf{spirouLog.py
  logger()} function) this function is the logger function thus like
  \textbf{debug\_start()} this isn't the level we want to debug.

\begin{verbatim}
    (Pdb) up
    > /scratch/Projects/spirou_py3/spirou_py3/INTROOT/SpirouDRS/spirouCore/spirouLog.py(136)logger()
    -> debug_start()

    (Pdb) list
    131                 TDATA_WARNING = 0
    132
    133         # deal with errors (if key is in EXIT_LEVELS) then exit after log/print
    134         if key in EXIT_LEVELS:
    135             if spirouConfig.Constants.DEBUG():
    136  ->             debug_start()
    137             else:
    138                 EXIT_TYPE(1)
    139
    140
    141     def printlogandcmd(message, key, human_time, dsec, option):
\end{verbatim}
\item
  Here again we have used the "up" command (as the \textbf{logger()}
  function level is not useful for debugging the exception. Thus we are
  now at the call to the \textbf{logger()} functino via the alias
  \textbf{WLOG()}. This shows what was happening before the log "error"
  was raised. (In this case it is due to "fits\_fn" having a value of
  "None").

\begin{verbatim}
    (Pdb) up
    > /scratch/Projects/spirou_py3/spirou_py3/INTROOT/SpirouDRS/spirouStartup/spirouStartup.py(259)initial_file_setup()
    -> WLOG('error', log_opt, [wmsg1, wmsg2, emsg.format(p['program'])])

    (Pdb) list
    254         # check that fitsfilename exists
    255         if fits_fn is None:
    256             wmsg1 = 'Argument Error: No fits file defined at run time argument'
    257             wmsg2 = '    format must be:'
    258             emsg = '    >>> {0}.py [FOLDER] [FILES]'
    259  ->         WLOG('error', log_opt, [wmsg1, wmsg2, emsg.format(p['program'])])
    260         if not os.path.exists(fits_fn):
    261             WLOG('error', log_opt, 'File : {0} does not exist'.format(fits_fn))
    262         # -------------------------------------------------------------------------
    263         # if we have prefixes defined then check that fitsfilename has them
    264         # if add_to_params is defined then add params to p accordingly
\end{verbatim}
\item
  At this stage it may be useful to be in interactive mode. Typing
  "interact" allows a python console to be opened (or ipython if it was
  detected that we were running in ipython). Below as with a standard
  python console we can explore the parameter space and indeed see that
  "fits\_fn" has a value of "None". Obviously this example is trivia as
  we intentionally forgot to add "night\_name" and "files" at run time.

\begin{verbatim}
(Pdb) interact
*interactive*

>>> log_opt
'cal_DARK_spirou'

>>> fits_fn is None
True
\end{verbatim}
\end{itemize}


    % Add a bibliography block to the postdoc
    
    
    
    \end{document}
