%%%%%%%%%%%%%%%%%%%%%%%%%%%%%%%%%%%%%%%%%%%%%%%%%%%%%%%%
%%
\chapter{Introduction}
\label{chapter:intro}
%%
%%%%%%%%%%%%%%%%%%%%%%%%%%%%%%%%%%%%%%%%%%%%%%%%%%%%%%%%

This documentation will cover the installation, data architecture, \ifdevguide the changes between the previous versions and this version\fi, using the DRS (with a working example), descriptions of the variables \ifdevguide and keywords for input and output FITS rec headers \fi, and the recipes \ifdevguide and module code \fi.

\noindent Variables are defined in detail in section \ref{ch:variables} and will be defined throughout via the following syntax: \definevariable{text:variable}{VARIABLE}. When referred to, one should take it as using the value set in section \ref{ch:variables} by default or in the file described in the variables description `Defined in' section. Clicking these variables will go to the appropriate variable description.

\noindent Certain sections will be written in code blocks, these imply text that is written into a text editor, the command shell console, or a python terminal/script. Below explains how one can distinguish these in this document. \\

\noindent The following denotes a line of text (or lines of text) that are to be edited in a text editor.
\begin{textbox}[title={Generic text file}]
<# A variable name that can be changes to a specific value>
@VARIABLE_NAME@ = "Variable Value"
\end{textbox}
\vspace{0.5cm}

\noindent These can also be shell scripts in a certain language:
\begin{bashbox}[title={For example in $\sim$/.bashrc}]
#!/usr/bin/bash
# Find out which console you are using
echo $0
# Set environment Hello
export Hello="Hello"
\end{bashbox}
\begin{cshbox}[title={For example in $\sim$/.tcshrc}]
#!/usr/bin/tcsh
# Find out which console you are using
echo $0
# Set environment Hello
setenv Hello "Hello"
\end{cshbox}
\vspace{0.5cm}

\noindent The following denotes a command to run in the command shell console 
\begin{cmdbox}
cd (*$\sim$*)/Downloads
\end{cmdbox}
\vspace{0.5cm}

\noindent The following denotes a command line print out
\begin{cmdboxprint}
 This is a print out in the command line
 produced by using the echo command
\end{cmdboxprint}
\vspace{0.5cm}

\noindent The following denotes a python terminal or python script
\begin{pythonbox}
import numpy as np
print("Hello world")
print("{0} seconds".format(np.sqrt(25)))
\end{pythonbox}
\vspace{0.5cm}

\ifdevguide
\noindent The following denotes \LaTeX code (in raw form and then compiled form) - this is used in Section \ref{ch:documentation}.
\begin{latexbox}
This is my \LaTeX code.
\end{latexbox}
\fi