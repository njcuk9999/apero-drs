%% Generated by Sphinx.
\def\sphinxdocclass{report}
\documentclass[a4paper,10pt,english]{report}
\ifdefined\pdfpxdimen
   \let\sphinxpxdimen\pdfpxdimen\else\newdimen\sphinxpxdimen
\fi \sphinxpxdimen=.75bp\relax

\PassOptionsToPackage{warn}{textcomp}
\usepackage[utf8]{inputenc}
\ifdefined\DeclareUnicodeCharacter
% support both utf8 and utf8x syntaxes
  \ifdefined\DeclareUnicodeCharacterAsOptional
    \def\sphinxDUC#1{\DeclareUnicodeCharacter{"#1}}
  \else
    \let\sphinxDUC\DeclareUnicodeCharacter
  \fi
  \sphinxDUC{00A0}{\nobreakspace}
  \sphinxDUC{2500}{\sphinxunichar{2500}}
  \sphinxDUC{2502}{\sphinxunichar{2502}}
  \sphinxDUC{2514}{\sphinxunichar{2514}}
  \sphinxDUC{251C}{\sphinxunichar{251C}}
  \sphinxDUC{2572}{\textbackslash}
\fi
\usepackage{cmap}
\usepackage[T1]{fontenc}
\usepackage{amsmath,amssymb,amstext}
\usepackage{babel}


\usepackage{amsmath,amsfonts,amssymb,amsthm}

\usepackage{fncychap}
\usepackage{sphinx}
\sphinxsetup{hmargin={0.7in,0.7in}, vmargin={1in,1in},         verbatimwithframe=true,         TitleColor={rgb}{0,0,0},         HeaderFamily=\rmfamily\bfseries,         InnerLinkColor={rgb}{0,0,1},         OuterLinkColor={rgb}{0,0,1}}
\fvset{fontsize=\small}
\usepackage{geometry}


% Include hyperref last.
\usepackage{hyperref}
% Fix anchor placement for figures with captions.
\usepackage{hypcap}% it must be loaded after hyperref.
% Set up styles of URL: it should be placed after hyperref.
\urlstyle{same}
\addto\captionsenglish{\renewcommand{\contentsname}{Contents:}}

\usepackage{sphinxmessages}
\setcounter{tocdepth}{1}


        %%%%%%%%%%%%%%%%%%%% Meher %%%%%%%%%%%%%%%%%%
        %%%add number to subsubsection 2=subsection, 3=subsubsection
        %%% below subsubsection is not good idea.
        \setcounter{secnumdepth}{3}
        %
        %%%% Table of content upto 2=subsection, 3=subsubsection
        \setcounter{tocdepth}{2}

        \usepackage{amsmath,amsfonts,amssymb,amsthm}
        \usepackage{graphicx}

        %%% reduce spaces for Table of contents, figures and tables
        %%% it is used "\addtocontents{toc}{\vskip -1.2cm}" etc. in the document
        \usepackage[notlot,nottoc,notlof]{}

        \usepackage{color}
        \usepackage{transparent}
        \usepackage{eso-pic}
        %% \usepackage{lipsum}

        \usepackage{footnotebackref} %%link at the footnote to go to the place of footnote in the text

        %% spacing between line
        \usepackage{setspace}
        %%%%\onehalfspacing
        %%%%\doublespacing
        \singlespacing


        %%%%%%%%%%% datetime
        \usepackage{datetime}

        \newdateformat{MonthYearFormat}{%
            \monthname[\THEMONTH], \THEYEAR}


        %% RO, LE will not work for 'oneside' layout.
        %% Change oneside to twoside in document class
        \usepackage{fancyhdr}
        \pagestyle{fancy}
        \fancyhf{}

        %%% Alternating Header for oneside
        \fancyhead[L]{\ifthenelse{\isodd{\value{page}}}{ \small \nouppercase{\leftmark} }{}}
        \fancyhead[R]{\ifthenelse{\isodd{\value{page}}}{}{ \small \nouppercase{\rightmark} }}

        %%% Alternating Header for two side
        %\fancyhead[RO]{\small \nouppercase{\rightmark}}
        %\fancyhead[LE]{\small \nouppercase{\leftmark}}

        %% for oneside: change footer at right side. If you want to use Left and right then use same as header defined above.
        \fancyfoot[R]{\ifthenelse{\isodd{\value{page}}}{{\tiny Meher Krishna Patel} }{\href{http://pythondsp.readthedocs.io/en/latest/pythondsp/toc.html}{\tiny PythonDSP}}}

        %%% Alternating Footer for two side
        %\fancyfoot[RO, RE]{\scriptsize Meher Krishna Patel (mekrip@gmail.com)}

        %%% page number
        \fancyfoot[CO, CE]{\thepage}

        \renewcommand{\headrulewidth}{0.5pt}
        \renewcommand{\footrulewidth}{0.5pt}

        \RequirePackage{tocbibind} %%% comment this to remove page number for following
        \addto\captionsenglish{\renewcommand{\contentsname}{Table of contents}}
        \addto\captionsenglish{\renewcommand{\listfigurename}{List of figures}}
        \addto\captionsenglish{\renewcommand{\listtablename}{List of tables}}
        % \addto\captionsenglish{\renewcommand{\chaptername}{Chapter}}


        %%reduce spacing for itemize
        \usepackage{enumitem}
        \setlist{nosep}

        %%%%%%%%%%% Quote Styles at the top of chapter
        \usepackage{epigraph}
        \setlength{\epigraphwidth}{0.8\columnwidth}
        \newcommand{\chapterquote}[2]{\epigraphhead[60]{\epigraph{\textit{#1}}{\textbf {\textit{--#2}}}}}
        %%%%%%%%%%% Quote for all places except Chapter
        \newcommand{\sectionquote}[2]{{\quote{\textit{``#1''}}{\textbf {\textit{--#2}}}}}
    

\title{APERO Documentation}
\date{Jan 02, 2020}
\release{0.6.010}
\author{Neil Cook}
\newcommand{\sphinxlogo}{\sphinxincludegraphics{apero_logo.png}\par}
\renewcommand{\releasename}{ }
\makeindex
\begin{document}

\pagestyle{empty}

        \pagenumbering{Roman} %%% to avoid page 1 conflict with actual page 1

        \begin{titlepage}
            \centering

            \vspace*{40mm} %%% * is used to give space from top
            \textbf{\Huge {APERO Documentation}}

            \vspace{5mm}
            \begin{figure}[!h]
                \centering
                \includegraphics[scale=1]{apero_logo.png}
            \end{figure}
            
            \vspace{5mm}
            \Large \textbf{Version 0.6.010}
            
            \vspace{5mm}
            \Large \textbf{{Neil Cook}}

            \vspace*{0mm}
            \small  Last updated : \MonthYearFormat\today


            %% \vfill adds at the bottom
            %% \vfill
            %% \small \textit{More documents are freely available at }{\href{http://pythondsp.readthedocs.io/en/latest/pythondsp/toc.html}{PythonDSP}}
        \end{titlepage}

        \clearpage
        \pagenumbering{roman}
        \tableofcontents
        %%\listoffigures
        %%\listoftables
        \pagenumbering{arabic}

        
\pagestyle{plain}
 
\pagestyle{normal}
\phantomsection\label{\detokenize{index::doc}}



\chapter{Installation}
\label{\detokenize{user/installation:installation}}\label{\detokenize{user/installation:id1}}\label{\detokenize{user/installation::doc}}

\section{Prerequisites}
\label{\detokenize{user/installation:prerequisites}}\label{\detokenize{user/installation:installation-prerequisites}}
APERO is tested with \sphinxhref{https://www.python.org/download/releases/3.0/}{python 3}

The following python modules are required:

\begin{sphinxVerbatim}[commandchars=\\\{\}]
\PYG{n}{astropy}
\PYG{n}{matplotlib}
\PYG{n}{numpy}
\PYG{n}{scipy}
\end{sphinxVerbatim}

The following python modules are recommended:

\begin{sphinxVerbatim}[commandchars=\\\{\}]
\PYG{n}{astroquery}
\PYG{n}{barycorrpy}
\PYG{n}{bottleneck}
\PYG{n}{ipdb}
\PYG{n}{numba}
\PYG{n}{pandas}
\PYG{n}{PIL}
\PYG{n}{tqdm}
\end{sphinxVerbatim}


\section{Download from GitHub}
\label{\detokenize{user/installation:download-from-github}}\label{\detokenize{user/installation:installation-download}}

\subsection{Clone}
\label{\detokenize{user/installation:clone}}
Clone from \sphinxhref{https://github.com/njcuk9999/apero-drs}{github}:

\begin{sphinxVerbatim}[commandchars=\\\{\}]
\PYG{o}{\PYGZgt{}} \PYG{n}{git} \PYG{n}{clone} \PYG{n}{https}\PYG{p}{:}\PYG{o}{/}\PYG{o}{/}\PYG{n}{github}\PYG{o}{.}\PYG{n}{com}\PYG{o}{/}\PYG{n}{njcuk9999}\PYG{o}{/}\PYG{n}{apero}\PYG{o}{\PYGZhy{}}\PYG{n}{drs}
\end{sphinxVerbatim}

This may take some time (in future most of the data required will be a separate download),
and we still have many (now redundant) files from the spirou\_py3 repository.


\subsection{Choose branch}
\label{\detokenize{user/installation:choose-branch}}\label{\detokenize{user/installation:installation-choose-branch}}
Change to the \sphinxtitleref{apero\sphinxhyphen{}drs} directory

Choose which branch:
\begin{itemize}
\item {} \begin{description}
\item[{master version}] \leavevmode
This is the version currently recommended for all general use.
It may not contain the most up\sphinxhyphen{}to\sphinxhyphen{}date features until long term support
and stability can be verified.

Change to this branch with:

\begin{sphinxVerbatim}[commandchars=\\\{\}]
\PYG{o}{\PYGZgt{}} \PYG{n}{git} \PYG{n}{checkout} \PYG{n}{master}
\PYG{o}{\PYGZgt{}} \PYG{n}{git} \PYG{n}{pull} \PYG{n}{origin} \PYG{n}{master}
\end{sphinxVerbatim}

\end{description}

\item {} \begin{description}
\item[{developer version}] \leavevmode
Note the developer version should have been tested and semi\sphinxhyphen{}stable but
not ready for full sets of processing and definitely not for release for
non\sphinxhyphen{}developers or for data put on archives. Some changes may not be
in this version that are in the working version.

Change to this branch with:

\begin{sphinxVerbatim}[commandchars=\\\{\}]
\PYG{o}{\PYGZgt{}} \PYG{n}{git} \PYG{n}{checkout} \PYG{n}{developer}
\PYG{o}{\PYGZgt{}} \PYG{n}{git} \PYG{n}{pull} \PYG{n}{origin} \PYG{n}{developer}
\end{sphinxVerbatim}

\end{description}

\item {} \begin{description}
\item[{working version}] \leavevmode
Note the working version will be the most up\sphinxhyphen{}to\sphinxhyphen{}date version but has not been
tested for stability \sphinxhyphen{} use at own risk.

Change to this branch with:

\begin{sphinxVerbatim}[commandchars=\\\{\}]
\PYG{o}{\PYGZgt{}} \PYG{n}{git} \PYG{n}{checkout} \PYG{n}{working}
\PYG{o}{\PYGZgt{}} \PYG{n}{git} \PYG{n}{pull} \PYG{n}{origin} \PYG{n}{working}
\end{sphinxVerbatim}

\end{description}

\end{itemize}


\section{Setup}
\label{\detokenize{user/installation:setup}}\label{\detokenize{user/installation:installation-setup}}

\subsection{Run the installation script}
\label{\detokenize{user/installation:run-the-installation-script}}
Change to the \sphinxtitleref{apero\sphinxhyphen{}drs} directory

Run the installation script:

\begin{sphinxVerbatim}[commandchars=\\\{\}]
\PYG{n}{python} \PYG{n}{setup}\PYG{o}{/}\PYG{n}{install}\PYG{o}{.}\PYG{n}{py}
\end{sphinxVerbatim}


\subsection{Step\sphinxhyphen{}by\sphinxhyphen{}step guide}
\label{\detokenize{user/installation:step-by-step-guide}}
Follow the step\sphinxhyphen{}by\sphinxhyphen{}step guide:
\begin{itemize}
\item {} 
A: User configuration path
\begin{quote}

This is the path where your configuration will be saved. If it doesn’t exist you will be prompted to create it. (This will be referred to as {\hyperref[\detokenize{misc/glossary:term-drs-uconfig}]{\sphinxtermref{\DUrole{xref,std,std-term}{DRS\_UCONFIG}}}}
from now on (default is \sphinxcode{\sphinxupquote{/home/user/apero/}})
\end{quote}

\item {} 
B: Instrument settings
\begin{quote}

Install {\hyperref[\detokenize{misc/glossary:term-instrument}]{\sphinxtermref{\DUrole{xref,std,std-term}{INSTRUMENT}}}}.
If yes it will install the instrument if not then it will not install the instrument. Currently only SPIRou is supported
\end{quote}

\item {} 
C: Set up paths
\begin{quote}

The first question will ask  whether to set up paths individually. If \sphinxtitleref{{[}Y{]}es}
it will allow you to set each path separately (i.e. for raw, tmp, reduced, calibDB etc). If \sphinxtitleref{{[}N{]}o}
you will just set one path and all folders (raw, tmp, reduced, calibDB etc)) will be created under this directory.
\end{quote}

\item {} 
D: Setting the directory/directories
\begin{quote}

Will prompt you to enter the directory path/paths (will ask you for each if you answered that paths be set up individually in step C above.
\end{quote}

\item {} 
E: Clean install
\begin{quote}

If you type {[}Y{]}es you will be prompted (later) to reset the directories this means any previous data in these directories will be removed. Note you can always say later to individual cases.
\end{quote}

\end{itemize}

\begin{sphinxadmonition}{warning}{Warning:}
Resetting a directory will remove all files/sub\sphinxhyphen{}directories from within these folders
\end{sphinxadmonition}

\begin{sphinxadmonition}{note}{Note:}
A to E will repeat for all installable instruments (To step up just one use the \sphinxtitleref{\textendash{}instrument} argument
\end{sphinxadmonition}


\subsection{Additional options}
\label{\detokenize{user/installation:additional-options}}
One will be prompted to give installation paths to various optional tools (currently {\hyperref[\detokenize{misc/glossary:term-ds9}]{\sphinxtermref{\DUrole{xref,std,std-term}{ds9}}}}
and {\hyperref[\detokenize{misc/glossary:term-pdflatex}]{\sphinxtermref{\DUrole{xref,std,std-term}{pdflatex}}}}
note the user will not be prompted if these were automatically found using the \sphinxtitleref{where}
command)


\section{Updating from github}
\label{\detokenize{user/installation:updating-from-github}}\label{\detokenize{user/installation:installation-update}}\begin{enumerate}
\sphinxsetlistlabels{\arabic}{enumi}{enumii}{}{.}%
\item {} 
Choose a branch (as in {\hyperref[\detokenize{user/installation:installation-choose-branch}]{\sphinxcrossref{\DUrole{std,std-ref}{Choose branch}}}})

\item {} 
Update the branch (pull from github):

\begin{sphinxVerbatim}[commandchars=\\\{\}]
\PYG{o}{\PYGZgt{}} \PYG{n}{git} \PYG{n}{pull} \PYG{n}{origin} \PYG{p}{\PYGZob{}}\PYG{n}{branch}\PYG{p}{\PYGZcb{}}
\end{sphinxVerbatim}

\item {} 
Update using the installation script:

\begin{sphinxVerbatim}[commandchars=\\\{\}]
\PYG{o}{\PYGZgt{}} \PYG{n}{python} \PYG{n}{setup}\PYG{o}{/}\PYG{n}{install}\PYG{o}{.}\PYG{n}{py} \PYG{o}{\PYGZhy{}}\PYG{o}{\PYGZhy{}}\PYG{n}{update}
\end{sphinxVerbatim}

\end{enumerate}

This will use all current settings and update the


\chapter{Using APERO}
\label{\detokenize{user/using_apero:using-apero}}\label{\detokenize{user/using_apero:id1}}\label{\detokenize{user/using_apero::doc}}
This section describes the process to use APERO.


\chapter{Known Issues}
\label{\detokenize{user/known_issues:known-issues}}\label{\detokenize{user/known_issues:id1}}\label{\detokenize{user/known_issues::doc}}
Currently known issues and problems with APERO.


\chapter{TODO}
\label{\detokenize{user/todo:todo}}\label{\detokenize{user/todo:id1}}\label{\detokenize{user/todo::doc}}
This is the currently list of items that need to still be completed.


\chapter{The recipes}
\label{\detokenize{user/recipes:the-recipes}}\label{\detokenize{user/recipes:recipes}}\label{\detokenize{user/recipes::doc}}
These are all the standard recipes to use with APERO.


\chapter{The Tools}
\label{\detokenize{user/tools:the-tools}}\label{\detokenize{user/tools:tools}}\label{\detokenize{user/tools::doc}}
These are all userful tools to use with APERO.


\chapter{Developer guide}
\label{\detokenize{index:developer-guide}}
Below is a guide for those developping APERO for the current set of instruments and for future instruments.


\section{Developer how to guide}
\label{\detokenize{dev/dev_main:developer-how-to-guide}}\label{\detokenize{dev/dev_main:dev-main}}\label{\detokenize{dev/dev_main::doc}}
The methodology and how things work in APERO.


\subsection{Adding a new constant}
\label{\detokenize{dev/adding_new_constant:adding-a-new-constant}}\label{\detokenize{dev/adding_new_constant:add-new-constant}}\label{\detokenize{dev/adding_new_constant::doc}}

\subsection{Adding a new keyword}
\label{\detokenize{dev/adding_new_keyword:adding-a-new-keyword}}\label{\detokenize{dev/adding_new_keyword:adding-new-keyword}}\label{\detokenize{dev/adding_new_keyword::doc}}

\subsection{Adding a new recipe}
\label{\detokenize{dev/adding_new_recipe:adding-a-new-recipe}}\label{\detokenize{dev/adding_new_recipe:add-new-recipe}}\label{\detokenize{dev/adding_new_recipe::doc}}

\subsection{Adding a new filetype}
\label{\detokenize{dev/adding_new_filetype:adding-a-new-filetype}}\label{\detokenize{dev/adding_new_filetype:add-new-filetype}}\label{\detokenize{dev/adding_new_filetype::doc}}

\subsection{Adding a new plot}
\label{\detokenize{dev/adding_new_plot:adding-a-new-plot}}\label{\detokenize{dev/adding_new_plot:add-new-plot}}\label{\detokenize{dev/adding_new_plot::doc}}

\chapter{Other}
\label{\detokenize{index:other}}\begin{itemize}
\item {} 
\DUrole{xref,std,std-ref}{genindex}

\item {} 
\DUrole{xref,std,std-ref}{modindex}

\item {} 
\DUrole{xref,std,std-ref}{search}

\end{itemize}


\section{Constants}
\label{\detokenize{misc/glossary:constants}}\label{\detokenize{misc/glossary:glossary}}\label{\detokenize{misc/glossary::doc}}\begin{description}
\item[{DRS\_UCONFIG\index{DRS\_UCONFIG@\spxentry{DRS\_UCONFIG}|spxpagem}\phantomsection\label{\detokenize{misc/glossary:term-drs-uconfig}}}] \leavevmode\begin{itemize}
\item {} 
The directory containing the users configurations files

\item {} 
default is \sphinxcode{\sphinxupquote{/home/user/apero/}}

\end{itemize}

\item[{INSTRUMENT\index{INSTRUMENT@\spxentry{INSTRUMENT}|spxpagem}\phantomsection\label{\detokenize{misc/glossary:term-instrument}}}] \leavevmode\begin{itemize}
\item {} 
This is the instrument used at a specific telescope. Some settings are instrument specific.

\item {} 
Currently supported instruments are::
SPIROU

\end{itemize}

\end{description}


\section{Glossary}
\label{\detokenize{misc/glossary:id1}}\begin{description}
\item[{ds9\index{ds9@\spxentry{ds9}|spxpagem}\phantomsection\label{\detokenize{misc/glossary:term-ds9}}}] \leavevmode\begin{itemize}
\item {} 
An astronomical imaging and data visualization application

\item {} 
see \sphinxhref{http://ds9.si.edu/site/Home.html}{ds9.si.edu}

\end{itemize}

\item[{pdflatex\index{pdflatex@\spxentry{pdflatex}|spxpagem}\phantomsection\label{\detokenize{misc/glossary:term-pdflatex}}}] \leavevmode\begin{itemize}
\item {} 
The pdf latex compiler

\item {} 
see \sphinxhref{https://www.latex-project.org/get/}{www.latex\sphinxhyphen{}project.org}

\end{itemize}

\end{description}



\renewcommand{\indexname}{Index}
\printindex
\end{document}