%%
%% This file can be distributed and/or modified under the
%% conditions of the LaTeX Project Public License, either version 1.3
%% of this license or (at your option) any later version.
%% The latest version of this license is in
%%   http://www.latex-project.org/lppl.txt
%% and version 1.3 or later is part of all distributions of LaTeX
%% version 2003/12/01 or later.
%% 
%% This work has the LPPL maintenance status "maintained".
%% 
%% The Current Maintainer of this work is Lars Madsen (daleif@imf.au.dk).
%%
%% $LastChangedDate: 2012-04-11 14:02:16 +0200 (Wed, 11 Apr 2012) $
%% $LastChangedRevision: 1483 $
%%
\begin{filecontents}{chapterexample.tex}
\let\clearforchapter\par % cheating, but saves some space
\chapter{A chapter title}
Nam dui ligula, fringilla a, euismod sodales, sollicitudin vel,
wisi. Morbi auctor lorem non justo. Nam lacus libero, pretium at,
lobortis vitae, ultricies et, tellus. Donec aliquet, tortor sed
accumsan bibendum, erat ligula aliquet magna, vitae ornare odio metus
a mi.
\par\fancybreak{$***$}\par
\chapter*{A non-numbered chapter title}
Nam dui ligula, fringilla a, euismod sodales, sollicitudin vel,
wisi. Morbi auctor lorem non justo. Nam lacus libero, pretium at,
lobortis vitae, ultricies et, tellus. Donec aliquet, tortor sed
accumsan bibendum, erat ligula aliquet magna, vitae ornare odio metus
a mi.  \thispagestyle{empty}
\end{filecontents}
\begin{filecontents*}{process.pl}
#!/usr/bin/perl
# licensed under the GPL, by Lars Madsen, 2008/03/25
use Getopt::Long;
my $f = '';
my $k = '';
my $p = '';
my $tmppdf  = 'tmp.pdf';
my $postfix = '-style';
GetOptions('f:s' => \$f,'k:s' => \$k,'p:s' => \$p);
my @styles = ();
my %pages  = ();
print <<END if ! $f;
Usage:

   $0 -f MemoirChapStyles.styles

END
exit if ! $f;

if ( $k ) { compile_file("$k$postfix"); exit ;}
open my $file ,'<', $f or die "Cannot open '$f': $!";
for my $l (<$file>) {
    chomp $l;
    next if $l =~ /^\s*$/;
    if ( $l =~ / page/ ) {
	($Page) = ( $l =~ / page (.*)/ ) ;
	$l =~ s/ page.*//;
	$pages{$l} = $Page;
    }
    push @styles,$l;
}
close $file;
for my $style ( @styles ) {
    compile_file($style);
}
print "done\n\n";
sub compile_file {
    my $style = shift;
    my @tmp = ();
    system("pdflatex", "$style.tex")          == 0 or warn "$!";
    system("pdfcrop", "$style.pdf","$tmppdf") == 0 or warn "$!";
    system("mv", "$tmppdf","$style.pdf")      == 0 or warn "$!";
    if ( $pages{$style} || $p ) {
	@tmp = split /\,/,  $pages{$style} ? $pages{$style} : $p ;
	for my $p ( @tmp ) {
	    system("pdftops", "-eps","-f","$p","-l","$p", "$style.pdf", "$style-$p.eps" )   == 0 or warn "$!";
	    warn "Created $style-$p.eps\n";
	}
    }
    else {
	system("pdftops", "-eps", "$style.pdf")   == 0 or warn "$!";
    }
    print "Done converting $style.pdf\n";
    return;
}
\end{filecontents*}
%$
\documentclass[a4paper,11pt,openany]{memoir}
\def\MyFileVersion{Version 1.7e, 2012/04/11}
\setlrmarginsandblock{2.5cm}{*}{1} 
\setulmarginsandblock{2.5cm}{2.5cm}{*}
\setmarginnotes{2.5mm}{2cm}{1em}
\checkandfixthelayout
\usepackage[latin1]{inputenc}
\usepackage[english]{babel}
\usepackage[T1]{fontenc}
\usepackage{
  calc,
  graphicx,
  url,
  fancyvrb,
  multicol,
  kvsetkeys
}
\usepackage[svgnames,dvipsnames]{xcolor}
\definecolor{felinesrcbgcolor}{rgb}{1,1,0.85}
\definecolor{felinesrcbgcolor}{rgb}{0.94,0.97,1}
\definecolor{felineframe}{rgb}{0.79,0.88,1}
\definecolor{myorange}{rgb}{1,0.375,0}

% \usepackage[draft]{fixme}
% \usepackage{fourier}
% \usepackage[scaled]{luximono}
\newcommand\starbreak{\fancybreak{\decosix\quad\decosix\quad\decosix}}

%\newcommand\starbreak{\fancybreak{$*\quad*\quad*$}}

\usepackage[scaled]{berasans}

\chapterstyle{ell}
\renewcommand\tocheadstart{}
\renewcommand\printtoctitle[1]{}

\raggedbottom
\fvset{frame=lines,
  framesep=3mm,
  framerule=3pt,
  fontsize=\small,
  rulecolor=\color{myorange},
  formatcom=\color{DarkGreen},
}
\newoutputstream{StyleList}
 \newoutputstream{OutputStyle}%
 \openoutputfile{\jobname.styles}{StyleList}
\def\OutputStylePostfix{-style}
\def\CurrentChapterStyle{}
\makeatletter
\kv@set@family@handler{MCS}{%
  \xdef\CurrentChapterStyle{#1}%
}
\define@key{MCS}{pages}{%\typeout{xxx: #1}
  \global\@namedef{MCS@pages@\CurrentChapterStyle}{#1}%
}
\define@key{MCS}{trim}{%\typeout{xxx: #1}
  \global\@namedef{MCS@trim@\CurrentChapterStyle}{#1}%
}
\def\defaulttrim{0 0 0 0}
\newif\ifSCS@full
\newcounter{MCS}
\newenvironment{@showchapterstyle}[1]{%
  \kvsetkeys{MCS}{#1}%
  \ifSCS@full%
    \edef\hest{\CurrentChapterStyle\OutputStylePostfix\space page \@nameuse{MCS@pages@\CurrentChapterStyle}}
    \addtostream{StyleList}{\hest}%
  \else%
    \addtostream{StyleList}{\CurrentChapterStyle\OutputStylePostfix}%
  \fi%
  \openoutputfile{\CurrentChapterStyle\OutputStylePostfix.tex}{OutputStyle}%
  \ifSCS@full%
    \addtostream{OutputStyle}{%
      \protect\let\protect\STARTCODE\relax^^J%
      \protect\let\protect\STOPCODE\relax^^J%
      \protect\STARTCODE%
    }%
  \else%
    \addtostream{OutputStyle}{%
      \protect\documentclass{memoir}^^J%
      \protect\let\protect\STARTCODE\relax^^J%
      \protect\let\protect\STOPCODE\relax^^J%
      \protect\STARTCODE%
    }%
  \fi%
  \writeverbatim{OutputStyle}%
}{%
  \endwriteverbatim\relax%
  \ifSCS@full%
    \addtostream{OutputStyle}{%
      \protect\STOPCODE%
    }
  \else% 
    \addtostream{OutputStyle}{%
      \protect\chapterstyle{\CurrentChapterStyle}^^J%
      \protect\STOPCODE^^J%
      \protect\setlength\afterchapskip{\onelineskip}^^J%
      \protect\setlength\beforechapskip{\onelineskip}^^J%
      \protect\usepackage{lipsum}^^J%
      \protect\begin{document}^^J%
      \protect\input{chapterexample.tex}^^J%
      \protect\end{document}%
    }%
  \fi%
  \closeoutputstream{OutputStyle}%
  \edef\FancyVerbStartString{\string\STARTCODE}%
  \edef\FancyVerbStopString{\string\STOPCODE}%
  \vskip\z@\@plus\bottomsectionskip
  \penalty\z@
  \vskip\z@\@plus -\bottomsectionskip
  \phantomsection
  \addcontentsline{toc}{section}{\CurrentChapterStyle}
  \VerbatimInput[label={\small Source for the \textsf{\CurrentChapterStyle} style}]{\CurrentChapterStyle-style.tex}%%
  \par\noindent%
  \IfFileExists{\CurrentChapterStyle\OutputStylePostfix.pdf}{%
    \fboxsep=4pt%
    \begin{adjustwidth}{-\fboxsep-\fboxrule}{-\fboxsep-\fboxrule}%
%      \begin{framed}%
      \@ifundefined{MCS@pages@\CurrentChapterStyle}{%
          \fcolorbox{felineframe}{felinesrcbgcolor}{%
            \includegraphics[width=\textwidth,clip]{%
              \CurrentChapterStyle\OutputStylePostfix}}%
        }{%
          \edef\nisse{\@nameuse{MCS@pages@\CurrentChapterStyle}}
          \@for\ITEM:=\nisse\do{
            \ifpdf%
              \fcolorbox{felineframe}{felinesrcbgcolor}{\includegraphics%
              [width=\textwidth,page=\ITEM,clip]%
              {\CurrentChapterStyle\OutputStylePostfix}}%
            \else%
              \fcolorbox{felineframe}{felinesrcbgcolor}{\includegraphics%
              [width=\textwidth,clip]%
              {\CurrentChapterStyle\OutputStylePostfix-\ITEM}}%
            \fi%
            \bigskip%
            \fancybreak{$***$}%
            \bigskip
          }%
        }%
%      \end{framed}%
    \end{adjustwidth}
  }{\fbox{File \CurrentChapterStyle-style.* does not exist}}
  \vskip1em plus 1em minus -.5em\noindent%
}
% the two actual environments, the stared one will let you add entire
% documents, while the unstared one will only display sniplets
\newenvironment{showchapterstyle}[1]{%
\SCS@fullfalse\@showchapterstyle{#1}}{\end@showchapterstyle}
\newenvironment{showchapterstyle*}[1]{%
\SCS@fulltrue\@showchapterstyle{#1}}{\end@showchapterstyle\SCS@fullfalse}
\newcommand\@Arg[1]{\textnormal{$\langle$\textit{#1}$\rangle$}}
\newcommand\@Args[1]{\texttt{\{\textnormal{$\langle$\textit{#1}$\rangle$}\}}}
\newcommand\Arg{\@ifstar{\@Args}{\@Arg}}
\renewcommand\cs[1]{\texttt{\textbackslash #1}}
\makeatother
\newenvironment{syntax}{%
  \vskip.5\onelineskip%
  \begin{adjustwidth}{0pt}{0pt}
    \parindent=0pt%
    \obeylines%
    \let\\=\relax%
  }{%
  \end{adjustwidth}%
  \vskip.5\onelineskip%
}
\newenvironment{syntax*}{%
  \vskip.5\onelineskip%
  \begin{adjustwidth}{0pt}{0pt}
    \parindent=0pt%
  }{%
  \end{adjustwidth}%
  \vskip.5\onelineskip%
}

\newtheorem{remark}{Remark}

\AtEndDocument{\closeoutputstream{StyleList}}
\pagestyle{plain}

\ifpdf
\usepackage[colorlinks]{hyperref}
\fi



\begin{document}

\title{Various chapter styles for the memoir class\thanks{\MyFileVersion}}
\author{Lars Madsen\thanks{Email: \protect\url{daleif@imf.au.dk}}}
\maketitle

The main idea behind this document is to demonstrate various either
contributed or inspired chapter styles for the memoir class.

If you have style you would like to contribute a style/implementation,
please send it with a minimal example to \url{daleif+memoir@imf.au.dk}
and I will include it into this document.

% \bigskip
% \starbreak

% \bigskip


\noindent The visual examples you will find later in this document
have all been made using external documents and included as images
(eps or pdf). As such, all images are scaled to have the same width as
the  text in this document, therefore some images are scaled down.

Also, please do not trust the spacing between the chapter title and
the start of the following text. This \verb+\afterchapskip+-spacing is
silently reduced (to \verb+\onelineskip+) in order to save space, the
same goes for \verb+\beforechapskip+.

% \starbreak

In any good chapter style design one should have given a thought at
both the normal numbered style as well as the unnumbered
style. Therefore the example text features both a numbered chapter and
an unnumbered. (I have relaxed \verb+\clearforchapter+ in order to
have both on the same side.)

The sample text used is
\VerbatimInput[
label={chapterexample.tex},
fontsize=\small
]{chapterexample.tex}

% \starbreak

If you want to use one of the styles presented in this document, 
then there is no need to start retyping it all your self. Simply
download the source for this document (\texttt{\jobname.tex}) from
\url{http://mirror.ctan.org/info/MemoirChapStyles/}. Run it
once through \LaTeX, then you will 
get a file called \Arg{Name of style}\texttt{-style.tex}, which is the
source code for example displaying that particular style. Then just
copy the code from there.

Please note that in the code you will find stuff like
\begin{Verbatim}
\let\STARTCODE\relax 
\let\STOPCODE\relax 
\STARTCODE
...
\STOPCODE  
\end{Verbatim}
This code can just be removed, it is used by this document as an easy
manner to include certain parts of a given source code.

\section*{Acknowledgement}

Acknowledgement goes (of course) to Peter Wilson for creating the
memoir class in the first place. But also to the people who
contributed with styles or comments: Danie Els, David Chadd, Pluton
(name used on \textsc{ctt}), Erik Quaeghebeur, Donald Arseneau plus
the those who posted memoir chapter styles on news groups, I hope it
is okay that I include them here.




\section*{TODO}
\label{sec:todo}

Have a look at the chapter styles offered by \texttt{fncychap} and
\texttt{titlesec}. 


\newpage

\chapter*{\contentsname}


\setlength\columnsep{8mm}
\begin{multicols}{2}
  \tableofcontents*
\end{multicols}

\newpage

\chapter{A little background}
\label{cha:little-background}

As you might already know the memoir class includes a feature to
switch the look and feel of a chapter title on a chapter to chapter
basis. This is achieved by using \verb+\chapterstyle+\Arg*{style}. The
most extreme use of this is seen in \emph{The Memoir class For
  Configurable Typesetting -- User Guide} by Peter Wilson, also know
as the \emph{Memoir manual}, \cite{memman}.


In general, \LaTeX\ classes use \verb+\@makechapterhead+ to print a
chapter title specified my \verb|chapter|, and
\verb+\@makeschapterhead+ for \verb+\chapter*+. In memoir Peter Wilson
made these two macros a bit more flexible than usual. The idea is
that for numbered chapters (i.e. \verb+\chapter+ and
$\texttt{secnumdepth}\geq\nobreak 0$) one should think of the chapter title as
build by:
\begin{Verbatim}
\chapterheadstart
\printchaptername \chapternamenum \printchapternum
\afterchapternum
\printchaptertitle{The title}  
\afterchaptertitle
\end{Verbatim}
For unnumbered (i.e. \verb+\chapter*+ and \verb+\chapter+ width
$\texttt{secnumdepth}<0$): 
\begin{Verbatim}
\chapterheadstart
\printchapternonum
\printchaptertitle{The title}
\afterchaptertitle  
\end{Verbatim}
Note that \verb+\printchaptertitle+ is the only macro that takes an
argument. At the start of every memoir chapter style these macros are
initialised to
\begin{Verbatim}
\renewcommand\chapterheadstart{\vspace*{\beforechapskip}}
\renewcommand\printchaptername{\chapnamefont \@chapapp}
\renewcommand\chapternamenum{\space}
\renewcommand\printchapternum{\chapnumfont \thechapter}
\renewcommand\afterchapternum{\par\nobreak\vskip \midchapskip}
\renewcommand\printchapternonum{}
\renewcommand\printchaptertitle[1]{\chaptitlefont #1}
\renewcommand\afterchaptertitle{\par\nobreak\vskip \afterchapskip}
\end{Verbatim}
In the design og these styles it is also worth noting that the
contents of \verb|\afterchapternum| is not included within
\verb|\printchapternonum|, which become important when designing the
non-numbered part of the style. If one does not remember to add say 
\verb|\par\nobreak\vskip \midchapskip| or similar to
\verb|\printchapternonum| the title part may be placed differently
when there is no number.

One might ask what \verb+\printchapternonum+ is good for when it is
always initialised to nothing. Well if a design need to one could use
it to insert a phantom width as wide as the chapter name plus number
would have been. If on the other hand one is creating a style where
the chapter name and number is actually typeset using
\verb+\printchaptertitle+ (like a framed one) then one could first
define a new if construction, say, \verb+\ifNoChapNum+ and then let
\verb+\printchapternonum+ set this to true and so on.


So one just have to change the ones one need. There are a few other
macros that are nice to know the meaning of. Remember that these are
\emph{not} reset at the start of a new chapter style.
\begingroup
\renewcommand\descriptionlabel[1]{\hspace\labelsep\cs{#1}}
\begin{description}\firmlist
\item[beforechapskip] length, self explanatory,usually set using
  \verb+\chapterheadstart+, default 50pt
\item[midchapskip] length, distance between the chapter name / number and the
title, usually set using \verb+\afterchapternum+, default 20pt
\item[afterchapskip] length, distance between the chapter title and
  the following text, usually set using \verb+\afterchaptertitle+,
  default 40pt
\item[chapnamefont] the font setting used for \emph{Chapter} or
  similar, default \verb+\normalfont\huge\bfseries+
\item[chapnumfont] same for the chapter number, default
  \verb+\normalfont\huge\bfseries+
\item[chaptitlefont] same for the chapter title, default
  \verb+\normalfont\Huge\bfseries+ 
\end{description}
\endgroup

In memoir a new chapter style is defined as
\begin{syntax}
\cs{makechapterstyle}\Arg*{name}\texttt{\{}  
\Arg{code}
\texttt{\}}
\end{syntax}
Where \Arg{code} is redefinitions of the macros mentioned
above. (Remember that if you redefine\linebreak \verb+\printchaptertitle+ then
you have to use \texttt{\#\#1} to represent the title.)
Activating a given style is done by simply issuing
\begin{syntax}
  \cs{chapterstyle}\Arg*{name}
\end{syntax}
By the way, if you happen to like a given style but wanted to, say,
add color to the chapter title, you could just refine
\verb+\chaptitlefont+ after you have issued \verb+\chapterstyle+. (Even
simpler to just use \verb+\addtodef\chaptitlefont{}{\color{nicered}}+.)

As a simple example, here is the code for the \texttt{section} chapter
style
\begin{Verbatim}[label={Source code for the \textsf{section} chapter style}]
\makechapterstyle{section}{%
  \renewcommand{\printchaptername}{}
  \renewcommand{\chapternamenum}{}
  \renewcommand{\chapnumfont}{\normalfont\Huge\bfseries}
  \renewcommand{\printchapternum}{\chapnumfont \thechapter\space}
  \renewcommand{\afterchapternum}{}
}
\end{Verbatim}



\clearpage

\chapter{Styles included in memoir}
\label{cha:defa-styl-incl}

First we have the six default chapterstyles in the memoir class. The
source code for these can be found in \texttt{memoir.cls}.


\begin{showchapterstyle}{default}
\end{showchapterstyle}

\newpage

\begin{showchapterstyle}{section}
\end{showchapterstyle}
\begin{showchapterstyle}{article}
\end{showchapterstyle}

\newpage

\begin{showchapterstyle}{reparticle}
\end{showchapterstyle}
\begin{showchapterstyle}{hangnum}
\end{showchapterstyle}

\newpage

\begin{showchapterstyle}{companion}
\end{showchapterstyle}
\begin{showchapterstyle}{demo}
\end{showchapterstyle}
If you want to use this in a different lanugage, then that a look at
the implementation in the memoir source. You will need your own
version of \cs{numtoName}

\begin{showchapterstyle}{bianchi}
\end{showchapterstyle}


\begin{showchapterstyle}{bringhurst}
\end{showchapterstyle}
\begin{showchapterstyle}{brotherton}
\end{showchapterstyle}
As with \texttt{demo}, you will need to define your own suitable
version of \cs{numtoName}


\newpage

\begin{showchapterstyle}{chappell}
\end{showchapterstyle}
\begin{showchapterstyle}{culver}
\end{showchapterstyle}
\begin{showchapterstyle}{dash}
\end{showchapterstyle}

\newpage

\begin{showchapterstyle}{demo2}
\end{showchapterstyle}

\newpage

\begin{showchapterstyle}{demo3}
\end{showchapterstyle}

\newpage

\begin{showchapterstyle}{ell}
\end{showchapterstyle}

\newpage

\begin{showchapterstyle}{ger}
\end{showchapterstyle}

\newpage

\begin{showchapterstyle}{lyhne}
\usepackage{graphicx}
\end{showchapterstyle}

\newpage

\begin{showchapterstyle}{madsen}
\usepackage{graphicx}
\end{showchapterstyle}

\newpage

\begin{showchapterstyle}{pedersen}
\usepackage{color,graphicx}
\definecolor{ared}{rgb}{.647,.129,.149}
\renewcommand\colorchapnum{\color{ared}}
\renewcommand\colorchaptitle{\color{ared}}
\end{showchapterstyle}

\newpage

\begin{showchapterstyle}{southall}
\end{showchapterstyle}
\begin{showchapterstyle}{thatcher}
\end{showchapterstyle}
\begin{showchapterstyle}{veelo}
\usepackage{graphicx}
\end{showchapterstyle}

The idea behind \emph{veelo} is that the black marker is printed to
the edge of the page and one is then able to see the star of the
chapter by just looking at the edge of the book.

\begin{showchapterstyle}{verville}
\end{showchapterstyle}

\newpage

\begin{showchapterstyle}{crosshead}
\end{showchapterstyle}
\begin{showchapterstyle}{dowding}
\end{showchapterstyle}
\begin{showchapterstyle}{komalike}
\end{showchapterstyle}
\begin{showchapterstyle}{ntglike}
\end{showchapterstyle}

\newpage

\begin{showchapterstyle}{tandh}
\end{showchapterstyle}
\begin{showchapterstyle}{wilsondob}
\end{showchapterstyle}






\chapter{Found or contributed styles}

Please note that most of the styles that were mentioned here in
earlier versions of this document, are now a part of memoir and
therefore removed.


%
By Alexander Grebenkov 2004/11/25, found via Google Groups on fido.ru.tex.
\begin{showchapterstyle}{AlexanderGrebenkov}
   \makechapterstyle{AlexanderGrebenkov}{%
  \renewcommand{\chapterheadstart}{\vspace*{\beforechapskip}\hrule\medskip}
  \renewcommand{\chapnamefont}{\normalfont\large\scshape}
  \renewcommand{\chapnumfont}{\normalfont\large\scshape}
  \renewcommand{\chaptitlefont}{\normalfont\large\scshape}
  \renewcommand{\printchaptername}{\S}
  \renewcommand{\chapternamenum}{ }
  \renewcommand{\printchapternum}{\chapnumfont \thechapter}
  \renewcommand{\afterchapternum}{. }
  \renewcommand{\afterchaptertitle}{\par\nobreak\medskip\hrule\vskip
\afterchapskip}
} 
\end{showchapterstyle}

\clearpage


\begin{showchapterstyle}{daleif1}
\usepackage{color,calc,graphicx,soul,fourier}
\definecolor{nicered}{rgb}{.647,.129,.149}
\makeatletter
\newlength\dlf@normtxtw
\setlength\dlf@normtxtw{\textwidth}
\def\myhelvetfont{\def\sfdefault{mdput}}
\newsavebox{\feline@chapter}
\newcommand\feline@chapter@marker[1][4cm]{%
  \sbox\feline@chapter{%
    \resizebox{!}{#1}{\fboxsep=1pt%
      \colorbox{nicered}{\color{white}\bfseries\sffamily\thechapter}%
    }}%
  \rotatebox{90}{%
    \resizebox{%
      \heightof{\usebox{\feline@chapter}}+\depthof{\usebox{\feline@chapter}}}%
    {!}{\scshape\so\@chapapp}}\quad%
  \raisebox{\depthof{\usebox{\feline@chapter}}}{\usebox{\feline@chapter}}% 
}
\newcommand\feline@chm[1][4cm]{%
  \sbox\feline@chapter{\feline@chapter@marker[#1]}%
  \makebox[0pt][l]{% aka \rlap
    \makebox[1cm][r]{\usebox\feline@chapter}%
  }}
\makechapterstyle{daleif1}{
  \renewcommand\chapnamefont{\normalfont\Large\scshape\raggedleft\so}
  \renewcommand\chaptitlefont{\normalfont\huge\bfseries\scshape\color{nicered}}
  \renewcommand\chapternamenum{}
  \renewcommand\printchaptername{}
  \renewcommand\printchapternum{\null\hfill\feline@chm[2.5cm]\par}
  \renewcommand\afterchapternum{\par\vskip\midchapskip}
  \renewcommand\printchaptertitle[1]{\chaptitlefont\raggedleft ##1\par}
}
\makeatother
\end{showchapterstyle}

%
Style build upon \texttt{VZ15b}, see later.
\begin{showchapterstyle}{{daleif3}}
\usepackage{fourier}
\makeatletter
\newif\iffelinenonum
\newcommand\MyNumToName[1]{%
  \ifcase#1\relax % case 0
  \or First\or Second\or Third%
  \else Not implemented\fi}
\makechapterstyle{daleif3}{
  \renewcommand\chapternamenum{}
  \renewcommand\printchaptername{}
  \renewcommand\chapnamefont{\small\itshape\centering} 
  \setlength\midchapskip{7pt}
  \renewcommand\printchapternum{%
    \par\chapnamefont\decofourleft\enspace%
    \ifanappendix
    \appendixname\space\thechapter%
    \else%
    \MyNumToName{\thechapter}\space\chaptername%
    \fi%
    \/\enspace\decofourright}
  \renewcommand\printchapternonum{\par\felinenonumtrue}
  \renewcommand\chaptitlefont{\huge\itshape\centering}
  \renewcommand\afterchapternum{%
    \par\nobreak\vskip-5pt%
  }
  \renewcommand\afterchaptertitle{%
    \par\vskip-2\midchapskip%
    \rule\textwidth\normalrulethickness
    \felinenonumfalse
    \nobreak\vskip\afterchapskip%
  }
}
\makeatother
\end{showchapterstyle}




Danie Els contributed the following style along with the BlueBox style
on page \pageref{BlueBox}.
\begin{showchapterstyle}{GreyNum}
\usepackage{fix-cm}
\usepackage{fourier}%................... Roman+math - Utopia
\usepackage[scaled=.92]{helvet}%........ Sans serif - Helvetica
\usepackage[T1]{fontenc}
\usepackage{color}
\definecolor{ChapGrey}{rgb}{0.6,0.6,0.6}
\newcommand{\LargeFont}{% Needs a 'stretchable' font
      \usefont{\encodingdefault}{\rmdefault}{b}{n}%
      \fontsize{60}{80}\selectfont\color{ChapGrey}}
\makeatletter
\makechapterstyle{GreyNum}{%
  \renewcommand{\chapnamefont}{\large\sffamily\bfseries\itshape}
  \renewcommand{\chapnumfont}{\LargeFont}
  \renewcommand{\chaptitlefont}{\Huge\sffamily\bfseries\itshape}
  \setlength{\beforechapskip}{0pt}
  \setlength{\midchapskip}{40pt}
  \setlength{\afterchapskip}{60pt}
  \renewcommand\chapterheadstart{\vspace*{\beforechapskip}}
  \renewcommand\printchaptername{%
    \begin{tabular}{@{}c@{}}
      \chapnamefont \@chapapp\\}
    \renewcommand\chapternamenum{\noalign{\vskip 2ex}}
    \renewcommand\printchapternum{\chapnumfont\thechapter\par}
    \renewcommand\afterchapternum{%
    \end{tabular}
    \par\nobreak\vskip\midchapskip}
  \renewcommand\printchapternonum{}
  \renewcommand\printchaptertitle[1]{%
    {\chaptitlefont{##1}\par}}
  \renewcommand\afterchaptertitle{\par\nobreak\vskip \afterchapskip}
}
\makeatother
\end{showchapterstyle}
Danie notes:
\begin{adjustwidth}{1em}{0pt}
  \itshape
  This looks a lot better with real italics sans-serif
  fonts such as Lucida Sans\\
  \verb|\usepackage[expert,vargreek]{lucidabr}%.. Lucida Bright + Expert (commercial)|
  \\
  or Myrad\\
  \verb|\usepackage{charter}%........... Roman      - Charter|\\
  \verb|\renewcommand{\sfdefault}{fmy}%. Sans serif - Myrad (Springer bundle)|
\end{adjustwidth}

% \starbreak

\newpage

This next style is inspired by a mail I recieved from Erik
Quaeghebeur. It took me a little while to actually get this working as
I wanted it to, partly because apparently there is a small issue
regarding \cs{thispagestyle} and \cs{pagestyle} as to which
\cs{chaptermark} gets used (I got around this by using the
\texttt{afterpage} package). This style is designed to be used with
\texttt{openleft} (i.e. chapters starting on even pages). And since
the design uses pagestyles, we need to show several seperate pages.


\begin{showchapterstyle*}{EQ,pages={2,4,6}}
\documentclass[openleft]{memoir}
\usepackage{calc}
\usepackage{afterpage}
\copypagestyle{EQ-pagestyle}{companion}
\setlength{\headwidth}{\textwidth}
\addtolength{\headwidth}{.382\foremargin}
\makerunningwidth{EQ-pagestyle}{\headwidth}
\makeheadposition{EQ-pagestyle}{flushright}{flushleft}{}{}
\makeevenhead{EQ-pagestyle}{\normalfont\bfseries\thepage}{}{\normalfont\bfseries\leftmark}
\makeoddhead{EQ-pagestyle}{\normalfont\bfseries\rightmark}{}{\normalfont\bfseries\thepage}
\newif\ifNoChapNum
\makeatletter
% chapterpage layout
\copypagestyle{EQ-chapterstyle}{EQ-pagestyle}
\makeheadposition{EQ-chapterstyle}{flushright}{flushleft}{}{}
\makeevenhead{EQ-chapterstyle}{%
  \normalfont\bfseries\thepage}{}{%
  \ifnum \c@secnumdepth>\m@ne%
    \ifNoChapNum%
      \raisebox{-4.5pt}[0pt][0pt]{\chapnamefont \rightmark}%
    \else%
      \raisebox{-4.5pt}[0pt][0pt]{\chapnamefont\@chapapp\ \thechapter}%
    \fi%
  \else%
    \raisebox{-4.5pt}[0pt][0pt]{\chapnamefont\rightmark}%
  \fi%
  }
\makeoddhead{EQ-chapterstyle}{\rightmark}{}{\normalfont\bfseries\thepage}
% build in the shorter headline
\@namedef{EQ-chapterstyleheadrule}{%
  \ifnum \c@secnumdepth>\m@ne%
    \ifNoChapNum%
      \settowidth\@tempdimc{\quad\chapnamefont\rightmark}%
    \else%
      \settowidth\@tempdimc{\quad\chapnamefont\@chapapp\ \thechapter}%
    \fi%
  \else%
  \settowidth\@tempdimc{\quad\chapnamefont\rightmark}%
  \fi%
  \setlength\@tempdimc{\headwidth-\@tempdimc}%
  \hrule\@width \@tempdimc\@height \normalrulethickness \vskip-\normalrulethickness%
}
\aliaspagestyle{chapter}{EQ-chapterstyle}
\pagestyle{EQ-pagestyle}
\makechapterstyle{EQ}{
  \renewcommand{\chapnamefont}{\raggedleft\bfseries\huge} 
  \renewcommand{\chapternamenum}{}
  \renewcommand\printchaptername{}
  \renewcommand\printchapternum{}
  \renewcommand\printchaptertitle[1]{%
    \ifnum \c@secnumdepth>\m@ne%
    \ifNoChapNum\else\chaptitlefont ##1\fi%
    \fi%
    \ifNoChapNum%
    \markboth{##1}{##1}%
    \fi%
    \afterpage{\global\NoChapNumfalse}%
  }
  \renewcommand\afterchapternum{}
  \renewcommand\afterchaptertitle{%
    \ifnum \c@secnumdepth>\m@ne%
    \ifNoChapNum\else\par\nobreak\vskip\afterchapskip\fi%
    \fi}
  \setlength\beforechapskip{15pt}
  \renewcommand\printchapternonum{\global\NoChapNumtrue}
  \renewcommand{\chaptitlefont}{\raggedleft\normalfont\Huge\bfseries}
}
\makeatother
\chapterstyle{EQ}
\begin{document}
\frontmatter
\chapter{Preface}

Some text at the beginning of a chapter. And we add a lot of text to
make sure that it spans more than one line.

\mainmatter

\chapter{A chapter title}
Some text at the beginning of a chapter. And we add a lot of text to
make sure that it spans more than one line.

\chapter*{A non-numbered chapter title}
Some text at the beginning of a chapter. And we add a lot of text to
make sure that it spans more than one line.

\end{document}
\end{showchapterstyle*}
Remember that the line you see is actually the header.

\newpage

\noindent
This next style is a modified version of a style requested on a danish
forum. 

\begin{showchapterstyle}{jenor}
\usepackage{xcolor,fix-cm}
\definecolor{numbercolor}{gray}{0.7}
\newif\ifchapternonum
\makechapterstyle{jenor}{
  \renewcommand\printchaptername{}
  \renewcommand\printchapternum{}
  \renewcommand\printchapternonum{\chapternonumtrue}
  \renewcommand\chaptitlefont{\fontfamily{pbk}\fontseries{db}%
    \fontshape{n}\fontsize{25}{35}\selectfont\raggedleft}
  \renewcommand\chapnumfont{\fontfamily{pbk}\fontseries{m}\fontshape{n}%
    \fontsize{1in}{0in}\selectfont\color{numbercolor}}
  \renewcommand\printchaptertitle[1]{%
    \noindent%
    \ifchapternonum%
    \begin{tabularx}{\textwidth}{X}%
    {\parbox[b]{\linewidth}{\chaptitlefont ##1}%
      \vphantom{\raisebox{-15pt}{\chapnumfont 1}}}
    \end{tabularx}%
    \else
    \begin{tabularx}{\textwidth}{Xl}
    {\parbox[b]{\linewidth}{\chaptitlefont ##1}} 
    & \raisebox{-15pt}{\chapnumfont \thechapter}%
    \end{tabularx}%
    \fi
    \par\vskip2mm\hrule
  }
} 
\end{showchapterstyle}


This next style originates from
\url{http://texblog.net/latex-archive/layout/fancy-chapter-tikz/} and
thus makes this interesting style available for memoir users.
\begin{showchapterstyle*}{texblogtikz,pages={1,3}}
\documentclass{memoir}
\usepackage[svgnames]{xcolor}
\usepackage{tikz}
% helper macros
\newcommand{\ChapWithNumber}[1]{
\begin{tikzpicture}[remember picture,overlay]
    \node[yshift=-3cm] at (current page.north west)
      {\begin{tikzpicture}[remember picture, overlay]
        \draw[fill=LightSkyBlue] (0,0) rectangle
          (\stockwidth,3cm);
        \node[anchor=east,xshift=.9\stockwidth,rectangle,
              rounded corners=20pt,inner sep=11pt,
              fill=MidnightBlue]
              {\color{white}\chapnamefont\thechapter\space #1};
       \end{tikzpicture}
      };
   \end{tikzpicture}
}
\newcommand{\ChapWithoutNumber}[1]{
  \begin{tikzpicture}[remember picture,overlay]
    \node[yshift=-3cm] at (current page.north west)
    {\begin{tikzpicture}[remember picture, overlay]
        \draw[fill=LightSkyBlue] (0,0) rectangle
        (\stockwidth,3cm);
        \node[anchor=east,xshift=.9\stockwidth,rectangle,
        rounded corners=20pt,inner sep=11pt,
        fill=MidnightBlue]
        {\color{white}\chapnamefont#1};
      \end{tikzpicture}
    };
  \end{tikzpicture}
}
\newif\ifnumberedchap
\numberedchaptrue
\makechapterstyle{texblogtikz}{
  \renewcommand\chapnamefont{\normalfont\sffamily\Huge\bfseries}
  \renewcommand\chapnumfont{\normalfont\sffamily\Huge\bfseries}
  \renewcommand\chaptitlefont{\normalfont\sffamily\Huge\bfseries}
  \renewcommand\chapternamenum{}
  \renewcommand{\afterchapternum}{}
  \renewcommand\printchaptername{}
  \renewcommand\printchapternum{}
  \renewcommand\printchapternonum{\global\numberedchapfalse}
  \renewcommand\printchaptertitle[1]{%
    \ifnumberedchap
    \ChapWithNumber{##1}
    \else
    \ChapWithoutNumber{##1}
    \fi
    \global\numberedchaptrue
  }
}
\chapterstyle{texblogtikz}
\aliaspagestyle{chapter}{empty} % just to save some space
\begin{document}
\chapter{A chapter title}
Lorem ipsum dolor sit amet, consectetuer adipiscing elit. Ut purus
elit, vestibulum ut, placerat ac, adipiscing vitae, felis. Curabitur
dictum gravida mauris. Nam arcu libero, nonummy eget, consectetuer id,
vulputate a, magna.


\chapter*{A non-numbered chapter title}
Nam dui ligula, fringilla a, euismod sodales, sollicitudin vel,
wisi. Morbi auctor lorem non justo. Nam lacus libero, pretium at,
lobortis vitae, ultricies et, tellus. Donec aliquet, tortor sed
accumsan bibendum, erat ligula aliquet magna, vitae ornare odio metus
a mi.  

\end{document}
\end{showchapterstyle*}

In reality, the blue box, covers the with of the paper, but do not
extend to the top op the paper, leaving a white ribbon.


% \starbreak

Here is a variation over the \texttt{texblogtikz} style, provided by
Verliya Gadis. Note that the text is being cut off on the right due to
the process used to create the sample images..

\begingroup

\setkeys{Gin}{trim=0 16cm 0 0}

\begin{showchapterstyle*}{verly,pages={1,3}}
\documentclass{memoir}
\setlrmarginsandblock{6cm}{3cm}{*}
\checkandfixthelayout
\usepackage[svgnames]{xcolor}
\usepackage{tikz}
% helper macros
\newcommand{\ChapWithNumber}[1]{
  \begin{tikzpicture}[remember picture,overlay]
    \node[yshift=-3cm] at (current page.north west)
    {\begin{tikzpicture}[remember picture, overlay]
	\draw[fill=gray!30!white] (0,-26) rectangle (5,5)
        (\stockwidth,3cm);
        \node[anchor=north,xshift=6cm,rectangle,
	rounded corners=20pt,inner sep=11pt,
	fill=gray]
	{\color{white}\chapnamefont\thechapter\space #1};
      \end{tikzpicture}};
  \end{tikzpicture}}
\newcommand{\ChapWithoutNumber}[1]{
  \begin{tikzpicture}[remember picture,overlay]
    \node[yshift=-3cm] at (current page.north west)
    {\begin{tikzpicture}[remember picture, overlay]
        \draw[fill=gray!30!white] (0,-26) rectangle (5,5)
        (\stockwidth,3cm);
        \node[anchor=north,xshift=6cm,rectangle,
        rounded corners=20pt,inner sep=11pt,
        fill=gray]{\color{white}\chapnamefont#1};
      \end{tikzpicture}};
  \end{tikzpicture}}
\newif\ifnumberedchap
\numberedchaptrue
\makechapterstyle{verly}{
  \renewcommand\chapnamefont{\normalfont\sffamily\Huge\bfseries}
  \renewcommand\chapnumfont{\normalfont\sffamily\Huge\bfseries}
  \renewcommand\chaptitlefont{\normalfont\sffamily\Huge\bfseries}
  \renewcommand\chapternamenum{}
  \renewcommand{\afterchapternum}{}
  \renewcommand\printchaptername{}
  \renewcommand\printchapternum{}
  \renewcommand\printchapternonum{\global\numberedchapfalse}
  \renewcommand\printchaptertitle[1]{%
    \ifnumberedchap\ChapWithNumber{##1}\else\ChapWithoutNumber{##1}\fi
    \global\numberedchaptrue
  }
}
\chapterstyle{verly}
\aliaspagestyle{chapter}{empty} % just to save some space
\begin{document}
\chapter{A chapter title}
Lorem ipsum dolor sit amet, consectetuer adipiscing elit. Ut purus
elit, vestibulum ut, placerat ac, adipiscing vitae, felis. Curabitur
dictum gravida mauris. Nam arcu libero, nonummy eget, consectetuer id,
vulputate a, magna.


\chapter*{A non-numbered chapter title}
Nam dui ligula, fringilla a, euismod sodales, sollicitudin vel,
wisi. Morbi auctor lorem non justo. Nam lacus libero, pretium at,
lobortis vitae, ultricies et, tellus. Donec aliquet, tortor sed
accumsan bibendum, erat ligula aliquet magna, vitae ornare odio metus
a mi.  

\end{document}
\end{showchapterstyle*}

\endgroup


% \starbreak

I made this next style for a book project to be published in Aarhus
University Press. The author requested a style without the word
\emph{Chapter}, and I'd like to do something slightly different,
something that might be useful for Bachelors projects and other
similar student projects. In this particular book, chapter titles are
at most two lines. So the design is made such that if it is two lines
then the second line stand on the same baseline as the number.

As it is for a book project the specific design depends on the font
used, the font size, and the line spread. 

Since this is a style without the word \emph{Chapter}, it is of course
a good idea to remove this from the headers. Here is an easy example
showing how to do this with the default page style:
\begin{verbatim}
\addtopsmarks{heading}{}{%
  \createmark{chapter}{both}{shownumber}{}{. \ }
}
\pagestyle{headings}
\end{verbatim}


\begin{showchapterstyle*}{hansen}
\documentclass[12pt]{memoir}
\usepackage[T1]{fontenc}
\usepackage{kpfonts}
\setSingleSpace{1.1}
\SingleSpacing
\usepackage{xcolor,calc}

\definecolor{chaptercolor}{gray}{0.8}
% helper macros
\newcommand\numlifter[1]{\raisebox{-2cm}[0pt][0pt]{\smash{#1}}}
\newcommand\numindent{\kern37pt}
\newlength\chaptertitleboxheight

\makechapterstyle{hansen}{
  \renewcommand\printchaptername{\raggedleft}
  \renewcommand\printchapternum{%
    \begingroup%
    \leavevmode%
    \chapnumfont%
    \strut%
    \numlifter{\thechapter}%
    \numindent%
    \endgroup%
  }
  \renewcommand*{\printchapternonum}{%
    \vphantom{\begingroup%
      \leavevmode%
      \chapnumfont%
      \numlifter{\vphantom{9}}%
      \numindent%
      \endgroup}
    \afterchapternum}
  \setlength\midchapskip{0pt}
  \setlength\beforechapskip{0.5\baselineskip}
  \setlength{\afterchapskip}{3\baselineskip}
  \renewcommand\chapnumfont{%
    \fontsize{4cm}{0cm}%
    \bfseries%
    \sffamily%
    \color{chaptercolor}%
  }
  \renewcommand\chaptitlefont{%
    \normalfont%
    \huge%
    \bfseries%
    \raggedleft%
  }%
  \settototalheight\chaptertitleboxheight{%
    \parbox{\textwidth}{\chaptitlefont \strut bg\\bg\strut}}
  \renewcommand\printchaptertitle[1]{%
    \parbox[t][\chaptertitleboxheight][t]{\textwidth}{%
      %\microtypesetup{protrusion=false}% add this if you use microtype
      \chaptitlefont\strut ##1\strut}%
  }
}
\chapterstyle{hansen}
\aliaspagestyle{chapter}{empty} % just to save some space
\begin{document}
\let\clearforchapter\par % cheating, but saves some space
\chapter{A chapter title}
Nam dui ligula, fringilla a, euismod sodales, sollicitudin vel,
wisi. Morbi auctor lorem non justo. Nam lacus libero, pretium at,
lobortis vitae, ultricies et, tellus. Donec aliquet, tortor sed
accumsan bibendum, erat ligula aliquet magna, vitae ornare odio metus
a mi.
\par\fancybreak{$***$}\par
\chapter*{A non-numbered chapter title}
Nam dui ligula, fringilla a, euismod sodales, sollicitudin vel,
wisi. Morbi auctor lorem non justo. Nam lacus libero, pretium at,
lobortis vitae, ultricies et, tellus. Donec aliquet, tortor sed
accumsan bibendum, erat ligula aliquet magna, vitae ornare odio metus
a mi.
\end{document}
\end{showchapterstyle*}






\chapter{Vincent Zoonekynd}
\label{sec:vincent-zoonekynd}

Some time ago Vincent Zoonekynd published a long list of general
chapter styles for \LaTeX, see
\url{http://zoonek.free.fr/LaTeX/LaTeX_samples_chapter/0.html}. 
In this section we implement several of these styles. Special thanks
to Danie Els for the BlueBox style (aka VZ39).

The styles are named after Vincent Zoonekynd (VZ) and the number on
the mentioned page.
\begin{showchapterstyle}{VZ14}
\makeatletter
\newcommand\thickhrulefill{\leavevmode \leaders \hrule height 1ex \hfill \kern \z@}
\setlength\midchapskip{10pt}
\makechapterstyle{VZ14}{
  \renewcommand\chapternamenum{}
  \renewcommand\printchaptername{}
  \renewcommand\chapnamefont{\Large\scshape}
  \renewcommand\printchapternum{%
    \chapnamefont\null\thickhrulefill\quad
    \@chapapp\space\thechapter\quad\thickhrulefill}
  \renewcommand\printchapternonum{%
    \par\thickhrulefill\par\vskip\midchapskip
    \hrule\vskip\midchapskip
  }
  \renewcommand\chaptitlefont{\Huge\scshape\centering}
  \renewcommand\afterchapternum{%
    \par\nobreak\vskip\midchapskip\hrule\vskip\midchapskip}
  \renewcommand\afterchaptertitle{%
    \par\vskip\midchapskip\hrule\nobreak\vskip\afterchapskip}
}
\makeatother
\end{showchapterstyle}
Variation over VZ15.
\begin{showchapterstyle}{VZ15b}
\usepackage{pifont,graphicx}
\newcommand\mylleaf{\ding{'247}}
\newcommand\myrleaf{\reflectbox{\mylleaf}}
\newcommand\MyNumToName[1]{%
  \ifcase#1\relax % case 0
  \or First\or Second\or Third%
  \else Not implemented\fi}
\makeatletter
\setlength\midchapskip{10pt}
\makechapterstyle{VZ15b}{
  \renewcommand\chapternamenum{}
  \renewcommand\printchaptername{}
  \renewcommand\chapnamefont{\Large\scshape}
  \renewcommand\printchapternum{%
    \chapnamefont\null\hfill\mylleaf\quad
    \MyNumToName{\thechapter}\space\@chapapp\quad\myrleaf\hfill\null}
  \renewcommand\printchapternonum{%
    \par\hrule\vskip\midchapskip}
  \renewcommand\chaptitlefont{\Huge\scshape\centering}
  \renewcommand\afterchapternum{%
    \par\nobreak\vskip\midchapskip\hrule\vskip\midchapskip}
  \renewcommand\afterchaptertitle{%
    \par\vskip\midchapskip\hrule\nobreak\vskip\afterchapskip}
}
\makeatother
\end{showchapterstyle}
Though I believe this style would look better without the lines. 

Variation over VZ21. Note the use of two different tabulars depending
upon the length of the title. Also note that we use the build-in
booktabs rules, and note that the thickness of these rules can be
individually adjusted. 
\begin{showchapterstyle}{VZ21}
\usepackage{calc,fourier}
\usepackage[T1]{fontenc}
\makeatletter
\setlength\midchapskip{7pt}
\makechapterstyle{VZ21}{
  \renewcommand\chapnamefont{\Large\scshape}
  \renewcommand\chapnumfont{\Large\scshape\centering}
  \renewcommand\chaptitlefont{\huge\bfseries\centering}
  \renewcommand\printchaptertitle[1]{%
    \setlength\tabcolsep{7pt}% used as indentation on both sides
    \settowidth\@tempdimc{\chaptitlefont ##1}%
    \setlength\@tempdimc{\textwidth-\@tempdimc-2\tabcolsep}%
    \chaptitlefont
    \ifdim\@tempdimc > 0pt\relax% one line
    \begin{tabular}{c}
      \toprule  ##1\\ \bottomrule
    \end{tabular}
    \else% two+ lines
    \begin{tabular}{%
        >{\chaptitlefont\arraybackslash}p{\textwidth-2\tabcolsep}}
      \toprule ##1\\ \bottomrule
    \end{tabular}
    \fi
  }
}
\makeatother
\end{showchapterstyle}
Next up is VZ23.
\begin{showchapterstyle}{VZ23}
\setlength\midchapskip{10pt}
\makechapterstyle{VZ23}{
  \renewcommand\chapternamenum{}
  \renewcommand\printchaptername{}
  \renewcommand\chapnumfont{\Huge\bfseries\centering}
  \renewcommand\chaptitlefont{\Huge\scshape\centering}
  \renewcommand\afterchapternum{%
    \par\nobreak\vskip\midchapskip\hrule\vskip\midchapskip}
  \renewcommand\printchapternonum{%
    \vphantom{\chapnumfont \thechapter}
    \par\nobreak\vskip\midchapskip\hrule\vskip\midchapskip}
}  
\end{showchapterstyle}
A variation over VZ34 (in the original the first cell in the tabular
adjusts to the width of the chapter number, here it does not).
\begin{showchapterstyle}{VZ34}
\usepackage{calc}
\newif\ifNoChapNumber
\makeatletter
\makechapterstyle{VZ34}{
  \renewcommand\chapternamenum{}
  \renewcommand\printchaptername{}
  \renewcommand\printchapternum{}
  \renewcommand\chapnumfont{\Huge\bfseries}
  \renewcommand\chaptitlefont{\Huge\bfseries\raggedright}
  \renewcommand\printchaptertitle[1]{%
    \begin{tabular}{@{}p{1cm}|!{\quad}p{\textwidth-1cm-2em-4\tabcolsep }}
      \ifNoChapNumber\relax\else\chapnumfont \thechapter\fi
      & \chaptitlefont ##1
    \end{tabular}
    \NoChapNumberfalse
  }
  \renewcommand\printchapternonum{\NoChapNumbertrue}
}
\end{showchapterstyle}
Variation over VZ39, contributed by Danie Els.\label{BlueBox}
\begin{showchapterstyle}{BlueBox}
\usepackage{fourier} % or what ever
\usepackage[scaled=.92]{helvet}%. Sans serif - Helvetica
\usepackage{color,calc}
\newsavebox{\ChpNumBox}
\definecolor{ChapBlue}{rgb}{0.00,0.65,0.65}
\makeatletter
\newcommand*{\thickhrulefill}{%
  \leavevmode\leaders\hrule height 1\p@ \hfill \kern \z@}
\newcommand*\BuildChpNum[2]{%
  \begin{tabular}[t]{@{}c@{}}
    \makebox[0pt][c]{#1\strut}  \\[.5ex]
    \colorbox{ChapBlue}{%
      \rule[-10em]{0pt}{0pt}%
      \rule{1ex}{0pt}\color{black}#2\strut
      \rule{1ex}{0pt}}%
  \end{tabular}}
\makechapterstyle{BlueBox}{%
  \renewcommand{\chapnamefont}{\large\scshape}
  \renewcommand{\chapnumfont}{\Huge\bfseries}
  \renewcommand{\chaptitlefont}{\raggedright\Huge\bfseries}
  \setlength{\beforechapskip}{20pt}
  \setlength{\midchapskip}{26pt}
  \setlength{\afterchapskip}{40pt}
  \renewcommand{\printchaptername}{}
  \renewcommand{\chapternamenum}{}
  \renewcommand{\printchapternum}{%
    \sbox{\ChpNumBox}{%
      \BuildChpNum{\chapnamefont\@chapapp}%
      {\chapnumfont\thechapter}}}
  \renewcommand{\printchapternonum}{%
    \sbox{\ChpNumBox}{%
      \BuildChpNum{\chapnamefont\vphantom{\@chapapp}}%
      {\chapnumfont\hphantom{\thechapter}}}}
  \renewcommand{\afterchapternum}{}
  \renewcommand{\printchaptertitle}[1]{%
    \usebox{\ChpNumBox}\hfill
    \parbox[t]{\hsize-\wd\ChpNumBox-1em}{%
      \vspace{\midchapskip}%
      \thickhrulefill\par
      \chaptitlefont ##1\par}}%
}
\end{showchapterstyle}
Style inspired by VZ43
\begin{showchapterstyle}{VZ43}
\usepackage{calc,color}
\newif\ifNoChapNumber
\newcommand\Vlines{%
  \def\VL{\rule[-2cm]{1pt}{5cm}\hspace{1mm}\relax}
  \VL\VL\VL\VL\VL\VL\VL}
\makeatletter
\setlength\midchapskip{0pt}
\makechapterstyle{VZ43}{
  \renewcommand\chapternamenum{}
  \renewcommand\printchaptername{}
  \renewcommand\printchapternum{}
  \renewcommand\chapnumfont{\Huge\bfseries\centering}
  \renewcommand\chaptitlefont{\Huge\bfseries\raggedright}
  \renewcommand\printchaptertitle[1]{%
    \Vlines\hspace*{-2em}%
    \begin{tabular}{@{}p{1cm} p{\textwidth-3cm}}%
      \ifNoChapNumber\relax\else%
      \colorbox{black}{\color{white}%
        \makebox[.8cm]{\chapnumfont\strut \thechapter}}
      \fi
      & \chaptitlefont ##1
    \end{tabular}
    \NoChapNumberfalse
  }
  \renewcommand\printchapternonum{\NoChapNumbertrue}
}
\makeatother
\end{showchapterstyle}

\begin{thebibliography}{9}
\bibitem{memman} Peter Wilson, \emph{The Memoir Class for Configurable
  Typesetting -- User Guide}, 2010.
\bibitem{VZ} Vincent Zoonekynd. On-line list of different chapter
  styles for \LaTeX. Available at
  \url{http://zoonek.free.fr/LaTeX/LaTeX_samples_chapter/0.html}. 
\end{thebibliography}

\end{document}

%%% Local Variables: 
%%% mode: latex
%%% TeX-master: t
%%% End: 
