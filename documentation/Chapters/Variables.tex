%%%%%%%%%%%%%%%%%%%%%%%%%%%%%%%%%%%%%%%%%%%%%%%%%%%%%%%%
%%
\ifdevguide
\chapter{Variables}
\label{ch:variables}
\else
\chapter{User modifiable variables}
\label{ch:variables}
\fi
%%
%%%%%%%%%%%%%%%%%%%%%%%%%%%%%%%%%%%%%%%%%%%%%%%%%%%%%%%%


To better understand the variables in the DRS we have laid out each variable in the following way:

\begin{itemize}

\item \ParameterEntry{Variable title}{Description of the variable}{VARIABLE\_NAME}
{Default Value}{The code used a variable is used in}{The place where the variable can be found}
{
Who should be able to change this variable, levels are as follows:
\begin{itemize}
	\item Public: Everyone (including the user)
	\item Private: Only the developer
\end{itemize}
}

\end{itemize}


%%%%%%%%%%%%%%%%%%%%%%%%%%%%%%%%%%%%%%%%%%%%%%%%%%%%%%%%
%%
\section{Variable file locations}
\label{ch:variables:location}
%%
%%%%%%%%%%%%%%%%%%%%%%%%%%%%%%%%%%%%%%%%%%%%%%%%%%%%%%%%

\ifdevguide
\subsection{User modifiable variables}
\fi

The variables are currently stored in two places. The first (\configtxtfile) contains constants that deal with initial set up. These were mentioned in Section \ref{ch:install:setup} and are located in \configtxtfilepath. \\

\noindent The other variables modify how the DRS runs. These are located in \constantsfile\, (located at \constantsfilepath).  \\


\ifdevguide
\subsection{Private variables}

\noindent In addition to the above (user modifiable public variable files) there are several files that will contain all constants that should not be changed by a user (i.e. static variables that are set and changed only in development). These are described below:

\begin{itemize}

	\item \textbf{Keywords:} The keywords for header input and output are stored in \spirouKeywords. This contains keyword definitions in the form of a python list:  \\

	\begin{pythonbox}
	kw_VARIABLE = ['KEYWORD', 'Default value', 'Comment']
	\end{pythonbox}

	\noindent where the 'KEYWORD' is the key in the FITs REC header file, with the value and comment defined in the next positions. i.e. in a FITs REC header reader one would expect

	\begin{thighlight}
	\begin{tabular}{l c r c l}
	KEYWORD & = & Default value & / Comment \\
	\end{tabular}
	\end{thighlight}


	\item \textbf{Constants and Pseudo-constants:} These are stored in \spirouCONST, they range from simple objects (strings, integers, float, lists, python dictionaries etc) to more complicated `pseudo-constants' that are constructed themselves from other constants. These are kept private (i.e. no mentioned in the user manual) as they should not need be changed by the average user.

\end{itemize}

\fi

%%%%%%%%%%%%%%%%%%%%%%%%%%%%%%%%%%%%%%%%%%%%%%%%%%%%%%%%
%%
\section{Global variables}
\label{ch:variables:global}
%%
%%%%%%%%%%%%%%%%%%%%%%%%%%%%%%%%%%%%%%%%%%%%%%%%%%%%%%%%


\begin{itemize}
% DRS_NAME
\ifdevguide
\item \ParameterEntry{DRS Name}
{Defines the data reduction software name. Value must be a valid string.}
{DRS\_NAME}{SPIROU}{All Recipes}{\spirouCONST}{Private}
\fi

% DRS_VERSION
\ifdevguide
\item \ParameterEntry{DRS Version}
{Defines the data reduction software version. Value must be a valid string.}
{DRS\_VERSION}{SPIROU}{All Recipes}{\spirouCONST}{Private}
\fi

% DRS_PLOT
\item \ParameterEntry{Plotting switch}
{Defines whether to show plots (A value of 1 to show plots, a value of 0 to not show plots). Value must be an integer (0 or 1) or boolean (True or False)}
{DRS\_PLOT}{1}{All Recipes}{\configtxtfile}{Public}

% PRINT_LEVEL
\item \ParameterEntry{Print message level}
{The level of messages to print, values can be as follows:
\begin{itemize}
	\item "all" -- prints all events
	\item "info" -- prints info, warning and error events
	\item "warning" -- prints warning and error events
	\item "error" -- print only error events
\end{itemize}
Value must be a valid string.
}
{PRINT\_LEVEL}{all}{All Recipes}{\configtxtfile}{Public}

% LOG_LEVEL
\item \ParameterEntry{Log message level}
{The level of messages to print, values can be as follows:
\begin{itemize}
	\item "all" -- prints all events
	\item "info" -- prints info, warning and error events
	\item "warning" -- prints warning and error events
	\item "error" -- print only error events
\end{itemize}
Value must be a valid string.
}
{LOG\_LEVEL}{all}{All Recipes}{\configtxtfile}{Public}


% ic_debug
\item \ParameterEntry{Debug mode}
{Enable various numeric debug codes (0 for no debug). Value must be an integer where:
\begin{itemize}
\item 0 = No debug
\item 1 = Level 1 debug \begin{todo}Define level 1 debug\end{todo}
\item 2 = Level 2 debug \begin{todo}Define level 2 debug\end{todo}
\end{itemize}
}
{ic\_debug}{0}{\callocRAW}{\constantsfile}{Public}
\DevNote{Should this be public?}

% ic_display_timeout
\item \ParameterEntry{Plot interval}
{Set the interval between plots in seconds (for certain interactive graphs). Value must be a valid float larger than zero.}
{ic\_display\_timeout}{0.5}{\callocRAW}{\constantsfile}{Public}
\DevNote{Should this be public?}

\end{itemize}


%%%%%%%%%%%%%%%%%%%%%%%%%%%%%%%%%%%%%%%%%%%%%%%%%%%%%%%%
%%
\section{Directory variables}
\label{ch:variables:directory}
%%
%%%%%%%%%%%%%%%%%%%%%%%%%%%%%%%%%%%%%%%%%%%%%%%%%%%%%%%%

\begin{itemize}

\item \ParameterEntry{The data directory}
{Defines the path to the data directory. Value must be a string containing a valid file location.}
{TDATA}{/drs/data/}{All Recipes}{\configtxtfile}{Public}

\item \ParameterEntry{The installation directory}
{Defines the instllation directory (\InstallDIR). Value must be a string containing a valid file location.}
{DRS\_ROOT}{/drs/INTROOT/}{All Recipes}{\configtxtfile}{Public}

\item \ParameterEntry{The raw data directory}
{Defines the directory that the reduced data will be saved to/read from. Value must be a string containing a valid file location.}
{DRS\_DATA\_RAW}{/drs/data/raw}{All Recipes}{\configtxtfile}{Public}

\item \ParameterEntry{The reduced data directory}
{Defines the directory that the reduced data will be saved to/read from. Value must be a string containing a valid file location.}
{DRS\_DATA\_REDUC}{/drs/data/reduced}{All Recipes}{\configtxtfile}{Public}

\item \ParameterEntry{The calibration database and calibration file directory}
{Defines the directory that the calibration files and database will be saved to/read from. Value must be a string containing a valid file location.}
{DRS\_CALIB\_DB}{/drs/data/calibDB}{All Recipes}{\configtxtfile}{Public}

\item \ParameterEntry{The log directory}
{Defines the directory that the log messages are stored in. Value must be a string containing a valid file location.}
{DRS\_DATA\_MSG}{/drs/data/msg}{All Recipes}{\configtxtfile}{Public}

\item \ParameterEntry{The working directory}
{Defines the working directory. Value must be a string containing a valid file location.}
{DRS\_DATA\_WORKING}{/drs/data/tmp/}{All Recipes}{\configtxtfile}{Public}

\end{itemize}


%%%%%%%%%%%%%%%%%%%%%%%%%%%%%%%%%%%%%%%%%%%%%%%%%%%%%%%%
%%
\section{Image variables}
\label{ch:variables:image}
%%
%%%%%%%%%%%%%%%%%%%%%%%%%%%%%%%%%%%%%%%%%%%%%%%%%%%%%%%%

\begin{itemize}

% Resize blue window (has to be defined manually)
\item 
\begin{minipage}[t]{\textwidth}
\textbf{Resizing blue window}

\begin{thighlight}
\textcolor{brown}{The blue window used in \calDARK. Each value must be a integer between 0 and the maximum array size in each dimension.} 

\begin{tabular}{>{\color{red}}l c l}
&&\\
ic\_ccdx\_blue\_low &=& 2048-200 \\
ic\_ccdx\_blue\_high &=& 2048-1500 \\
ic\_ccdy\_blue\_low &=& 2048-20 \\
ic\_ccdy\_blue\_high &=& 2048-350 \\
&&\\
\textcolor{blue}{Used in:}  & \multicolumn{2}{p{10cm}}{\calDARK} \\
\textcolor{blue}{Defined in:} & \multicolumn{2}{p{10cm}}{\constantsfile} \\
\ifdevguide
\textcolor{blue}{Level:} & \multicolumn{2}{p{10cm}}{Public} \\
\fi
\end{tabular}
\end{thighlight}
\end{minipage}


% Resize red window (has to be defined manually)
\item 
\begin{minipage}[t]{\textwidth}
\textbf{Resizing red window}

\begin{thighlight}
\textcolor{brown}{The blue window used in \calDARK. Each value must be a integer between 0 and the maximum array size in each dimension.} 

\begin{tabular}{>{\color{red}}l c l}
&&\\
ic\_ccdx\_red\_low  & =  & 2048-20 \\
ic\_ccdx\_red\_high &  = &  2048-1750 \\
ic\_ccdy\_red\_low  & =  & 2048-1570 \\
ic\_ccdy\_red\_high &  = &  2048-1910 \\
&&\\
\textcolor{blue}{Used in:}  & \multicolumn{2}{p{10cm}}{\calDARK} \\
\textcolor{blue}{Defined in:} & \multicolumn{2}{p{10cm}}{\constantsfile} \\
\ifdevguide
\textcolor{blue}{Level:} & \multicolumn{2}{p{10cm}}{Public} \\
\fi
\end{tabular}
\end{thighlight}
\end{minipage}



% Resize image (has to be defined manually)
\item 
\begin{minipage}[t]{\textwidth}
\textbf{Resizing red window}

\begin{thighlight}
\textcolor{brown}{The blue window used in \calDARK. Each value must be a integer between 0 and the maximum array size in each dimension.} 

\begin{tabular}{>{\color{red}}l c l}
&&\\
ic\_ccdx\_low &=& 5 \\
ic\_ccdx\_high &=& 2040 \\
ic\_ccdy\_low &=& 5 \\
ic\_ccdy\_high &=& 1935 \\

\textcolor{blue}{Used in:}  & \multicolumn{2}{p{10cm}}{\AllRecipes} \\
\textcolor{blue}{Defined in:} & \multicolumn{2}{p{10cm}}{\constantsfile} \\
\ifdevguide
\textcolor{blue}{Level:} & \multicolumn{2}{p{10cm}}{Public} \\
\fi
\end{tabular}
\end{thighlight}
\end{minipage}

\end{itemize}


%%%%%%%%%%%%%%%%%%%%%%%%%%%%%%%%%%%%%%%%%%%%%%%%%%%%%%%%
%%
\clearpage
\newpage
\section{Fiber variables}
\label{ch:variables:fiber}
%%
%%%%%%%%%%%%%%%%%%%%%%%%%%%%%%%%%%%%%%%%%%%%%%%%%%%%%%%%

These variables are defined for each type of fiber and thus are defined as a python dictionary of values \ifdevguide (read using the python `eval' function) \fi. As such they all must contain the same dictionary keys (currently `AB', `A', `B' and `C'). 

\DevNote{For python to combine these at run time the suffix `\_fpall' must be used (thus once a fiber is defined the code will know to extract the key before the suffix). i.e. for variable `nbfib\_fpall' and a fiber `AB' the extracted parameter will be `nbfib' with the value in the dictionary corresponding to the `AB' key.}

\begin{itemize}

% nbfib
\item \ParameterEntry{Number of fibers}
{This describes the number of fibers of a given type. Must be a python dictionary with identical keys to all other fiber parameters (each value must be an integer).}
{nbfib\_fpall}
{\lstinline[style=pythoninline]| \{'AB':2, 'A':1, 'B':1, 'C':1\} |}
{\callocRAW}{\constantsfile}{Public}

% ic_first_order_jump_fpall
\item \ParameterEntry{Order skip number}
{Describes the number of orders to skip at the start of an image. Must be a python dictionary with identical keys to all other fiber parameters (each value must be an integer).}
{ic\_first\_order\_jump\_fpall}
{\lstinline[style=pythoninline]| \{'AB':2, 'A':0, 'B':0, 'C':0\} |}
{\callocRAW}{\constantsfile}{Public}

% ic_locnbmaxo_fpall
\item \ParameterEntry{Maximum order numbers}
{Describes the maximum allowed number of orders. Must be a python dictionary with identical keys to all other fiber parameters (each value must be an integer).}
{ic\_locnbmaxo\_fpall}
{\lstinline[style=pythoninline]| \{'AB':72, 'A':36, 'B':36, 'C':36\} |}
{\callocRAW}{\constantsfile}{Public}

% qc_loc_nbo_fpall
\item \ParameterEntry{Number of orders to fit (QC)}
{Quality control parameter for the number of orders on fiber to fit. Must be a python dictionary with identical keys to all other fiber parameters (each value must be an integer).}
{qc\_loc\_nbo\_fpall}
{\lstinline[style=pythoninline]| \{'AB':72, 'A':36, 'B':36, 'C':36\} |}
{\callocRAW}{\constantsfile}{Public}
\DevNote{Should this be merged with `ic\_locnbmaxo\_fpall'?}


% fib_type_fpall
\item \ParameterEntry{Fiber types for this fiber}
{The fiber type(s) -- as a list -- for this fiber. Must be a python dictionary with identical keys to all other fiber parameters (each value must be a list of strings).}
{fib\_type\_fpall}
{\lstinline[style=pythoninline]| \{'AB':["AB"], 'A':["A"], 'B':["B"], 'C':["C"]\} |}
{\calFFraw}{\constantsfile}{Public}
\DevNote{This must not be needed but is in here due to a loop in \calFFraw}

% ic_ext_range1_fpall
\item \ParameterEntry{Half-zone extraction width (left/top)}
{The pixels are extracted from the center of the order out to the edges - defined by `top' and `bottom' - width (illuminated part of the order) - this number defines the \textbf{top} side (if one requires a symmetric extraction around the order fit both range 1 and range 2 -- below -- should be the same). This can also be used to extract A and B separately (where the fit order is defined at the center of the AB pair). Must be a python dictionary with identical keys to all other fiber parameters.}
{ic\_ext\_range1\_fpall}
{\lstinline[style=pythoninline]| \{'AB':14.5, 'A':0.0, 'B':14.5, 'C':7.5\} |}
{\calFFraw}{\constantsfile}{Public}
\DevNote{Formally this was called `plage1' in \calFFraw}

% ic_ext_range2_fpall
\item \ParameterEntry{Half-zone extraction width (right/bottom)}
{The pixels are extracted from the center of the order out to the edges - defined by `top' and `botto'm - width (illuminated part of the order) - this number defines the \textbf{bottom} side (if one requires a symmetric extraction around the order fit both range 1 and range 2 -- below -- should be the same). This can also be used to extract A and B separately (where the fit order is defined at the center of the AB pair). Must be a python dictionary with identical keys to all other fiber parameters.}
{ic\_ext\_range2\_fpall}
{\lstinline[style=pythoninline]| \{'AB':14.5, 'A':14.5, 'B':0.0, 'C':7.5\} |}
{\calFFraw}{\constantsfile}{Public}
\DevNote{Formally this was called `plage2' in \calFFraw}

% ic_ext_range_fpall
\item \ParameterEntry{Half-zone extraction width for full extraction}
{The pixels are extracted from the center of the order out to the edges - defined by the left and right - or top and bottom - width (illuminated part of the order). In \calextractRAW both sides of the fit order are extracted at with the same width (symmetric). Must be a python dictionary with identical keys to all other fiber parameters.}
{ic\_ext\_range\_fpall}
{\lstinline[style=pythoninline]| \{'AB':14.5, 'A':14.5, 'B':14.5, 'C':7.5\} |}
{\calextractRAW}{\constantsfile}{Public}
\DevNote{Formally this was called `plage' in \calextractRAW}

% loc_file_fpall
\item \ParameterEntry{Localization fiber for extraction  }
{Defines the localization fiber to use for each fiber type. This is the file in calibDB that is used i.e. the keyword \masterCALIBDBfile used will be \`LOC\_\{loc\_file\_fpall\}' (e.g. for fiber=`AB' use `LOC\_AB'). Must be a python dictionary with identical keys to all other fiber parameters.}
{loc\_file\_fpall}
{\lstinline[style=pythoninline]| \{'AB':'AB', 'A':'AB', 'B':'AB', 'C':'C'\} |}
{\calextractRAW}{\constantsfile}{Public}

% orderp_file_fpall
\item \ParameterEntry{Order profile fiber for extraction}
{Defines the order profile fiber to use for each fiber type. This is the file in calibDB that is used i.e. the keyword \masterCALIBDBfile used will be \`ORDER\_PROFILE\_\{orderp\_file\_fpall\}' (e.g. for fiber=`AB' use `ORDER\_PROFILE\_AB'). Must be a python dictionary with identical keys to all other fiber parameters.}
{orderp\_file\_fpall}
{\lstinline[style=pythoninline]| \{'AB':'AB', 'A':'AB', 'B':'AB', 'C':'C'\} |}
{\calextractRAW}{\constantsfile}{Public}

% ic_ext_d_range_fpall
\item \ParameterEntry{Half-zone extract width \calDRIFTRAW}
{The size in pixels of the extraction away from the order localization fit (to the top and bottom) - defines the illuminated area of the order for extraction. Must be a python dictionary with identical keys to all other fiber parameters.}
{ic\_ext\_d\_range\_fpall}
{\lstinline[style=pythoninline]| \{'AB':14.0, 'A':14.0, 'B':14.0, 'C':7.0\} |}
{\calDRIFTRAW}{\constantsfile}{Public}
\DevNote{Formally this was called `ic\_extnbsig' in \calDRIFTRAW}

\end{itemize}


%%%%%%%%%%%%%%%%%%%%%%%%%%%%%%%%%%%%%%%%%%%%%%%%%%%%%%%%
%%
\clearpage
\newpage
\section{Dark calibration variables}
\label{ch:variables:dark}
%%
%%%%%%%%%%%%%%%%%%%%%%%%%%%%%%%%%%%%%%%%%%%%%%%%%%%%%%%%

\begin{itemize}

% dark_qmin
\item \ParameterEntry{Lower percentile for dead pixel stats}
{This defines the lower percentile to be logged for the fraction of dead pixels statistics. Value must be an integer between 0 and 100 (1 sigma below the mean is $\sim$16).}
{dark\_qmin}{5}
{\calDARK}{\constantsfile}{Public}

% dark_qmax
\item \ParameterEntry{Upper percentile for dead pixel stats}
{This defines the upper percentile to be logged for the fraction of dead pixels statistics. Value must be an integer between 0 and 100 (1 sigma above the mean is $\sim$84).}
{dark\_qmax}{95}
{\calDARK}{\constantsfile}{Public}

% histo\_bins
\item \ParameterEntry{Dark stat histogram bins}
{Defines the number of bins to use in the dark histogram plot. Value must be a positive integer.}
{histo\_bins}{200}
{\calDARK}{\constantsfile}{Public}

% histo_range_low
\item \ParameterEntry{Lower bound for the Dark stat histogram}
{Defines the lower bound for the dark statistic histogram. Value must be a float less than (no equal to) the value of `histo\_range\_high'}
{histo\_range\_low}{-0.5}
{\calDARK}{\constantsfile}{Public}

% histo_range_high
\item \ParameterEntry{Upper bound for the Dark stat histogram}
{Defines the upper bound for the dark statistic histogram. Value must be a float greater than (not equal to) the value of `histo\_range\_low'}
{histo\_range\_high}{5}
{\calDARK}{\constantsfile}{Public}

% dark_cutlimit
\item \ParameterEntry{Bad pixel cut limit}
{Defines the bad pixel cut limit in ADU/s. 
\begin{equation}
badpixels = (image > \text{dark\_cut\_limit}) \text{ OR } (\text{non-finite})
\end{equation}}
{dark\_cutlimit}{100.0}
{\calDARK}{\constantsfile}{Public}

\end{itemize}

% % 
% \item \ParameterEntry{}
% {}
% {}{}
% {}{}{Public}