%%%%%%%%%%%%%%%%%%%%%%%%%%%%%%%%%%%%%%%%%%%%%%%%%%%%%%%%
%%
\chapter{Installation}
\label{chapter:installation}
%%
%%%%%%%%%%%%%%%%%%%%%%%%%%%%%%%%%%%%%%%%%%%%%%%%%%%%%%%%

%%%%%%%%%%%%%%%%%%%%%%%%%%%%%%%%%%%%%%%%%%%%%%%%%%%%%%%%
%%
\section{Introduction}
\label{ch:install:installintro}
%%
%%%%%%%%%%%%%%%%%%%%%%%%%%%%%%%%%%%%%%%%%%%%%%%%%%%%%%%%

Once finialised the installation should just be a download, run setup.py and configure the DRS directories, however, during development the following stages are required.

\begin{note}
Currently the download repositry on github is private and requires a github account, and the user to be added to the list of collaborators. To be added to the collaborators please email \url{neil.james.cook@gmail.com} with your github username.
\end{note}







%%%%%%%%%%%%%%%%%%%%%%%%%%%%%%%%%%%%%%%%%%%%%%%%%%%%%%%%
%%
\section{Download}
\label{ch:install:installDownload}
%%
%%%%%%%%%%%%%%%%%%%%%%%%%%%%%%%%%%%%%%%%%%%%%%%%%%%%%%%%

Get the latest version of the DRS (for \instrument version \MyCodeVersion). Use any of the following ways:

\begin{itemize}
\item manually download from here: \url{\DRSLatestURL}

\item use Git:
\begin{cmdbox}
git checkout (*\DRSGitURL*)
\end{cmdbox}

\item use SVN:
\begin{cmdbox}
svn checkout  (*\DRSGitURL*)
\end{cmdbox}

\item use ssh:
\begin{cmdbox}
scp -r (*\DRSsshURL*)
\end{cmdbox}

\end{itemize}


%%%%%%%%%%%%%%%%%%%%%%%%%%%%%%%%%%%%%%%%%%%%%%%%%%%%%%%%
%%
\clearpage
\newpage
\section{Prerequisites}
\label{ch:install:prerequisites}
%%
%%%%%%%%%%%%%%%%%%%%%%%%%%%%%%%%%%%%%%%%%%%%%%%%%%%%%%%%

It is recommended to install the latest version of Anaconda python distribution, available for Windows, \mac and Linux (here: \url{https://www.anaconda.com/download/}). However one can run the DRS on a native python installation. \\

\noindent We recommend python 3 over python 2 for long term continued support (however the latest version of the DRS supports the newest versions of python 2.7).

\begin{note}
Before installing the DRS you must have one of the following:
\begin{itemize}
\item Latest version of Anaconda (for python 2 or python 3) --- RECOMMENDED
\item An Up-to-date version of python (python 2 or python 3)
\end{itemize}
\end{note}


% -------------------------------------------------------
\subsection{Anaconda python distribution}
\label{ch:install:prerequisites:anaconda}
% -------------------------------------------------------

A valid version of the Anaconda python distribution (for python2 or python 3)
\noindent Currently tested version of python are:
\begin{itemize}
\item Python 2.7.13 and Anaconda 4.4.0
\item Python 3.6.3 and Anaconda 5.0.1 --- RECOMMENDED
\end{itemize}



% -------------------------------------------------------
\subsection{Separate python installation}
\label{ch:install:prerequisites:separate_python}
% -------------------------------------------------------

An up-to-date version of python (either python 2 or python 3) and the following python modules (with version of python they were tested with).
\begin{itemize}
\item Python 3.6
	\begin{itemize}
	\item \Program{astropy} (tested with version 2.0.2)
	\item \Program{matplotlib} (tested with version 2.1.0)
	\item \Program{numpy} (tested with version 1.13.3)
	\item and the following built-in modules (comes with python): \Program{datetime}, \Program{filecmp}, \Program{glob}, \Program{os}, \Program{pkg\_resources}, \Program{shutil}, \Program{sys}, \Program{time}, \Program{warnings}
	\end{itemize}
\item Python 2.7
	\begin{itemize}
	\item astropy (tested with version 1.3.2)
	\item matplotlib (tested with version 2.0.2)
	\item numpy (tested with version 1.12.1)
	\item and the following built-in modules (comes with python): \Program{\twound{future}\twound}, \Program{collections}, \Program{datetime}, \Program{filecmp}, \Program{glob}, \Program{os}, \Program{pkg\_resources}, \Program{shutil}, \Program{sys}, \Program{time}, \Program{warnings}
	\end{itemize}
\end{itemize}



%%%%%%%%%%%%%%%%%%%%%%%%%%%%%%%%%%%%%%%%%%%%%%%%%%%%%%%%
%%
\clearpage
\newpage
\section{Installation Linux and macOS}
\label{ch:install:installunix}
%%
%%%%%%%%%%%%%%%%%%%%%%%%%%%%%%%%%%%%%%%%%%%%%%%%%%%%%%%%

Currently the DRS has to be installed manually. This involves the following steps:
\begin{enumerate}
\item Extraction (Section \ref{ch:install:installunix:extraction})
\item Modify environmental settings (Section \ref{ch:install:installunix:environ_settings})
\item Make recipes executable (Section \ref{ch:install:installunix:executable})
\end{enumerate}

% -------------------------------------------------------
\subsection{Extraction}
\label{ch:install:installunix:extraction}
% -------------------------------------------------------

The first step is to extract the DRS into a folder (the \InstallDIR).

\noindent Do this by using the following commands:
\begin{cmdbox}
cd (*\InstallDIRcmd*)
unzip DRS.zip
\end{cmdbox}

% -------------------------------------------------------
\subsection{Modify environmental settings}
\label{ch:install:installunix:environ_settings}
% -------------------------------------------------------

The next step is to modify your PATH and PYTHONPATH environmental variables (to include the \InstallDIR. This depends which shell you are using (type `\lstinline[style=bashstyle]{echo $0}' to find out which).

\begin{itemize}
	\item In bash open the `.bashrc' text file in your home ($\sim$) directory (or create it if it doesn't exist)

	\begin{bashbox}
	@export@ <PATH>=(*\InstallDIR*)/bin/:<$PATH>

	@export@ <PYTHONPATH>=(*\InstallDIR*):(*\InstallDIR*)/bin/:<$PYTHONPATH>
	\end{bashbox}

	\item In csh /tcsh open the `.cshrc' or `.tcshrc' text file in your home ($\sim$) directory (or create it if it doesn't exist) 

	\begin{cshbox}
	@setenv@ <PATH> (*\InstallDIR*)/bin/:<${PATH}>

	@setenv@ <PYTHONPATH> (*\InstallDIR*):(*\InstallDIR*)/bin/:<${PYTHONPATH}>
	\end{cshbox}

\end{itemize}

% -------------------------------------------------------
\subsection{Make recipes executable}
\label{ch:install:installunix:executable}
% -------------------------------------------------------

\noindent To run the recipes from the command line (without starting python) one must make them executable. Do this by using the following command:
\begin{cmdbox}
chmod +x (*\InstallDIRcmd*)/bin/*.py
\end{cmdbox}



%%%%%%%%%%%%%%%%%%%%%%%%%%%%%%%%%%%%%%%%%%%%%%%%%%%%%%%%
%%
\clearpage
\newpage
\section{Installation Windows}
\label{ch:install:install_win}
%%
%%%%%%%%%%%%%%%%%%%%%%%%%%%%%%%%%%%%%%%%%%%%%%%%%%%%%%%%

This is very similar currently to the Linux/\mac installation (in the future a `.exe' file will be given).

\begin{enumerate}
\item Extract to \InstallDIR with your favourite unzipping softwear.
\item Add \InstallDIR to your PYTHONPATH (Section \ref{ch:install:install_win:environ_settings})
\end{enumerate}

% -------------------------------------------------------
\subsection{How to modify environmental settings in windows}
\label{ch:install:install_win:environ_settings}
% -------------------------------------------------------

This process is a little more convoluted than on Linux or \mac system.

\begin{enumerate}
\item Go to `My computer > Properties > Advanced System Settings > Enviromental Variables'.

\item if under system variable `PythonPath' exists click edit and add `\InstallDIR;'\, to the end.

\noindent i.e.

\begin{textbox}
C:\Python27;(*\InstallDIR*);
\end{textbox}

\item if under system variables `PythonPath' does not exist create a new variable called `PythonPath' and add:

\begin{textbox}
%PYTHONPATH%;(*\InstallDIR*);(*\InstallDIR*)\bin\;
\end{textbox}

\end{enumerate}

\noindent For problems/troubleshooting see here: \url{https://stackoverflow.com/questions/3701646/how-to-add-to-the-pythonpath-in-windows-7}.



%%%%%%%%%%%%%%%%%%%%%%%%%%%%%%%%%%%%%%%%%%%%%%%%%%%%%%%%
%%
\clearpage
\newpage
\section{Setting up the DRS}
\label{ch:install:setup}
%%
%%%%%%%%%%%%%%%%%%%%%%%%%%%%%%%%%%%%%%%%%%%%%%%%%%%%%%%%

Before running the DRS one must set the data paths. \\

\noindent The `config.txt' file is located in the \InstallDIR in the config folder.

i.e. at \InstallDIR{/config/config.txt} \\

\noindent The following keywords \textbf{must} be changed (and must be a valid path):
\begin{thighlight}
\begin{table}[H]
\begin{tabular}{p{4cm} p{0.05cm} p{2.5cm} p{0.05cm} p{4.5cm}}
\definevariable{TDATA}            & = & /drs/data/        & / & Define the DATA directory\\
&&&&\\
\definevariable{DRS\_ROOT}         & = & /drs/INTROOT/     & / & Define the installation directory (\InstallDIR) \\
\definevariable{DRS\_DATA\_RAW}     & = & /drs/data/raw     & / & Define the folder with the raw data files in \\
\definevariable{DRS\_DATA\_REDUC}   & = & /drs/data/reduced & / & Define the directory that the reduced data should be saved to/read from \\
\definevariable{DRS\_CALIB\_DB}     & = & /drs/data/calibDB & / & Define the directory that the calibration files should be saved to/read from \\
\definevariable{DRS\_DATA\_MSG}     & = & /drs/data/msg     & / & Define the directory that the log messages are stored in \\
\definevariable{DRS\_DATA\_WORKING} & = & /drs/data/tmp/    & / & Define the working directory \\
\end{tabular}
\end{table}

\vspace{0.1cm}
\noindent The directories here are for linux and \mac\, systems another example would be `/home/user/INTROOT' for the \InstallDIR\, directory. 

\noindent On Windows machines this would be equivalent to `C:\textbackslash{Users}\textbackslash{<username>}\textbackslash{INTROOT}' in Windows Vista, 7, 8 and 10 or `C:\textbackslash{Documents and Settings}\textbackslash{<username>}\textbackslash{INTROOT}' on early versions of Windows. \\
\end{thighlight}
\vspace{0.25cm}

\noindent The following keywords can be changed: \\
\begin{thighlight}
\begin{table}[H]
\begin{tabular}{>{\color{red}}l c r c p{5cm}}
\definevariable{DRS\_PLOT}    & = & 1     & / & Whether to show plots \\
\definevariable{PRINT\_LEVEL} & = & "all" & / & Level at which to print \\
\definevariable{LOG\_LEVEL}   & = & "all" & / & Level at which to log in log file \\
\end{tabular}
\end{table}

\noindent For the `\definevariable{PRINT\_LEVEL} and \definevariable{LOG\_LEVEL} keywords the values are set as follows:
\begin{itemize}
	\item "all" -- prints all events
	\item "info" -- prints info, warning and error events
	\item "warning" -- prints warning and error events
	\item "error" -- print only error events
\end{itemize}
\end{thighlight}

%%%%%%%%%%%%%%%%%%%%%%%%%%%%%%%%%%%%%%%%%%%%%%%%%%%%%%%%
%%
\clearpage
\newpage
\section{Validating Installation on Linux and macOS}
\label{ch:install:validating_installunix}
%%
%%%%%%%%%%%%%%%%%%%%%%%%%%%%%%%%%%%%%%%%%%%%%%%%%%%%%%%%

\begin{note}
One must install the DRS (Section \ref{ch:install:installunix}) AND set up the DRS (Section \ref{ch:install:setup}) before validation will be successful.
\end{note}

\noindent There are four ways to run the DRS in Linux and \mac (thus four ways to verify installation was correct).

\begin{itemize}

\item To validate running from command line type:
\begin{cmdbox}
(*\calvalidate*)
\end{cmdbox}

\item To validate running from python/ipython from the command line type:
\begin{cmdbox}
python (*\calvalidate*)
ipython (*\calvalidate*)
\end{cmdbox}

\item To validate running from ipython, open ipython and type:
\begin{pythonbox}
@run@ (*\calvalidate*)
\end{pythonbox}

\item To validate running from import from python/ipython, open python/ipython and type:
\begin{pythonbox}
import cal_validate_spirou
cal_validate_spirou.main()
\end{pythonbox}

\end{itemize}

\noindent If validation is successful the following should appear:

\begin{cmdboxprint}
HH:MM:SS.S -   || *****************************************
HH:MM:SS.S -   || * SPIROU @(#) Geneva Observatory (0.0.1)
HH:MM:SS.S -   || *****************************************
HH:MM:SS.S -   ||(dir_data_raw)      DRS_DATA_RAW=/scratch/Projects/spirou_py3/data/raw
HH:MM:SS.S -   ||(dir_data_reduc)    DRS_DATA_REDUC=/scratch/Projects/spirou_py3/data/reduced
HH:MM:SS.S -   ||(dir_calib_db)      DRS_CALIB_DB=/scratch/Projects/spirou_py3/data/calibDB
HH:MM:SS.S -   ||(dir_data_msg)      DRS_DATA_MSG=/scratch/Projects/spirou_py3/data/msg
HH:MM:SS.S -   ||(print_level)       PRINT_LEVEL=all         %(error/warning/info/all)
HH:MM:SS.S -   ||(log_level)         LOG_LEVEL=all         %(error/warning/info/all)
HH:MM:SS.S -   ||(plot_graph)        DRS_PLOT=1            %(def/undef/trigger)
HH:MM:SS.S -   ||(used_date)         DRS_USED_DATE=undefined
HH:MM:SS.S -   ||(working_dir)       DRS_DATA_WORKING=/scratch/Projects/spirou_py3/data/tmp/
HH:MM:SS.S -   ||                    DRS_INTERACTIVE is not set, running on-line mode
HH:MM:SS.S -   ||
HH:MM:SS.S -   ||Validation successful. DRS installed corrected.
\end{cmdboxprint}

%%%%%%%%%%%%%%%%%%%%%%%%%%%%%%%%%%%%%%%%%%%%%%%%%%%%%%%%
%%
\clearpage
\newpage
\section{Validating Installation on Windows}
\label{ch:install:validating_installwin}
%%
%%%%%%%%%%%%%%%%%%%%%%%%%%%%%%%%%%%%%%%%%%%%%%%%%%%%%%%%

\begin{note}
One must install the DRS (Section \ref{ch:install:install_win}) AND set up the DRS (Section \ref{ch:install:setup}) before validation will be successful.
\end{note}

\noindent In windows there are currently 3 ways to run the RS (running in python/ipython).

\begin{itemize}
\item To validate running from python/ipython from the command line type:
\begin{cmdbox}
python (*\calvalidate*)
ipython (*\calvalidate*)
\end{cmdbox}

\item To validate running from ipython, open ipython and type:
\begin{pythonbox}
@run@ (*\calvalidate*)
\end{pythonbox}

\item To validate running from import from python/ipython, open python/ipython and type:
\begin{pythonbox}
import cal_validate_spirou
cal_validate_spirou.main()
\end{pythonbox}

\end{itemize}

\noindent If validation is successful the following should appear:
\begin{cmdboxprint}
HH:MM:SS.S -   || *****************************************
HH:MM:SS.S -   || * SPIROU @(#) Geneva Observatory (0.0.1)
HH:MM:SS.S -   || *****************************************
HH:MM:SS.S -   ||(dir_data_raw)      DRS_DATA_RAW=/scratch/Projects/spirou_py3/data/raw
HH:MM:SS.S -   ||(dir_data_reduc)    DRS_DATA_REDUC=/scratch/Projects/spirou_py3/data/reduced
HH:MM:SS.S -   ||(dir_calib_db)      DRS_CALIB_DB=/scratch/Projects/spirou_py3/data/calibDB
HH:MM:SS.S -   ||(dir_data_msg)      DRS_DATA_MSG=/scratch/Projects/spirou_py3/data/msg
HH:MM:SS.S -   ||(print_level)       PRINT_LEVEL=all         %(error/warning/info/all)
HH:MM:SS.S -   ||(log_level)         LOG_LEVEL=all         %(error/warning/info/all)
HH:MM:SS.S -   ||(plot_graph)        DRS_PLOT=1            %(def/undef/trigger)
HH:MM:SS.S -   ||(used_date)         DRS_USED_DATE=undefined
HH:MM:SS.S -   ||(working_dir)       DRS_DATA_WORKING=/scratch/Projects/spirou_py3/data/tmp/
HH:MM:SS.S -   ||                    DRS_INTERACTIVE is not set, running on-line mode
HH:MM:SS.S -   ||
HH:MM:SS.S -   ||Validation successful. DRS installed corrected.
\end{cmdboxprint}