%%%%%%%%%%%%%%%%%%%%%%%%%%%%%%%%%%%%%%%%%%%%%%%%%%%%%%%%
%%
\chapter{Introduction}
\label{chapter:intro}
%%
%%%%%%%%%%%%%%%%%%%%%%%%%%%%%%%%%%%%%%%%%%%%%%%%%%%%%%%%



%%%%%%%%%%%%%%%%%%%%%%%%%%%%%%%%%%%%%%%%%%%%%%%%%%%%%%%%
%%
\section{Code blocks}
\label{ch:intro:codeblocks}
%%
%%%%%%%%%%%%%%%%%%%%%%%%%%%%%%%%%%%%%%%%%%%%%%%%%%%%%%%%

Certain sections will be written in code blocks, these imply text that is written into a text editor, the command shell console, or a python terminal/script. Below explains how one can distinguish these in this document. \\

\noindent The following denotes a line of text (or lines of text) that are to be edited in a text editor.
\begin{textbox}
<# A variable name that can be changes to a specific value>
@VARIABLE_NAME@ = "Variable Value"
\end{textbox}

\noindent These can also be shell scripts in a certain language:
\begin{bashbox}
#!/usr/bin/bash
# Find out which console you are using
echo $0
# Set environment Hello
export Hello="Hello"
\end{bashbox}
\begin{cshbox}
#!/usr/bin/tcsh
# Find out which console you are using
echo $0
# Set environment Hello
setenv Hello "Hello"
\end{cshbox}

\noindent The following denotes a command to run in the command shell console 
\begin{cmdbox}
cd (*$\sim$*)/Downloads
\end{cmdbox}

\noindent The following denotes a command line print out
\begin{cmdboxprint}
 This is a print out in the command line
 produced by using the echo command
\end{cmdboxprint}

\noindent The following denotes a python terminal or python script
\begin{pythonbox}
import numpy as np
print("Hello world")
print("{0} seconds".format(np.sqrt(25)))
\end{pythonbox}
